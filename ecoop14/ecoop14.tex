\documentclass[runningheads,a4paper]{llncs}

\usepackage{amssymb}
\usepackage{listings}
\setcounter{tocdepth}{3}
\usepackage{graphicx}

\usepackage{url}
\newcommand{\keywords}[1]{\par\addvspace\baselineskip
\noindent\keywordname\enspace\ignorespaces#1}

\newcommand{\todo}[1]{\textbf{[TODO: #1]}}
\newcommand{\keyw}[1]{\texttt{\textbf{#1}}}

\begin{document}

\title{Type-Directed, Whitespace-Delimited Parsing for Embedded DSLs}

\author{Cyrus Omar \and Benjamin Chung \and Darya Kurilova \and\\
Alex Potanin$^{1}$ \and Jonathan Aldrich}
\institute{Carnegie Mellon University\\
\path|{comar, bwchung, darya, aldrich}@cs.cmu.edu|
and
\path|alex@ecs.vuw.ac.nz|$^{1}$}

\maketitle


\begin{abstract}
%The abstract should summarize the contents of the paper and should
%contain at least 70 and at most 150 words. It should be written using the
%\emph{abstract} environment.

Domain-specific languages improve ease-of-use, expressiveness and
verifiability, 
but defining and using different 
DSLs within a single application remains difficult.  
We introduce an approach for embedded DSLs where 1) whitespace delimits DSL-governed blocks, and 2) the parsing and type checking phases occur in tandem so that the expected type of the block determines which domain-specific parser governs that block.
We argue that this approach occupies a sweet spot, providing   
high expressiveness and ease-of-use while maintaining safe composability. We introduce the design, provide examples and describe an ongoing implementation of this strategy in the Wyvern programming language. We also discuss how a more conventional keyword-directed strategy for parsing of DSLs can arise as a special case of this type-directed strategy.

\keywords{We would like to encourage you to list your keywords within
the abstract section}
\end{abstract}

\section{Introduction}

Capabilities have been recently gaining new attention as a promising mechanism for controlling access to resources in systems and languages~\cite{miller03,drossopoulou07,dimoulas14,devriese16}.
A \textit{capability} is an unforgeable token that can be used by its bearer to perform some operation on a resource \cite{dennis66}.
In a \textit{capability-safe} language, all resources must be accessed through capabilities, and a resource-access capability must be obtained from a source that already has it: ``only connectivity begets connectivity'' \cite{miller03}.
For example, a \kwat{logger} component that provides a logging service would need to be initialized with a capability providing the ability to append to the log file.

Capability-safe languages thus prohibit the \textit{ambient authority} that is present in non-capability-safe languages.
An implementation of a \kwat{logger} in OCaml or Java, for example, does not need to be passed a capability at initialization time; it can simply import the appropriate file-access library and open the log file for appending itself.
Critically, a malicious implementation of such a component could also delete the log, read from another file, or exfiltrate logging information over the network.
Other mechanisms such as sandboxing can be used to limit the effects of such malicious components, but recent work has found that Java's sandbox (for example) is difficult to use and is therefore often misused~\cite{coker15, maass16}.

In practice, reasoning about resource use in capability-based systems is mostly done informally.
But if capabilities are useful for \textit{informal} reasoning, shouldn't they also aid in \textit{formal} reasoning?
Recent work by \citet{drossopoulou07} sheds some light on this question by presenting a logic that formalizes capability-based reasoning about trust between objects.
Two other trains of work, rather than formalize capability-based reasoning itself, reason about how capabilities may be used.
\citet{dimoulas14} developed a formalism for reasoning about which components may use a capability and which may influence (perhaps indirectly) the use of a capability.
\citet{devriese16} formulate an effect parametricity theorem that limits the effects of an object based on the capabilities it possesses, and then use logical relations to reason about capability use in higher-order settings.
Overall, this prior work presents new formal systems for reasoning about capability use, or reasoning about new properties using capabilities.

We are interested in a different question: can capabilities be used to enhance formal reasoning that is currently done without relying on capabilities?
In other words, what value do capabilities add to existing formal reasoning?

To answer this question, we decided to pick a simple and practical formal reasoning system, and see if capability-based reasoning could help.
A natural choice for our investigation is effect systems~\cite{nielson99}.
Effect systems are a relatively simple formal reasoning approach, and keeping things simple will help to make the difference made by capabilities more obvious.
Furthermore, effects have an intuitive link to capabilities: in a system that uses capabilities to protect resources, an expression can only have an effect on a resource if it is given a capability to do so.

How could capabilities help with effects?
One challenge to the wider adoption of effect systems is their annotation overhead~\cite{rytz12}.
For example, Java's checked exception system is a kind of effect system, and is often criticised for being cumbersome~\cite{Kiniry2006}.
Effect inference can be used to reduce the annotations required~\cite{koka14}, but this has significant drawbacks: understanding error messages that arise through effect inference requires a detailed understanding of the internal structure of the code, not just its interface.
Capabilities are a promising alternative for reducing the overhead of effect annotations, as suggested by the following example:

\begin{lstlisting}
import log : String -> Unit with effect File.write

e
\end{lstlisting}

In the code above, written in a capability-safe language, what can we infer about the effects on resources (e.g. the file system or network) of evaluating \kwat{e}?
Since we are in a capability-safe language, \kwat{e} has no ambient authority, and so the only way it can have any effect on resources is via the \kwat{log} function it imports.
Note that this reasoning requires nothing about \kwat{e} other than that it obeys the rules of a capability-safe language; in particular, we don't require any effect annotations within \kwat{e}, and we don't need to analyze its structure as an effect inference would have to do.
Also note that \kwat{e} might be arbitrarily large, perhaps consisting of an entire program that we have downloaded from a source that we trust enough to allow it to write to a log, but that we don't trust to access any other resources.
Thus in this scenario, capabilities can be used to reason ``for free'' about the effect of a large body of code based on a few annotations on the components it imports.

The central intuition that we formalize in this paper is this: the effect of an unannotated expression can be given a bound based on the effects latent in variables that are in scope.
Of course, there are challenges to solve on the way, most notably involving higher-order programs: how can we generalize this intuition if \kwat{log} takes a higher-order argument?
If \kwat{e} evaluates not to unit but to a function, what can we infer about that function's effects?

In the remainder of this paper, we will formalize these ideas and explore these questions.
To demonstrate, we introduce a pair of languages: the operation calculus $\opercalc$ (Section 3) and the capability calculus $\epscalc$ (Section 4).
$\opercalc$ is a typed lambda calculus with a simple notion of capabilities and their operations, in which all code is effect-annotated.
Relaxing this requirement, we then introduce $\epscalc$, which permits the nesting of unannotated code inside annotated code in a controlled, capability-safe manner.
A safe inference about the unannotated code can be made by inspecting the capabilities passed into it from its annotated surroundings.
We then show how $\epscalc$ can model practical situations, presenting a range of examples to illustrate the benefits of a capability-flavoured effect system.

Throughout this paper we give motivating examples in a capability-safe language very similar to \textit{Wyvern} as presented by \citet{nistor13}.
A more thorough discussion of the language and how it can be translated into the calculi is given in section 4.

% !TEX root = plas14.tex

\section{Introduction / Motivation}

Weaknesses categorized as \emph{injection vulnerabilities} have topped
rankings of the most critical web application vulnerabilities for
several years in a row amongst industry groups \cite{owasp, cwsans}.
Web application vulnerabilities achieve top rankings by being quite
onerous. Injection vulnerabilities are tedious and
difficult to prevent, and are often easily exploitable with
severe technical consequences.

\todo{handlebars stuff here or close to this?}

Injection vulnerabilities can occur anywhere that user input may be erroneously
executed as code. In essence, an injection attack on a web application
occurs when a user---anyone with access to the application and a
vulnerable input field---submits maliciously crafted input that has
been made to appear and execute as code in a specific browser context.
The term \emph{browser context} refers to the browser's parsing state
when processing a string. If the browser is parsing a string as HTML,
it is in an HTML context. Likewise, if the browser has determined
a string is CSS and is processing it using its CSS engine, we call
the context the CSS context.

To ensure user inputs are not erroneously executed as a code, the inputs
must be properly encoded before they are placed in a vulnerable context. Using 
the correct encoding for a given context ensures the input will be recognized as 
data and not code. Traditionally, developers have had to create sanitizers to perform 
this encoding process themselves and then must place each sanitizer manually within
their application's code to prevent injection vulnerabilities. Many researchers have
noted that developers have significant trouble making and placing sanitizers manually \todo{cite these studies}. While the problem of creating correct sanitizers has largely been solved by the use of mainstream security libraries and frameworks, there is no accepted solution for sanitizer placement.

To prevent mistakes in placing sanitizers, several researchers have presented approaches 
that automatically place sanitizers in the correct locations for existing programs. 
The earliest fully automatic approach was presented by Samuel \emph{et al}., which
generates type constraints for contexts and solves those constraints to automatically place
sanitizers in small Google Closure programs \cite{Samuel:2011}. This 
approach may not scale to large, complicated applications due to its dependency on 
a custom constraint solver. Livshits \emph{et al}. instead use a dataflow analysis that
statically places sanitizers in existing code where possible \cite{Livshits:2013}. Where the 
type of the input cannot be determined statically, this approach must dynamically instrument
the program to determine the proper sanitizer outside of the program's existing logic. This
approach also cannot currently handle cases where inputs of different types are concatenated.

Both approaches support source sensitivity and sink sensitivity. \emph{Sink sensitivity} encodes user inputs based on the sink that consumes them, but cannot on its own allow different security policies to be enforced based on the source of input. \emph{Source sensitivity} allows for policies to be enforced based on the source of the input. 

We present a scalable approach that utilizes Type-Specific Languages (TSLs) \todo{cite ecoop paper} to allow developers using the Wyvern programming language \todo{cite something more for Wyvern?}
to safely splice user inputs into their webpages. TSLs allow the developer to
specify their webpages using a \todo{fill in a high level TSL and injection prevention description}.

Our approach makes the following contributions:

\todo{My wording here sucks...}
\begin{itemize}
\item We safely handle the concatenation of inputs of multiple types by requiring the developer to express their intent through statically checkable type conversion annotations. A failure to do so is a compile error.
\item We do not introduce dynamic checks into programs aside from those inherent in the program's structure at implementation time.
\item Existing programs developed using popular template systems may be integrated without change.
\item Our approach utilizes sink sensitivity by default. As a result, Wyvern applications implement the most secure tactic for preventing injection vulnerabilities in a given context unless the developer intentionally weakens the outcome. The ability to weaken the outcome is how we implement source sensitivity.
\end{itemize}
\subsection{Motivating Example}

\todo{Revise, rewrite, add.}

Web blogs are a convenient
example of such an application. A web blog, or simply a blog, is a
website that promotes discussion and information sharing on a specific
topic or a set of topics and that consists of discrete entries called posts.
Typically, a blog author will write posts by typing up the message
and formatting it in a text field that is available only to the blog's
authors. This message is later published for public consumption. The
post author formats their post by using HTML and CSS. However, the
post author does not need access to the full subset of HTML, and there
is in fact a lot of incentive for blog hosting providers (e.g. Blogger
\cite{blogger}) to limit the HTML/CSS tags that are available. In other words, sometimes the developer wants to override the default of non-executability, but control execution by limiting the set of tags that are interpreted by the browser.

Due to their relative simplicity and ubiquity, we use XSS and SQLi
vulnerabilities as the platform for illustrating our approach. The
approach generalizes to other web application injection vulnerabilities.
Figure \ref{fig:Motivating-Example-Code} shows a simple example that
contains an XSS and SQLi vulnerability in a PHP-like psuedocode. This
sample presents the user with a prompt to enter her username. When
the user enters her username and hits the submit button, a SQL query
is executed to get her last login time from a database table named
UserLog. Her last login time is then printed in a message below
the form, along with the username she entered. 

A SQLi vulnerability exists because the username the user entered
is concatenated into the SQL statement exactly as it was entered.
If the user had entered SQL code as their username instead, the extra
SQL statements they entered would be executed by the database. For
example, entering \texttt{';DROP TABLE UserLog; -{}- }as the username
would cause the UserLog table to be deleted from the database. The
attacker could have just as easily injected SQL to perform operations
such as adding fake data to the database, dumping the contents of
the database to try to steal password hashes, etc. 

The XSS vulnerability is equally simple.
The XSS vulnerability exists because the user name is being directly
written to a context the attacker can leverage to execute Javascript.
An attacker could enter the username \texttt{<script> alert(``XSS!''); </script>},
thus causing a Javascript pop-up to appear. The XSS exploit could
have also read the user's cookie and sent it to a different website,
defaced the website to turn it into a phishing form, or set-up a \texttt{document.onkeypress}
event handler to keylog the webpage. These exploits don't need to
be entered into the form directly.

The form uses the HTTP GET method, which allows the attacker to enter
his exploit into the URL using the \texttt{usr} parameter. As a
result, the attacker could craft a URL that contains the exploit and
trick a user into clicking the malicious link, thus causing the user's
browser to execute the exploit instead.


\begin{figure}
\begin{lstlisting}
let user_input_s = readline_user
let author_input_s = readline_author
let user_input : HTML = parse_user(user_input_s)
let author_input : HTML_A = parse_author(author_input_s)
let page : HTML = ~
   >html
      >title Best Author's Blog
      >body
      >h1 Post Title
      >div
         < admin_input : HTML_A (* htmla_to_html(author_input)*)
      >h2 Comments
      >div
         < user_input
      ... 
\end{lstlisting}
\caption{A Wyvern Example}
\label{fig:wyvern-example}
\end{figure}


\begin{figure}
\begin{lstlisting}
casetype HTML_A
   IFrame of HTMLA
   Div of HTMLA
   ...

casetype HTML
   Div of HTML
   ...
\end{lstlisting}
\caption{Wyvern HTML Type}
\label{fig:wyvern-example}
\end{figure}

% !TEX root = plas14.tex

\section{Approach}

Approach goes here.
% !TEX root = ecoop14.tex

\begin{figure}
\begin{lstlisting}[mathescape]
p $\rightarrow$ 'objtype'$^=$ ID$^>$ NEWLINE$^>$ objdecls$^>$ NEWLINE$^>$ metadatadecl$^>$ NEWLINE$^>$ p$^=$
p $\rightarrow$ 'casetype'$^=$ ID$^>$ NEWLINE$^>$ casedecls$^>$ NEWLINE$^>$ metadatadecl$^>$ NEWLINE$^>$ p$^=$
p $\rightarrow$ e$^=$

metadatadecl $\rightarrow$ 'metadata'$^=$ '='$^>$ e$^>$

e $\rightarrow$ $\overline{\texttt{e}}$$^=$
e $\rightarrow$ $\widetilde{\texttt{e}}$['~']$^=$ NEWLINE$^>$ chars$^>$
e $\rightarrow$ $\widetilde{\texttt{e}}$['new']$^=$ NEWLINE$^>$ d$^>$
e $\rightarrow$ $\widetilde{\texttt{e}}$['case(' $\overline{\texttt{e}}$ ')']$^=$ NEWLINE$^>$ c$^>$

$\overline{\texttt{e}}$ $\rightarrow$ ID$^=$
$\overline{\texttt{e}}$ $\rightarrow$ 'fn'$^=$ ID$^>$ ':'$^>$ type$^>$ '=>'$^>$ $\overline{\texttt{e}}$$^=$
$\overline{\texttt{e}}$ $\rightarrow$ $\overline{\texttt{e}}$$^=$ '('$^>$ $\overline{\texttt{al}}$$^=$ ')'$^>$
$\overline{\texttt{e}}$ $\rightarrow$ '('$^=$ $\overline{\texttt{e}}$$^>$ ','$^>$ $\overline{\texttt{e}}$$^>$ ')'$^>$
$\overline{\texttt{e}}$ $\rightarrow$ 'let'$^=$ ID$^>$ ':'$^>$ type$^>$ '='$^>$ e$^>$ NEWLINE$^>$ $\overline{\texttt{e}}$$^=$
$\overline{\texttt{e}}$ $\rightarrow$ $\overline{\texttt{e}}$$^=$ '.'$^>$ ID$^>$
$\overline{\texttt{e}}$ $\rightarrow$ type$^=$ '.'$^>$ ID$^>$ '('$^>$ $\overline{\texttt{e}}$$^>$ ')'$^>$
$\overline{\texttt{e}}$ $\rightarrow$ $\overline{\texttt{e}}$$^=$ ':'$^>$ type$^>$
$\overline{\texttt{e}}$ $\rightarrow$ 'valAST'$^=$ '('$^>$ $\overline{\texttt{e}}$$^>$ ')'$^>$
$\overline{\texttt{e}}$ $\rightarrow$ type$^=$ '.'$^>$ 'metadata'$^>$
$\overline{\texttt{e}}$ $\rightarrow$ inlinelit$^=$

$\widetilde{\texttt{e}}$[fwd] $\rightarrow$ fwd$^=$
$\widetilde{\texttt{e}}$[fwd] $\rightarrow$ 'fn'$^=$ ID$^>$ ':'$^>$ type$^>$ '=>'$^>$ $\widetilde{\texttt{e}}$[fwd]$^>$
$\widetilde{\texttt{e}}$[fwd] $\rightarrow$ $\widetilde{\texttt{e}}$[fwd]$^=$ '('$^>$ $\overline{\texttt{al}}$$^>$ ')'$^>$
$\widetilde{\texttt{e}}$ $\rightarrow$ '('$^=$ $\overline{\texttt{e}}$$^>$ ','$^>$ $\widetilde{\texttt{e}}$$^>$ ')'$^>$
$\widetilde{\texttt{e}}$ $\rightarrow$ '('$^=$ $\widetilde{\texttt{e}}$$^>$ ','$^>$ $\overline{\texttt{e}}$$^>$ ')'$^>$
$\widetilde{\texttt{e}}$[fwd] $\rightarrow$ 'let'$^=$ ID$^>$ ':'$^>$ type$^>$ '='$^>$ e$^>$ NEWLINE$^>$ $\widetilde{\texttt{e}}$[fwd]$^=$
$\widetilde{\texttt{e}}$[fwd] $\rightarrow$ $\overline{\texttt{e}}$$^=$ '('$^>$ $\widetilde{\texttt{al}}$[fwd]$^>$ ')'$^>$
$\widetilde{\texttt{e}}$[fwd] $\rightarrow$ $\widetilde{\texttt{e}}$[fwd]$^=$ '.'$^>$ ID$^>$
$\widetilde{\texttt{e}}$[fwd] $\rightarrow$ type$^=$ '.'$^>$ ID$^>$ '('$^>$ $\widetilde{\texttt{e}}$[fwd]$^>$ ')'$^>$
$\widetilde{\texttt{e}}$[fwd] $\rightarrow$ $\widetilde{\texttt{e}}$[fwd]$^=$ ':'$^>$ type$^>$
$\widetilde{\texttt{e}}$[fwd] $\rightarrow$ 'valAST'$^=$ '('$^>$ $\widetilde{\texttt{e}}$[fwd]$^>$ ')'$^>$

d $\rightarrow$ $\varepsilon$
d $\rightarrow$ 'val'$^=$ ID$^>$ ':'$^>$ type$^>$ '='$^>$ e$^>$ NEWLINE$^>$ d$^=$
d $\rightarrow$ 'def'$^=$ ID$^>$ '('$^>$ argsig$^>$ ')'$^>$ ':'$^>$ type$^>$ '='$^>$ e$^>$ NEWLINE$^>$ d$^=$

c $\rightarrow$ ID$^=$ '('$^>$ ID$^>$ ')'$^>$ '=>'$^>$ e$^>$
c $\rightarrow$ ID$^=$ '('$^>$ ID$^>$ ')'$^>$ '=>'$^>$ e$^>$ NEWLINE$^>$ c$^=$

$\overline{\texttt{al}}$ $\rightarrow$ $\varepsilon$ | $\overline{\texttt{al}}_{\texttt{nonempty}}$$^=$
$\overline{\texttt{al}}_{\texttt{nonempty}}$ $\rightarrow$ $\overline{\texttt{e}}$$^=$ | $\overline{\texttt{e}}$$^=$ ','$^>$ $\overline{\texttt{al}}_{\texttt{nonempty}}$$^>$

$\widetilde{\texttt{al}}$[fwd] $\rightarrow$ $\widetilde{\texttt{e}}$[fwd]$^=$
$\widetilde{\texttt{al}}$[fwd] $\rightarrow$ $\widetilde{\texttt{e}}$[fwd]$^=$ ','$^>$ $\overline{\texttt{al}}_{\texttt{nonempty}}$$^>$
$\widetilde{\texttt{al}}$[fwd] $\rightarrow$ $\overline{\texttt{e}}$$^=$ ','$^>$ $\widetilde{\texttt{al}}$[fwd]$^>$

inlinelit $\rightarrow$ chars1['`']$^=$ | chars1[''']$^=$ | chars1['"']$^=$ | ...
inlinelit $\rightarrow$ chars2['{', '}']$^=$ | chars2['<', '>']$^=$ | chars2['[', ']']$^=$ | ...
inlinelit $\rightarrow$ numlit$^=$
\end{lstlisting}
\caption{Concrete Syntax}
\label{f-grammar}
\end{figure}


% !TEX root = ecoop14.tex

\section{Implementation}
\label{s:implementation}
The Wyvern implementation is written in Java, based around a custom recursive-descent parser, with self-hosting left to future work.

Whitespace based parsing is implemented with a custom stateful lexer as in Python, producing indent and dedent tokens. The token stream produced by the lexer is then passed into the Wyvern parser. When a language treansition occurs, the Wyvern core parser extracts a substream from the current token stream, using either indent and dedent or any of the TSL delimiters to indicate where the substream should begin or end. This substream is then passed to the extension parser as an argument. By subdividing the token stream, the parsers can avoid complicated issues with delegation of responsibillity caused by a single shared stream. 

In order to invoke the correct extension parser, the Wyvern compiler reqires a typing context to be present when parsing. To implement this, we combine the typechecking and parsing stages of the compiler, so that typechecking happens incrementally as the source is parsed. Once the first stage of parsing is complete and all Wyvern expressions are known, the Wyvern constructs are typechecked. Then, types for TSL blocks are inferred from the local type context, and the associated parsers are invoked in stage 2 on the substreams inside the TSL blocks.

Extension parsers are added though the interpreters Java interop, which allows Wyvern types to be structural subtypes of Java interfaces. Using this system, we convert type metaobjects into Java objects extending the Java Parser interface. Then, they are used just as if they were defined in Java code.
% !TEX root = ecoop14.tex
\section{Related Work}
\label{s:related}

Closely related to our approach of type-driven parsing is a concurrent paper by Ichikawa et al.~\cite{Ichikawa:2014:CUO:2584469.2577092} that presents \textit{protean operators}. The paper describes the \textit{ProteaJ} language, based on Java, which allows a programmer to define flexible
%(with any number of parameters or using any kind of pattern)
operators annotated with named types. Syntactic conflict is resolved by looking at the expected type. Conflicts may still arise when the expected type matches two protean operators; in this case ProteaJ allows the programmer to explicitly disambiguate, as in other systems.  In contrast, by associating parsers with types, our approach avoids all conflicts, achieving a stricter notion of modularity at the cost of some expressiveness (we only consider delimited literals -- these may define operators inside, but we cannot support custom  operator syntax directly at the top level). We also give a type theoretic foundation for our approach.
%Additionally, a programmer is allowed to define operator precedences to help guide the parser in resolving potential conflicts. The implementation of ProteaJ and a case study involving a combination of arithmetic operators and file path literals languages (with a number of obvious syntactic conflicts such as \lstinline{/}) are presented in great detail. Our paper concentrates on the formal aspects of type-specific languages including a formal system and how such approach can be made as general as possible --- providing a nice complement to the work of Ichikawa et al.

%\todo{http://confluence.jetbrains.com/display/Kotlin/Type-safe+Groovy-style+builders} NO?

%\todo{staging parsers}

%\todo{language boxes work discussed at Parsing workshop~\cite{Diekmann:2013}}

Another way to approach language extensibility is to go a level of abstraction above parsing, as is done via metaprogramming and macro facilities, with Scheme and other Lisp-style languages' hygienic macros being the 'gold standard' for hygiene. In those languages, macros are written in the language itself and use its simple syntax -- parentheses universally serve as expression delimiters (although proposals for whitespace as a substitute for parentheses have been made \cite{srfi-49}). Our work is inspired by this flexibility, but aims to support richer syntax as well as maintain a static type discipline. Wyvern's use of types to trigger parsing  avoids the overhead of invoking macros explicitly by name, and makes it easier to compose TSLs declaratively. Static macro systems also exist. For instance, OJ (previously, OpenJava)~\cite{Tatsubori00openjava:a} provides a macro system based on a meta-object protocol, and Backstage Java~\cite{Palmer:2011:BJM:2048066.2048137}, Template Haskell \cite{sheard2002template} and Converge~\cite{Tratt:2008:DSL:1391956.1391958} also employ compile-time meta-programming, the latter with some support for whitespace delimited blocks.  Each of these systems provide macro-style rewriting of source code, but they provide at most limited extension of language parsing. String literals can be reinterpreted, but splicing is not hygienic if this is done.

Other systems aim at providing forms of syntax extension that change the host language, as opposed to our whitespace-delimited approach.  For example, Camlp4 \cite{camlp4} is a preprocessor for OCaml that can be used to extend the concrete syntax of the language with parsers and extensible grammars.  SugarJ \cite{Erdweg:2011:SLL:2048147.2048199} supports syntactic extension of the Java language by adding libraries. Wyvern differs from these approach in that the core language is not extended directly, so conflicts cannot arise at link-time.

Scoping TSLs to expressions of a single type comes at the expense of some flexibility, but we believe that many uses of domain-specific languages are of this form already. A previous approach has considered type-based disambiguation of parse forests for supporting quotation and anti-quotation of arbitrary object languages~\cite{bravenboer2005generalized}. Our work is similar in spirit, but does not rely on generation of parse forests and associates grammars with types, rather than types with grammar productions.  This provides stronger modularity guarantees and is arguably simpler. 
 C\# expression trees \cite{Csharp} are similar in that, when the type of a term is, e.g., \li{Expression<T->T'>}, it is parsed as a quotation. However, like the work just mentioned, this is \emph{specifically} to support quotations. Our work supports quotations as one use case amongst many.
 
Many approaches to syntax extension, such as XJ~\cite{DBLP:conf/scam/ClarkSW08} are keyword-delimited in some form. We believe that a type-directed approach is more seamless and natural, coinciding with how one would build in language support directly. These approaches also differ in that they either do not support hygienic expansion, or have not specified it in the simple manner that we have.

In terms of work on safe language composition, Schwerdfeger and van Wyk~\cite{Schwerdfeger:2009:VCD:1542476.1542499} proposed a solution that make strong safety guarantees provided that the languages comply with certain grammar restrictions, concerning first and follow sets of the host language and the added new languages. It also relied on strongly named entry tokens, as with keyword delimited approaches. Our approach does not impose any such restrictions while still making safety guarantees.%Techniques that limit the kinds of syntax that can be introduced, to guarantee that ambiguities cannot occur, must introduce constraints that limit expressiveness and can be difficult to reason about, and still require disambiguation tokens (e.g. \cite{Schwerdfeger:2009:VCD:1542476.1542499}).


Domain-specific language frameworks and language workbenches, such as Spoofax \cite{KatsVisser2010}, Ens\={o}~\cite{enso} and others~\cite{van1992pregmatic}, also provide a possible solution for the language extension task. They provide support for generating new programming languages and tooling in a modular manner.  The Marco language \cite{lee:2012:marco} similarly provides macro definition at a level of abstraction that is largely independent of the target language. In these approaches, each TSL is \emph{external} relative to the host language; in contrast, Wyvern focuses on \emph{internal} extensibility, improving interoperability and composability.

Ongoing work on projectional editors (e.g., \cite{mps,Diekmann:2013}) uses a special graphical user interface to allow the developer to implicitly mark where the extensions are placed in the code, essentially directly specifying the underlying ASTs. This solution to the language extension problem is of considerable interest to us, but remains relatively understudied formally. It is likely that a type-oriented approach to projectional editing, inspired by that described herein, could be fruitful. 

We were informed by our previous work on Active Code Completion (ACC), which associates code completion palettes with types~\cite{omar2012active}, much as we associate parsers with types. ACC palettes could be used for defining a TSL syntax for types in a complementary manner. In ACC that syntax
is immediately translated to Java syntax at edit time, while this work
integrates with the language, so the syntax is retained with the code. ACC supports more general interaction modes than just textual syntax, situated between our approach and projectional editors.

% !TEX root = ecoop14.tex
%\section{Conclusion} % and Future Work}
%\label{s:conclusion}

%In this paper, we described how extensible parsing in Wyvern makes for
%a solid platform to support whitespace-delimited, type-directed embedded DSLs or \textit{Type-Specific Languages (TSLs)} for short.
%In the
%future, we aim to implement a wide variety of TSLs in Wyvern tweaking
%our approach and implementation thereof to provide a comprehensive example of
%supporting multiple interacting TSLs in a safe and easy-to-use manner.

% \todo{tie features to goals}

% \todo{implementation and validation plans}

\section*{Acknowledgements}
We thank the anonymous reviewers, Joshua Sunshine, Filipe Milit\~ao and Eric Van Wyk for helpful comments and discussions, and acknowledge the support of the United States Air Force Research Laboratory and the National Security Agency lablet contract \#H98230-14-C-0140, as well as the Royal Society of New Zealand Marsden Fund. Cyrus Omar was supported by an NSF Graduate Research Fellowship.


\bibliographystyle{abbrv}
\bibliography{biblio}

\end{document}
