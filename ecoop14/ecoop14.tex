\documentclass[runningheads]{llncs}
\usepackage{amsmath}
\usepackage{latexsym}
\usepackage{verbatim}
\usepackage[T1]{fontenc}
%\usepackage[defaultmono]{droidmono}
\usepackage{proof,amssymb,enumerate}
\usepackage{math-cmds}
\usepackage{listings}
%\setcounter{tocdepth}{3}
%\renewcommand*\ttdefault{txtt}
\usepackage[scaled]{beramono}
\usepackage{xcolor}
\usepackage{graphicx}
\usepackage{url}
\newcommand{\keywords}[1]{\par\addvspace\baselineskip
\noindent\keywordname\enspace\ignorespaces#1}

\def\implies{\Rightarrow}
\newcommand{\todo}[1]{\textbf{[TODO: #1]}}
\newcommand{\keyw}[1]{\textbf{#1}}

\newtheorem{thm}{Theorem}
\newtheorem{dfn}{Definition}

\lstset{tabsize=2, 
basicstyle=\ttfamily\scriptsize, 
commentstyle=\itshape\rmfamily, 
numbers=left, numberstyle=\scriptsize\color{gray}\ttfamily, language=java,moredelim=[il][\sffamily]{?},mathescape=true,showspaces=false,showstringspaces=false,xleftmargin=15pt,escapeinside={(@}{@)}, morekeywords=[1]{let,in,fn,var,type,rec,fold,unfold,letrec,alloc,ref,application,policy,external,component,connects,to,meth,val,where,return,group,by,within,count,connect,with,attr,keyword}}
\lstloadlanguages{Java,VBScript,XML,HTML}


\begin{document}

\title{Safely-Composable Type-Specific Languages}
\author{~}
\institute{~}
%\author{Cyrus Omar \and Darya Kurilova \and Ligia Nistor \and Benjamin Chung \and\\
%Alex Potanin$^{1}$ \and Jonathan Aldrich}
%\institute{Carnegie Mellon University\todo{Should Alex's affiliation be CMU too?}\\

%NB! Alex says: I don't mind either way. CMU (since they paid me when I was there!)
%or VUW (Victoria University of Wellington) as long as my
%email stays: alex@ecs.vuw.ac.nz . Thanks!

%\texttt{\scriptsize \{comar, darya, lnistor, bwchung, aldrich\}@cs.cmu.edu}
%and
%$^{1}$\texttt{\scriptsize alex@ecs.vuw.ac.nz}}

\maketitle

\begin{abstract}
%The abstract should summarize the contents of the paper and should
%contain at least 70 and at most 150 words. It should be written using the
%\emph{abstract} environment.
Programming languages often include specialized notation for common datatypes (e.g. lists) and some also build in support for specific specialized datatypes (e.g. regular expressions), but user-defined types must use general-purpose notations. Frustration with this causes developers to use strings, rather than structured representations, with alarming frequency, leading to correctness, performance, security and usability  problems.
Allowing developers to modularly extend a language with new notations could help address these issues. Unfortunately, prior mechanisms either lack expressive power or suffer from composability problems: individually-unambiguous extensions introduce ambiguities if used together. We introduce an approach where user-defined types can define \emph{type-specific languages} that determine how \emph{generalized literals}, containing arbitrary syntax, are parsed specifically when they appear where an expression of that type is expected. This type-driven dispatch protocol guarantees safe composability. We provide evidence in support of  this approach and specify it with a bidirectional type system for an emerging language: Wyvern.

%
%Domain-specific languages can improve ease-of-use, expressiveness and verifiability, but defining and using different DSLs within a single application remains difficult.
%
%We introduce an approach for embedding DSLs in a common host language where the type of a piece of domain-specific code can specify which grammar governs it. Because this grammar is type-specific, but the block is delimited by the host language, we can guarantee that link-time conflicts cannot arise. These grammars can recursively include top-level expressions using special entry tokens that guarantee that the composition of the type-specific language and the host language is also sound. We argue that this approach occupies a previously-unexplored sweet spot providing high expressiveness and ease-of-use while guaranteeing safety. We introduce the design, provide examples, sketch the safety theorems and describe an ongoing implementation of this strategy in the Wyvern programming language.
%
%Domain-specific languages improve ease-of-use, expressiveness and
%verifiability, 
%but defining and using different 
%DSLs within a single application remains difficult.  
%We introduce an approach for embedded DSLs where 1) whitespace delimits DSL-governed blocks, and 2) the parsing and type checking phases occur in tandem so that the expected type of the block determines which domain-specific parser governs that block.
%We argue that this approach occupies a sweet spot, providing   
%high expressiveness and ease-of-use while maintaining safe composability. We introduce the design, provide examples and describe an ongoing implementation of this strategy in the Wyvern programming language. We also discuss how a more conventional keyword-directed strategy for parsing of DSLs can arise as a special case of this type-directed strategy.
%
\keywords{extensible languages; parsing; bidirectional typechecking}
\end{abstract}

\section{Introduction}

Capabilities have been recently gaining new attention as a promising mechanism for controlling access to resources in systems and languages~\cite{miller03,drossopoulou07,dimoulas14,devriese16}.
A \textit{capability} is an unforgeable token that can be used by its bearer to perform some operation on a resource \cite{dennis66}.
In a \textit{capability-safe} language, all resources must be accessed through capabilities, and a resource-access capability must be obtained from a source that already has it: ``only connectivity begets connectivity'' \cite{miller03}.
For example, a \kwat{logger} component that provides a logging service would need to be initialized with a capability providing the ability to append to the log file.

Capability-safe languages thus prohibit the \textit{ambient authority} that is present in non-capability-safe languages.
An implementation of a \kwat{logger} in OCaml or Java, for example, does not need to be passed a capability at initialization time; it can simply import the appropriate file-access library and open the log file for appending itself.
Critically, a malicious implementation of such a component could also delete the log, read from another file, or exfiltrate logging information over the network.
Other mechanisms such as sandboxing can be used to limit the effects of such malicious components, but recent work has found that Java's sandbox (for example) is difficult to use and is therefore often misused~\cite{coker15, maass16}.

In practice, reasoning about resource use in capability-based systems is mostly done informally.
But if capabilities are useful for \textit{informal} reasoning, shouldn't they also aid in \textit{formal} reasoning?
Recent work by \citet{drossopoulou07} sheds some light on this question by presenting a logic that formalizes capability-based reasoning about trust between objects.
Two other trains of work, rather than formalize capability-based reasoning itself, reason about how capabilities may be used.
\citet{dimoulas14} developed a formalism for reasoning about which components may use a capability and which may influence (perhaps indirectly) the use of a capability.
\citet{devriese16} formulate an effect parametricity theorem that limits the effects of an object based on the capabilities it possesses, and then use logical relations to reason about capability use in higher-order settings.
Overall, this prior work presents new formal systems for reasoning about capability use, or reasoning about new properties using capabilities.

We are interested in a different question: can capabilities be used to enhance formal reasoning that is currently done without relying on capabilities?
In other words, what value do capabilities add to existing formal reasoning?

To answer this question, we decided to pick a simple and practical formal reasoning system, and see if capability-based reasoning could help.
A natural choice for our investigation is effect systems~\cite{nielson99}.
Effect systems are a relatively simple formal reasoning approach, and keeping things simple will help to make the difference made by capabilities more obvious.
Furthermore, effects have an intuitive link to capabilities: in a system that uses capabilities to protect resources, an expression can only have an effect on a resource if it is given a capability to do so.

How could capabilities help with effects?
One challenge to the wider adoption of effect systems is their annotation overhead~\cite{rytz12}.
For example, Java's checked exception system is a kind of effect system, and is often criticised for being cumbersome~\cite{Kiniry2006}.
Effect inference can be used to reduce the annotations required~\cite{koka14}, but this has significant drawbacks: understanding error messages that arise through effect inference requires a detailed understanding of the internal structure of the code, not just its interface.
Capabilities are a promising alternative for reducing the overhead of effect annotations, as suggested by the following example:

\begin{lstlisting}
import log : String -> Unit with effect File.write

e
\end{lstlisting}

In the code above, written in a capability-safe language, what can we infer about the effects on resources (e.g. the file system or network) of evaluating \kwat{e}?
Since we are in a capability-safe language, \kwat{e} has no ambient authority, and so the only way it can have any effect on resources is via the \kwat{log} function it imports.
Note that this reasoning requires nothing about \kwat{e} other than that it obeys the rules of a capability-safe language; in particular, we don't require any effect annotations within \kwat{e}, and we don't need to analyze its structure as an effect inference would have to do.
Also note that \kwat{e} might be arbitrarily large, perhaps consisting of an entire program that we have downloaded from a source that we trust enough to allow it to write to a log, but that we don't trust to access any other resources.
Thus in this scenario, capabilities can be used to reason ``for free'' about the effect of a large body of code based on a few annotations on the components it imports.

The central intuition that we formalize in this paper is this: the effect of an unannotated expression can be given a bound based on the effects latent in variables that are in scope.
Of course, there are challenges to solve on the way, most notably involving higher-order programs: how can we generalize this intuition if \kwat{log} takes a higher-order argument?
If \kwat{e} evaluates not to unit but to a function, what can we infer about that function's effects?

In the remainder of this paper, we will formalize these ideas and explore these questions.
To demonstrate, we introduce a pair of languages: the operation calculus $\opercalc$ (Section 3) and the capability calculus $\epscalc$ (Section 4).
$\opercalc$ is a typed lambda calculus with a simple notion of capabilities and their operations, in which all code is effect-annotated.
Relaxing this requirement, we then introduce $\epscalc$, which permits the nesting of unannotated code inside annotated code in a controlled, capability-safe manner.
A safe inference about the unannotated code can be made by inspecting the capabilities passed into it from its annotated surroundings.
We then show how $\epscalc$ can model practical situations, presenting a range of examples to illustrate the benefits of a capability-flavoured effect system.

Throughout this paper we give motivating examples in a capability-safe language very similar to \textit{Wyvern} as presented by \citet{nistor13}.
A more thorough discussion of the language and how it can be translated into the calculi is given in section 4.

% !TEX root = plas14.tex

\section{Introduction / Motivation}

Weaknesses categorized as \emph{injection vulnerabilities} have topped
rankings of the most critical web application vulnerabilities for
several years in a row amongst industry groups \cite{owasp, cwsans}.
Web application vulnerabilities achieve top rankings by being quite
onerous. Injection vulnerabilities are tedious and
difficult to prevent, and are often easily exploitable with
severe technical consequences.

\todo{handlebars stuff here or close to this?}

Injection vulnerabilities can occur anywhere that user input may be erroneously
executed as code. In essence, an injection attack on a web application
occurs when a user---anyone with access to the application and a
vulnerable input field---submits maliciously crafted input that has
been made to appear and execute as code in a specific browser context.
The term \emph{browser context} refers to the browser's parsing state
when processing a string. If the browser is parsing a string as HTML,
it is in an HTML context. Likewise, if the browser has determined
a string is CSS and is processing it using its CSS engine, we call
the context the CSS context.

To ensure user inputs are not erroneously executed as a code, the inputs
must be properly encoded before they are placed in a vulnerable context. Using 
the correct encoding for a given context ensures the input will be recognized as 
data and not code. Traditionally, developers have had to create sanitizers to perform 
this encoding process themselves and then must place each sanitizer manually within
their application's code to prevent injection vulnerabilities. Many researchers have
noted that developers have significant trouble making and placing sanitizers manually \todo{cite these studies}. While the problem of creating correct sanitizers has largely been solved by the use of mainstream security libraries and frameworks, there is no accepted solution for sanitizer placement.

To prevent mistakes in placing sanitizers, several researchers have presented approaches 
that automatically place sanitizers in the correct locations for existing programs. 
The earliest fully automatic approach was presented by Samuel \emph{et al}., which
generates type constraints for contexts and solves those constraints to automatically place
sanitizers in small Google Closure programs \cite{Samuel:2011}. This 
approach may not scale to large, complicated applications due to its dependency on 
a custom constraint solver. Livshits \emph{et al}. instead use a dataflow analysis that
statically places sanitizers in existing code where possible \cite{Livshits:2013}. Where the 
type of the input cannot be determined statically, this approach must dynamically instrument
the program to determine the proper sanitizer outside of the program's existing logic. This
approach also cannot currently handle cases where inputs of different types are concatenated.

Both approaches support source sensitivity and sink sensitivity. \emph{Sink sensitivity} encodes user inputs based on the sink that consumes them, but cannot on its own allow different security policies to be enforced based on the source of input. \emph{Source sensitivity} allows for policies to be enforced based on the source of the input. 

We present a scalable approach that utilizes Type-Specific Languages (TSLs) \todo{cite ecoop paper} to allow developers using the Wyvern programming language \todo{cite something more for Wyvern?}
to safely splice user inputs into their webpages. TSLs allow the developer to
specify their webpages using a \todo{fill in a high level TSL and injection prevention description}.

Our approach makes the following contributions:

\todo{My wording here sucks...}
\begin{itemize}
\item We safely handle the concatenation of inputs of multiple types by requiring the developer to express their intent through statically checkable type conversion annotations. A failure to do so is a compile error.
\item We do not introduce dynamic checks into programs aside from those inherent in the program's structure at implementation time.
\item Existing programs developed using popular template systems may be integrated without change.
\item Our approach utilizes sink sensitivity by default. As a result, Wyvern applications implement the most secure tactic for preventing injection vulnerabilities in a given context unless the developer intentionally weakens the outcome. The ability to weaken the outcome is how we implement source sensitivity.
\end{itemize}
\subsection{Motivating Example}

\todo{Revise, rewrite, add.}

Web blogs are a convenient
example of such an application. A web blog, or simply a blog, is a
website that promotes discussion and information sharing on a specific
topic or a set of topics and that consists of discrete entries called posts.
Typically, a blog author will write posts by typing up the message
and formatting it in a text field that is available only to the blog's
authors. This message is later published for public consumption. The
post author formats their post by using HTML and CSS. However, the
post author does not need access to the full subset of HTML, and there
is in fact a lot of incentive for blog hosting providers (e.g. Blogger
\cite{blogger}) to limit the HTML/CSS tags that are available. In other words, sometimes the developer wants to override the default of non-executability, but control execution by limiting the set of tags that are interpreted by the browser.

Due to their relative simplicity and ubiquity, we use XSS and SQLi
vulnerabilities as the platform for illustrating our approach. The
approach generalizes to other web application injection vulnerabilities.
Figure \ref{fig:Motivating-Example-Code} shows a simple example that
contains an XSS and SQLi vulnerability in a PHP-like psuedocode. This
sample presents the user with a prompt to enter her username. When
the user enters her username and hits the submit button, a SQL query
is executed to get her last login time from a database table named
UserLog. Her last login time is then printed in a message below
the form, along with the username she entered. 

A SQLi vulnerability exists because the username the user entered
is concatenated into the SQL statement exactly as it was entered.
If the user had entered SQL code as their username instead, the extra
SQL statements they entered would be executed by the database. For
example, entering \texttt{';DROP TABLE UserLog; -{}- }as the username
would cause the UserLog table to be deleted from the database. The
attacker could have just as easily injected SQL to perform operations
such as adding fake data to the database, dumping the contents of
the database to try to steal password hashes, etc. 

The XSS vulnerability is equally simple.
The XSS vulnerability exists because the user name is being directly
written to a context the attacker can leverage to execute Javascript.
An attacker could enter the username \texttt{<script> alert(``XSS!''); </script>},
thus causing a Javascript pop-up to appear. The XSS exploit could
have also read the user's cookie and sent it to a different website,
defaced the website to turn it into a phishing form, or set-up a \texttt{document.onkeypress}
event handler to keylog the webpage. These exploits don't need to
be entered into the form directly.

The form uses the HTTP GET method, which allows the attacker to enter
his exploit into the URL using the \texttt{usr} parameter. As a
result, the attacker could craft a URL that contains the exploit and
trick a user into clicking the malicious link, thus causing the user's
browser to execute the exploit instead.


\begin{figure}
\begin{lstlisting}
let user_input_s = readline_user
let author_input_s = readline_author
let user_input : HTML = parse_user(user_input_s)
let author_input : HTML_A = parse_author(author_input_s)
let page : HTML = ~
   >html
      >title Best Author's Blog
      >body
      >h1 Post Title
      >div
         < admin_input : HTML_A (* htmla_to_html(author_input)*)
      >h2 Comments
      >div
         < user_input
      ... 
\end{lstlisting}
\caption{A Wyvern Example}
\label{fig:wyvern-example}
\end{figure}


\begin{figure}
\begin{lstlisting}
casetype HTML_A
   IFrame of HTMLA
   Div of HTMLA
   ...

casetype HTML
   Div of HTML
   ...
\end{lstlisting}
\caption{Wyvern HTML Type}
\label{fig:wyvern-example}
\end{figure}

% !TEX root = plas14.tex

\section{Approach}

Approach goes here.
% !TEX root = ecoop14.tex
\section{Statics}
\todo{perhaps the figures in approach.tex could be used in this section with the reference to the motivating example (Darya)}

\todo{This section on Statics is going to go and simply be integrated into approach in the right places, right? Right now it is just a list of figures and some text descrbining parts of them, not a coherent section as such, right?}

\begin{figure}[t]
\centering
\begin{minipage}{.53\textwidth}
  \centering
\[
\begin{array}{ll}
\keyw{casetype}\ & Exp=\\
& \ \ \ Var\ of\ ID\\
& \bnfalt \ Lam\ of\ ID\ *\ Ty\ *\ Exp\\
& \bnfalt \ App\ of\ Exp\ *\ Exp\\
& \cdots\\
& \bnfalt \ FromTS\ of\ Exp\ *\ Exp\\
& \bnfalt \ Error 
\end{array}
\]
\end{minipage}%
\begin{minipage}{.47\textwidth}
  \centering
\[
\begin{array}{ll}
\keyw{casetype}\ & Ty=\\
& \ \ \ Var\ of\ ID\\
& \bnfalt \ Arrow\ of\ Ty*Ty\\
\\
\\
\\
\\
\end{array}
\]
\end{minipage}
\caption{Syntax Trees of Expressions and Types}
\label{fig:synExpTy}
\end{figure}


\begin{figure}
\centering
\[
\begin{array}{c}

\infer[\textit{RT-objtype}]
          {\renewcommand{\arraystretch}{1}
	    \begin{array}{r}
	    \Delta; \Gamma \vdash  \keyw{objtype}~ t~=\{{\omega}, \keyw{metaobject}=e\}; \rho: \tau'\leadsto\\
            \keyw{objtype}~ t~=\{{\omega}, \keyw{metaobject}=\hat{e}\}; \hat{\rho}
            \end{array}
       }
	  {\Delta \vdash \omega & \Delta; \Gamma \vdash e \Rightarrow \tau \leadsto \hat{e} & \Delta, t:\{\omega, \hat e:\tau\}; \Gamma \vdash \rho :\tau'\leadsto \hat{\rho} }
	   \\[3ex] 


\infer[\textit{RT-casetype}]
          {\renewcommand{\arraystretch}{1}
	    \begin{array}{r}
	    \Delta; \Gamma \vdash  \keyw{casetype}~ t~=\{\chi, \keyw{metaobject}=e\}; \rho :\tau' \leadsto \\
            \keyw{casetype}~ t~=\{\chi, \keyw{metaobject}=\hat{e}\};\hat{\rho}
            \end{array}
       }
	  {\Delta \vdash \chi & \Delta; \Gamma \vdash e \Rightarrow \tau \leadsto \hat{e} & \Delta, t:\{\chi, \hat e:\tau\}; \Gamma \vdash \rho :\tau'\leadsto \hat{\rho} }
	   \\[3ex] 


\infer[\textit{RT-e}]
	{\Delta; \Gamma \vdash  e:\tau \leadsto \hat{e}} 
	{\Delta; \Gamma \vdash e \Rightarrow \tau \leadsto \hat{e}}\\[3ex]

\infer[\textit{C-decl}]
	{\Delta; \Gamma \vdash  C~\keyw{of}~\tau} 
	{\Delta \vdash \tau   }\\[3ex]

\infer[\textit{C-decls}]
	{\Delta; \Gamma \vdash  \chi_1 \bnfalt \chi_2} 
	{\Delta; \Gamma \vdash \chi_1 & \Delta; \Gamma \vdash \chi_2 & \text{dom}(\chi_1) \intersect \text{dom}(\chi_2) = \emptyset}\\[3ex]

\infer[\textit{O-val}]
	{\Delta; \Gamma \vdash \keyw{val}~ f:\tau \ \texttt{ok} }
	{\Delta \vdash \tau} \\[3ex]
	
\infer[\textit{O-def}]
	{\Delta; \Gamma \vdash \keyw{def}~ m:\tau \ \texttt{ok} }
	{\Delta \vdash \tau } \\[3ex]

\infer[\textit{O-defs}]
	{\Delta; \Gamma \vdash \omega_1\ \omega_2  }
	{\Delta \vdash \omega_1 & \Delta \vdash \omega_2 & \text{dom}(\omega_1) \intersect \text{dom}(\omega_2) = \emptyset } \\[3ex]

\infer[\textit{Syn2Check}]
	{\Delta; \Gamma \vdash  e \Leftarrow \tau \leadsto \hat{e}} 
	{\Delta;\Gamma \vdash e \Rightarrow \tau \leadsto \hat{e}   }\\[3ex]
	
\infer[\textit{T-varx}]
	{\Delta,\Gamma \vdash x\Rightarrow\tau } 
	{x:\tau \in \Gamma }\\[3ex]

\infer[\textit{T-abs}]
	{\Delta; \Gamma \vdash  \boldsymbol\lambda x{:}\tau . e \Leftarrow \tau \rightarrow \tau_1 \leadsto \boldsymbol\lambda x{:}\tau .\hat{e}} 
	{\Delta; \Gamma, x:\tau \vdash e\Leftarrow \tau_1 \leadsto \hat{e}  & \Delta\vdash \tau}\\[3ex]

\infer[\textit{T-appl}]
	{\Delta; \Gamma \vdash  e(e_1) \Rightarrow \tau_2  \leadsto \hat{e}(\hat{e}_1) } 
	{\Delta; \Gamma \vdash e \Rightarrow \tau_1 \rightarrow \tau_2  \leadsto \hat{e}  & \Gamma \vdash e_1 \Leftarrow \tau_1 \leadsto \hat{e}_1 }\\[3ex]

\infer[\textit{T-introcase}]
	{\Delta; \Gamma \vdash  t.C(e) \Rightarrow t  \leadsto t.C(\hat{e}) } 
	{t:\{\chi, e_0:\tau\} \in \Delta & C\ \keyw{of}\ \tau' \in \chi &\Delta; \Gamma \vdash e \Leftarrow \tau'  \leadsto \hat{e}}\\[3ex]

\infer[\textit{T-elimcase}]
	{\Delta; \Gamma \vdash  \keyw{case}~(e)~\{ c \} \Rightarrow \tau'  \leadsto \keyw{case}~(\hat{e})~\{ c \} } 
	{\Delta; \Gamma \vdash e \Rightarrow t  \leadsto \hat{e}  & t:\{ \chi,e_0:\tau\} \in \Delta & c:\chi \Rightarrow \tau'}\\[3ex]

\infer[\textit{T-casehelper1}]
	{\Delta; \Gamma \vdash  C(\chi)\Rightarrow e : C\ \keyw{of}\ \tau \Rightarrow \tau' \leadsto C(\chi)\Rightarrow \hat{e} : C\ \keyw{of}\ \tau} 
	{\Delta; \Gamma, x:\tau \vdash e \Rightarrow \tau' \leadsto \hat{e}}\\[3ex]

\infer[\textit{T-casehelper2}]
	{\Delta; \Gamma \vdash  c_1 \bnfalt c_2: \chi_1 \bnfalt \chi_2 \Rightarrow \tau' } 
	{\Delta; \Gamma \vdash c_1:\chi_1 \Rightarrow \tau' & \Delta; \Gamma \vdash c_2:\chi_2 \Rightarrow \tau'}\\[3ex]


\end{array}
\]
\label{fig:statics1}
\caption{Static Semantics Rules}
\end{figure}

\begin{figure}
\centering
\[
\begin{array}{c}

\infer[\textit{T-new}]
	{\Delta; \Gamma \vdash \keyw{new}\ \{ d \} \Leftarrow  t \leadsto \keyw{new}\ \{\hat d\}}
	{ t:\{\omega, \hat e:\tau\} \in \Delta & \Delta;\Gamma \vdash d \Leftarrow \omega \leadsto \hat d & t\neq TokenStream} \\[3ex]

\infer[\textit{DT-val}]
	{\Delta; \Gamma \vdash \keyw{val}~ f:\tau = e \Leftarrow \keyw{val}~ f:\tau  \leadsto \keyw{val}~ f:\tau = \hat{e}}
	{\Delta \vdash \tau &\Delta; \Gamma \vdash e \Leftarrow \tau \leadsto \hat{e} } \\[3ex]
	
\infer[\textit{DT-def}]
	{\Delta; \Gamma \vdash \keyw{def}~ m:\tau = e \Leftarrow \keyw{def}~ m:\tau \leadsto \keyw{def}~ m:\tau = \hat{e} }
	{\Delta \vdash \tau  & \Delta; \Gamma \vdash e  \Leftarrow \tau \leadsto \hat{e} } \\[3ex]

	
\infer[\textit{DT-defs}]
	{\Delta; \Gamma \vdash d_1\ d_2 \Leftarrow \omega_1\ \omega_2 }
	{\Delta; \Gamma \vdash d_1 \Leftarrow \omega_1 &  \Delta; \Gamma \vdash d_2 \Leftarrow \omega_2 } \\[3ex]


\infer[\textit{T-field}]
	{\Delta; \Gamma \vdash  e.f \Rightarrow \tau' \leadsto \hat{e}.f} 
	{\Delta; \Gamma \vdash e \Rightarrow t \leadsto \hat{e} & t:\{\omega, e_0:\tau\}\in \Delta & \keyw{val}\ f:\tau' \in \omega  }\\[3ex]

 
\infer[\textit{T-def }]
	{\Delta; \Gamma \vdash  e.m \Rightarrow \tau' \leadsto \hat{e}.m} 
	{\Delta; \Gamma \vdash e \Rightarrow t \leadsto \hat{e} & t: \{\omega, e_0:\tau\} \in \Delta & \keyw{def}\ m:\tau' \in \omega }\\[3ex]

\infer[\textit{T-ascribe}]
	{\Delta; \Gamma  \vdash  e:\tau \Rightarrow \tau \leadsto \hat{e}:\tau}
	{\Delta \vdash \tau & \Delta; \Gamma \vdash e \Leftarrow \tau \leadsto \hat{e} } \\[3ex]

\infer[\textit{T-valAST}]
        {\Delta; \Gamma \vdash \keyw{valAST}(e) \Rightarrow Exp \leadsto \keyw{valAST}(\hat{e}) }
	{\Delta; \Gamma \vdash e \Rightarrow \tau \leadsto \hat{e}} \\[3ex]

\infer[\textit{T-metaobject}]
        {\Delta; \Gamma \vdash t.\keyw{metaobject} \Rightarrow \tau   }
	{t:\{\_, e_0:\tau\} \in \Delta} \\[3ex]


\infer[\textit{T-fromTS}]
	  {\Delta; \Gamma' \vdash \keyw{fromTS}[\Gamma](e_1,e_2) \Leftarrow \tau \leadsto \hat{e} }
	  {\renewcommand{\arraystretch}{1}
	    \begin{array}{r}
	    \Delta;\Gamma' \vdash e_1:TokenStream ~~~~~~ \Delta;\Gamma' \vdash e_2:Token\\
            \texttt{parseConcrete(}e_1,e_2\texttt{)}\ \texttt{is}\ e ~~~~~~\Delta; \Gamma', \Gamma \vdash e \Leftarrow \tau \leadsto \hat{e}
            \end{array}
       } \\[3ex]  

\infer[\textit{T-literal}]
	  {\Delta; \Gamma \vdash \lfloor literal \rfloor \Leftarrow t \leadsto \hat{e} }
	  {\renewcommand{\arraystretch}{1}
	    \begin{array}{r}
	    \Delta;\Gamma \vdash t.\keyw{metaobject}.parser\Leftarrow Parser \leadsto \hat{e}_p ~~~~~ \texttt{TokenStream(}\lfloor literal \rfloor \texttt{)}\ \texttt{is}\ \hat{e}_{ts}\\
            \hat{e}_p.parse(\hat{e}_{ts}, Token.EOS(())) \Downarrow_{\Delta} (\hat{e}', \hat e_{ts}') ~~~~~  e \triangleleft_\Gamma \hat{e}'~~~~~ \Delta;\Gamma\vdash e\Leftarrow t \leadsto \hat{e} ~~~~~ \hat e_{ts}'\ \texttt{empty}
            \end{array}
       } \\[3ex]   
\end{array}
\]
\label{fig:statics2}
\caption{Static Semantics Rules 2}
\end{figure}

\begin{figure}
\centering
\[
\infer[\textit{T-new-hat}]
	{\Delta; \Gamma \vdash \keyw{new}\ \{ \hat d \} :  t }
	{ t:\{\omega, \hat e:\tau\} \in \Delta & \Delta;\Gamma \vdash \hat d : \omega}
\]
\caption{Statics for $\hat e$}
\label{fig:staticsHat}
\end{figure}

\begin{figure}[t]
\centering
\begin{minipage}{.53\textwidth}
  \centering
   \[
\begin{array}{c}

\infer[\textit{DExp-Var}]
	{ x \triangleleft_{\Gamma} Exp.Var(\hat{e})   }
	{ ID(x)\ \text{is}\ \hat{e}} \\[3ex]

\infer[\textit{DExp-Lam}]
	{ \boldsymbol\lambda x{:}\tau . e'_1 \triangleleft_{\Gamma} Exp.Lam( \hat{e}, \tau, \hat{e}_1 )  }
	{ID(x)\ \text{is}\ \hat{e} & e'_1 \triangleleft_{\Gamma} \hat{e}_1  } \\[3ex]

\infer[\textit{DExp-App}]
	{ e'_1(e'_2)  \triangleleft_{\Gamma} Exp.App(\hat{e}_1,\hat{e}_2) }
	{ e'_1 \triangleleft_{\Gamma} \hat{e}_1  & e'_2 \triangleleft_{\Gamma} \hat{e}_2   } \\[3ex]

\infer[\textit{DExp-Literal}]
	{ \lfloor literal \rfloor \triangleleft_{\Gamma} Exp.Literal( \hat{e}_{ts} )  }
	{ \text{literal of}\ \hat{e}_{ts}\ \text{is}\ \lfloor literal \rfloor  } \\[3ex]

\infer[\textit{DExp-FromTS}]
          {\renewcommand{\arraystretch}{1}
	    \begin{array}{r}
	    fromTS[\Gamma](e_1,e_2) \triangleleft_{\Gamma}\\
            Exp.FromTS(\hat{e}_1,\hat{e}_2)
            \end{array}
       }
	  {e_1 \triangleleft_{\Gamma} \hat{e}_1 & e_2 \triangleleft_{\Gamma} \hat{e}_2  }
	   \\[3ex] 

\infer[\textit{DTy-Var}]
	{ t \triangleleft Ty.Var(\hat{e})   }
	{ ID(t)\ \text{is}\ \hat{e}} \\[3ex]

\infer[\textit{DTy-Arrow}]
	{ \tau_1 \rightarrow \tau_2 \triangleleft Ty.Arrow(\hat{e}_1,\hat{e}_2 )  }
	{ \tau_1 \triangleleft \hat{e}_1 & \tau_2 \triangleleft \hat{e}_2 } \\[3ex]
   
\end{array}
\]
\label{fig:dereification}
\caption{Dereification Rules}
\end{minipage}%
\vline
\begin{minipage}{.47\textwidth}
  \centering
  \[
\begin{array}{c}
\infer[\textit{RExp-Var}]
	{ x \triangleright Exp.Var(\hat{e})   }
	{ ID(x)\ \text{is}\ \hat{e}} \\[3ex]

\infer[\textit{RExp-Lam}]
	{ \boldsymbol\lambda x{:}\tau . \hat{e}'_1 \triangleright Exp.Lam( \hat{e}, \tau, \hat{e}_1 )  }
	{ID(x)\ \text{is}\ \hat{e} & \hat{e}'_1 \triangleright \hat{e}_1  } \\[3ex]

\infer[\textit{RExp-App}]
	{ \hat{e}'_1(\hat{e}'_2)  \triangleright Exp.App(\hat{e_1},\hat{e}_2) }
	{ \hat{e}'_1 \triangleright \hat{e}_1  & \hat{e}'_2 \triangleright \hat{e}_2   } \\[3ex]

\infer[\textit{RExp-Literal}]
	{ \lfloor literal \rfloor \triangleright Exp.Literal( \hat{e}_{ts} )  }
	{ \text{literal of}\ \hat{e}_{ts}\ \text{is}\ \lfloor literal \rfloor  } \\[3ex]

\infer[\textit{RExp-FromTS}]
          {\renewcommand{\arraystretch}{1}
	    \begin{array}{r}
	    fromTS[\Gamma](e_1,e_2) \triangleright\\
           Exp.FromTS(\hat{e}_1,\hat{e}_2)
            \end{array}
       }
	  {e_1 \triangleright \hat{e}_1 & e_2 \triangleright \hat{e}_2  }
	   \\[3ex] 

\infer[\textit{RTy-Var}]
	{ t \triangleright Ty.Var(\hat{e})   }
	{ ID(t)\ \text{is}\ \hat{e}} \\[3ex]

\infer[\textit{RTy-Arrow}]
	{ \tau_1 \rightarrow \tau_2 \triangleright Ty.Arrow(\hat{e}_1,\hat{e}_2 )  }
	{ \tau_1 \triangleright \hat{e}_1 & \tau_2 \triangleright \hat{e}_2 } \\[3ex]
\end{array}
\]
\label{fig:reification}
\caption{Reification Rules}
\end{minipage}
\end{figure}

The judgement 

\fbox{$\Delta; \Gamma \vdash e\Rightarrow \tau \leadsto \hat{e}$} 
\\
\noindent
means that from the type context $\Delta$ and the variable context $\Gamma$ we synthesize the type $\tau$ for $e$. The  expression $e$ possibly containing $\lfloor literal \rfloor$ forms is transformed into the expression $\hat{e}$ without literals.

The judgement 

\fbox{$\Delta; \Gamma \vdash e \Leftarrow \tau \leadsto \hat{e}$} 

means that we check $e$ against the type $\tau$ and the expression $e$ is transformed into the expression $\hat{e}$. 

The rule \textit{RT-objtype} checks that the declaration of the object type $t$ is well-formed and the type of the expression $e$ is the same as the type of $t$'s metaobject.

The rule \textit{RT-casetype} checks that the declaration of the sum type $t$ is well-formed and the type of the expression $e$ is the same as the type of $t$'s metaobject.

The rule \textit{RT-e} is the corresponding static rule for the third case of the program $\rho$ of the abstract syntax.

The rule \textit{C-decl} checks that the type $\tau$ that is referenced by the name $C$ belongs to the type context $\Delta$.

The rule \textit{C-decls} allows a case type to have multiple cases, where each case will be checked by the rule \textit{C-decl}. We have to make sure that there are no two cases with the same names; we do this by checking that the domains of the $\chi$s are disjoint.

The rule \textit{O-val} checks that the type of a field ($\keyw{val}$) belongs to the type context $\Delta$ and thus makes the declaration of a value well-formed (\texttt{ok}).

The rule \textit{O-def} checks that the type of a method ($\keyw{def}$) belongs to the type context $\Delta$ and thus makes the declaration of a method well-formed (\texttt{ok}).

The rule \textit{O-defs} allows multiple declarations of values $val$ and methods $def$ to appear one after the other. Each declaration will either be checked by the rule \textit{O-val} or \textit{O-def}. We have to make sure that there are no two values or methods with the same name; we do this by checking that the domains of the $\omega$s are disjoint.

The \textit{Syn2Check} rule mediates between synthesis and type checking. This rule states that syntesis is more powerful than type checking.

The rule \textit{T-varx} synthesizes the type of the variable $x$ to be $\tau$, after checking that the variable context $\Gamma$ contains the $x:\tau$ declaration. 

The rule \textit{T-abs} checks the type of the lambda abstraction and states what are the conditions for the type checking to go through.

The rule \textit{T-appl} synthesizes the type of the application of one expression to the other and states what are the premises needed for this synthesis to happen.

The rule \textit{T-introcase} introduces a case of the sum type and syntesizes the type of the resulting expression. To make it simpler, we precede the name of the case by the sum type that includes that case.

The rules \textit{T-elimcase} synthesizes the type of the resulting expression of a particular case of a sum type. Note that all the cases of the same sum type should synthesize to the same type. The rules  \textit{T-casehelper1} and \textit{T-casehelper2} help with the type checking of $c:\chi \Rightarrow \tau'$ in the rule \textit{T-elimcase}. Rule \textit{T-casehelper1} is used when $c$ is of the kind (matches) $C(\chi)\Rightarrow e$ , while rule \textit{T-casehelper2} is used when $c$ is of the kind $c_1 \bnfalt c_2$.

The rule \textit{T-new} checks the type of a \keyw{new} expression. This rule is used to construct a \keyw{new} expression and it follows the convention of bidirectional type systems, where type-checking is used for constructors of types. We need the additional premise $t\neq TokenStream$ because we do not want to allow users to create their own token streams. This premise is related to hygienic macros \cite{DBLP:conf/esop/HermanW08} and that we do not want identifiers (names of fields or methods) defined by users to be shadowed by identifiers defined in a type. We have to differentiate between the two and we do this by annotating the code that comes from the user with $TokenStream$. 
The rule \textit{T-new-hat} from Figure \ref{fig:staticsHat} allows the type $t$ to be $TokenStream$ because we need to allow the compiler to create token streams; the users are the ones who write expressions of the form $e$, while the compiler produces expressions of the form $\hat{e}$ which have already been translated from a specific TSL to Wyvern. 

The rule \textit{DT-val} checks the type of a field \keyw{val} that is instantiated to the expression $e$.

The rule \textit{DT-def} checks the type of a method \keyw{def} that is instantiated to the expression $e$.

The rule \textit{DT-defs} allows for multiple instatiations of fields or methods to take place one after the other. Each instantiation is checked with the rule \textit{DT-val} or \textit{DT-def}.

The rule \textit{T-field} synthesizes the type of the field of an expression. The premise mentions that the field is declared as a value \keyw{val}.

The rule \textit{T-def}  synthesizes the type of the method (\keyw{def}) of an expression. 

The rule \textit{T-ascribe} ascribes (attributes) the type $\tau$ to $e$, after checking in the premise of the rule that the type of $e$ is $\tau$.

The rule \textit{T-valAST} is used to transform an expression $e$ into an abstract syntax tree of type $Exp$. The definition of type $Exp$ is given in Figure \ref{fig:synExpTy}. 

The rule \textit{T-metaobject} synthesizes the type of the \keyw{metaobject} of the type $t$ by checking that $t$ is in the type context $\Delta$ and it has the right type.

The rule \textit{T-fromTS} checks the type of the \keyw{fromTS} expression. Expression $e_1$ represents a token stream, $e_2$ is a delimiter token and $e$ is the concrete expression that $e_1$ parses to using $e_2$ as end delimiter.

The rule \textit{T-literal} is a crucial rule of the our system. The conclusion checks that the $\lfloor literal \rfloor$ expression has the type $t$ and that it is translated to the expression $\hat{e}$ that does not contain a $\lfloor literal \rfloor$ expression. The premise checks that the $parser$ method of the \keyw{metaobject} of the $t$ type is of type $Parser$. Instead of using $t.\keyw{metaobject}.parser$ we use $\hat{e}_p$ in the rest of the rule. The premise continues by denoting the token stream of the $\lfloor literal \rfloor$ expression by $\hat{e}_{ts}$. When the token stream is parsed with the method $parse$ that has as second argument a token used as a signal of the stream ending, it evaluates to $(\hat{e}',\hat{e}'_{ts})$. The expression $\hat{e}'$ is the actual result of the parsing and it does not contain a $\lfloor literal \rfloor$ expression, with $\hat{e}'_{ts}$ being the remainder of the token stream. Since we parse the token stream until the last token, $\hat{e}'_{ts}$ will be empty. Note that only expressions with a hat $\hat{}$ contain a $\lfloor literal \rfloor$ expression. The expression $\hat{e}'$ is then dereificated to the expression $e$. The dereification rules are found in Figure \ref{fig:dereification} and the reification rules are in Figure \ref{fig:reification}. 

Dereification means transforming an expression of type $Exp$ or $Ty$ (from Figure \ref{fig:synExpTy}) into an expression that can be found in the abstract syntax Figure \ref{fig:core2-syntax}. The judgement for dereification is \fbox{$e_1 \triangleleft_{\Gamma} e_2$}, where $e_1$ is a form of the abstract syntax and $e_2$ is of type $Exp$ or $Ty$ . Reification is the reverse of dereification, where an expression found in the abstract syntax is transformed into an expression that the parser generates, of type $Exp$ or $Ty$. The judgement for reification is \fbox{$e_1 \triangleright_{\Gamma} e_2$}, where $e_1$ is a form of the abstract syntax and $e_2$ is of type $Exp$ or $Ty$ .
 Expression $e$ is recursively translated to the $\hat{e}$ expression by using the same rule \textit{T-literal}. This rule might not terminate in the general case because the $parse$ function in the premise might not terminate, but termination of the parsing process is well studied \cite{DBLP:conf/sle/KrishnanW12} and the techniques can be applied to our system. 

\begin{figure}
\centering
\[
\begin{array}{c}
\infer[\textit{Dyn-Meta}]
	{t.\keyw{metaobject} \xmapsto[\Delta]{} e} 
	{t:\{\_,e:\tau \} \in \Delta}\\[3ex]

\infer[\textit{Dyn-valAST1}]
	{\keyw{valAST}(\hat{e}) \xmapsto[\Delta]{} \keyw{valAST}(\hat{e}') } 
	{\hat{e} \xmapsto[\Delta]{} \hat{e}'}
~~~~~~~~
\infer[\textit{Dyn-valAST2}]
	{\keyw{valAST}(\hat{e}) \xmapsto[\Delta]{} \hat{e}' } 
	{\hat{e}\ \text{val} &\hat{e} \triangleright \hat{e}' }\\[3ex]
\end{array}
\]
\label{fig:dynsemantics}
\caption{Dynamic Semantics Rules}
\end{figure}

The most interesting dynamic semantics rules are presented in Figure \ref{fig:dynsemantics} (due to lack of space we do not present all the dynamic semantics rules). The $\keyw{metaobject}$ of a type steps to the expression $e$ representing the metaobject. The two dynamic rules related to $\keyw{valAST}$ describe how to step from an expression of the kind $\keyw{valAST}(\hat{e})$ to an expression of type $Exp$ (of Figure \ref{fig:synExpTy}).



% !TEX root = ecoop14.tex

\section{Implementation}
\label{s:implementation}
The Wyvern implementation is written in Java, based around a custom recursive-descent parser, with self-hosting left to future work.

Whitespace based parsing is implemented with a custom stateful lexer as in Python, producing indent and dedent tokens. The token stream produced by the lexer is then passed into the Wyvern parser. When a language treansition occurs, the Wyvern core parser extracts a substream from the current token stream, using either indent and dedent or any of the TSL delimiters to indicate where the substream should begin or end. This substream is then passed to the extension parser as an argument. By subdividing the token stream, the parsers can avoid complicated issues with delegation of responsibillity caused by a single shared stream. 

In order to invoke the correct extension parser, the Wyvern compiler reqires a typing context to be present when parsing. To implement this, we combine the typechecking and parsing stages of the compiler, so that typechecking happens incrementally as the source is parsed. Once the first stage of parsing is complete and all Wyvern expressions are known, the Wyvern constructs are typechecked. Then, types for TSL blocks are inferred from the local type context, and the associated parsers are invoked in stage 2 on the substreams inside the TSL blocks.

Extension parsers are added though the interpreters Java interop, which allows Wyvern types to be structural subtypes of Java interfaces. Using this system, we convert type metaobjects into Java objects extending the Java Parser interface. Then, they are used just as if they were defined in Java code.
% !TEX root = ecoop14.tex

\section{Discussion}\label{s:discussion}
We have presented a minimal but complete language design that we believe is particularly elegant, practical and theoretically well-motivated. The key to this is our organization of language extensions around types, rather than around grammar fragments.

There are several directions that remain to be explored:
\begin{itemize}
\item TSL Wyvern does not support polymorphic types, like \li{'a list} in our first example. Were we to add support for them, we would expect that the type constructor (\li{list}) would determine the syntax, not the particular type. Thus, we may fundamentally be proposing \emph{type constructor specific languages}.
\item Similarly, TSL Wyvern does not support abstract types. It may be useful to include the ability to associate metadata with an abstract type, much in the same way that we associate metadata with a named type here.
\item TSLs as described here allow one to give an alternative syntax for introductory term forms, but elimination forms cannot be defined directly. There are two directions we may wish to go to support this:
\begin{enumerate}
\item Pattern matching is a powerful feature supported by an increasing number of languages. Pattern syntax is similar to term syntax. It may be possible for a TSL definition to include parse functions for ``literal-like'' forms appearing in patterns, elaborating them to pattern terms rather than expression terms.
\item Keywords are more useful when defining custom elimination forms (e.g. \verb|if| based on \verb|case|). It may be possible to support ``typed syntax macros'' using the same hygiene mechanisms we described here.
\end{enumerate}
\item We do not provide TSLs with the ability to diverge based on the type of a spliced expression. This might be useful if, for example, our HTML TSL wanted to treat spliced strings differently from other spliced HTML terms. For polymorphic types, we might also wish to diverge based on the type index.
\item We may wish to design less restrictive shadowing constraints, so that TSLs can introduce variables directly into the scope of a spliced expression if they explicitly wish to (bypassing the need for the client to provide a function for the TSL to call). The community may wish to discuss whether this is worth the cost in terms of difficulty of determining where a variable has been bound.
\item We need to provide further empirical validation. This may benefit from the integration of TSLs into existing languages other than Wyvern.
\item We need to consider broader IDE support -- custom syntax benefits from custom editor support, and it may be possible to design IDEs that dispatch to type metadata in much the way the typechecker does in this paper. Our informal considerations of existing IDE extension mechanisms suggests that this may be non-trivial.
\end{itemize}

%
%As an example, consider control flow operators like \verb|if|. This can be defined as a polymorphic method of the \verb|bool| type with signature $(\texttt{unit} \rightarrow \alpha, \texttt{unit} \rightarrow \alpha) \rightarrow \alpha$. That is, it takes the two branches as functions and chooses which to invoke based on the value of the boolean, using perhaps a more primitive control flow operator, like case analysis, or even a Church encoding of booleans as functions. In Wyvern, the branches could be packaged together into a type, \verb|IfBranches|, with an associated grammar that accepts the two branches as unwrapped expressions. Thus, \verb|if| could be defined entirely in a library and used as follows: 
%
%\begin{lstlisting}
%if(guard, ~)
%  then
%    <any Wyvern>
%  else
%    <any Wyvern>
%\end{lstlisting}
%
%For methods like \verb|if| where constructing the argument explicitly will almost never be done, it may be useful to mark the method in a way that allows Wyvern to assume it is being called with a TSL argument immediately following its use. This would eliminate the need for the \verb|(~)| portion, supporting even more conventional notation.
% We have not considered this possibility in detail.

%\paragraph{Explicit Delimiters}
%Throughout this paper, DSLs have been delimited by whitespace. This allows arbitrary syntax within DSLs, since no delimiters need to be reserved to indicate the end of the DSL and thus there is no need for escaping internal uses of these delimiters. In cases where DSL expressions are expected to be reasonably short, such as the \lstinline{URL} example, or where delimiters are more natural than whitespace, such as for array or dictionary literals, it may be desirable to support other forms of delimited ``DSL literals''. 
%
%One possible strategy for this is to reserve a number of common delimiter forms, such as quotation marks and  forms of braces, as equivalent DSL literal forms. The traditional meaning of these delimiters, such as quotation marks for strings and square brackets for lists, would then simply be convention in Wyvern. That is, the following expressions, as well as several similar ones, would be precisely equivalent (the programmer could choose the most convenient form, given the enclosed term):
%\begin{verbatim}
%  f("http://github.com/wyvernlang")  
%  f([http://github.com/wyvernlang])
%\end{verbatim}
%
%Alternatively, types could specify the set of permitted delimiters so that conventions can be enforced by the compiler, improving identifiability. We have not yet explored either of these possibilities in detail, nor explored options that allow \emph{arbitrary} type-specified delimiters (a naive strategy for which would require that the first phase of parsing also be type-directed, which we wish to avoid).
%
%\paragraph{Interaction with Subtyping}
%
%The mechanism described here does not consider the case where multiple subtypes of a base type define a grammar. This can be resolved in several ways. Our plan in full Wyvern, which includes subtyping, is to use the \emph{declared} type's grammar (if a subtype's grammar is desired, an explicit type annotation on the tilde can be used). Alternatively, we could attempt to parse against all relevant subtypes, only requiring explicit disambiguation when ambiguities arise.


%% NB! Confusing given our nesting delimiters, so dropped.
%Finally, following Python and some other whitespace delimited
%languages, we may wish to allow parenthesized expressions to avoid the need for
%following the indentation level. This is still subject to debate and,
%as we try to express more and more DSL kinds in Wyvern, we may decide
%to enforce indentation levels even inside the parentheses.
% !TEX root = ecoop14.tex
\section{Related Work}
\label{s:related}

Closely related to our approach of type-driven parsing is a concurrent paper by Ichikawa et al.~\cite{Ichikawa:2014:CUO:2584469.2577092} that presents \textit{protean operators}. The paper describes the \textit{ProteaJ} language, based on Java, which allows a programmer to define flexible
%(with any number of parameters or using any kind of pattern)
operators annotated with named types. Syntactic conflict is resolved by looking at the expected type. Conflicts may still arise when the expected type matches two protean operators; in this case ProteaJ allows the programmer to explicitly disambiguate, as in other systems.  In contrast, by associating parsers with types, our approach avoids all conflicts, achieving a stricter notion of modularity at the cost of some expressiveness (we only consider delimited literals -- these may define operators inside, but we cannot support custom  operator syntax directly at the top level). We also give a type theoretic foundation for our approach.
%Additionally, a programmer is allowed to define operator precedences to help guide the parser in resolving potential conflicts. The implementation of ProteaJ and a case study involving a combination of arithmetic operators and file path literals languages (with a number of obvious syntactic conflicts such as \lstinline{/}) are presented in great detail. Our paper concentrates on the formal aspects of type-specific languages including a formal system and how such approach can be made as general as possible --- providing a nice complement to the work of Ichikawa et al.

%\todo{http://confluence.jetbrains.com/display/Kotlin/Type-safe+Groovy-style+builders} NO?

%\todo{staging parsers}

%\todo{language boxes work discussed at Parsing workshop~\cite{Diekmann:2013}}

Another way to approach language extensibility is to go a level of abstraction above parsing, as is done via metaprogramming and macro facilities, with Scheme and other Lisp-style languages' hygienic macros being the 'gold standard' for hygiene. In those languages, macros are written in the language itself and use its simple syntax -- parentheses universally serve as expression delimiters (although proposals for whitespace as a substitute for parentheses have been made \cite{srfi-49}). Our work is inspired by this flexibility, but aims to support richer syntax as well as maintain a static type discipline. Wyvern's use of types to trigger parsing  avoids the overhead of invoking macros explicitly by name, and makes it easier to compose TSLs declaratively. Static macro systems also exist. For instance, OJ (previously, OpenJava)~\cite{Tatsubori00openjava:a} provides a macro system based on a meta-object protocol, and Backstage Java~\cite{Palmer:2011:BJM:2048066.2048137}, Template Haskell \cite{sheard2002template} and Converge~\cite{Tratt:2008:DSL:1391956.1391958} also employ compile-time meta-programming, the latter with some support for whitespace delimited blocks.  Each of these systems provide macro-style rewriting of source code, but they provide at most limited extension of language parsing. String literals can be reinterpreted, but splicing is not hygienic if this is done.

Other systems aim at providing forms of syntax extension that change the host language, as opposed to our whitespace-delimited approach.  For example, Camlp4 \cite{camlp4} is a preprocessor for OCaml that can be used to extend the concrete syntax of the language with parsers and extensible grammars.  SugarJ \cite{Erdweg:2011:SLL:2048147.2048199} supports syntactic extension of the Java language by adding libraries. Wyvern differs from these approach in that the core language is not extended directly, so conflicts cannot arise at link-time.

Scoping TSLs to expressions of a single type comes at the expense of some flexibility, but we believe that many uses of domain-specific languages are of this form already. A previous approach has considered type-based disambiguation of parse forests for supporting quotation and anti-quotation of arbitrary object languages~\cite{bravenboer2005generalized}. Our work is similar in spirit, but does not rely on generation of parse forests and associates grammars with types, rather than types with grammar productions.  This provides stronger modularity guarantees and is arguably simpler. 
 C\# expression trees \cite{Csharp} are similar in that, when the type of a term is, e.g., \li{Expression<T->T'>}, it is parsed as a quotation. However, like the work just mentioned, this is \emph{specifically} to support quotations. Our work supports quotations as one use case amongst many.
 
Many approaches to syntax extension, such as XJ~\cite{DBLP:conf/scam/ClarkSW08} are keyword-delimited in some form. We believe that a type-directed approach is more seamless and natural, coinciding with how one would build in language support directly. These approaches also differ in that they either do not support hygienic expansion, or have not specified it in the simple manner that we have.

In terms of work on safe language composition, Schwerdfeger and van Wyk~\cite{Schwerdfeger:2009:VCD:1542476.1542499} proposed a solution that make strong safety guarantees provided that the languages comply with certain grammar restrictions, concerning first and follow sets of the host language and the added new languages. It also relied on strongly named entry tokens, as with keyword delimited approaches. Our approach does not impose any such restrictions while still making safety guarantees.%Techniques that limit the kinds of syntax that can be introduced, to guarantee that ambiguities cannot occur, must introduce constraints that limit expressiveness and can be difficult to reason about, and still require disambiguation tokens (e.g. \cite{Schwerdfeger:2009:VCD:1542476.1542499}).


Domain-specific language frameworks and language workbenches, such as Spoofax \cite{KatsVisser2010}, Ens\={o}~\cite{enso} and others~\cite{van1992pregmatic}, also provide a possible solution for the language extension task. They provide support for generating new programming languages and tooling in a modular manner.  The Marco language \cite{lee:2012:marco} similarly provides macro definition at a level of abstraction that is largely independent of the target language. In these approaches, each TSL is \emph{external} relative to the host language; in contrast, Wyvern focuses on \emph{internal} extensibility, improving interoperability and composability.

Ongoing work on projectional editors (e.g., \cite{mps,Diekmann:2013}) uses a special graphical user interface to allow the developer to implicitly mark where the extensions are placed in the code, essentially directly specifying the underlying ASTs. This solution to the language extension problem is of considerable interest to us, but remains relatively understudied formally. It is likely that a type-oriented approach to projectional editing, inspired by that described herein, could be fruitful. 

We were informed by our previous work on Active Code Completion (ACC), which associates code completion palettes with types~\cite{omar2012active}, much as we associate parsers with types. ACC palettes could be used for defining a TSL syntax for types in a complementary manner. In ACC that syntax
is immediately translated to Java syntax at edit time, while this work
integrates with the language, so the syntax is retained with the code. ACC supports more general interaction modes than just textual syntax, situated between our approach and projectional editors.

% !TEX root = ecoop14.tex
%\section{Conclusion} % and Future Work}
%\label{s:conclusion}

%In this paper, we described how extensible parsing in Wyvern makes for
%a solid platform to support whitespace-delimited, type-directed embedded DSLs or \textit{Type-Specific Languages (TSLs)} for short.
%In the
%future, we aim to implement a wide variety of TSLs in Wyvern tweaking
%our approach and implementation thereof to provide a comprehensive example of
%supporting multiple interacting TSLs in a safe and easy-to-use manner.

% \todo{tie features to goals}

% \todo{implementation and validation plans}

\section*{Acknowledgements}
We thank the anonymous reviewers, Joshua Sunshine, Filipe Milit\~ao and Eric Van Wyk for helpful comments and discussions, and acknowledge the support of the United States Air Force Research Laboratory and the National Security Agency lablet contract \#H98230-14-C-0140, as well as the Royal Society of New Zealand Marsden Fund. Cyrus Omar was supported by an NSF Graduate Research Fellowship.


\bibliographystyle{abbrv}
\bibliography{biblio}

\end{document}
