% !TEX root = ecoop14.tex

\section{Implementation}
\label{s:implementation}
As of this writing, our implementation of the techniques described herein is ongoing based on a fork of the public Wyvern implementation. The Wyvern implementation is written in Java, based around a custom recursive-descent parser, with self-hosting left as future work. Our continued use of the Wyvern custom parser preceded our awareness of Adams' formalism (which does not have a public implementation currently). Thus, it is implemented with a stateful lexer as in Python, producing \li{INDENT} and \li{DEDENT} tokens. The token stream produced by the lexer is then passed into the Wyvern parser. When a language transition occurs, the Wyvern core parser extracts a substream from the current token stream, using either \li{INDENT} and \li{DEDENT} or any of the TSL delimiters to indicate where the substream should begin or end. This substream is then passed to the extension parser as an argument. By subdividing the token stream, the parsers can avoid complicated issues with delegation of responsibility caused by a single shared stream. We anticipate shifting to a more elegant system based on what we have specified here by the time the paper is presented. 
Some extension parsers are added though the interpreter's Java interoperability support.

In order to invoke the correct extension, we combine the typechecking and parsing stages of the compiler, so that typechecking happens incrementally as semantically distinct portions of the source are parsed. Once the first stage of parsing is complete and all Wyvern expressions are known, the Wyvern constructs are typechecked. Then, types for TSL blocks are inferred from the local type context, and the associated parsers are invoked in the next stage on the substreams inside the TSL blocks. This process then continued recursively, as TSL blocks can contain Wyvern code that contains TSLs and etc., until all expressions have been parsed and typechecked. This current implementation is as described in this paper.