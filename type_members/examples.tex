\documentclass{llncs}

\usepackage{listings}
\usepackage{amssymb}
\usepackage[margin=.9in]{geometry}
\usepackage{amsmath}
%\usepackage{amsthm}
\usepackage{mathpartir}
\usepackage{color,soul}


\newtheorem{subcase}{SubCase}
\numberwithin{subcase}{case}
\numberwithin{case}{theorem}
\numberwithin{case}{lemma}




\lstdefinestyle{custom_lang}{
  xleftmargin=\parindent,
  showstringspaces=false,
  basicstyle=\ttfamily,
  keywordstyle=\bfseries
}

\lstset{emph={%  
    val, def, type, new, z%
    },emphstyle={\bfseries \tt}%
}

\begin{document}

\section{Type Members Examples}

\subsection{Removing Environment Narrowing}

A central issue in similar type systems thus far is maintaining 
type soundness in the presence of environment narrowing. Our 
proposed solution is to remove environment narrowing altogether.

An environment is narrowed when during reduction a variable is 
found to have a more precise type than originally assumed. This 
is common when passing arguments to a method, or when returning 
from a method call. In fact these are the two areas that environment 
narrowing must be removed. Below is a small example where we attempt to 
remove environment narrowing.
\begin{lstlisting}[mathescape, style=custom_lang]
Food = {z $\Rightarrow$ type IsFood : $\bot$ ... $\top$}

Recipe = {z $\Rightarrow$ type F : $\bot$ .. Food;
               def cook(){ ... } : z.F;}
               
Sushi = {z $\Rightarrow$ type IsFood : $\bot$ .. $\top$
              type IsSushi : $\bot$ .. $\top$}

SushiRecipe = {z $\Rightarrow$ type F : $\bot$ .. Sushi;
                    def cook()
                    {
                      new {z $\Rightarrow$ type IsFood : $\bot$ .. $\top$
                                type IsSushi : $\bot$ .. $\top$}
                    } : z.F;}
\end{lstlisting}
Above we have defined several types. In our current syntax 
it is not possible to define types like this, we just use this 
as short hand during the following examples for brevity. Bellow 
is an example using the above types, demonstrating the evaluation 
of method calls.
\begin{lstlisting}[mathescape, style=custom_lang]
...
def prepareFood (r : Recipe){
  r.cook();
} : Food;

def main(){
  prepareFood(new SushiRecipe());
}: $\top$;
...
\end{lstlisting}
We now look at the evaluation of above \texttt{main} method.
\begin{center}
\texttt{prepareFood(new SushiRecipe());} $\rightarrow$ \texttt{prepareFood($l$);}
\end{center}
where $l$ is the location in 
the store containing the newly created \texttt{SushiRecipe} object.
\begin{center}
\texttt{prepareFood($l$);} $\rightarrow$ ($l\unlhd$\texttt{Recipe.cook()}) 
$\unlhd$ \texttt{Food}
\end{center}
Here we have substituted the argument of the \texttt{prepareFood} method 
into the method body. We attach a persistent upcast to $l$ of the original 
argument type to avoid narrowing. We also do this to the entire returned body 
of the method
\subsection{Path Equality}

\bibliographystyle{plain}
\bibliography{bib}

\end{document} 
