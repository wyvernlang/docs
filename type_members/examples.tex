\documentclass{llncs}

\usepackage{listings}
\usepackage{amssymb}
\usepackage[margin=.9in]{geometry}
\usepackage{amsmath}
%\usepackage{amsthm}
\usepackage{mathpartir}
\usepackage{color,soul}
\usepackage{graphicx}

%\theoremstyle{definition}
%%\newtheorem{case1}{Case1}
\spnewtheorem{casethm}{Case}[theorem]{\itshape}{\rmfamily}
\spnewtheorem{subcase}{Subcase}{\itshape}{\rmfamily}
\numberwithin{subcase}{casethm}
\numberwithin{casethm}{theorem}
\numberwithin{casethm}{lemma}





\lstdefinestyle{custom_lang}{
  xleftmargin=\parindent,
  showstringspaces=false,
  basicstyle=\ttfamily,
  keywordstyle=\bfseries
}

\lstset{emph={%  
    val, def, type, new, z%
    },emphstyle={\bfseries \tt}%
}

\begin{document}

\section{Counter Examples to Preservation}
	\label{s:examples}

\subsection{Term Membership Restriction}
	\label{s:term_mem}
Figure \ref{f:mem} gives the \emph{Membership} judgement. 
$\Gamma \vdash e \ni \sigma$ says that an expression $e$ 
has $\sigma_i$ as a member of its type in environment $\Gamma$. 
There are two rules for membership. By \textsc{M-Path} a 
variable $x$ has a member of type $[x/\texttt{z}]\sigma_i$ 
if $\sigma_i$ is a member of the type of $x$. By \textsc{M-Exp}
an expression $e$ has member $\sigma_i$ if $\sigma_i$ is 
a member of $e$'s type, and \texttt{z} does not occur 
within $\sigma_i$.

This is reasonable since we cannot substitute a non-value 
expression into a selection type such as $\texttt{z}.L$. 
This does however present a counter-example to preservation. 
Given two types $X$ and $Y$,

\begin{lstlisting}[mathescape, style=custom_lang]
Y = {z $\Rightarrow$
     val l : $\top$
     def m : Y(y:Y){
       val a = new {};
       y
     }
    }
\end{lstlisting}
\begin{lstlisting}[mathescape, style=custom_lang]
X = {z $\Rightarrow$
     type L : $\top$ .. $\top$
     val l : z.L
     def m : Y(y:Y){
       y
     }
    }
\end{lstlisting}

the following expression is well typed.
\begin{lstlisting}[mathescape, style=custom_lang]
val x = new X(l = z);
val y = new Y(l = (val z = new {}; z));
y.m(x).l
\end{lstlisting}
We can see that this is well-typed, and in particular that membership 
holds for \texttt{y.m(x)} and \texttt{l}. This reduces to 
\begin{lstlisting}[mathescape, style=custom_lang]
(val a = new {}; x).l
\end{lstlisting}
which is not well-typed since \texttt{val a = new {}; x} has type 
\texttt{X}, and \texttt{X.l} contains a \texttt{z} reference 
\texttt{z.L}.

After patching we allow expressions to maintain original type. 
Expressions are then typed with respect to the original type.
This is given in Figure \ref{f:syntax} as $e \unlhd T$ where 
$T$ is the original type. The patched reduction is given in 
Figure \ref{f:red}. The amended method reduction \textsc{R-Meth} is 
shown below.
\begin{mathpar}
\inferrule
  {\mu \vdash v_1 \leadsto l \\
  	\mu(l) = \{\texttt{z} \Rightarrow ...,m:T(x:S)=e,...\}}
  {\mu \; | \; v_1.m_U(v_2) \;\rightarrow \mu \; | \; [l/\texttt{z},v_2 \unlhd S/x]e \unlhd U}
  \quad (\textsc {R-Meth})
\end{mathpar}
This is very similar to the standard method reduction. The two main 
differences are the introduction of the $path$ function, and the 
inclusion of the original types in the returned expression. For now 
ignore the $path$ function, this is explained in \ref{s:patheq}.

We retrieve the method for the receiver from the store, and substitute 
the method parameter into the body. The method parameter retains it's 
original type ($v_2 \unlhd S$) as does the entire returned body 
$[v_1/\texttt{z},v_2 \unlhd S/x]e \unlhd T$. Using this rule we 
can attempt to evaluate our original example.
\begin{lstlisting}[mathescape, style=custom_lang]
val x = new X(l = z);
val y = new Y(l = (val z = new {}; z));
y.m(x).l
\end{lstlisting}
reduces to
\begin{lstlisting}[mathescape, style=custom_lang]
((val a = new {}; (x $\unlhd$ Y))$\unlhd$ Y).l
\end{lstlisting}
Now we can treat the method body as having type \texttt{Y} and 
we can determine membership of \texttt{l} for \texttt{Y} rather 
than \texttt{X}.

\subsection{Expansion Lost}
For preservation to hold, we need to ensure that types are expandable 
to lists of declarations. This is captured in Figure \ref{f:wfe}. It is 
possible for expansion to be lost due to a combination of environment 
narrowing and intersection types.
First we informally define what we mean by intersection types. An intersection 
type is the subtype intersection of two types. That is
$S \wedge T <: S$ and $S \wedge T <: T$. Structurally this means 
$S \wedge T$ contains a union of the members of $S$ and $T$ taking the 
highest common subtype of any common members. 
Loss of expansion is 
shown in \cite{Amin:2012}, but is briefly covered again below along with 
an example of our proposed fix.
\begin{lstlisting}[mathescape, style=custom_lang]
X = {z $\Rightarrow$
     type A : $\bot$ .. z.B
     type B : $\bot$ .. $\top$
    }
Y = {z $\Rightarrow$
     type A : $\bot$ .. $\top$
     type B : $\bot$ .. z.A
    }
\end{lstlisting}
While both these types are expandable, their intersection is not.
\begin{lstlisting}[mathescape, style=custom_lang]
X $\wedge$ Y = {z $\Rightarrow$
         type A : $\bot$ .. z.B
         type B : $\bot$ .. z.A
        }
\end{lstlisting}
While we could not type this normally, we can create this 
scenario during environmental narrowing that results in 
an a type without an expansion, providing a counter-example 
to preservation.
\begin{lstlisting}[mathescape, style=custom_lang]
W = {z $\Rightarrow$
     type A : $\bot$ .. $\top$
     type B : $\bot$ .. $\top$
    }
\end{lstlisting}
Now we demonstrate how we can construct a well-typed expression 
that becomes ill-formed as the result of evaluation.
\begin{lstlisting}[mathescape, style=custom_lang]
val x = new {z $\Rightarrow$ type K : $\bot$ .. X};
val y = new {z $\Rightarrow$ type L : $\bot$ .. Y};
\end{lstlisting}
We do this by first constructing two objects from the 
\texttt{X} and \texttt{Y} types we previously defined.
Now we construct another object using these types.
\begin{lstlisting}[mathescape, style=custom_lang]
val c = new {z $\Rightarrow$
         def meth1 : $\top$ (i : {z $\Rightarrow$ type L : $\bot$ .. W}){
           new {z $\Rightarrow$
                  def meth2 : $\top$ (j : i.L $\wedge$ x.K){j}
               }
         }
        };
c.meth1(y)
\end{lstlisting}
Looking at the previous expression, we can see it is well-typed, 
and the type \texttt{i.L} $\wedge$ \texttt{x.K} in particular is 
well-formed and expanding.
Evaluating the above expression results in the following.
\begin{lstlisting}[mathescape, style=custom_lang]
new {z $\Rightarrow$ def meth2 : $\top$ (j : y.L $\wedge$ x.K){j}}
\end{lstlisting}
Now the type \texttt{y.L} $\wedge$ \texttt{x.K} gives us the loss of 
expansion we were attempting to construct. In our new patched calculus, 
this problem does not occur because the original type of the \texttt{i} 
parameter is maintained. So if we re-evaluate the original expression 
with our type system, we get the following.
\begin{lstlisting}[mathescape, style=custom_lang]
val d = new {z $\Rightarrow$
             def meth2 : $\top$ (j : (y $\unlhd$ {z $\Rightarrow$ type L : $\bot$ .. W}).L $\wedge$ x.K){j}}
\end{lstlisting}
The type \texttt{(y} $\unlhd$ \texttt{\{z} $\Rightarrow$ 
\texttt{type L :} $\bot$ \texttt{.. W\}).L} $\wedge$ \texttt{x.K} is now 
both well-formed and expanding because it is the original type 
from our original expression.



%\subsection{Expansion Lost Redux}
%	\label{s:term_mem2}
%First we define the following types.
%\begin{lstlisting}[mathescape, style=custom_lang]
%X = {z $\Rightarrow$
%     def m(x : $\top$){var y = new {z $\Rightarrow$}}:$\top$
%    }
%Y = {z $\Rightarrow$
%     type L : $\bot$ .. $\top$
%     def m(x : $\top$){var y = new {z $\Rightarrow$}}:z.L
%    }
%\end{lstlisting}
%Then we attempt to evaluate the following expression.
%\begin{lstlisting}[mathescape, style=custom_lang]
%var a = new {z $\Rightarrow$
%              def meth(x:$\top$){val b = new Y}:X
%            };
%val c = new {z $\Rightarrow$};
%a.meth(c).m(c)
%\end{lstlisting}
%This reduces to \texttt{a.meth(c).m(c)} which has type $\top$. 
%Applying \textsc{R-Meth} we get \texttt{(val b = new Y $\unlhd$ X).m(c)}
%which has type $\top$ and so is still well typed. This eventually reduces 
%to \texttt{[z/b $\unlhd$ X](var y = new \{z $\Rightarrow$\}) $\unlhd$ z.L}.
%Since during reduction on a method call of the form $v \unlhd T$
%we retrieve the return type of the object ($v$) and 
%not the type ($T$), the reduced expression has type \texttt{b $\unlhd$ X.L}. 
%This type is allowed since \texttt{b $\unlhd$ X} is a path, but 
%it does present a case of narrowing.

\subsection{Well-Formedness Lost}

We can also show that the combination of environment narrowing 
and intersection types can result in a loss of type well-formedness. 
For obvious reasons we enforce a subtype relationship on the lower 
and upper bounds of type members. Environment narrowing 
can cause perfectly well-formed types to become ill-formed by breaking 
this subtype relationship during evaluation. First lets define a few 
simple types.
\begin{lstlisting}[mathescape, style=custom_lang]
S = {z $\Rightarrow$
     type A : $\bot$ .. List
    }
T = {z $\Rightarrow$
     type A : Integer .. $\top$
    }
\end{lstlisting}
If we try and use the intersection of \texttt{S} and \texttt{T}
we get. 
\begin{lstlisting}[mathescape, style=custom_lang]
S $\wedge$ T = {z $\Rightarrow$
         type A : Integer .. List
        }
\end{lstlisting}
Clearly this is ill-formed. We might use it to conclude 
that \texttt{Integer} subtypes \texttt{List}. We now show how 
we can construct an example that appears to be well-formed, but 
through evaluation we end up the this exact type. First define 
three more types.
\begin{lstlisting}[mathescape, style=custom_lang]
U = {z $\Rightarrow$
     type A : $\bot$ .. $\top$
    }
X = {z $\Rightarrow$
     type B : $\bot$ .. U
    }
Y = {z $\Rightarrow$
     type B : $\bot$ .. S
    }
\end{lstlisting}
Since \texttt{S} subtypes \texttt{U}, it follows that 
\texttt{Y} subtypes \texttt{X}.
Now consider the following example.
\begin{lstlisting}[mathescape, style=custom_lang]
val a = new {z $\Rightarrow$
              def badNarrow(x : X){
                new {z $\Rightarrow$ type L : $\bot$ .. x.B $\wedge$ T}
              }
            };
val y = new Y;
a.badNarrow(y);
\end{lstlisting}
Initially \texttt{x.B $\wedge$ T} is well-formed when \texttt{x} 
has type \texttt{X}. After a few steps of evaluation we end up with 
\begin{lstlisting}[mathescape, style=custom_lang]
new {z $\Rightarrow$ type L : $\bot$ .. y.B $\wedge$ T}
\end{lstlisting}
\texttt{y.B $\wedge$ T} now gives us the ill-formed type of 
our earlier example.


\subsection{Path Equality}
\label{s:patheq}
The example below illustrates the problem with path equality.
\begin{lstlisting}[mathescape, style=custom_lang]
val b = new {z $\Rightarrow$
             type L : $\top$ .. $\top$
             val l : z.L = b};
val a = new {z $\Rightarrow$
             val i : {z $\Rightarrow$
                      type L : $\bot$ .. $\top$
                      val l : z.L} = b
             def meth : $\top$ (x : z.i.L){x}};
a.meth(a.i.l)
\end{lstlisting}
\texttt{a.i.l} reduces to \texttt{b.l} which has type \texttt{b.L}. 
There is no way for us to ensure that \texttt{b.L} <: \texttt{a.i.L}.
This presents a path equality problem for preservation.

Typing paths correctly requires that we maintain some information 
about the original paths. For this reason we don't evaluate paths 
in the calculus. We still however need to retrieve the correct object 
the path is pointing to during method reduction and object initialization. 
This is done by the \emph{path} function given in Figure \ref{f:path}. 
With a path and a store, we can find the object referenced by that path.
Below is the evaluation of the above example without evaluating paths.
\begin{lstlisting}[mathescape, style=custom_lang]
val b = new {z $\Rightarrow$
             type L : $\top$ .. $\top$
             val l : z.L = b};
val a = new {z $\Rightarrow$
             val i : {z $\Rightarrow$
                      type L : $\bot$ .. $\top$
                      val l : z.L} = b
             def meth : $\top$ (x : z.i.L){x}};
a.meth(a.i.l)
\end{lstlisting}
Reduces to \texttt{a.meth(a.i.l)} which type checks. \texttt{a.meth(a.i.l)}
reduces to \texttt{a.i.l}

\subsection{Path Equality Redux}
\begin{lstlisting}[mathescape, style=custom_lang]
val n = new {z $\Rightarrow$ type X : $\top$ .. $\top$};
val a = new {z $\Rightarrow$
             type Y : {z $\Rightarrow$ type X : $\bot$ .. $\top$} .. {z $\Rightarrow$ type X : $\bot$ .. $\top$}
             val i : z.Y = n};
val b = new {z $\Rightarrow$
             type L : $\bot$ .. {z $\Rightarrow$ type Y : {z $\Rightarrow$ type X : $\bot$ .. $\top$} .. 
                                          {z $\Rightarrow$ type X : $\bot$ .. $\top$}
                                 val i : z.Y}
             val j : z.L = a};
val c = new {z $\Rightarrow$
             val k : {z $\Rightarrow$
                      type L : $\bot$ .. {z $\Rightarrow$ type Y : {z $\Rightarrow$ type X : $\bot$ .. $\top$} .. 
                                                   {z $\Rightarrow$ type X : $\bot$ .. $\top$}
                                          val i : z.Y}
                      val j : z.L} = b
             def m : $\top$ (x : z.k.L){
                 val y = new {z $\Rightarrow$ val l : x.Y = x.i 
                            val m : x.i.X = new {z $\Rightarrow$}}; y
             }};
c.m(b.j)
\end{lstlisting}
%\begin{lstlisting}[mathescape, style=custom_lang]
%c.m(c.k.j) $\rightarrow$ c.m(b.j)
%\end{lstlisting} 
\texttt{b.j} has type \texttt{b.L}, therefore we need 
$\texttt{b.L} <: \texttt{c.k.L}$
\begin{lstlisting}[mathescape, style=custom_lang]
c.m(b.j) $\rightarrow$ c.m(a)
c.m(a) $\rightarrow$ (val y = new {z $\Rightarrow$ val l : (a $\unlhd$ c.k.L).Y = (a $\unlhd$ c.k.L).i 
                            val m : (a $\unlhd$ c.k.L).i.X = new {z $\Rightarrow$}}; y) $\unlhd$ $\top$
\end{lstlisting}

\newpage

\section{Contravariant Lower Bounds and Subtype Transitivity}
Below is a simple example that demonstrates that 
when you allow subtype contravariance on the lower bounds 
of type members in the presence of two environments, 
transitivity can fail, and subsequently so does preservation
due to an ill-typed store. First we consider three types.
\begin{lstlisting}[mathescape, style=custom_lang]
A = {z $\Rightarrow$ type N : $\bot$ .. $\top$}

B = {z $\Rightarrow$ type N : $\bot$ .. $\top$
          def meth1(x : $\top$){return new{z $\Rightarrow$}}:$\top$}
         
S = {z $\Rightarrow$ type L : A .. $\top$
          type M : z.L .. $\top$
          val f : A
          def meth2(x : $\top$){...}:A}
         
T = {z $\Rightarrow$ type L : A .. $\top$
          type M : z.L .. $\top$
          val f : z.L
          def meth2(x : $\top$){...}:z.L}
         
U = {z $\Rightarrow$ type L : B .. $\top$
          type M : z.L .. $\top$
          val f : z.L
          def meth2(x : $\top$){...}:z.L}
\end{lstlisting}
Clearly $\texttt{B} <: \texttt{A}$, and subsequently
$\texttt{S} <: \texttt{T}$ and $\texttt{T} <: \texttt{U}$. 
However \texttt{S} does not subtype \texttt{U}, so subtyping 
is not transitive. The following example demonstrates how this 
can then lead to an ill-formed store. First we define another type
\begin{lstlisting}[mathescape, style=custom_lang]
X = {z $\Rightarrow$ val badField : U}
\end{lstlisting}
Now consider this program.
\begin{lstlisting}[mathescape, style=custom_lang]
new X {val badField = new {z $\Rightarrow$ def createT (x : $\top$){
                                  return new S{val f = new A}
                                }:T}.createT(...)}
\end{lstlisting}
Evaluating this we can see that what initially 
looks like a reasonable program that in fact type 
checks fine actually results in a bad field assignment 
and thus an ill-formed store. The above code fragment 
evaluates to the following.
\begin{lstlisting}[mathescape, style=custom_lang]
new X {val badField = l1.createT(...)}
\end{lstlisting}
Where \texttt{l1} is a location in the store 
that contains the newly created object. At this 
point the store is still fine.
Further evaluation of the method call gives us this.
\begin{lstlisting}[mathescape, style=custom_lang]
new X {val badField = (new S{val f = new A} $\unlhd$ T)}
\end{lstlisting}
We now evaluate the $new A$ and $new S$ expressions to get.
\begin{lstlisting}[mathescape, style=custom_lang]
new X {val badField = (l3 $\unlhd$ T)}
\end{lstlisting}
Where \texttt{l3} and \texttt{l2} (which is not shown) 
point to the new \texttt{S} and \texttt{A} objects 
in the store respectively. Now one more step of completes
the evaluation and results in the ill-formed store.
\begin{lstlisting}[mathescape, style=custom_lang]
l4
\end{lstlisting}
Where \texttt{l4} is the location in the store containing 
the newly created \texttt{X} object. The type for \texttt{badField}
in \texttt{X} is \texttt{U}, while the type of the object 
it is initialized to (\texttt{l3}) is \texttt{S}, which as we showed 
previously is not a subtype of \texttt{U}.

Since this only affects the lower bounds on type members, 
this should never result in an incorrect method or field call, 
however it does violate the restrictions enforced by lower bounds.

\subsection{Proposed Type Wrapper Solution}
As mentioned previously, this does not cause 
a problem with method and field accesses since 
these are always checked against upper bounds which 
will always be above any subtype. In the above 
example the issue arises from the lower bounds 
of the \texttt{L} type member in each type. So 
while \texttt{A} does not subtype \texttt{z.L} 
in \texttt{U}, it does subtype the upper bound of
\texttt{z.L}, and there would be no way to construct 
an example where it does not. So we know that 
programs that are initially well-typed should all 
behave in an expected manner even if they enter 
an ill-typed state during evaluation. In order 
to avoid this ill-typed state, we make use of our
existing syntax, the explicit up-cast of expressions, 
$e \unlhd T$. While transitivity in general does not 
hold, we can try and avoid transitivity altogether.
Below we re-assess the previous example.
\begin{lstlisting}[mathescape, style=custom_lang]
new X {val badField = new {z $\Rightarrow$ def createT (x : $\top$){
                                  return new S{val f = new A}
                                }:T}.createT(...)}
\end{lstlisting}
\begin{center}$\downarrow$\end{center}
\begin{lstlisting}[mathescape, style=custom_lang]
new X {val badField = l1.createT(...)}
\end{lstlisting}
\begin{center}$\downarrow$\end{center}
\begin{lstlisting}[mathescape, style=custom_lang]
new X {val badField = (new S{val f = new A} $\unlhd$ T)}
\end{lstlisting}
\begin{center}$\downarrow$\end{center}
\begin{lstlisting}[mathescape, style=custom_lang]
new X {val badField = (l3 $\unlhd$ T)}
\end{lstlisting}
At this point we would normally strip the type 
\texttt{T} away from the field initializer before 
storing it in the store. Now we leave it there. 
If we were initializing \texttt{badField} with 
a field access, we would insert an explicit 
upcast to the type of the field. Below is an 
example of this.
\begin{lstlisting}[mathescape, style=custom_lang]
new X {val badField = new {z $\Rightarrow$ val fieldT : T}
                             (val fieldT = new S(val f = new A)).fieldT}
\end{lstlisting}
Here we have initialized \texttt{badField} to a 
field call \texttt{fieldT} on a new object. 
Evaluation is similar to the previous example.
\begin{lstlisting}[mathescape, style=custom_lang]
new X {val badField = new {z $\Rightarrow$ val fieldT : T}
                             (val fieldT = l1).fieldT}
\end{lstlisting}
Where \texttt{l1} is the location in the store containing the 
new \texttt{S} object. Now this evaluates to
\begin{lstlisting}[mathescape, style=custom_lang]
new X {val badField = l2.fieldT}
\end{lstlisting}
Where \texttt{l2} is the location in the store 
containing our new object that has the \texttt{fieldT} 
field. Now continuing evaluation, normally we would simply 
retrieve the location stored in \texttt{fieldT} and store
it in \texttt{badField}, but now we rather retrieve 
\texttt{l1} (the location stored in \texttt{fieldT}) 
and explicitly upcast it to the type of \texttt{fieldT}, 
\texttt{T}. Continuing evaluation we get the following 
\begin{lstlisting}[mathescape, style=custom_lang]
l3
\end{lstlisting}
Where \texttt{l3} is the location containing our new
\texttt{X} object. The difference is that the expression 
stored in \texttt{badField} is \texttt{l2 $\unlhd$ T}, 
which means that the store is not ill-typed.



\bibliographystyle{plain}
\bibliography{bib}

\end{document}