\section{Environment Narrowing and Transitiviy}

The full DOT calculus contains several powerful concepts that each individually introduce complexity to any proof of soundness. These concepts are environment narrowing, the modification (specifically narrowing) of contexts while maintaining well-formedness, intersection types, recursive self types and path dependent types. There have been several iterations of the DOT calculus and more recent ones have made trade offs in order to attain a soundness proof more readily. Primarily complex paths (paths that may include not only variables but field accesses) have been removed from the most recent iterations.

In this paper we attempt to derive soundness for a version of the DOT calculus that features further restrictions. Primarily we attempt to solve the environment narrowing issue in the absence of intersection types but while retaining recursive self types.

One of the central issues in deriving soundness without employing a series of confusing proof techniques is proving transitivity in the presence of \emph{environment narrowing} and intersection types. The issue here is the mutual dependence that the transitivity proof has on member lookup. This is directly caused by the presence of environment narrowing. 
