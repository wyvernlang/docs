\section{Introduction}
After many years of work Dependent Object Types (DOT) was proven sound in 2016 \cite{nada and tiark 2016}, however the proof remained excruciatingly complex until recently when a simple modular proof was developed that separated the intertwined concepts of types and values to give a much more satisfying and extensible proof \cite{marianna stuff}. While the full proof elegantly deals with multiple intertwined concepts, experimentation does not always require the full complexity of DOT. The key DOT properties are as follows:
\begin{enumerate}
\item
Intersection and Union types: DOT makes heavy use of intersection types to emulate record types and model Scala traits.
\item
Recursive Self Types: Recursive self types are used to allow full advantage of path dependent types, and are required to fully capture parametric polymorphism.
\item
Path Dependent Types: Types can be constructed by performing selections on values. While more recent editions of DOT limit type selections on values ($x.L$), a fuller calculus ultimately includes type selections on complex paths, or chains of value selections ($x.f_1 \ldots .f_n.L$). Complex paths in selection types allows the modelling of concepts like classes and modules.
\end{enumerate}
Each of these features provides powerful expressiveness to a language, and as such DOT benefits from an economy of concepts. The drawback as we have already acknowledged is a complex entangling of these concepts when attempting to derive a soundness proof. 


In this paper we explore how to limit these features to gain an even simpler soundness proof, consider what trade-offs can be made while still maintaining an interesting set of features. We subsequently look at ways to introduce more complex paths while maintaining soundness.


