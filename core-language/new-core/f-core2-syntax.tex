\begin{figure}
\centering
\[
\begin{array}{lll}

e    & \bnfdef & x \\
     & \bnfalt & \boldsymbol\lambda x{:}\tau . e \\
     & \bnfalt & e(e) \\
     & \bnfalt & \boldsymbol\Lambda t . e \\
     & \bnfalt & e[\tau] \\
     & \bnfalt & \keyw{new}~ \tau \{ \overline{d} \} \\	% t may be inferred
     & \bnfalt & e.m \\
     & \bnfalt & e.f \\
     & \bnfalt & e.f = e \\
\\[1ex]

\tau & \bnfdef & t\\
     & \bnfalt & e.t\\				% e must be pure
     & \bnfalt & \forall t . \tau\\
     & \bnfalt & \tau[\overline{t{=}\tau}]\\
     & \bnfalt & \tau \rightarrow \tau \\
     & \bnfalt & \keyw{obj}~ t . \overline{\tau_d}\\ % internal syntax makes recursive type explicit
\\[1ex]

\end{array}
\begin{array}{lll}
~~~~
\end{array}
\begin{array}{lll}
	 
d    & \bnfdef & \keyw{var}~ f:\tau = e \\
     & \bnfalt & \keyw{def}~ m:\tau = e \\
     & \bnfalt & \keyw{type}~ t~ = \tau  \\
\\[1ex]

\tau_d & \bnfdef & \keyw{def}~ m:\tau \\
       & \bnfalt & \keyw{type}~ t  \\
       & \bnfalt & \keyw{type}~ t~ = \tau  \\
\\[1ex]

\sigma & \bnfdef & \tau \\
       & \bnfalt & \keyw{obj}~ t . \overline{\sigma_d} \\
\\[1ex]

\sigma_d & \bnfdef & \keyw{var}~ f:\tau \\
         & \bnfalt & \tau_d \\

\end{array}
\]
\caption{Featherweight Wyvern Syntax}
\label{fig:core2-syntax}
\end{figure}


%%%%%%%%%%%%%%%%%%%%%%%%%% TAGS %%%%%%%%%%%%%%%%%%%%%%%%%%%%
\begin{figure}
\centering
\[
\begin{array}{lll}

e    & \bnfdef & \ldots \\
     & \bnfalt & \keyw{case}(e)~ \overline{p => e} \\ % if e is a var then its type is updated
\\[1ex]

p    & \bnfdef & \tau \\
     & \bnfalt & \keyw{default} \\
\\[1ex]

\tau & \bnfdef & \ldots\\
     & \bnfalt & \tau \& \tau\\		% for combining tags
	 
\end{array}
\begin{array}{lll}
~~~~
\end{array}
\begin{array}{lll}
	 
d    & \bnfdef & \ldots\\
     & \bnfalt & \keyw{tag}~ t:\tau ~ [\keyw{case}~\keyw{of}~\tau] ~ [\keyw{comprises}~ \overline{\tau}] \\
\\[1ex]

\tau_d & \bnfdef & \ldots \\
       & \bnfalt & \keyw{tag}~ t:\tau ~ [\keyw{case}~\keyw{of}~\tau] ~ [\keyw{comprises}~ \overline{\tau}] \\
\\[1ex]

\end{array}
\]
\caption{Featherweight Wyvern Tag Syntax}
\label{fig:tag-syntax}
\end{figure}


%%%%%%%%%%%%%%%%%%%%%%%%%% MODULES %%%%%%%%%%%%%%%%%%%%%%%%%%%%
\begin{figure}
\centering
\[
\begin{array}{lll}

f   & \bnfdef & m \bnfalt c \\% a file is a module or component

% a component is a unit of distribution

% a module is a unit of compilation

m   & \keyw{module} qname \keyw{in} uri \\

e    & \bnfdef & \ldots \\
     & \bnfalt & \keyw{case}(e)~ \overline{p => e} \\
\\[1ex]

p    & \bnfdef & \tau \\
     & \bnfalt & \keyw{default} \\
\\[1ex]

\tau & \bnfdef & \ldots\\
     & \bnfalt & \tau \& \tau\\		% for combining tags
	 
\end{array}
\begin{array}{lll}
~~~~
\end{array}
\begin{array}{lll}
	 
d    & \bnfdef & \ldots\\
     & \bnfalt & \keyw{tag}~ t:\tau ~ [\keyw{case}~\keyw{of}~\tau] ~ [\keyw{comprises}~ \overline{\tau}] \\
\\[1ex]

\tau_d & \bnfdef & \ldots \\
       & \bnfalt & \keyw{tag}~ t:\tau ~ [\keyw{case}~\keyw{of}~\tau] ~ [\keyw{comprises}~ \overline{\tau}] \\
\\[1ex]

\end{array}
\]
\caption{Featherweight Wyvern Module Syntax}
\label{fig:module-syntax}
\end{figure}



%%%%%%%%%%%%%%%%%%%%%%%%%% META-OBJECTS (FIX ME) %%%%%%%%%%%%%%%%%%%%%%%%%%%%


\begin{figure}
\centering
\[
\begin{array}{lll}

e    & \bnfdef & \ldots \\
     & \bnfalt & 'dsl' \\
\\[1ex]

\tau & \bnfdef & t\\
     & \bnfdef & \tau.t\\
     & \bnfalt & \tau \rightarrow \tau \\
\\[1ex]
	 
\end{array}
\begin{array}{lll}
~~~~
\end{array}
\begin{array}{lll}
	 
d    & \bnfdef & \keyw{var}~ f:\tau = e \\
     & \bnfalt & \keyw{def}~ m:\tau = e \\
     & \bnfalt & \keyw{type}~ t~ = \{ \overline{\tau_d}, \keyw{metaobject}=e \} \\
     & \bnfalt & \keyw{type}~ t~ = \{ \tau \}  \\
\\[1ex]

\tau_d   & \bnfdef & \keyw{def}~ m:\tau \\
\\[1ex]

\sigma & \bnfdef & \tau \\
       & \bnfalt & \{ \overline{\sigma_d} \} \\
\\[1ex]

\sigma_d & \bnfdef & \keyw{var}~ f:\tau \\
         & \bnfalt & \tau_d \\

\end{array}
\]
\caption{Featherweight Wyvern Metaobject Syntax}
\label{fig:meta-syntax}
\end{figure}
