\section{A Class-Based Language}

This version adds classes and shows how to rewrite them in terms of
more primitive constructs.  It is a true subset of the real Wyvern
language.

Note that \{\} are used in the abstract syntax as an abbreviation for
an indented block.


\begin{figure*}
\[
\begin{array}{ccc}
\begin{array}{lll}
e    & \bnfdef & x \\
     & \bnfalt & \boldsymbol\lambda x{:}\tau . e \\
     & \bnfalt & e(e) \\
     & \bnfalt & \keyw{new}~ \{ \overline{d} \} \\
     & \bnfalt & e.f \\
     & \bnfalt & e.f = e\\
     & \bnfalt & e.m \\
\\[1ex]

\tau & \bnfdef & t \\
     & \bnfalt & \tau \rightarrow \tau \\
\\[1ex]

\tau_d  & \bnfdef & \keyw{meth}~ m:\tau \\
\\[1ex]

\end{array}
&
\begin{array}{lll}
d    & \bnfdef & \keyw{var}~ f:\tau = e \\
     & \bnfalt & \keyw{meth}~ m:\tau = e \\
     & \bnfalt & \keyw{type}~ t ~ \{ \overline{\tau_d} \} \\
%    & \bnfalt & \keyw{type}~ t = \tau \\
     & \bnfalt & \keyw{class}~ c ~\{~ \overline{cd};~ \overline{d} ~\} \\
\\[1ex]

cd   & \bnfdef & \keyw{class var}~ f:\tau = e \\
     & \bnfalt & \keyw{class meth}~ m:\tau = e \\
\\[1ex]
\end{array}
&
\begin{array}{lll}
\sigma   & \bnfdef & \tau \\
     & \bnfalt & \{\overline{\sigma_{cd}}\} \\
\\[1ex]

\sigma_{cd}  & \bnfdef & \keyw{class var}~ f : \tau \\
     & \bnfalt & \keyw{class meth}~ m : \tau \\
     & \bnfalt & \sigma_d \\
\\[1ex]

\sigma_d  & \bnfdef & \keyw{var}~ f : \tau \\
     & \bnfalt & \keyw{type}~ t ~ \{\overline{\tau_d}\} \\
     & \bnfalt & \keyw{class}~ c ~ \{ ~ \overline{\sigma_{cd}}, ~ \overline{\sigma_d} ~ \} \\
     & \bnfalt & \tau_d \\
\\[1ex]
\end{array}

\end{array}
\]
\caption{Syntax of Featherweight Wyvern with Classes}
\label{f-core3-syntax}
\end{figure*}



\clearpage


\begin{figure}
\centering
\[
\begin{array}{c}
\infer[\textit{DT-type}]
	{\Gamma, \_ \vdash \keyw{type}~ t ~ \{\overline{\tau_d}\} :: \keyw{type}~ t ~ \{\overline{\tau_d}\} } 
	{}\\[3ex]
\infer[\textit{DT-class}]
	{\Gamma, \_ \vdash \keyw{class}~ c ~ \{\overline{cd}, \overline d\} :: \keyw{class}~ c ~ \{\overline{cd}, \overline d\} } 
	{\Gamma, \sigma_{this} \vdash \overline d :: \overline{\sigma_d} & \Gamma, \sigma_{this} \vdash \overline{cd} :: \overline{\sigma_{cd}} & \sigma_{this} = \{\overline{\sigma_{cd}}, \overline{\sigma_d} \} }\\[3ex]
\infer[\textit{DT-class-var}]
	{\Gamma, \sigma_{this} \vdash \keyw{class var} ~ f : \tau_1 = e_1 :: \keyw{class var}~ f : \tau_1 } 
	{\Gamma, \sigma_{this} \vdash e : \tau_2 & \tau_2 <: \tau_1 }\\[3ex]
\infer[\textit{DT-class-meth}]
	{\Gamma, \sigma_{this} \vdash \keyw{class meth} ~ m : \tau_1 = e_1 :: \keyw{class meth}~ m : \tau_1 } 
	{\Gamma, \sigma_{this} \vdash e : \tau_2 & \tau_2 <: \tau_1 }\\[3ex]
\infer[\textit{T-new}]
	{\Gamma, \sigma_{this} \vdash \keyw{new}~ \{ \overline{d_1} \} : \{ \overline{\tau_d} \}}
	{\Gamma,\sigma_{this} \vdash \overline{d_1} :: \overline{\sigma_{d_1}} & \{\overline{\sigma_{d_1}}\} \union \sigma_{this} <: \{\overline{\tau_d} \} } \\[3ex]
\infer[\textit{T-class-field}]
	{\Gamma, \sigma_{this} \vdash  e.f:\tau_1} 
	{\Gamma, \sigma_{this} \vdash e : \sigma_1 & \sigma_1 = \{\keyw{class var}~ f:\tau_1,... \} }\\[3ex]
\infer[\textit{T-class-meth}]
	{\Gamma, \sigma_{this} \vdash  e.m:\tau_1} 
	{\Gamma, \sigma_{this} \vdash e : \tau_1 & \tau_1 = \{\keyw{class meth}~ m:\tau_1=e_1,... \} }
\end{array}
\]
\caption{Static Semantics Rules Core 3}
\end{figure}


\begin{figure}

$trans(\keyw{class} ~ c ~ \{\overline{cd}; \overline{d}\}) \equiv \keyw{type}~ c = \tau_i; ~ \keyw{var}~ c:\tau_c = e$ \\
$~~~~~~~ \textrm{where}$ \\
$~~~~~~~~~~~~~~ \tau_c = \{\overline{\keyw{meth}~ m:\tau}\}$ \\
$~~~~~~~~~~~~~~~~~~~~~ \textrm{where}~ \keyw{meth}~ m:\tau \in \tau_c$ \\
$~~~~~~~~~~~~~~~~~~~~~~~~~~~~ \textrm{iff}~ \keyw{class}~ \keyw{meth}~ m:\tau = e \in \overline{cd}$ \\
%$~~~~~~~~~~~~~~ \sigma_c = \tau_c \union \{\overline{\keyw{var}~ f:\tau}\}$ \\
%$~~~~~~~~~~~~~~~~~~~~~\textrm{where}~ \keyw{var}~ f:\tau \in \sigma_c ~\textrm{iff}~ \keyw{class}~ \keyw{var}~ f:\tau = e \in \overline{cd}$ \\
$~~~~~~~~~~~~~~ \tau_i = \{\overline{\keyw{meth}~ m:\tau}\}$ \\
$~~~~~~~~~~~~~~~~~~~~~\textrm{where}~ \keyw{meth}~ m:\tau \in \tau_i ~\textrm{iff}~ \keyw{meth}~ m:\tau = e \in \overline{d}$ \\
%${~~~~~~~~~~~~~~ \sigma_i = \tau_i \union \{\overline{\keyw{type}~ t ~ \{\overline{\tau_d}\}}\} \union \{\overline{\keyw{var}~ f:\tau}\}}$ \\
%$~~~~~~~~~~~~~~~~~~~~~ \textrm{where}~ \keyw{type}~ t ~ \{\overline{\tau_d}\} \in \sigma_i ~\textrm{iff}~ \keyw{type}~ t ~ \{\overline{\tau_d}\} \in \overline{d},$ \\
%$~~~~~~~~~~~~~~~~~~~~~ \textrm{and}~ \keyw{var}~ f:\tau \in \sigma_i ~\textrm{iff}~ \keyw{var}~ f:\tau = e \in \overline{d}$ \\
$~~~~~~~~~~~~~~ \overline{d_{cl}} = \{\overline{\keyw{meth}~ m:\tau = e}\} \union \{\overline{\keyw{var}~ f:\tau = e}\}$ \\
$~~~~~~~~~~~~~~~~~~~~~ \textrm{where}~ \keyw{meth}~ m:\tau = e \in \overline{d_{cl}}$\\
$~~~~~~~~~~~~~~~~~~~~~~~~~~~~ \textrm{iff}~ \keyw{class}~ \keyw{meth}~ m:\tau = e \in \overline{cd}$ \\
$~~~~~~~~~~~~~~~~~~~~~ \textrm{and}~ \keyw{var}~ f:\tau = e \in \overline{d_{cl}}$\\
$~~~~~~~~~~~~~~~~~~~~~~~~~~~~ \textrm{iff}~ \keyw{class}~ \keyw{var}~ f:\tau = e \in \overline{cd}$ \\
$~~~~~~~~~~~~~~ \overline{d'_{cl}} = [\keyw{new} ~ \{\overline{d} \oplus \overline{d'}\} ~ / ~ \keyw{new} ~ \{\overline{d'}\}] ~ \overline{d_{cl}}$ \\
$~~~~~~~~~~~~~~ \overline{d''_{cl}} = trans(\overline{d'_{cl}}$) \\
$~~~~~~~~~~~~~~ e = \keyw{new} \{ \overline{d''_{cl}} \}$ \\
%$~~~~~~~~~~~~~~ \Gamma \vdash e : \tau$

\caption{Translation of a Class from FWC to FW}
\label{f-core3-translate-function}

\end{figure}


\clearpage


\subsection{Example Program and Translation by Darya}

\begin{figure}
  \centering
\begin{lstlisting}
class Option
	var quantity : int = 0
	var price : int = 0
	meth exercise : int =
		this.quantity * this.price

	class var totalQuantityIssued : int = 0
	class meth issue : int -> int -> Option =
		fn q : int =>
			fn p : int =>
				totalQuantityIssued = 
					totalQuantityIssued + q
				new
					var quantity : int = q
					var price : int = p

var optn : Option = Option.issue(100, 50)
var ret : int = optn.exercise

\end{lstlisting}
\caption{An \texttt{Option} Class in Featherweight Wyvern with Classes}
\label{f-example}
\end{figure}
\begin{figure}
  \centering
\begin{lstlisting}
type Option =
	meth exercise : int

type OptionC =
	meth issue : int -> int -> Option

var Option : OptionC =
	new
		var totalQuantityIssued : int = 0
		meth issue : int -> int -> Option =
			fn q : int =>
				fn p : int =>
					totalQuantityIssued =
						totalQuantityIssued + q
					new
						var quantity : int = q
						var price : int = p
						meth exercise : int =
							this.quantity * this.price

var optn : Option = Option.issue(100, 50)
var ret : int = optn.exercise
\end{lstlisting}
\caption{\texttt{Option} Class Translated to Featherweight Wyvern}
\label{f-example-translated}
\end{figure}


\clearpage


\subsection{Example Program in the Class-Based Language (By Jonathan)}

\begin{figure}[t]
  \centering
\begin{lstlisting}
type t
	meth add : int -> t
	meth get : int
	
class c
	var f : int
	meth add : int -> c =
		fn x : int =>
			this.f = this.f + x
			this
	meth get : int
	meth equals : c -> bool =
		fn other : c =>
			this.f == other.get // cannot access other.f in type c
								// because this core doesn't support ADTs
	class var nc : int = 1
	class meth make : int -> c =
		fn x:int =>
			nc = nc + 1
			new
				var f:int = x
				meth get : int = this.f
			
val o : c = c.make(4)
val o2 : t = o.add(2)
var x6 : int = o.get
\end{lstlisting}
\caption{Example Program in Featherweight Wyvern}
\label{f-core3-example}
\end{figure}


\subsection{Translation of the Program to the Core Method-Based
  Language (By Jonathan)}

\begin{figure}
  \centering
  
\begin{lstlisting}
type t = rec t2.
	meth add : int -> t2
	meth get : int

type c = rec c2.
	meth add : int -> c2
	meth get : int
	meth equals : c2 -> bool
		
type c_internal = rec ci2. // not necessary, but a convenient abbreviation
	var f : int
	meth add : int -> c
	meth get : int
	meth equals : c -> bool

type c_class = rec cl2.  // not necessary
	meth make : int -> c
	
type c_class_internal = rec cli2.  // not necessary
	var nc : int
	meth make : int -> c

val c : c_class
	= new c_class_internal
		var nc : int = 1
		meth make : int -> c =
			fn x:int =>
				nc = nc + 1
				new c_internal
					var f : int = x
					meth get : int = this.f
					meth add : int -> c =
						fn x : int =>
							this.f = this.f + x
							this
					meth equals : c -> bool =
						fn other : c =>
							this.f == other.get // cannot access other.f in type c

val o : c = c.make(4)
val o2 : t = o.add(2)
var x6 : int = o.get
\end{lstlisting}

\caption{Example Program in FW Translated to OO Core without Classes}
\label{f-core3-translated-example}

\end{figure}


Limitations: this language only supports objects, not ADTs.  For ADTs we need bounded type members.


\clearpage


\subsection{Tasks}

\begin{itemize}

 \item write some examples!
 \item define stripClass, rewriteNew, and a way of computing $\tau_i$
 \item add lots of conveniences as sugar
 \item in rule R-class, meth c needs to return the same object each time, so cache it in a field.
 \item give complete rewriting rules (R*) to the core language
 \item give complete typing rules, and prove that well-typed source programs translate to well-typed core programs.  Is it possible to prove a property related to the uniform access principle and/or state encapsulation?
 \item consider ``class type t = ...''
 \item no abstract class members
 \item no class class members because a class is always a class member; if you want a class in an object use a val $+$ type 

\end{itemize}
