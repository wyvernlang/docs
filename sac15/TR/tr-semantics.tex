%!TEX root=sac15-tr.tex
\section{Bidirectional Typechecking and Elaboration}
\label{tr-semantics}

The top-level judgement in the system is the compilation judgement:
$$\AXC{$d\sim(\Psi;\Theta)$ ~~~~ $\emptyset;\emptyset\vdash_{\Theta_0\Theta}^{\Psi}e\rightsquigarrow i\Rightarrow \tau$}      \RightLabel{(compile)}
\UIC{$ d;e\sim(\Psi;\Theta)\rightsquigarrow i:\tau$}
\DP$$

The compilation judgment will compile a external Wyvern program to internal Wyvern program by parsing declarations, $d$, and transforming the expression $e$, representing the Wyvern program body, into $i$. 

The judgement $d \sim (\Psi, \Theta)$, defined in Figure \ref{typechecking-elaboration}, generates a corresponding TSM context, $\Psi$, and named type context, $\Theta$, from $d$. The syntax for these contexts is given in Figure \ref{typechecking-environment}. We refer to the type declarations in the prelude, some of which were shown in Fig. \ref{exp-prelude}, by using the named type context $\Theta_0$ in our rules.

Judgements $\Delta; \Gamma \vdash_\Theta^\Phi e \leadsto i \Rightarrow \tau$ and  $\Delta; \Gamma \vdash_\Theta^\Phi e \leadsto i \Leftarrow \tau$ can be read ``under kinding context $\Delta$, typing context $\Gamma$, named type context $\Theta$ and TSM context $\Phi$, external term $e$ elaborates to internal term $i$ and (synthesizes / analyzes against) type $\tau$''. This is a \emph{bidirectionally typed elaboration semantics}, used to elaborate TSL literals and TSM applications, which appear only in the external language, to an internal language, which includes only the core operations in the language. The semantics follows that given in \cite{TSLs}, so we do not repeat the rules for the core operations or TSL literals. The rules for the only new form, $\textbf{eaptsm}[s, body]$, are given in Figure \ref{expkw-kwstatics} and described below.