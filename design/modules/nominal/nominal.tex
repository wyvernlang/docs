\documentclass{llncs}

\usepackage{listings}
\usepackage{proof}
\usepackage{amssymb}
\usepackage[margin=.9in]{geometry}
\usepackage{amsmath}

\lstset{tabsize=3, basicstyle=\ttfamily\small, commentstyle=\itshape\rmfamily, numbers=left, numberstyle=\tiny, language=java,moredelim=[il][\sffamily]{?},mathescape=true,showspaces=false,showstringspaces=false,columns=fullflexible,xleftmargin=15pt,escapeinside={(@}{@)}, morekeywords=[1]{objtype,module,import,let,in,fn,var,type,rec,fold,unfold,letrec,alloc,ref,application,policy,external,component,connects,to,meth,val,where,return,group,by,within,count,connect,with,attr,html,head,title,style,body,div,keyword,unit,def}}
\lstloadlanguages{Java,VBScript,XML,HTML}

\newcommand{\keywadj}[1]{\mathtt{#1}}
\newcommand{\keyw}[1]{\keywadj{#1}~}
\newcommand{\reftt}{\mathtt{ref}~}
\newcommand{\Reftt}{\mathtt{Ref}~}
\newcommand{\inttt}{\mathtt{int}~}
\newcommand{\Inttt}{\mathtt{Int}~}
\newcommand{\stepsto}{\leadsto}

\begin{document}


\section{Modules}

\subsection{Syntax}

\[
\begin{array}{lll}
\begin{array}{lllr}
e & ::= & x & expressions \\
& | & \keywadj{new}(x \Rightarrow d) \\
& | & e.m(e)\\
&&\\
d & ::= & \epsilon & declarations \\
  & |   & \keyw{def} m(x:\tau):\tau = e; d \\
  & |   & \keyw{type} t = \tau; d \\
&&\\
\tau & ::= & \{ x \Rightarrow \sigma \} & types \\
     & |   & x.t \\
&&\\
\sigma & ::= & \epsilon & decl.~ types \\
       & |   & \keyw{def} m:\tau \rightarrow \tau; \sigma & (method~type)\\
       & |   & \keyw{abstype} t; \sigma & (abstract~type~member)\\
       & |   & \keyw{abstype} t = \tau; \sigma & (concrete~type~member) \\
&&\\
\Gamma & :: = & \varnothing & contexts\\
& | & \Gamma,~x : \tau\\
\end{array}
& ~~~~~~
&
\begin{array}{lllr}
v & ::= & \keywadj{new}(x \Rightarrow d) & values\\
&&\\
E & ::= & \epsilon & environments\\
  & |   & E, x \rightarrow v \\
&&\\
\end{array}
\end{array}
\]

\newpage

\subsection{Semantics}

In general, the semantic rules are similar to---or simplified from---Amin et al.'s
OOPLSA 2014 paper.

One exception is that the dynamic semantics is given with a small-step semantics that uses an environment $E$ rather than using substitution.  Here $E$ includes (possibly renamed) variables for every binding in the program.  We could optionally go further and use an abstract machine so that we have a stack of method frames, each of which keeps a set of variable bindings of its own.

$\fbox{$E \triangleright e \hookrightarrow E \triangleright e$}$
\[
\begin{array}{c}
\infer[\textsc{(E-Call)}]
  {E \triangleright v.m(v') \hookrightarrow E' \triangleright [y/x,y'/x']e'}
  {v = \keywadj{new}(x \Rightarrow d) ~~~~
  lookup(m,d) = \keyw{def} m(x':\tau):\tau = e' ~~~~
  y,y' \not\in domain(E) ~~~~
  E' = E, y \rightarrow v, y' \rightarrow v'}%\\[5ex]
\end{array}
\]

\end{document}