\documentclass{llncs}

\usepackage{listings}
\usepackage{proof}
\usepackage{amssymb}
\usepackage[margin=.9in]{geometry}
\usepackage{amsmath}
\usepackage[english]{babel}
\usepackage[utf8]{inputenc}
\usepackage{enumitem}
\usepackage{filecontents}
\usepackage{calc}
\usepackage[linewidth=0.5pt]{mdframed}
\usepackage{changepage}
\usepackage{tabto}
\allowdisplaybreaks

\usepackage{fancyhdr}
\renewcommand{\headrulewidth}{0pt}
\pagestyle{fancy}
 \fancyhf{}
\rhead{\thepage}

\lstset{tabsize=3, basicstyle=\ttfamily\small, commentstyle=\itshape\rmfamily, numbers=left, numberstyle=\tiny, language=java,moredelim=[il][\sffamily]{?},mathescape=true,showspaces=false,showstringspaces=false,columns=fullflexible,xleftmargin=5pt,escapeinside={(@}{@)}, morekeywords=[1]{objtype,module,import,let,in,fn,var,type,rec,fold,unfold,letrec,alloc,ref,application,policy,external,component,connects,to,meth,val,where,return,group,by,within,count,connect,with,attr,html,head,title,style,body,div,keyword,unit,def}}
\lstloadlanguages{Java,VBScript,XML,HTML}

\newcommand{\keywadj}[1]{\mathtt{#1}}
\newcommand{\keyw}[1]{\keywadj{#1}~}

\newcommand{\kw}[1]{\keyw{ #1 }}
\newcommand{\kwa}[1]{\keywadj{ #1 }}
\newcommand{\reftt}{\mathtt{ref}~}
\newcommand{\Reftt}{\mathtt{Ref}~}
\newcommand{\inttt}{\mathtt{int}~}
\newcommand{\Inttt}{\mathtt{Int}~}
\newcommand{\stepsto}{\leadsto}
\newcommand{\todo}[1]{\textbf{[#1]}}
\newcommand{\intuition}[1]{#1}
\newcommand{\hyphen}{\hbox{-}}

%\newcommand{\intuition}[1]{}

\newlist{pcases}{enumerate}{1}
\setlist[pcases]{
  label=\fbox{\textit{Case}}\protect\thiscase\textit{:}~,
  ref=\arabic*,
  align=left,
  labelsep=0pt,
  leftmargin=0pt,
  labelwidth=0pt,
  parsep=0pt
}
\newcommand{\pcase}[1][]{

  \if\relax\detokenize{#1}\relax
    \def\thiscase{}
  \else
    \def\thiscase{~\fbox{#1:}}
  \fi
  \item
}

\newcommand{\thm}[3]{
	\begin{large}
		\bf{#1}
	\end{large} \\\\
	\fbox{Statement.} ~ #2
	\fbox{Proof.}~ #3 \qed
}

\newcommand{\proofcase}[2]{
	\begin{adjustwidth}{1.5em}{0pt}
		\fbox{Case.}~~#1. \\ ~#2
	\end{adjustwidth}
}

\newcommand{\subcase}[1] {
	\begin{adjustwidth}{2.2em}{0pt}
		\underline{Subcase.} #1
	\end{adjustwidth}
}

\newcommand{\stmt}[1] {

\begin{adjustwidth}{2.5em}{0pt}
	#1
\end{adjustwidth}

}
\newcommand{\type}[2]{
	#1~\keyw{with} #2
}

\newcommand{\newd}[0]{
	\keywadj{new}_d~x \Rightarrow \overline{d = e}
}

\newcommand{\newsig}[0]{
	\keywadj{new}_\sigma~x \Rightarrow \overline{\sigma = e}
}


\begin{document}


\section{Higher Order Example}

\subsection{Program 1}

\begin{lstlisting}
resource File

def writeFile(v: Int): Unit with File.write =
   File.write(v)

def dolt(f : Int $\rightarrow$ Unit with $\varnothing$): Unit =
   f(0)

def unlabeled(): Unit =
   dolt(writeFile)

unlabeled()
\end{lstlisting}

\noindent
Globally this gives the correct result (the effect is $\kwa{File.write}$ but locally is not correct because $\kwa{writeFile}$ is being passed to $\kwa{dolt}$, and there is a mismatch between the effect signatures. \\

\subsection{Program 2}

\noindent
If we de-sugar the program by ``one layer'', we get the following.

\begin{lstlisting}

let x$_1$ = new$_\sigma$x $\Rightarrow$ {
   def writeFile(v: Int): Unit with File.write =
      File.write(v)
} in

let x$_2$ = new$_\sigma$x $\Rightarrow$ {
   def dolt(obj: {f: Int $\rightarrow$ Unit with $\varnothing$}): Unit with $\varnothing$ =
      f(0)
} in

let x$_3$ = new$_d$x $\Rightarrow$ {
   def unlabelled(): Unit =
      x$_2$.dolt(x$_1$)
} in

x$_3$.unlabelled()
\end{lstlisting}

\noindent
To typecheck $x_3$ your choice of $\Gamma'$ will need both $x_1$ and $x_2$. Now, $\kwa{capture}(x_1) = \kwa{File.write}$ and $\kwa{capture}(x_2) = \varnothing$. Therefore $\kwa{capture}(\Gamma') = \varepsilon_c = \kwa{File.write}$. However, in the premise of \textsc{C-NewObj}, $\kwa{capture}(x_2) \supseteq \varepsilon_c$ is NOT true so it wouldn't typecheck. Since your choice of $\Gamma'$ needs at least $x_2$ to be well-formed, then it won't typecheck under any choice of $\Gamma'$.

\subsection{Program 3}

If we translate the let expressions we get the following:

\begin{lstlisting}
new$_\sigma$x $\Rightarrow$ {
    def _dummy1(x$_1$: { writeFile: Int $\rightarrow$ Unit with File.write }): Unit with File.write =
      
        new$_d$x $\Rightarrow$ {
            def _dummy2(x$_2$: {dolt: {f: Int $\rightarrow$ Unit with $\varnothing$} $\rightarrow$ Unit}): Unit =
                
                new$_d$x $\Rightarrow$ {
                    def _dummy3(x$_3$: {unlabelled: Unit $\rightarrow$ Unit}): Unit =
                        x$_3$.unlabelled()
                        
                }._dummy3(new$_d$x $\Rightarrow$ { def unlabelled(): Unit =
                                          x$_2$.dolt(x$_1$) })
            
        }._dummy2(new$_d$x $\Rightarrow$ { def dolt(obj: { f: Int $\rightarrow$ Unit with $\varnothing$ }): Unit =
                                 f(0) })
      
}._dummy1(new$_\sigma$x $\Rightarrow$ { def writeFile(v: Int): Unit with File.write =
                         File.write(v)   })
\end{lstlisting}

\section{Weirdness Without Well-Formedness}


\begin{lstlisting}
let $x_1$ = new$_\sigma$ x $\Rightarrow$ {
   def example(y$_1$: {meth: Unit $\rightarrow$ Unit with File.write}): Unit with File.write =
       y$_1$.meth()
} in

let $x_2$ = new$_\sigma$ x $\Rightarrow$ {
   def meth(y$_2$: Unit): Unit with $\varnothing$ = unit
} in

$x_1$.example($x_2$)
\end{lstlisting}

\noindent
This program $e$ is a counter-example to the (naive) Use Lemma because $\varnothing \vdash e : \kwa{Unit}~with~\kwa{File.write}$, but $\kwa{File.write} \not\subseteq \kwa{capture}(\varnothing)$. Two possible solutions:

\begin{enumerate}
	\item The type system does a check to make sure types only reference known resources (which would require you to typecheck in the context $\Gamma = \kwa{File} : \{ \kwa{File} \}$).
	\item Add a condition to the Use Lemma like ``the program under consideration is closed under $\Gamma$'', where a suitable definition of closed would mean that the program above, typechecked in $\varnothing$, doesn't count because $\kwa{File}$ occurs free in the type of $y_1$.
\end{enumerate}


\section{Higher-Order W/ Currying}

\subsection{}

\begin{lstlisting}
def go(a: Int, b: Int): Unit with File.write =
    File.write
    
def fixsecond(f: Int $\rightarrow$ Unit with $\varnothing$): Unit with $\varnothing$ =
    f(0)

def fixfirst(f: Int $\rightarrow$ Int $\rightarrow$ Unit with $\varnothing$): (Int $\rightarrow$ Unit with $\varnothing$) with $\varnothing$ =
    fixsecond(f(0))
    
def main(): Unit =
    fixfirst(go)
\end{lstlisting}

\subsection{}

\begin{lstlisting}
let $x_1$ = new$_\sigma$x $\Rightarrow$ {
    def go(a: Int): {f: Int $\rightarrow$ Unit with File.write} with $\varnothing$ =
        new$_\sigma$x $\Rightarrow$ {
            def go(b: Int): Unit with File.write =
                File.write
        }
} in

let $x_2$ = new$_\sigma$x $\Rightarrow$ {
    def fixsecond(obj: {go: Int $\rightarrow$ Unit with $\varnothing$}): Unit with $\varnothing$ =
        obj.go(0)
}

let $x_3$ = new$_\sigma$x $\Rightarrow$ {
    def fixfirst(obj: {go: Int $\rightarrow$ Int $\rightarrow$ Unit with $\varnothing$}): {go: Int $\rightarrow$ Unit} with $\varnothing$ =
        obj.go(0)
} in

let $x_4$ = new$_d$x $\Rightarrow$ {
    def main(): Unit =
        $x_3$.fixfirst($x_1$)
} in

$x_4$.main()
\end{lstlisting}

\end{document}





