\documentclass{llncs}

\usepackage{listings}
\usepackage{proof}
\usepackage{amssymb}
\usepackage[margin=.9in]{geometry}
\usepackage{amsmath}
\usepackage[english]{babel}
\usepackage[utf8]{inputenc}
\usepackage{enumitem}
\usepackage{filecontents}
\usepackage{calc}
\usepackage[linewidth=0.5pt]{mdframed}
\usepackage{changepage}
\allowdisplaybreaks

\usepackage{fancyhdr}
\renewcommand{\headrulewidth}{0pt}
\pagestyle{fancy}
 \fancyhf{}
\rhead{\thepage}

\lstset{tabsize=3, basicstyle=\ttfamily\small, commentstyle=\itshape\rmfamily, numbers=left, numberstyle=\tiny, language=java,moredelim=[il][\sffamily]{?},mathescape=true,showspaces=false,showstringspaces=false,columns=fullflexible,xleftmargin=5pt,escapeinside={(@}{@)}, morekeywords=[1]{objtype,module,import,let,in,fn,var,type,rec,fold,unfold,letrec,alloc,ref,application,policy,external,component,connects,to,meth,val,where,return,group,by,within,count,connect,with,attr,html,head,title,style,body,div,keyword,unit,def}}
\lstloadlanguages{Java,VBScript,XML,HTML}

\newcommand{\keywadj}[1]{\mathtt{#1}}
\newcommand{\keyw}[1]{\keywadj{#1}~}

\newcommand{\kw}[1]{\keyw{ #1 }}
\newcommand{\kwa}[1]{\keywadj{ #1 }}
\newcommand{\reftt}{\mathtt{ref}~}
\newcommand{\Reftt}{\mathtt{Ref}~}
\newcommand{\inttt}{\mathtt{int}~}
\newcommand{\Inttt}{\mathtt{Int}~}
\newcommand{\stepsto}{\leadsto}
\newcommand{\todo}[1]{\textbf{[#1]}}
\newcommand{\intuition}[1]{#1}
\newcommand{\hyphen}{\hbox{-}}

%\newcommand{\intuition}[1]{}

\newlist{pcases}{enumerate}{1}
\setlist[pcases]{
  label=\fbox{\textit{Case}}\protect\thiscase\textit{:}~,
  ref=\arabic*,
  align=left,
  labelsep=0pt,
  leftmargin=0pt,
  labelwidth=0pt,
  parsep=0pt
}
\newcommand{\pcase}[1][]{

  \if\relax\detokenize{#1}\relax
    \def\thiscase{}
  \else
    \def\thiscase{~\fbox{#1:}}
  \fi
  \item
}

\newcommand{\thm}[3]{
	\begin{large}
		\bf{#1}
	\end{large} \\\\
	\fbox{Statement.} ~ #2
	\fbox{Proof.}~ #3 \qed
}

\newcommand{\proofcase}[2]{
	\begin{adjustwidth}{1.5em}{0pt}
		\fbox{Case.}~~#1. \\ ~#2
	\end{adjustwidth}
}

\newcommand{\subcase}[1] {
	\begin{adjustwidth}{2.2em}{0pt}
		\underline{Subcase.} #1
	\end{adjustwidth}
}

\newcommand{\stmt}[1] {

\begin{adjustwidth}{2.5em}{0pt}
	#1
\end{adjustwidth}

}
\newcommand{\type}[2]{
	#1~\keyw{with} #2
}

\newcommand{\unit}[0]{ \kwa{unit} }

\newcommand{\Unit}[0]{ \kwa{Unit} }

\newcommand{\fx}[1]{ \kwa{effects}(#1) }

\newcommand{\hofx}[1]{ \kwa{ho \hyphen effects}(#1) }

\newcommand{\safe}[2]{ \kwa{safe}(#1, #2) }

\newcommand{\hosafe}[2]{ \kwa{ho \hyphen safe}(#1, #2) }

\newcommand{\arr}[3]{
	#1 \rightarrow_{#3} #2
}

\newcommand{\newd}[0]{
	\keywadj{new}_d~x \Rightarrow \overline{d = e}
}

\newcommand{\newsig}[0]{
	\keywadj{new}_\sigma~x \Rightarrow \overline{\sigma = e}
}

\newcommand{\import}[4]{
	\keywadj{import}(#1)~#2 = #3~\kw{in} #4
}

\newcommand{\annot}[2]{
	\keywadj{annot}(#1, #2)
}

\newcommand{\erase}[1]{
	\keywadj{erase}(#1)
}

\newcommand{\poly}[3]{
	\forall #1 : #2 . #3
}

\newcommand\defn{\mathrel{\overset{\makebox[0pt]{\mbox{\normalfont\tiny\sffamily def}}}{=}}}


\begin{document}

\section{Grammar}

\[
\begin{array}{lll}

\begin{array}{lllr}

e & ::= & ~ & \bf{exprs.} \\
	& | & x & variable \\
	& | & v & value \\
	& | & e ~ e & application \\
	& | & e.\pi & operation~call \\
	& | & \poly{t}{\tau}{e} & type~polymorphism \\
	&&\\

v & ::= & ~ & \bf{values} \\
	& | & r & resource~literal \\
	& | & \lambda x: \tau.e & abstraction \\
	&&\\
	
\hat e & ::= & ~ & \bf{annotated~exprs.} \\
	& | & x & variable \\
	& | & \hat v & value \\
	& | & \hat e ~ \hat e & application \\
	& | & \hat e.\pi & operation~call\\
	& | & \poly{t}{\hat \tau}{\hat e} & type~polymorphism \\
	& | & \poly{\epsilon}{\varepsilon}{\hat e} & effect~polymorphism \\
	& | & \import{\varepsilon_s}{x}{\hat e}{e} & import\\
	&&\\

\hat v & ::= & ~ & \bf{annotated~values} \\
	& | & r & resource~literal \\
	& | & \lambda x: \hat \tau.\hat e & abstraction \\
	&&\\

\end{array}

& ~~~~~~~~&

\begin{array}{lllr}

\tau & ::= & ~ & \bf{types} \\
	& | & t & type~variable \\
	& | & \{ \bar r \} & effect~set \\
	& | & \tau \rightarrow \tau & arrow \\ \
	&&\\

\hat \tau & ::= & ~ & \bf{annotated ~types} \\
		& | & t & type~variable \\
		& | & \{ \bar r \} & resource~set \\
		& | & \hat \tau \rightarrow_{\varepsilon} \hat \tau & annotated~arrow\\
		&&\\

\varepsilon & ::= & ~ &\bf{effects} \\
	& | & \epsilon & effect~variable \\
	& | & \{ \overline{r.\pi} \} & effect~set \\
	&&\\

\Gamma & ::= & ~ & \bf{contexts} \\
				& | & \varnothing & empty~ctx. \\
				& | & \Gamma, x: \tau & var.~binding \\
				& | & \Gamma, t : \tau & type~var.~binding \\
				&&\\
				
\hat \Gamma & ::= & ~ & \bf{annotated~contexts}\\
				& | & \varnothing & empty~ctx. \\
				& | & \hat \Gamma, x: \hat \tau & var.~binding \\
				& | & \hat \Gamma, \epsilon : \varepsilon & effect~var.~binding \\
				& | & \hat \Gamma, t: \hat \tau & type~var.~binding \\
				&&\\

\end{array}

\end{array}
\]

\section{Functions}

\subsection*{Definition ($\kwa{annot :: \tau \times \varepsilon \rightarrow \hat \tau}$)}

\begin{enumerate}
	\item $\annot{t}{\_} = t$
	\item $\annot{\{ \bar r \}}{\_} = \{ \bar r \}$
	\item $\annot{\tau_1 \rightarrow \tau_2}{\varepsilon} = \annot{\tau_1}{\varepsilon} \rightarrow_{\varepsilon} \annot{\tau_2}{\varepsilon}$
\end{enumerate}


\subsection*{Definition ($\kwa{annot :: e \times \varepsilon \rightarrow \hat e}$)}

\begin{enumerate}
	\item $\kwa{annot}(x, \_) = e$
	\item $\kwa{annot}(r, \_) = r$
	\item $\kwa{annot}(\lambda x: \tau.e, \varepsilon) = \lambda x: \kwa{annot}(\tau, \varepsilon) . \kwa{annot}(e, \varepsilon)$
	\item $\kwa{annot}(e_1~e_2, \varepsilon) = \kwa{annot}(e_1) \kwa{annot}(e_2)$
	\item $\kwa{annot}(e.\pi, \varepsilon) = \kwa{annot}(e).\pi$
	\item $\annot{\poly{t}{\tau}{e}}{\varepsilon} = \poly{t}{\annot{\tau}{\varepsilon}}{\annot{e}{\varepsilon}}$
	\item $\annot{\poly{\epsilon}{\varepsilon'}{e}}{\varepsilon} = \poly{\epsilon}{\varepsilon'}{\annot{e}{\varepsilon}}$
\end{enumerate}

\subsection*{Definition ($\kwa{annot :: \Gamma \times \varepsilon \rightarrow \hat \Gamma}$)}

\begin{enumerate}
	\item $\kwa{annot}(\varnothing, \_) = \varnothing$
	\item $\kwa{annot}((\Gamma, x: \tau), \varepsilon) = \kwa{annot}(\Gamma, \varepsilon), x: \kwa{annot}(\tau, \varepsilon)$
	\item $\kwa{annot}((\Gamma, t: \tau), \varepsilon) = \kwa{annot}(\Gamma, \varepsilon), x: \kwa{annot}(\tau, \varepsilon)$
\end{enumerate}

\subsection*{Definition ($\kwa{erase :: \hat \tau \rightarrow \tau}$)}

\begin{enumerate}
	\item $\erase{t} = e$
	\item $\kwa{erase}(\{ \bar r \}, \_) = \{ \bar r \}$
	\item $\kwa{erase}(\hat \tau_1 \rightarrow_{\varepsilon} \hat \tau_2) = \kwa{erase}(\hat \tau_1) \rightarrow \kwa{erase}(\hat \tau_2)$
\end{enumerate}

\subsection*{Definition ($\kwa{erase :: \hat e \rightarrow e}$)}

\begin{enumerate}
	\item $\kwa{erase}(x) = x$
	\item $\kwa{erase}(r) = r$
	\item $\kwa{erase}(\lambda x: \hat \tau.\hat e) = \lambda x: \kwa{erase}(\hat \tau).\kwa{erase}(\hat e)$
	\item $\kwa{erase}(e_1~e_2) = \kwa{erase}(e_1) \kwa{erase}(e_2)$
	\item $\kwa{erase}(e.\pi) = \kwa{erase}(e).\pi$
	\item $\erase{\poly{t}{\hat \tau}{\hat e}} = \poly{t}{\erase{\hat \tau}}{\erase{\hat e}}$
	\item $\erase{\poly{\epsilon}{\varepsilon}{\hat e}} = \poly{\epsilon}{\varepsilon}{\erase{\hat e}}$
\end{enumerate}


\subsection*{Definition ($\kwa{erase :: \hat \Gamma \rightarrow \Gamma}$)}

\begin{enumerate}
	\item $\erase{\varnothing} = \varnothing$
	\item $\erase{\hat \Gamma, x: \hat \tau} = \erase{\hat \Gamma}, x: \erase{\hat \tau}$
	\item $\erase{\Gamma, t: \hat \tau} = \erase{\Gamma}, x: \erase{\hat \tau}$
	\item $\erase{\Gamma, \epsilon: \varepsilon} = \erase{\Gamma}$
\end{enumerate}


\subsection*{Definition ($\kwa{effects :: \hat \tau \rightarrow \varepsilon}$)}

\begin{enumerate}
	\item $\fx{t} = \varnothing$
	\item $\fx{\{ \bar r \}} = \{ r.\pi \mid r \in \bar r, \pi \in \Pi \}$
	\item $\fx{\hat \tau_1 \rightarrow_{\varepsilon} \hat \tau_2} = \kwa{ho \hyphen effects}(\hat \tau_1) \cup \varepsilon \cup \kwa{effects}(\hat \tau_2)$
\end{enumerate}

\subsection*{Definition ($\kwa{ho \hyphen effects :: \hat \tau \rightarrow \varepsilon}$)}

\begin{enumerate}
	\item $\hofx{t} = \varnothing$
	\item $\hofx{\{ \bar r \}} = \varnothing$
	\item $\hofx{\hat \tau_1 \rightarrow_{\varepsilon} \hat \tau_2} = \kwa{effects}(\hat \tau_1) \cup \kwa{ho \hyphen effects}(\hat \tau_2)$
\end{enumerate}

\subsection*{Definition ($\kwa{substitution :: \hat e \times \hat v \times \hat v \rightarrow \hat e}$)}

The notation $[\hat v/x]\hat e$ is short-hand for $\kwa{substitution}(\hat e, \hat v, x)$. This function is partial, because the third input must be a variable. We adopt the usual renaming conventions to avoid accidental capture.

\begin{enumerate}
	\item $[\hat v/y]x = \hat v$, if $x = y$
	\item $[\hat v/y]x = x$, if $x \neq y$
	\item $[\hat v/y](\lambda x: \hat \tau.\hat e) = \lambda x: \hat \tau.[\hat v/y]\hat 
e$, if $y \neq x$ and $y$ does not occur free in $\hat e$
\item $[\hat v/y](\hat e_1~\hat e_2) = ([\hat v/y]\hat e_1) ([\hat v/y]\hat e_2)$ 
	\item $[\hat v/y](\hat e.\pi) = ([\hat v/y]\hat e).\pi$
	\item $[\hat v/y](\poly{t}{\hat \tau}{\hat e}) = \poly{t}{\hat \tau}{[\hat v/y]\hat e}$, if $y \neq t$ and $y$ does not occur free in $\hat e$
	\item $[\hat v/y](\poly{\epsilon}{\varepsilon}{\hat e}) = \poly{\epsilon}{\varepsilon}{[\hat v/y]\hat e}$, if $y \neq \epsilon$ and $y$ does not occur free in $\hat e$
	\item $[\hat v/y](\import{\varepsilon_s}{x}{\hat e}{e}) = \import{\varepsilon_s}{x}{[\hat v/y]\hat e}{e}$
\end{enumerate}

\noindent
When performing multiple substitutions the notation $[\hat v_1/x_1, \hat v_2/x_2]\hat e$ is used as shorthand for $[\hat v_2/x_2]([\hat v_1/x_1]\hat e)$ (note the order of the variables has been flipped; the substitutions occur as they are written, left-to-right).


\section{Static Rules}

\fbox{$\Gamma \vdash e: \tau$}

\[
\begin{array}{c}


\infer[\textsc{(T-Var)}]
	{\Gamma, x: \tau \vdash x: \tau}
	{}
~~~
\infer[\textsc{(T-Resource)}]
	{\Gamma, r: \{ r \} \vdash r : \{ r \}}
	{}

~~~
\infer[\textsc{(T-Abs)}]
	{\Gamma \vdash \lambda x: \tau_1.e : \tau_1 \rightarrow \tau_2}
	{\Gamma, x: \tau_1 \vdash e: \tau_2}\\[2ex]
	
\infer[\textsc{(T-App)}]
	{\Gamma \vdash e_1~e_2: \tau_3}
	{\Gamma \vdash e_1: \tau_2 \rightarrow \tau_3 & \Gamma \vdash e_2: \tau_2}
~~~
\infer[\textsc{(T-OperCall)}]
	{\Gamma \vdash e.\pi: \kwa{Unit}}
	{\Gamma \vdash e: \{ \bar r \}} \\[2ex]


\end{array}
\]

\noindent
\fbox{$\hat \Gamma \vdash \hat e: \hat \tau~\kw{with} \varepsilon$}

\[
\begin{array}{c}

\infer[\textsc{($\varepsilon$-Var)}]
	{ \hat \Gamma, x:\tau \vdash x: \tau~\kw{with} \varnothing }
	{}
~~~
\infer[\textsc{($\varepsilon$-Resource)}]
 	{ \hat \Gamma, r: \{ r \} \vdash r : \{ r \}~\kw{with} \varnothing }
 	{}\\[2ex]
 	
~~~
	\infer[\textsc{($\varepsilon$-Abs)}]
	{ \hat \Gamma \vdash \lambda x:\tau_2 . \hat e : \hat \tau_2 \rightarrow_{\varepsilon_3} \hat \tau_3~\kw{with} \varnothing }
	{ \hat \Gamma, x: \hat \tau_2 \vdash \hat e: \hat \tau_3~\kw{with} \varepsilon_3 }\\[2ex]
	
\infer[\textsc{($\varepsilon$-App)}]
	{ \hat \Gamma \vdash \hat e_1 \hat e_2 : \hat \tau_3~\kw{with} \varepsilon_1 \cup \varepsilon_2 \cup \varepsilon  }
	{ \hat \Gamma \vdash \hat e_1: \hat \tau_2 \rightarrow_{\varepsilon} \hat \tau_3~\kw{with} \varepsilon_1 & \hat \Gamma \vdash \hat e_2: \hat \tau_2~\kw{with} \varepsilon_2 } \\[2ex]


\infer[\textsc{($\varepsilon$-OperCall)}]
	{ \hat \Gamma \vdash \hat e.\pi: \kw{Unit} \kw{with} \{ r.\pi \mid r \in \bar r \} }
	{ \hat \Gamma \vdash \hat e: \{ \bar r \}}
	~~~

\infer[\textsc{($\varepsilon$-Subsume)}]
	{\hat \Gamma \vdash e: \tau' ~\kw{with} \varepsilon'}
	{\hat \Gamma \vdash e: \tau ~\kw{with} \varepsilon & \tau <: \tau' & \varepsilon \subseteq \varepsilon'}\\[2ex]

\infer[\textsc{($\varepsilon$-Import)}]
	{ \hat \Gamma \vdash \kwa{import}(\varepsilon)~x = \hat e~\kw{in} e: \kwa{annot}(\tau, \varepsilon)~\kw{with} \varepsilon \cup \varepsilon_1 }
{{\def\arraystretch{1.4}
  \begin{array}{c}
\kwa{effects}(\hat \tau) \cup \hofx{\annot{\tau}{\varnothing}}\subseteq \varepsilon \\
\hat \Gamma \vdash \hat e: \hat \tau~\kw{with} \varepsilon_1 ~~~~~~ \kwa{ho \hyphen safe}(\hat \tau, \varepsilon) ~~~~~~ x: \kwa{erase}(\hat \tau) \vdash e: \tau
  \end{array}}} 
 
 
\end{array}
\]

\noindent
$\fbox{$\kwa{safe(\tau, \varepsilon)}$}$

\[
\begin{array}{c}

\infer[\textsc{(Safe-Resource)}]
	{\kwa{safe}(\{ \bar r \}, \varepsilon)}
	{}
~~~~~
\infer[\textsc{(Safe-Unit)}]
	{\kwa{safe}(\kwa{Unit}, \varepsilon)}
	{} \\[3ex]

\infer[\textsc{(Safe-Arrow)}]
	{\kwa{safe}(\hat \tau_1 \rightarrow_{\varepsilon'} \hat \tau_2, \varepsilon)}
	{\varepsilon \subseteq \varepsilon' & \kwa{ho \hyphen safe}(\hat \tau_1, \varepsilon) & \kwa{safe}(\hat \tau_2, \varepsilon)} \\[3ex]

\end{array}
\]

\noindent
$\fbox{$\kwa{ho \hyphen safe(\hat \tau, \varepsilon)}$}$

\[
\begin{array}{c}

\infer[\textsc{(HOSafe-Resource)}]
	{ \kwa{ho \hyphen safe}( \{ \bar r \}, \varepsilon)} 
	{}
	~~~~~~
\infer[\textsc{(HOSafe-Unit)}]
	{ \kwa{ho \hyphen safe}( \kwa{Unit}, \varepsilon)} 
	{}\\[3ex]

\infer[\textsc{(HOSafe-Arrow)}]
	{ \kwa{ho \hyphen safe}( \hat \tau_1 \rightarrow_{\varepsilon'} \hat \tau_2, \varepsilon ) }
	{ \kwa{safe}(\hat \tau_1, \varepsilon)  & \kwa{ho \hyphen safe}(\hat \tau_2, \varepsilon) }\\[3ex]

\end{array}
\]

\noindent
$\fbox{$\hat \tau <: \hat \tau$}$

\[
\begin{array}{c}

\infer[(\textsc{S-Effects})]
	{\hat \tau_1 \rightarrow_{\varepsilon} \hat \tau_2 <: \hat \tau_1' \rightarrow_{\varepsilon'} \hat \tau_2'}
	{\varepsilon \subseteq \varepsilon' & \hat \tau_2 <: \hat \tau_2' & \hat \tau_1' <: \hat \tau_1}
~~~
\infer[(\textsc{S-ResourceSet})]
	{\{ \bar r_2 \} <: \{ \bar r_1 \}}
	{r \in \bar r_2 \implies r \in \bar r_1}


\end{array}
\]

\section{Dynamic Rules}

\noindent
\fbox{$\hat e \longrightarrow \hat e~|~\varepsilon$}

\[
\begin{array}{c}

\infer[\textsc{(E-App1)}]
	{\hat e_1 \hat e_2 \longrightarrow \hat e_1' \hat e_2~|~\varepsilon}
	{\hat e_1 \longrightarrow \hat e_1'~|~\varepsilon}
	~~~~~~
\infer[\textsc{(E-App2)}]
	{\hat v_1 \hat e_2 \longrightarrow \hat v_1 \hat e_2'~|~\varepsilon} 
	{\hat e_2 \longrightarrow \hat e_2'~|~\varepsilon}
~~~~~~
\infer[\textsc{(E-App3)}]
	{ (\lambda x: \hat \tau.\hat e) \hat v_2 \longrightarrow [\hat v_2/x]\hat e~|~\varnothing }
	{}\\[2ex]
	
\infer[\textsc{(E-OperCall1)}]
	{\hat e.\pi \longrightarrow \hat e'.\pi~|~\varepsilon }
	{\hat e \rightarrow \hat e'~|~\varepsilon}
		
	~~~~~~
	
\infer[\textsc{(E-OperCall2)}]
	{r.\pi \longrightarrow \kwa{unit}~|~\{ r.\pi \} }
	{r \in R & \pi \in \Pi}
	 \\[4ex]
	 
\infer[\textsc{(E-TypePoly)}]
	{\poly{t}{\tau}{e} \longrightarrow [\tau/t]e}
	{}
	~~~
\infer[\textsc{(E-FxPoly)}]
	{\poly{\epsilon}{\varepsilon}{e} \longrightarrow [\varepsilon/\epsilon]e}
	{} \\[2ex]
	 
\infer[\textsc{(E-Import1)}]
	{\import{\varepsilon_s}{x}{\hat e}{e} \longrightarrow \import{\varepsilon_s}{x}{\hat e'}{e}~|~\varepsilon'}
	{\hat e \longrightarrow \hat e'~|~\varepsilon'}\\[4ex]

\infer[\textsc{(E-Import2)}]
	{\import{\varepsilon_s}{x}{\hat e}{e} \longrightarrow [\hat v/x]\annot{e}{\varepsilon_s}~|~\varnothing}
	{}

\end{array}
\]


\section{Encodings}

\subsection{$\bot$}
The bottom type is defined as $\kwa{\bot \defn \varnothing}$, which is the literal for an empty set of resources.

\[
\begin{array}{c}


\infer[\textsc{(T-$\bot$)}]
	{\Gamma \vdash \bot : \varnothing}
	{}
~~~~~~
\infer[\textsc{($\varepsilon$-$\bot$)}]
	{\Gamma \vdash \bot : \varnothing~\kw{with} \varnothing}
	{}

\end{array}
\]

\subsection{$\unit,~\Unit$}

\noindent
Define $\kwa{unit = \lambda x: \varnothing.x}$, i.e. the function which takes an empty set of resources and returns it. We shall refer to its type, which is $\kwa{\varnothing \rightarrow_{\varnothing} \varnothing}$, as $\Unit$. It has various properties befitting $\unit$.

\begin{enumerate}
	\item $\unit$ cannot be invoked as $\varnothing$ is uninhabited.
	\item $\unit$ is a value.
	\item The only term with type $\Unit$ is $\unit$.
	\item $\vdash \unit : \Unit$ by using \textsc{$\varepsilon$-Abs} and \textsc{$\varepsilon$-Var}.
	\item $\kwa{effects}(\Unit) = \kwa{ho \hyphen effects}(\Unit) = \varnothing$
	\item $\kwa{safe}(\Unit, \varepsilon)$ and $\kwa{ho \hyphen safe}(\Unit, \varepsilon)$
\end{enumerate}

\[
\begin{array}{c}


\infer[\textsc{(T-Unit)}]
	{\Gamma \vdash \unit: \Unit}
	{}
~~~~~~
\infer[\textsc{($\varepsilon$-Unit)}]
	{\Gamma \vdash \unit: \Unit~\kw{with} \varnothing}
	{}

\end{array}
\]

\end{document}










