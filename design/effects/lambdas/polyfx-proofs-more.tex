\documentclass{llncs}

\usepackage{listings}
\usepackage{proof}
\usepackage{amssymb}
\usepackage[margin=.9in]{geometry}
\usepackage{amsmath}
\usepackage[english]{babel}
\usepackage[utf8]{inputenc}
\usepackage{enumitem}
\usepackage{filecontents}
\usepackage{calc}
\usepackage[linewidth=0.5pt]{mdframed}
\usepackage{changepage}
\allowdisplaybreaks

\usepackage{fancyhdr}
\renewcommand{\headrulewidth}{0pt}
\pagestyle{fancy}
 \fancyhf{}
\rhead{\thepage}

\lstset{tabsize=3, basicstyle=\ttfamily\small, commentstyle=\itshape\rmfamily, numbers=left, numberstyle=\tiny, language=java,moredelim=[il][\sffamily]{?},mathescape=true,showspaces=false,showstringspaces=false,columns=fullflexible,xleftmargin=5pt,escapeinside={(@}{@)}, morekeywords=[1]{objtype,module,import,let,in,fn,var,type,rec,fold,unfold,letrec,alloc,ref,application,policy,external,component,connects,to,meth,val,where,return,group,by,within,count,connect,with,attr,html,head,title,style,body,div,keyword,unit,def}}
\lstloadlanguages{Java,VBScript,XML,HTML}

\newcommand{\keywadj}[1]{\mathtt{#1}}
\newcommand{\keyw}[1]{\keywadj{#1}~}

\newcommand{\kw}[1]{\keyw{ #1 }}
\newcommand{\kwa}[1]{\keywadj{ #1 }}
\newcommand{\reftt}{\mathtt{ref}~}
\newcommand{\Reftt}{\mathtt{Ref}~}
\newcommand{\inttt}{\mathtt{int}~}
\newcommand{\Inttt}{\mathtt{Int}~}
\newcommand{\stepsto}{\leadsto}
\newcommand{\todo}[1]{\textbf{[#1]}}
\newcommand{\intuition}[1]{#1}
\newcommand{\hyphen}{\hbox{-}}

%\newcommand{\intuition}[1]{}

\newlist{pcases}{enumerate}{1}
\setlist[pcases]{
  label=\fbox{\textit{Case}}\protect\thiscase\textit{:}~,
  ref=\arabic*,
  align=left,
  labelsep=0pt,
  leftmargin=0pt,
  labelwidth=0pt,
  parsep=0pt
}
\newcommand{\pcase}[1][]{

  \if\relax\detokenize{#1}\relax
    \def\thiscase{}
  \else
    \def\thiscase{~\fbox{#1:}}
  \fi
  \item
}

\newcommand{\thm}[3]{
	\begin{large}
		\bf{#1}
	\end{large} \\\\
	\fbox{Statement.} ~ #2
	\fbox{Proof.}~ #3 \qed
}

\newcommand{\proofcase}[2]{
	\begin{adjustwidth}{1.5em}{0pt}
		\fbox{Case.}~~#1. \\ ~#2
	\end{adjustwidth}
}

\newcommand{\subcase}[1] {
	\begin{adjustwidth}{2.2em}{0pt}
		\underline{Subcase.} #1
	\end{adjustwidth}
}

\newcommand{\stmt}[1] {

\begin{adjustwidth}{2.5em}{0pt}
	#1
\end{adjustwidth}

}
\newcommand{\type}[2]{
	#1~\keyw{with} #2
}

\newcommand{\unit}[0]{ \kwa{unit} }

\newcommand{\Unit}[0]{ \kwa{Unit} }

\newcommand{\fx}[1]{ \kwa{effects}(#1) }

\newcommand{\hofx}[1]{ \kwa{ho \hyphen effects}(#1) }

\newcommand{\safe}[2]{ \kwa{safe}(#1, #2) }

\newcommand{\hosafe}[2]{ \kwa{ho \hyphen safe}(#1, #2) }

\newcommand{\wf}[1]{ \kwa{WF}(#1) }

\newcommand{\fv}[1]{ \kwa{free \hyphen vars}(#1) }

\newcommand{\arr}[3]{
	#1 \rightarrow_{#3} #2
}

\newcommand{\newd}[0]{
	\keywadj{new}_d~x \Rightarrow \overline{d = e}
}

\newcommand{\newsig}[0]{
	\keywadj{new}_\sigma~x \Rightarrow \overline{\sigma = e}
}

\newcommand{\import}[4]{
	\keywadj{import}(#1)~#2 = #3~\kw{in} #4
}

\newcommand{\annot}[2]{
	\keywadj{annot}(#1, #2)
}

\newcommand{\erase}[1]{
	\keywadj{erase}(#1)
}

\newcommand{\reannot}[2]{
	\keywadj{reannot}(#1, #2)
}

\newcommand{\poly}[2]{
	\forall #1. #2
}

\newcommand{\polycap}[3]{
	\forall #1. #2~ \kw{caps} #3
}

\newcommand{\ispoly}[1]{
	\kwa{is \hyphen poly}(#1)
}

\newcommand{\lub}[1]{
	\kwa{lub}(#1)
}

\newcommand{\ub}[1]{
	\kwa{ub}(#1)
}

\newcommand\defn{\mathrel{\overset{\makebox[0pt]{\mbox{\normalfont\tiny\sffamily def}}}{=}}}


\begin{document}

\noindent
Notation: if $\hat \Gamma \vdash \fx{\hat \tau} = \varepsilon$, we write $\fx{\hat \Gamma, \hat \tau} = \varepsilon$.

\hrulefill

\begin{lemma}[Effect-VarAbstraction]
If $\hat \Gamma \vdash \fx{[\varnothing/\Phi]\hat \tau} = \varepsilon$, then $\hat \Gamma, \Phi \subseteq \varnothing \vdash \fx{\hat \tau} = \varepsilon$.
\end{lemma}

\hrulefill

\begin{lemma}[Effect-VarWeakening] If the following are true:
\begin{enumerate}
	\item $\hat \Gamma, \Phi \subseteq \varepsilon_1, \hat \Delta \vdash \fx{\hat \tau} = \varepsilon_0$
	\item $\Phi \notin \kwa{dom}(\hat \Delta)$
	\item $\hat \Gamma \vdash \varepsilon_1 \subseteq \varepsilon_2$
\end{enumerate}

\noindent
Then $\hat \Gamma, \Phi \subseteq \varepsilon_2, \hat \Delta \vdash \fx{\hat \tau} = \varepsilon_0 \cup \varepsilon_0'$, where $\varepsilon_0' \subseteq \varepsilon_2 - \varepsilon_1$.


\end{lemma}

\hrulefill

\begin{lemma}[Approximation I]
If the following are true:
\begin{enumerate}
	\item $\hat \Gamma \vdash \hat \tau$
	\item $\hat \Gamma \vdash \fx{\hat \Gamma, \hat \tau} \subseteq \varepsilon_s$
	\item $\hat \Gamma \vdash \hosafe{\hat \tau}{\varepsilon_s}$
\end{enumerate}

\noindent
then $\hat \Gamma \vdash \hat \tau <: \reannot{\hat \tau}{\varepsilon_s}$

\end{lemma}

\begin{proof}

\noindent
By mutual induction with the Approximation II lemma on the form of $\hat \tau$.\\

\fbox{Case: $\hat \tau = \{ \overline{r} \}$} Since $\reannot{\{ \overline{r} \}}{\varepsilon_s} = \{ \overline{r} \}$, we can obtain the theorem conclusion by applying $\textsc{S-Reflexive}$. \\

\fbox{Case: $\hat \tau = \forall X <: \hat \tau_1. \hat \tau_2~\kw{caps} \varepsilon_3$}  The theorem conclusion can be written as $\hat \Gamma \vdash (\forall \Phi \subseteq \varepsilon_1. \hat \tau_2~\kw{caps} \varepsilon_3) <: (\forall \Phi \subseteq \varepsilon_1. \reannot{\hat \tau_2}{\varepsilon_s}~\kw{caps} \varepsilon_3)$. To establish this, we use $\textsc{S-PolyFx}$, which requires us to establish the following premises:

\begin{enumerate}
	\setcounter{enumi}{3}
	\item $\hat \Gamma \vdash \varepsilon \subseteq \varepsilon$
	\item $\hat \Gamma, \Phi \subseteq \varepsilon_1 \vdash \varepsilon_3 \subseteq \varepsilon_3$
	\item $\hat \Gamma, \Phi \subseteq \varepsilon_1 \vdash \hat \tau_2 <: \reannot{\hat \tau_2}{\varepsilon_s}$
\end{enumerate}

\noindent
(4) and (5) are true by reflexivity. Therefore, to establish the theorem conclusion, it is sufficient to establish (6). By inversion on (3), we know (7, 8).

\begin{enumerate}
	\setcounter{enumi}{6}
	\item $\hat \Gamma \vdash \varepsilon_1 \subseteq \varepsilon_s$
	\item $\hat \Gamma, \Phi \subseteq \varepsilon_1 \vdash \hosafe{\hat \tau_2}{\varepsilon_s}$
\end{enumerate}

\noindent
By inversion on (2), we know $\fx{\hat \Gamma, \hat [\varnothing/\Phi]\hat \tau_2} \subseteq \varepsilon_s$. By the \textsc{Effect-VarAbstraction} lemma, this is the same as $\fx{(\hat \Gamma, \Phi \subseteq \varnothing), \hat \tau_2} \subseteq \varepsilon_s$. By the \textsc{Effect-VarWeakening} lemma, we know that $\fx{(\hat \Gamma, \Phi \subseteq \varepsilon_1), \hat \tau} \subseteq \varepsilon_s \cup \varepsilon_1$. Because of (7), $\varepsilon_s \cup \varepsilon_1 = \varepsilon_s$. Therefore, we have (9):

\begin{enumerate}
	\setcounter{enumi}{8}
	\item $\hat \Gamma \vdash \fx{(\hat \Gamma, \Phi \subseteq \varepsilon_1), \hat \tau_2} \subseteq \varepsilon_s$
\end{enumerate}

\noindent
Also, since $\hat \Gamma \vdash \hat \tau$, we know that $\hat \Gamma \vdash \hat \tau_2$. Trivially, $\hat \Gamma, \Phi \subseteq \varepsilon_1 \vdash \hat \tau_2$. With this judgement, as well as (8, 9), we can apply the inductive assumption of Approximation I, giving judgement (6).

\end{proof}

\end{document}