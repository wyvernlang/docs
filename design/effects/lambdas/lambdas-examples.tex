\documentclass{llncs}

\usepackage{listings}
\usepackage{proof}
\usepackage{amssymb}
\usepackage[margin=.9in]{geometry}
\usepackage{amsmath}
\usepackage[english]{babel}
\usepackage[utf8]{inputenc}
\usepackage{enumitem}
\usepackage{filecontents}
\usepackage{calc}
\usepackage[linewidth=0.5pt]{mdframed}
\usepackage{changepage}
\usepackage{tabto}
\allowdisplaybreaks

\usepackage{fancyhdr}
\renewcommand{\headrulewidth}{0pt}
\pagestyle{fancy}
 \fancyhf{}
\rhead{\thepage}

\lstset{tabsize=3, basicstyle=\ttfamily\small, commentstyle=\itshape\rmfamily, numbers=left, numberstyle=\tiny, language=java,moredelim=[il][\sffamily]{?},mathescape=true,showspaces=false,showstringspaces=false,columns=fullflexible,xleftmargin=5pt,escapeinside={(@}{@)}, morekeywords=[1]{objtype,module,import,let,in,fn,var,type,rec,fold,unfold,letrec,alloc,ref,application,policy,external,component,connects,to,meth,val,where,return,group,by,within,count,connect,with,attr,html,head,title,style,body,div,keyword,unit,def}}
\lstloadlanguages{Java,VBScript,XML,HTML}

\newcommand{\keywadj}[1]{\mathtt{#1}}
\newcommand{\keyw}[1]{\keywadj{#1}~}

\newcommand{\kw}[1]{\keyw{ #1 }}
\newcommand{\kwa}[1]{\keywadj{ #1 }}
\newcommand{\reftt}{\mathtt{ref}~}
\newcommand{\Reftt}{\mathtt{Ref}~}
\newcommand{\inttt}{\mathtt{int}~}
\newcommand{\Inttt}{\mathtt{Int}~}
\newcommand{\stepsto}{\leadsto}
\newcommand{\todo}[1]{\textbf{[#1]}}
\newcommand{\intuition}[1]{#1}
\newcommand{\hyphen}{\hbox{-}}

%\newcommand{\intuition}[1]{}

\newlist{pcases}{enumerate}{1}
\setlist[pcases]{
  label=\fbox{\textit{Case}}\protect\thiscase\textit{:}~,
  ref=\arabic*,
  align=left,
  labelsep=0pt,
  leftmargin=0pt,
  labelwidth=0pt,
  parsep=0pt
}
\newcommand{\pcase}[1][]{

  \if\relax\detokenize{#1}\relax
    \def\thiscase{}
  \else
    \def\thiscase{~\fbox{#1:}}
  \fi
  \item
}

\newcommand{\thm}[3]{
	\begin{large}
		\bf{#1}
	\end{large} \\\\
	\fbox{Statement.} ~ #2
	\fbox{Proof.}~ #3 \qed
}

\newcommand{\proofcase}[2]{
	\begin{adjustwidth}{1.5em}{0pt}
		\fbox{Case.}~~#1. \\ ~#2
	\end{adjustwidth}
}

\newcommand{\subcase}[1] {
	\begin{adjustwidth}{2.2em}{0pt}
		\underline{Subcase.} #1
	\end{adjustwidth}
}

\newcommand{\stmt}[1] {

\begin{adjustwidth}{2.5em}{0pt}
	#1
\end{adjustwidth}

}
\newcommand{\type}[2]{
	#1~\keyw{with} #2
}

\newcommand{\unit}[0]{ \kwa{unit} }

\newcommand{\Unit}[0]{ \kwa{Unit} }

\newcommand{\arr}[3]{
	#1 \rightarrow_{#3} #2
}

\newcommand{\module}[0]{
\kwa{import}(\varepsilon)~x = \hat e~\kwa{in}~e
}

\newcommand{\newd}[0]{
	\keywadj{new}_d~x \Rightarrow \overline{d = e}
}

\newcommand{\newsig}[0]{
	\keywadj{new}_\sigma~x \Rightarrow \overline{\sigma = e}
}


\begin{document}

\section{Examples}



\subsection{Safe Logger}
\begin{lstlisting}
import(File.append)
   log = $\lambda$x:Unit . File.append
   in log unit
\end{lstlisting}

\noindent
Firstly $\vdash \lambda x:\kwa{Unit~.~File.append : Unit \rightarrow_{File.append} Unit~\kw{with} \varnothing}$ by \textsc{$\varepsilon$-Abs}. Call this type $\hat \tau$. $\kwa{effects}(\hat \tau) = \{ \kwa{File.append} \}$ and its erasure is $\Unit \rightarrow \Unit$. Now $\kwa{log} : \Unit \rightarrow \Unit \vdash \kwa{log}~\unit : \Unit$ by \textsc{T-Abs}. \\

\noindent
By definition $\kwa{ho \hyphen safe(Unit \rightarrow_{File.append} Unit, \{ \kwa{File.append} \})}$  iff $\kwa{safe}(\Unit, \{ \kwa{File.append} \})$ and \\ $\kwa{ho \hyphen safe}(\Unit, \{ \kwa{File.append} \})$. The two conjuncts are true by \textsc{Safe-Unit} and \textsc{HOSafe-unit}.\\

\subsection{Logger Uses Undefined Capability}
\begin{lstlisting}
import(File.append)
   log = $\lambda$x:Unit . File.write
   in log unit
\end{lstlisting}

\noindent
Firstly $\kwa{\vdash \lambda x:Unit.~File.write: Unit \rightarrow_{File.write} Unit~with~\varnothing}$ by \textsc{$\varepsilon$-Abs}. Now: \\

\noindent
$\kwa{effects}(\kwa{Unit \rightarrow_{File.write} Unit})$ \\
$= \{ \kwa{File.write} \} \cup \kwa{effects}(\Unit) \cup \kwa{ho \hyphen effects}(\Unit)$ \\
$= \{ \kwa{File.write} \}$ \\

\noindent
But $\{ \kwa{File.write} \} \neq \{ \kwa{File.append} \}$, the set of capabilities declared by the module. Hence this program doesn't typecheck. \\

\subsection{Higher-Order Safe}
\begin{lstlisting}
import(File.append)
   getLogger = $\lambda$x:Unit. ($\lambda$y:Unit. File.append)
in (getLogger unit) unit
\end{lstlisting}

\noindent
Firstly, $\kwa{x:Unit \vdash \lambda y:Unit .~File.append : \Unit \rightarrow_{File.append} \Unit ~with ~ \varnothing} $ by \textsc{$\varepsilon$-Abs}. Then by \textsc{$\varepsilon$-Abs} again, $\kwa{\vdash \lambda x:Unit.~(\lambda y:Unit.~File.append): Unit \rightarrow_{\varnothing} Unit \rightarrow_{File.append} Unit}~\kw{with} \varnothing$. This is our $\hat \tau$. \\

\noindent
The set of effects declared by the module is $\varepsilon = \{ \kwa{File.append} \}$. This needs to be the same as $\kwa{effects}(\hat \tau)$. \\

\noindent
$\kwa{effects}(\hat \tau)$ \\
$= \kwa{effects}(\kwa{Unit \rightarrow_{\varnothing} Unit \rightarrow_{File.append} Unit})$ \\
$ = \kwa{ho \hyphen effects(Unit) \cup \varnothing \cup \kwa{effects}(Unit \rightarrow_{File.append} Unit)}$ \\
$ = \kwa{effects(Unit \rightarrow_{File.append} Unit)}$ \\
$ = \kwa{ho \hyphen effects(Unit) \cup \{ File.append \} \cup effects(Unit)}$ \\
$ = \{ \kwa{File.append} \}$\\

\noindent
We also need higher-order safety.\\\\
$\kwa{ho \hyphen safe(Unit \rightarrow_{\varnothing} Unit \rightarrow_{File.append} Unit, \{ File.append \})}$\\
$\kwa{\equiv safe(Unit, \{ File.append \}) \land ho \hyphen safe(Unit \rightarrow_{File.append} Unit, \{ File.append \})}$ \\
$\kwa{\equiv ho \hyphen safe(Unit \rightarrow_{File.append} Unit, \{ File.append \})}$ \\
$\equiv \kwa{safe(Unit, \{ File.append \}) \land ho \hyphen safe(Unit, \{ File.append \})}$ \\
$\equiv \kwa{True}$\\

\noindent
Lastly, $\kwa{erase(Unit \rightarrow_{\varnothing} Unit \rightarrow_{File.append} Unit) = Unit \rightarrow Unit \rightarrow Unit}$. By using \text{T-App} twice we have \\ $\kwa{getLogger: Unit \rightarrow Unit \rightarrow Unit \vdash (getLogger~unit)~unit: Unit}$.\\

\subsection{Higher-Order Unsafe}

In this example the module leaks a capability for appending to a file, violating its signature.

\begin{lstlisting}
import($\varnothing$)
   getLogger = $\lambda$x:Unit . ($\lambda$y:Unit . File.append)
in (getLogger unit) unit
\end{lstlisting}

\noindent
By \textsc{$\varepsilon$-Abs}, $\kwa{\vdash \lambda x:Unit.~(\lambda y:Unit.~File.append): Unit \rightarrow_{\varnothing} Unit \rightarrow_{File.append} Unit}~\kw{with} \varnothing$. This is our $\hat \tau$. The set of effects declared by the module is $\varnothing$. This needs to be the same as $\kwa{effects}(\hat \tau)$. \\

\noindent
$\kwa{effects}(\hat \tau)$ \\
$= \kwa{effects}(\kwa{Unit \rightarrow_{\varnothing} Unit \rightarrow_{File.append} Unit})$ \\
$ = \kwa{ho \hyphen effects(Unit) \cup \varnothing \cup \kwa{effects}(Unit \rightarrow_{File.append} Unit)}$ \\
$ = \kwa{effects(Unit \rightarrow_{File.append} Unit)}$ \\
$ = \kwa{ho \hyphen effects(Unit) \cup \{ File.append \} \cup effects(Unit)}$ \\
$ = \{ \kwa{File.append} \} \neq \varnothing$\\

\noindent
So the example fails to typecheck.


\subsection{Higher-Order Unsafe 2}

In this example we pass in a function which writes and appends to a file. However, the signature expects it to only be appending.

\begin{lstlisting}
import(File.append)
    logger = $\lambda$f:Unit $\rightarrow_{\kwa{File.append}}$ Unit. f unit
in logger ($\lambda$x:Unit. let y = File.append in File.write)
\end{lstlisting}

\noindent
It desugars into this program.

\begin{lstlisting}
import(File.append)
    logger = $\lambda$f:Unit $\rightarrow_{\kwa{File.append}}$ Unit. f unit
in logger ($\lambda$x:Unit. ($\lambda$y:Unit. File.append) File.write)
\end{lstlisting}

\noindent
By \textsc{$\varepsilon$-App} we have $\kwa{f : Unit \rightarrow_{File.append} Unit \vdash f~unit: Unit~\kw{with} \{ File.append \}}$. Then by \textsc{$\varepsilon$-Abs} we have $\kwa{\vdash logger: (Unit \rightarrow_{File.append} Unit) \rightarrow_{varnothing} Unit~\kw{with} \varnothing }$. This is our $\hat \tau$. \\

\noindent
The set of effects declared by this module is $\kwa{\{ File.append \}}$. We need this to be the same as $\kwa{effects}(\hat \tau)$. By definition, \\

\noindent
$\kw{effects((Unit \rightarrow_{File.append} Unit) \rightarrow_{varnothing} Unit)}$\\
$\kw{= ho \hyphen effects(Unit \rightarrow_{File.append} Unit) \cup \varnothing \cup effects(Unit)}$ \\
$\kw{= ho \hyphen effects(Unit \rightarrow_{File.append} Unit)}$ \\
$\kw{= effects(Unit) \cup ho \hyphen effects(Unit)}$ \\
$\kw{= \varnothing \subseteq \{File.append\}}$ \\

\noindent
Here is the derivation of higher-order safety. \\
$\kw{ho \hyphen safe((Unit \rightarrow_{File.append} Unit) \rightarrow_{varnothing} Unit, \{File.append\})}$ \\
$\kw{\equiv safe(Unit \rightarrow_{File.append} Unit, \{File.append\}) \land ho \hyphen safe(Unit, \{File.append\})}$ \\
$\kw{\equiv safe(Unit \rightarrow_{File.append} Unit, \{File.append\})}$ \\
$\kw{\equiv \{File.write\} \subseteq \{File.write\} \land safe(Unit, \{ File.write \}) \land ho \hyphen safe(Unit , \{ File.write \})}$ \\
$\equiv \kw{True}$\\

\noindent
Now by \textsc{T-Abs} we have $\kwa{logger: Unit \rightarrow Unit, x: Unit \vdash \lambda y: Unit.~File.append: Unit \rightarrow Unit}$. By \textsc{T-Abs} again we have $\kwa{logger: Unit \rightarrow Unit \vdash \lambda x:Unit.(\lambda y:Unit.~File.write): Unit \rightarrow Unit \rightarrow Unit}$. \\

\noindent
\textbf{So this example typechecks (but it shouldn't)}

\subsection{Higher-Order Unsafe 3}

In this example we pass in a function which returns a function which writes to a file (not allowed).

\begin{lstlisting}
import($\varnothing$)
    logger = $\lambda$f:Unit $\rightarrow_{\varnothing}$ Unit  $\rightarrow_{\varnothing}$ Unit. f unit unit
in logger ($\lambda$x:Unit. ($\lambda$y:Unit. File.write))
\end{lstlisting}

\noindent
Firstly by \textsc{$\varepsilon$-App} we have $\kwa{f:Unit \rightarrow_{\varnothing} Unit  \rightarrow_{\varnothing} Unit \vdash f~unit~unit: Unit~\kw{with} \varnothing}$. Then by \textsc{$\varepsilon$-Abs} we have $\kwa{\vdash logger: Unit \rightarrow_{\varnothing} Unit \rightarrow_{\varnothing} Unit \rightarrow_{\varnothing} Unit}$. This is our $\hat \tau$.\\

\noindent
Secondly $\kwa{effects}(\hat \tau) = \varnothing$ which is the same as the set of effects declared by this module. This is also trivially higher-order safe. \\

\noindent
Now $\kwa{erase(Unit \rightarrow_{\varnothing} Unit  \rightarrow_{\varnothing} Unit) = Unit \rightarrow Unit \rightarrow Unit}$ and by using \textsc{T-App} and \textsc{T-Abs} we have \\ $\kwa{logger: Unit \rightarrow Unit \rightarrow Unit \vdash logger (\lambda x:Unit. (\lambda y:Unit. File.write)): Unit }$. \\

\noindent
\textbf{So this example typechecks (but it shouldn't)}

\end{document}




