\documentclass{llncs}

\usepackage{listings}
\usepackage{proof}
\usepackage{amssymb}
\usepackage[margin=.9in]{geometry}
\usepackage{amsmath}
\usepackage[english]{babel}
\usepackage[utf8]{inputenc}
\usepackage{enumitem}
\usepackage{filecontents}
\usepackage{calc}
\usepackage[linewidth=0.5pt]{mdframed}
\usepackage{changepage}
\usepackage{tabto}
\allowdisplaybreaks

\usepackage{fancyhdr}
\renewcommand{\headrulewidth}{0pt}
\pagestyle{fancy}
 \fancyhf{}
\rhead{\thepage}

\lstset{tabsize=3, basicstyle=\ttfamily\small, commentstyle=\itshape\rmfamily, numbers=left, numberstyle=\tiny, language=java,moredelim=[il][\sffamily]{?},mathescape=true,showspaces=false,showstringspaces=false,columns=fullflexible,xleftmargin=5pt,escapeinside={(@}{@)}, morekeywords=[1]{objtype,module,import,let,in,fn,var,type,rec,fold,unfold,letrec,alloc,ref,application,policy,external,component,connects,to,meth,val,where,return,group,by,within,count,connect,with,attr,html,head,title,style,body,div,keyword,unit,def}}
\lstloadlanguages{Java,VBScript,XML,HTML}

\newcommand{\keywadj}[1]{\mathtt{#1}}
\newcommand{\keyw}[1]{\keywadj{#1}~}

\newcommand{\kw}[1]{\keyw{ #1 }}
\newcommand{\kwa}[1]{\keywadj{ #1 }}
\newcommand{\reftt}{\mathtt{ref}~}
\newcommand{\Reftt}{\mathtt{Ref}~}
\newcommand{\inttt}{\mathtt{int}~}
\newcommand{\Inttt}{\mathtt{Int}~}
\newcommand{\stepsto}{\leadsto}
\newcommand{\todo}[1]{\textbf{[#1]}}
\newcommand{\intuition}[1]{#1}
\newcommand{\hyphen}{\hbox{-}}

%\newcommand{\intuition}[1]{}

\newlist{pcases}{enumerate}{1}
\setlist[pcases]{
  label=\fbox{\textit{Case}}\protect\thiscase\textit{:}~,
  ref=\arabic*,
  align=left,
  labelsep=0pt,
  leftmargin=0pt,
  labelwidth=0pt,
  parsep=0pt
}
\newcommand{\pcase}[1][]{

  \if\relax\detokenize{#1}\relax
    \def\thiscase{}
  \else
    \def\thiscase{~\fbox{#1:}}
  \fi
  \item
}

\newcommand{\thm}[3]{
	\begin{large}
		\bf{#1}
	\end{large} \\\\
	\fbox{Statement.} ~ #2
	\fbox{Proof.}~ #3 \qed
}

\newcommand{\proofcase}[2]{
	\begin{adjustwidth}{1.5em}{0pt}
		\fbox{Case.}~~#1. \\ ~#2
	\end{adjustwidth}
}

\newcommand{\subcase}[1] {
	\begin{adjustwidth}{2.2em}{0pt}
		\underline{Subcase.} #1
	\end{adjustwidth}
}

\newcommand{\stmt}[1] {

\begin{adjustwidth}{2.5em}{0pt}
	#1
\end{adjustwidth}

}
\newcommand{\type}[2]{
	#1~\keyw{with} #2
}

\newcommand{\unit}[0]{ \kwa{unit} }

\newcommand{\Unit}[0]{ \kwa{Unit} }

\newcommand{\fx}[1]{ \kwa{effects}(#1) }

\newcommand{\hofx}[1]{ \kwa{ho \hyphen effects}(#1) }

\newcommand{\safe}[2]{ \kwa{safe}(#1, #2) }

\newcommand{\hosafe}[2]{ \kwa{ho \hyphen safe}(#1, #2) }

\newcommand{\arr}[3]{
	#1 \rightarrow_{#3} #2
}

\newcommand{\module}[0]{
\kwa{import}(\varepsilon)~x = \hat e~\kwa{in}~e
}

\newcommand{\newd}[0]{
	\keywadj{new}_d~x \Rightarrow \overline{d = e}
}

\newcommand{\newsig}[0]{
	\keywadj{new}_\sigma~x \Rightarrow \overline{\sigma = e}
}


\begin{document}


\begin{lemma}
If $\varepsilon \subseteq \kwa{effects}(\hat \tau)$ and $\kwa{ho \hyphen safe}(\hat \tau, \varepsilon)$ then $\hat \tau <: \kwa{annot}(\kwa{erase}(\hat \tau), \varepsilon)$.
\end{lemma}

\noindent
\textbf{Counterexample.} Let $\hat \tau = \Unit \rightarrow_{a} \Unit$ and $\varepsilon = \varnothing$. Note $\kwa{annot(erase}(\Unit \rightarrow_{a} \Unit), \varnothing) = \Unit \rightarrow_{\varnothing} \Unit$ \\

\noindent
$\kwa{effects}(\Unit \rightarrow_{a} \Unit) = \{ a \}\supseteq \varepsilon = \varnothing$ \\

\noindent
$\kwa{ho \hyphen safe}(\Unit \rightarrow_{a} \Unit, \varnothing) = \kwa{safe}(\Unit, \varnothing) \land \kwa{ho \hyphen safe}(\Unit, \varnothing) = \kwa{True}$ \\

\noindent
The lemma applies, so $\Unit \rightarrow_{a} \Unit <: \Unit \rightarrow_{\varnothing} \Unit$. This implies that $\{ a \} \subseteq \varnothing$. \\

\hrulefill


\begin{lemma}
If (1) $\kwa{effects}(\hat \tau) \subseteq \varepsilon$ and (2) $\kwa{ho \hyphen safe}(\hat \tau, \varepsilon)$ then $\hat \tau <: \kwa{annot}(\kwa{erase}(\hat \tau), \varepsilon)$.
\end{lemma}

\noindent
\textbf{Counterexample.}  Let $\hat \tau = ((\Unit \rightarrow_{a} \Unit) \rightarrow_{\varnothing} \Unit) \rightarrow_{\varnothing} \Unit$ and $\varepsilon = \{ a \}$. \\

\noindent
\textit{Proof of (1).}

	$\fx{((\Unit \rightarrow_{a} \Unit) \rightarrow_{\varnothing} \Unit) \rightarrow_{\varnothing} \Unit}$
	
	$= \hofx{(\Unit \rightarrow_{a} \Unit) \rightarrow_{\varnothing} \Unit)} \cup \fx{\Unit} $
	
	$= \fx{\Unit \rightarrow_{a} \Unit} \cup \fx{\Unit}$
	
	$= \{ a \}$, therefore (1) is true. \\
	
\noindent
\textit{Proof of (2).}

	$\hosafe{((\Unit \rightarrow_{a} \Unit) \rightarrow_{\varnothing} \Unit) \rightarrow_{\varnothing} \Unit}{\{ a \}}$
	
	$= \safe{(\Unit \rightarrow_{a} \Unit) \rightarrow_{\varnothing} \Unit}{\{ a \}} \land \hosafe{\Unit}{\{ a \}}$
	
	$= \safe{(\Unit \rightarrow_{a} \Unit) \rightarrow_{\varnothing} \Unit}{\{ a \}}$
	
	This is untrue as $\{ a \} \nsubseteq \varnothing$, so this is not a valid counterexample.
	
%\begin{lemma}
%If $\kwa{effects}(\hat \tau) \subseteq \varepsilon$ and $\kwa{ho \hyphen safe}%(\hat \tau, \varepsilon)$ then $\hat \tau <: \kwa{annot}(\kwa{erase}(\hat \tau), \varepsilon)$.
%\end{lemma}


%\begin{lemma}
%If $\varepsilon \subseteq \kwa{ho \hyphen effects}(\hat \tau)$ and $\kwa{safe}(\hat \tau, \varepsilon)$ then $\kwa{annot}(\kwa{erase}(\hat \tau), \varepsilon) <: \hat \tau$.
%\end{lemma}


%\begin{proof} By a case analysis. \\

%\noindent
%\fbox{$\hat \tau = \hat \tau_1 \rightarrow_{\varepsilon} \hat \tau_2$, %$\kwa{effects}(\hat \tau) \subseteq \varepsilon$, $\kwa{ho \hyphen safe}(\hat %\tau, \varepsilon)$}

%\noindent
%If $\kwa{annot(erase}(\hat \tau_1), \varepsilon) <: \hat \tau$ and $\hat \tau_2 <: %\kwa{annot(erase}(\hat \tau_2), \varepsilon)$ then $\hat \tau <: %\kwa{annot(erase}(\hat \tau), \varepsilon)$ by \textsc{S-Effects}. We proceed with %showing these two premises hold. \\


%\noindent
%\fbox{$\hat \tau = \hat \tau_1 \rightarrow_{\varepsilon} \hat \tau_2$, %$\varepsilon \subseteq \kwa{ho \hyphen effects}(\hat \tau)$, $\kwa{safe}(\hat %\tau, \varepsilon)$}

%\noindent
%If $\hat \tau_1 <: \kwa{annot(erase}(\hat \tau_2), \varepsilon)$ and %$\kwa{annot(erase}(\hat \tau_2), \varepsilon) <: \hat \tau_2$ then %$\kwa{annot(erase}(\hat \tau), \varepsilon) <: \hat \tau$ by \textsc{S-Effects}. %We proceed with showing these two premises hold. \\




%\end{proof}

\end{document}
