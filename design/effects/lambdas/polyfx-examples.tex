\documentclass{llncs}

\usepackage{listings}
\usepackage{proof}
\usepackage{amssymb}
\usepackage[margin=.9in]{geometry}
\usepackage{amsmath}
\usepackage[english]{babel}
\usepackage[utf8]{inputenc}
\usepackage{enumitem}
\usepackage{filecontents}
\usepackage{calc}
\usepackage[linewidth=0.5pt]{mdframed}
\usepackage{changepage}
\allowdisplaybreaks

\usepackage{fancyhdr}
\renewcommand{\headrulewidth}{0pt}
\pagestyle{fancy}
 \fancyhf{}
\rhead{\thepage}

\lstset{tabsize=3, basicstyle=\ttfamily\small, commentstyle=\itshape\rmfamily, numbers=left, numberstyle=\tiny, language=java,moredelim=[il][\sffamily]{?},mathescape=true,showspaces=false,showstringspaces=false,columns=fullflexible,xleftmargin=5pt,escapeinside={(@}{@)}, morekeywords=[1]{objtype,module,import,let,in,fn,var,type,rec,fold,unfold,letrec,alloc,ref,application,policy,external,component,connects,to,meth,val,where,return,group,by,within,count,connect,with,attr,html,head,title,style,body,div,keyword,unit,def}}
\lstloadlanguages{Java,VBScript,XML,HTML}

\newcommand{\keywadj}[1]{\mathtt{#1}}
\newcommand{\keyw}[1]{\keywadj{#1}~}

\newcommand{\kw}[1]{\keyw{ #1 }}
\newcommand{\kwa}[1]{\keywadj{ #1 }}
\newcommand{\reftt}{\mathtt{ref}~}
\newcommand{\Reftt}{\mathtt{Ref}~}
\newcommand{\inttt}{\mathtt{int}~}
\newcommand{\Inttt}{\mathtt{Int}~}
\newcommand{\stepsto}{\leadsto}
\newcommand{\todo}[1]{\textbf{[#1]}}
\newcommand{\intuition}[1]{#1}
\newcommand{\hyphen}{\hbox{-}}

%\newcommand{\intuition}[1]{}

\newlist{pcases}{enumerate}{1}
\setlist[pcases]{
  label=\fbox{\textit{Case}}\protect\thiscase\textit{:}~,
  ref=\arabic*,
  align=left,
  labelsep=0pt,
  leftmargin=0pt,
  labelwidth=0pt,
  parsep=0pt
}
\newcommand{\pcase}[1][]{

  \if\relax\detokenize{#1}\relax
    \def\thiscase{}
  \else
    \def\thiscase{~\fbox{#1:}}
  \fi
  \item
}

\newcommand{\thm}[3]{
	\begin{large}
		\bf{#1}
	\end{large} \\\\
	\fbox{Statement.} ~ #2
	\fbox{Proof.}~ #3 \qed
}

\newcommand{\proofcase}[2]{
	\begin{adjustwidth}{1.5em}{0pt}
		\fbox{Case.}~~#1. \\ ~#2
	\end{adjustwidth}
}

\newcommand{\subcase}[1] {
	\begin{adjustwidth}{2.2em}{0pt}
		\underline{Subcase.} #1
	\end{adjustwidth}
}

\newcommand{\stmt}[1] {

\begin{adjustwidth}{2.5em}{0pt}
	#1
\end{adjustwidth}

}
\newcommand{\type}[2]{
	#1~\keyw{with} #2
}

\newcommand{\unit}[0]{ \kwa{unit} }

\newcommand{\Unit}[0]{ \kwa{Unit} }

\newcommand{\fx}[1]{ \kwa{effects}(#1) }

\newcommand{\hofx}[1]{ \kwa{ho \hyphen effects}(#1) }

\newcommand{\safe}[2]{ \kwa{safe}(#1, #2) }

\newcommand{\hosafe}[2]{ \kwa{ho \hyphen safe}(#1, #2) }

\newcommand{\arr}[3]{
	#1 \rightarrow_{#3} #2
}

\newcommand{\newd}[0]{
	\keywadj{new}_d~x \Rightarrow \overline{d = e}
}

\newcommand{\newsig}[0]{
	\keywadj{new}_\sigma~x \Rightarrow \overline{\sigma = e}
}

\newcommand{\import}[4]{
	\keywadj{import}(#1)~#2 = #3~\kw{in} #4
}

\newcommand{\annot}[2]{
	\keywadj{annot}(#1, #2)
}

\newcommand{\erase}[1]{
	\keywadj{erase}(#1)
}

\newcommand{\poly}[2]{
	\forall #1. #2
}

\newcommand\defn{\mathrel{\overset{\makebox[0pt]{\mbox{\normalfont\tiny\sffamily def}}}{=}}}


\begin{document}

\section{Map Function}

\subsection*{Pseudo-Wyvern}
\begin{lstlisting}
def map(f: A $\rightarrow_{\phi}$ B, l: List[A]): List[B] with $\phi$ =
	if isnil l then []
	else cons (f (head l)) (map (tail l f))
\end{lstlisting}

\subsection*{$\lambda$-Calculus}
\begin{lstlisting}
map = $\lambda \phi$. $\lambda$A. $\lambda$B.
  $\lambda$f: A$\rightarrow_{\phi}$B.
    (fix ($\lambda$map: List[A] $\rightarrow$ List[B]).
      $\lambda$l: List[A].
        if isnil l then []
        else cons (f (head l)) (map (tail l f)))
\end{lstlisting}

\subsection*{Typing}

\begin{itemize}
	\item This has the type: $\forall \phi. \forall A. \forall B. (A \rightarrow_{\phi} B)  \rightarrow_{\varnothing} \kwa{List}[A] \rightarrow_{\phi} \kwa{List}[B]~ \kw{with} \varnothing$.
	\item $\kwa{map}~\varnothing$ is a pure version of map.
	\item $\kwa{map}~\{ \kwa{File.*} \}$ is a version of map which can perform operations on $\kwa{File}$.
\end{itemize}

\section{Dependency Injection}

\subsection*{Pseudo-Wyvern}

An HTTPServer module provides a single $\kwa{init}$ method which returns a $\kwa{Server}$ that responds to HTTP requests on the supplied socket.
\begin{lstlisting}
module HTTPServer

def init(out: A <: {File, Socket}): $\kwa{Str}\rightarrow_{A.write}\Unit$ with $\varnothing$ =
   $\lambda$ msg: Str.
      if (msg == ``POST'') then out.write(``post response'')
      else if (msg == ``GET'') then out.write(``get response'')
      else out.write(``client error 400'')
\end{lstlisting}

\noindent
The main module calls $\kwa{HTTPServer.init}$ with the $\kwa{Socket}$ it should be writing to.

\begin{lstlisting}
module Main
require HTTPServer, Socket

def main(): Unit =
   HTTPServer.init(Socket) ``GET /index.html''
\end{lstlisting}

\noindent
The testing module calls $\kwa{HTTPServer.init}$ with a $\kwa{LogFile}$, perhaps so the responses of the server can be tested offline.

\begin{lstlisting}
module Testing
require HTTPServer, LogFile

def testSocket():  =
   HTTPServer.init(LogFile) ``GET /index.html''
\end{lstlisting}


\subsection*{$\lambda$-Calculus}

\noindent
The HTTPServer module:
\begin{lstlisting}
HTTPServer = $\lambda$x: Unit.
   $\lambda$A <: {File, Socket}.
      $\lambda$out: A.
         $\lambda$msg: Str. A.write
\end{lstlisting}

\noindent
The Main module:

\begin{lstlisting}
Main = $\lambda$hs: HTTPServer. $\lambda$sock: Socket.
   $\lambda$x: Unit.
      (hs sock) ``GET /index.html''
\end{lstlisting}

\noindent
The Testing module:

\begin{lstlisting}
Testing = $\lambda$hs: HTTPServer. $\lambda$lf: LogFile.
   $\lambda$x: Unit.
      (hs lf) ``GET /index.html''
\end{lstlisting}

\subsection*{Types}

\begin{itemize}
	\item $\kwa{HTTPServer.init}$ has the type $\lambda A <: \{ \kwa{File, Socket} \}.~A \rightarrow_{\varnothing} \kwa{Str} \rightarrow_{A.write} \Unit$
\end{itemize}

%\section{File-Backed Data Structure}

%\section{Accessing Database via Expert}

\end{document}





