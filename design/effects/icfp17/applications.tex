
\section{Applications}

In this section we show how the capability-based design of $\epscalc$ can assist in reasoning about the effects and behaviour of a program. We present several scenarios which demonstrate unsafe behaviour or a particular developer story. This takes the form of writing a Wyvern program, expressing it in $\epscalc$ using the techniques of the last section, and then explaining how the rules of $\epscalc$ apply. 







































\subsection{Unannotated Client}

There is a single primitive capability $\kwa{File}$. A $\kwa{Logger}$ module possessing this capability exposes a function $\kwa{log}$ which incurs $\kwa{File.write}$ when executed. The $\kwa{Client}$ module possesses the $\kwa{Logger}$ module and exposes a function $\kwa{run}$ which invokes $\kwa{Logger.log}$, incurring $\kwa{File.write}$. While $\kwa{Logger}$ has been annotated, $\kwa{Client}$ has not --- if $\kwa{Client.run}$ is executed, what effects might it have? Code for this example is given below. 

\begin{lstlisting}
resource module Logger
require File

def log(): Unit with {File.append} =
    File.append(``message logged'')
\end{lstlisting}

\begin{lstlisting}
module Client
require Logger

def run(): Unit =
   Logger.log()
\end{lstlisting}

\begin{lstlisting}
resource module Main
require File
instantiate Logger(File)
instantiate Client(Logger)

Client.run()
\end{lstlisting}

The desugared version is given below. It first creates two functions, $\kwa{MakeLogger}$ and $\kwa{MakeClient}$, which instantiate the $\kwa{Logger}$ and $\kwa{Client}$ modules. Lines 1-3 define $\kwa{MakeLogger}$. When given a $\kwa{File}$, it returns a function representing $\kwa{Logger.log}$. Lines 5-8 define $\kwa{MakeClient}$. When given a $\kwa{Logger}$, it returns a function representing $\kwa{Client.run}$. Lines 10-15 define $\kwa{MakeMain}$ which returns a function which, when executed, instantiates all other modules and invokes the code in the body of $\kwa{Main}$. Program execution begins on line $17$, where $\kwa{Main}$ is given the initial capabilities --- which, in this case, is just $\kwa{File}$.

\begin{lstlisting}
let MakeLogger =
   ($\lambda$f: File.
      $\lambda$x: Unit. f.append) in
          
let MakeClient =
   ($\lambda$logger: Unit $\rightarrow_{ \{ \kwa{File.append} \}}$ Unit.
      import(File.append) l = logger in
         $\lambda$x: Unit. l unit) in
          
let MakeMain =
   ($\lambda$f: File.
      $\lambda$x: Unit.
         let LoggerModule = MakeLogger f in
         let ClientModule = MakeClient LoggerModule in
         ClientModule unit) in

(MakeMain File) unit
\end{lstlisting}

The interesting part  is on line $7$ where the unannotated code selects $\{ \kwa{File.append} \}$ as its authority. This is exactly the effects of the logger, i.e. $\kwa{effects}(\Unit \rightarrow_{\{\kwa{File.append}\}} \Unit) = \{ \kwa{File.append} \}$. The code also satisfies the higher-order safety predicates, and the body of the $\kwa{import}$ expression typechecks in the empty context. Therefore, the unannotated code typechecks by \textsc{$\varepsilon$-Import} with approximate effects $\kwa{\{ \kwa{File.append} \}}$.


























































\subsection{Unannotated Library}

The next example inverts the roles of the last scenario: now, the annotated $\kwa{Client}$ wants to use the unannotated $\kwa{Logger}$. $\kwa{Logger}$ captures $\kwa{File}$ and exposes a single function $\kwa{log}$ which incurs the $\kwa{File.append}$ effect. $\kwa{Client}$ has a function $\kwa{run}$ which executes $\kwa{Logger.log}$, incurring its effects. $\kwa{Client.run}$ is annotated with $\varnothing$.

\begin{lstlisting}
resource module Logger
require File

def log(): Unit =
    File.append(``message logged'')
\end{lstlisting}

\begin{lstlisting}
resource module Client
require Logger

def run(): Unit with {File.append} =
   Logger.log()
\end{lstlisting}

\begin{lstlisting}
resource module Main
require File
instantiate Logger(File)
instantiate Client(Logger)

Client.run()
\end{lstlisting}

A desugaring is given below. On lines 3-4, the unannotated code is wrapped in an $\kwa{import}$ expression selecting $\{ \kwa{File.append} \}$ as its authority. The implementation of $\kwa{Logger}$ actually abides by this selected authority, but it has the authority to perform any operation on $\kwa{File}$, so it could, in general, invoke any of them. \textsc{$\varepsilon$-Import} rejects this example because the imported capability has the type $\{ \File \}$ and $\fx{\{ \File \}} = \{ \kwa{File}.* \} \not\subseteq \{ \kwa{ File.append } \}$.

\begin{lstlisting}
let MakeLogger =
   ($\lambda$f: File.
      import(File.append) f = f in
         $\lambda$x: Unit. f.append) in

let MakeClient =
	($\lambda$logger: Logger.
	   $\lambda$x: Unit. logger unit) in

let MakeMain =
   ($\lambda$f: File.
      let LoggerModule = MakeLogger f in
      let ClientModule = MakeClient LoggerModule in
      ClientModule unit) in

(MakeMain File) unit
\end{lstlisting}

The only way for this to typecheck would be to annotate $\kwa{Client.run}$ as having every effect on $\File$. This demonstrates how the effect-system of $\epscalc$ approximates unannotated code: it simply considers it as having every effect which could be incurred on those resources in scope, which here is $\kwa{File}.*$.


















\subsection{Higher-Order Effects}

In this scenario, $\kwa{Main}$ instantiates the $\kwa{Plugin}$ module, which instantiates the $\kwa{Malicious}$ module. $\kwa{Plugin}$ exposes a function $\kwa{run}$ that should incur no effects. However, the implementation tries to read from $\kwa{File}$ by wrapping the operation inside a function and passing it to $\kwa{Malicious}$, which invokes $\kwa{File.read}$ in a higher-order way.

\begin{lstlisting}
module Malicious

def log(f: Unit $\rightarrow$ Unit):Unit
   f()
\end{lstlisting}

\begin{lstlisting}
module Plugin
instantiate Malicious

def run(f: {File}): Unit with $\varnothing$
   Malicious.log($\lambda$x:Unit. f.read)
\end{lstlisting}

\begin{lstlisting}
resource module Main
require File
instantiate Plugin

Plugin.run(File)
\end{lstlisting}

This example shows how higher-order effects can obfuscate potential security risks. On line 3 of $\kwa{Malicious}$, the argument to $\kwa{log}$ has type $\Unit \rightarrow \Unit$. The body of $\kwa{log}$ types with the \textsc{T-}rules, which do not approximate effects. It is not clear from inspecting the unannotated code that a $\kwa{File.read}$ will be incurred. To realise this requires one to examine the source code of both $\kwa{Plugin}$ and $\kwa{Malicious}$.

A desugared version is given below. On lines 5-6, the $\kwa{Malicious}$ code selects its authority as $\varnothing$, to be consistent with the annotation on $\kwa{Plugin.run}$. This example is rejected by \textsc{$\varepsilon$-Import}. When the unannotated code is annotated with $\varnothing$, it has type $\{ \File \} \rightarrow_{\varnothing} \Unit$. The higher-order effects of this type are $\kwa{File.*}$, which is not contained in the selected authority $\varnothing$ --- hence, \textsc{$\varepsilon$-Import} safely rejects. To get this example to typecheck, the $\kwa{import}$ expression has to select $\kwa{ \{ File.* \}}$ as its authority. 

\begin{lstlisting}
let MakePlugin =
	($\lambda$x: Unit.
	   let MakeMalicious =
	      ($\lambda$x: Unit.
	         import($\varnothing$) y=unit in
	            $\lambda$f: {File}. f.append) in
      let LoggerModule = MakeLogger unit in
      $\lambda$f: {File}. LoggerModule f) in

let MakeMain =
   ($\lambda$f: {File}.
      $\lambda$x: Unit.
         let ClientModule = MakeClient unit in
         ClientModule f) in

(MakeMain File) unit
\end{lstlisting}














~





























\subsection{Resource Leak}

This is another example which obfuscates an unsafe effect by invoking it in a higher-order manner. The setup is the same, except the function which $\kwa{Plugin}$ passes to $\kwa{Malicious}$ now returns $\kwa{File}$ when invoked. $\kwa{Malicious}$ uses this function to obtain $\kwa{File}$ and directly invokes $\kwa{read}$ upon it.

\begin{lstlisting}
module Malicious

def log(f: Unit $\rightarrow$ File):Unit
   f().read
\end{lstlisting}

\begin{lstlisting}
module Plugin
instantiate Malicious

def run(f: {File}): Unit with $\varnothing$
   Malicious.log($\lambda$x:Unit. f)
\end{lstlisting}

\begin{lstlisting}
resource module Main
require File
instantiate Plugin

Plugin.run(File)
\end{lstlisting}

The desugaring is given below. The unannotated code in $\kwa{Malicious}$ is given on lines 5-6. The selected authority is $\varnothing$, to be consistent with the annotation on $\kwa{Plugin}$. Nothing is being imported, so the $\kwa{import}$ binds a name $\kwa{y}$ to $\unit$. This example is rejected by \textsc{$\varepsilon$-Import} because the premise $\varepsilon = \fx{\hat \tau} \cup \hofx{\annot{\tau}{\varepsilon}}$ is not satisfied. In this case, $\varepsilon = \varnothing$ and $\tau = \kwa{ (Unit \rightarrow \{ File \}) \rightarrow Unit }$. Then $\annot{\tau}{\varepsilon} = \kwa{ (Unit \rightarrow_{\varnothing} \{ File \}) \rightarrow_{\varnothing} Unit }$ and $\hofx{\annot{\tau}{\varepsilon}} = \{ \kwa{File.*} \}$. Thus, the premise cannot be satisfied and the example safely rejects.


\begin{lstlisting}
let MakePlugin =
	($\lambda$x: Unit.
	   let MakeMalicious =
	      ($\lambda$x: Unit.
	         import($\varnothing$) y=unit in
	            $\lambda$f: Unit $\rightarrow$ {File}. f().read) in
      let LoggerModule = MakeLogger unit in
      $\lambda$f: {File}. LoggerModule f) in

let MakeMain =
   ($\lambda$f: {File}.
      $\lambda$x: Unit.
         let ClientModule = MakeClient unit in
         ClientModule f) in

(MakeMain File) unit
\end{lstlisting}






