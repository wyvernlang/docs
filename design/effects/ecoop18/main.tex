\documentclass[a4paper,UKenglish]{lipics-v2016}
%This is a template for producing LIPIcs articles. 
%See lipics-manual.pdf for further information.
%for A4 paper format use option "a4paper", for US-letter use option "letterpaper"
%for british hyphenation rules use option "UKenglish", for american hyphenation rules use option "USenglish"
% for section-numbered lemmas etc., use "numberwithinsect"
 
\usepackage{microtype}%if unwanted, comment out or use option "draft"

%\graphicspath{{./graphics/}}%helpful if your graphic files are in another directory

\bibliographystyle{plainurl}% the recommended bibstyle

% Author macros::begin %%%%%%%%%%%%%%%%%%%%%%%%%%%%%%%%%%%%%%%%%%%%%%%%
\title{Capabilities: Effects for Free}
%\titlerunning{Capabilities: Effects for Free} %optional, in case that the title is too long; the running title should fit into the top page column

%% Please provide for each author the \author and \affil macro, even when authors have the same affiliation, i.e. for each author there needs to be the  \author and \affil macros
\author{Authors omitted for double-blind review.}
%\author[1]{Aaron Craig}
%\author[1]{Alex Potanin}
%\author[1]{Lindsay Groves}
%\author[2]{Jonathan Aldrich}
%\affil[1]{ECS, VUW\\
%  \texttt{aaroncraig@protonmail.ch, alex@ecs.vuw.ac.nz, lindsay@ecs.vuw.ac.nz}}
%\affil[2]{ISR, CMU\\
%  \texttt{jonathan.aldrich@cs.cmu.edu}}
%\authorrunning{A. Craig et al} %mandatory. First: Use abbreviated first/middle names. Second (only in severe cases): Use first author plus 'et. al.'

%\Copyright{Aaron Craig, Alex Potanin, Lindsay Groves, and Jonathan Aldrich}%mandatory, please use full first names. LIPIcs license is "CC-BY";  http://creativecommons.org/licenses/by/3.0/

%\subjclass{300 Software and its engineering, Semantics}
%Dummy classification -- please refer to \url{http://www.acm.org/about/class/ccs98-html}}% mandatory: Please choose ACM 1998 classifications from http://www.acm.org/about/class/ccs98-html . E.g., cite as "F.1.1 Models of Computation". 
%\keywords{capabilities, effects}% mandatory: Please provide 1-5 keywords
% Author macros::end %%%%%%%%%%%%%%%%%%%%%%%%%%%%%%%%%%%%%%%%%%%%%%%%%

%Editor-only macros:: begin (do not touch as author)%%%%%%%%%%%%%%%%%%%%%%%%%%%%%%%%%%
\EventEditors{John Q. Open and Joan R. Acces}
\EventNoEds{2}
\EventLongTitle{42nd Conference on Very Important Topics (CVIT 2016)}
\EventShortTitle{CVIT 2016}
\EventAcronym{CVIT}
\EventYear{2016}
\EventDate{December 24--27, 2016}
\EventLocation{Little Whinging, United Kingdom}
\EventLogo{}
\SeriesVolume{42}
\ArticleNo{23}
% Editor-only macros::end %%%%%%%%%%%%%%%%%%%%%%%%%%%%%%%%%%%%%%%%%%%%%%%

\usepackage{graphicx}
\usepackage{times}
\usepackage{bm}
\usepackage{color}
\usepackage{ebproof} % For proof trees
\usepackage{listings} % For code snippets
\usepackage{proof} % For inference rules.
\usepackage[ruled]{algorithm2e}
\usepackage{amssymb}
\usepackage{amsmath}

\definecolor{grey}{gray}{0.92}

\lstset{
tabsize=3,
basicstyle=\ttfamily\small, commentstyle=\itshape\rmfamily, 
backgroundcolor=\color{grey},
numbers=left,
numberstyle=\tiny,
language=java,
moredelim=[il][\sffamily]{?},
mathescape=true,
showspaces=false,
showstringspaces=false,
columns=fullflexible,
escapeinside={(@}{@)}, morekeywords=[1]{def, if, then, else, with, val, module, instantiate, require, let, in}}
\lstloadlanguages{Java,VBScript,XML,HTML}

%using \kwa outside math mode
\newcommand{\kwat}[1]{$\kwa{#1}$}

% Hyphens
\newcommand{\hyphen}{\hbox{-}}

% For defining derived forms.
\newcommand\defn{\mathrel{\overset{\makebox[0pt]{\mbox{\normalfont\tiny\sffamily def}}}{=}}}

% Constants, types.
\newcommand{\unit}{\kwa{unit}}
\newcommand{\Unit}{\kwa{Unit}}
\newcommand{\File}{\kwa{File}}
\newcommand{\Socket}{\kwa{Socket}}

% Keywords.
\newcommand{\kwa}[1]{\mathtt{#1}}
\newcommand{\kw}[1]{\mathtt{#1}~}

% Expressions.
\newcommand{\import}[4]{\kwa{import}(#1)~#2 = #3~\kw{in} #4}
\newcommand{\letxpr}[3]{\kw{let} #1 = #2~\kw{in} #3}	

% Functions in the type theory.
\newcommand{\annot}[2]{\kwa{annot}(#1, #2)}
\newcommand{\erase}[1]{\kwa{erase}(#1)}
\newcommand{\fx}[1]{\kwa{effects}(#1)}
\newcommand{\hofx}[1]{\kwa{ho \hyphen effects}(#1)}

% Safety predicates in the type theory.
\newcommand{\safe}[2]{\kwa{safe}(#1, #2)}
\newcommand{\hosafe}[2]{\kwa{ho \hyphen safe}(#1, #2)}

% Names of the calculi.
\newcommand{\opercalc}{\kwa{OC}}
\newcommand{\epscalc}{\kwa{CC}}

\renewcommand{\algorithmcfname}{ALGORITHM}
\SetAlFnt{\small}
\SetAlCapFnt{\small}
\SetAlCapNameFnt{\small}
\SetAlCapHSkip{0pt}
\IncMargin{-\parindent}

%% Bibliography style
\bibliographystyle{splncs04}
 % Contains the packages and other data common to both paper (main.tex) and supplementary material (proofs.tex).

% Document starts
\begin{document}

\maketitle

%% Abstract
%% Note: \begin{abstract}...\end{abstract} environment must come
%% before \maketitle command
\begin{abstract}
Object capabilities are increasingly used to reason informally about the properties of secure systems. Can capabilities also aid in \textit{formal} reasoning? To answer this question, we examine a calculus that uses effects to capture resource use and extend it with a rule that captures the essence of capability-based reasoning. We demonstrate that capabilities provide a way to reason for free about effects: we can bound the effects of an expression based on the capabilities to which it has access.  This reasoning is ``free'' in that it relies only on type-checking (not effect-checking); does not require the programmer to add effect annotations within the expression; nor does it require the expression to be analysed for its effects. Our result sheds light on the essence of what capabilities provide and suggests useful ways of integrating lightweight capability-based reasoning into languages.
\end{abstract}

\section{Introduction}

Capabilities have been recently gaining new attention as a promising mechanism for controlling access to resources in systems and languages~\cite{miller03,drossopoulou07,dimoulas14,devriese16}.
A \textit{capability} is an unforgeable token that can be used by its bearer to perform some operation on a resource \cite{dennis66}.
In a \textit{capability-safe} language, all resources must be accessed through capabilities, and a resource-access capability must be obtained from a source that already has it: ``only connectivity begets connectivity'' \cite{miller03}.
For example, a \kwat{logger} component that provides a logging service would need to be initialized with a capability providing the ability to append to the log file.

Capability-safe languages thus prohibit the \textit{ambient authority} that is present in non-capability-safe languages.
An implementation of a \kwat{logger} in OCaml or Java, for example, does not need to be passed a capability at initialization time; it can simply import the appropriate file-access library and open the log file for appending itself.
Critically, a malicious implementation of such a component could also delete the log, read from another file, or exfiltrate logging information over the network.
Other mechanisms such as sandboxing can be used to limit the effects of such malicious components, but recent work has found that Java's sandbox (for example) is difficult to use and is therefore often misused~\cite{coker15, maass16}.

In practice, reasoning about resource use in capability-based systems is mostly done informally.
But if capabilities are useful for \textit{informal} reasoning, shouldn't they also aid in \textit{formal} reasoning?
Recent work by \citet{drossopoulou07} sheds some light on this question by presenting a logic that formalizes capability-based reasoning about trust between objects.
Two other trains of work, rather than formalize capability-based reasoning itself, reason about how capabilities may be used.
\citet{dimoulas14} developed a formalism for reasoning about which components may use a capability and which may influence (perhaps indirectly) the use of a capability.
\citet{devriese16} formulate an effect parametricity theorem that limits the effects of an object based on the capabilities it possesses, and then use logical relations to reason about capability use in higher-order settings.
Overall, this prior work presents new formal systems for reasoning about capability use, or reasoning about new properties using capabilities.

We are interested in a different question: can capabilities be used to enhance formal reasoning that is currently done without relying on capabilities?
In other words, what value do capabilities add to existing formal reasoning?

To answer this question, we decided to pick a simple and practical formal reasoning system, and see if capability-based reasoning could help.
A natural choice for our investigation is effect systems~\cite{nielson99}.
Effect systems are a relatively simple formal reasoning approach, and keeping things simple will help to make the difference made by capabilities more obvious.
Furthermore, effects have an intuitive link to capabilities: in a system that uses capabilities to protect resources, an expression can only have an effect on a resource if it is given a capability to do so.

How could capabilities help with effects?
One challenge to the wider adoption of effect systems is their annotation overhead~\cite{rytz12}.
For example, Java's checked exception system is a kind of effect system, and is often criticised for being cumbersome~\cite{Kiniry2006}.
Effect inference can be used to reduce the annotations required~\cite{koka14}, but this has significant drawbacks: understanding error messages that arise through effect inference requires a detailed understanding of the internal structure of the code, not just its interface.
Capabilities are a promising alternative for reducing the overhead of effect annotations, as suggested by the following example:

\begin{lstlisting}
import log : String -> Unit with effect File.write

e
\end{lstlisting}

In the code above, written in a capability-safe language, what can we infer about the effects on resources (e.g. the file system or network) of evaluating \kwat{e}?
Since we are in a capability-safe language, \kwat{e} has no ambient authority, and so the only way it can have any effect on resources is via the \kwat{log} function it imports.
Note that this reasoning requires nothing about \kwat{e} other than that it obeys the rules of a capability-safe language; in particular, we don't require any effect annotations within \kwat{e}, and we don't need to analyze its structure as an effect inference would have to do.
Also note that \kwat{e} might be arbitrarily large, perhaps consisting of an entire program that we have downloaded from a source that we trust enough to allow it to write to a log, but that we don't trust to access any other resources.
Thus in this scenario, capabilities can be used to reason ``for free'' about the effect of a large body of code based on a few annotations on the components it imports.

The central intuition that we formalize in this paper is this: the effect of an unannotated expression can be given a bound based on the effects latent in variables that are in scope.
Of course, there are challenges to solve on the way, most notably involving higher-order programs: how can we generalize this intuition if \kwat{log} takes a higher-order argument?
If \kwat{e} evaluates not to unit but to a function, what can we infer about that function's effects?

In the remainder of this paper, we will formalize these ideas and explore these questions.
To demonstrate, we introduce a pair of languages: the operation calculus $\opercalc$ (Section 3) and the capability calculus $\epscalc$ (Section 4).
$\opercalc$ is a typed lambda calculus with a simple notion of capabilities and their operations, in which all code is effect-annotated.
Relaxing this requirement, we then introduce $\epscalc$, which permits the nesting of unannotated code inside annotated code in a controlled, capability-safe manner.
A safe inference about the unannotated code can be made by inspecting the capabilities passed into it from its annotated surroundings.
We then show how $\epscalc$ can model practical situations, presenting a range of examples to illustrate the benefits of a capability-flavoured effect system.

Throughout this paper we give motivating examples in a capability-safe language very similar to \textit{Wyvern} as presented by \citet{nistor13}.
A more thorough discussion of the language and how it can be translated into the calculi is given in section 4.

\section{Operation Calculus ($\opercalc$)}

$\opercalc$ extends the simply-typed lambda calculus with a notion of primitive capabilities and their operations.
Every function is annotated with the effects it may incur.
Its static rules associate a type and a set of effects to well-formed programs.
Defining $\opercalc$ will introduce the notations and concepts needed to understand $\epscalc$, which allows developers to omit annotations from some expressions and uses capability-based reasoning to bound the effects of those expressions.

In a capability-safe language, ``only connectivity begets connectivity'' \cite{miller06}: all access to a capability must derive from previous access.
To prevent an infinite regress, there are a set of primitive capabilities passed into the program by the system environment.
These primitive capabilities provide operations for manipulating resources in the system environment.
For example, $\kwa{File}$ might provide read/write operations on a particular file in the file system.
For convenience, we often conflate primitive capabilities with the resources they manipulate, referring to both as resources.
An effect in $\opercalc$ is a particular operation invoked on some resource; for example, $\kwa{File.write}$.
Functions in an $\opercalc$ program are (conservatively) annotated with the effects they may incur when invoked. Annotations might be given in accordance with the principle of least authority to specify the maximum authority a component may exercise.
When this authority is exceeded, an effect system like that of $\opercalc$ will reject the program, signaling an unsafe implementation.
For example, consider the pair of modules\footnote{Our formal grammar, below, does not include this \textit{Wyvern}-like module syntax, but we can model the \kwat{logger} functor as a function and the \kwat{client} module as a record (which is itself encodable using functions).} in Figure \ref{fig:opercalc_motivating}: the $\kwa{logger}$ module possesses a $\File$ capability and exposes a single function $\kwa{log}$.
The $\kwa{client}$ has a single function $\kwa{run}$ which, when passed a $\kwa{Logger}$, will invoke $\kwa{Logger.log}$.

\begin{figure}[h]
\vspace{-5pt}

\begin{lstlisting}
module def logger(f:{File}):Logger

def log(): Unit with {File.append} =
    f.read
\end{lstlisting}

\begin{lstlisting}
module client

def run(l: Logger): Unit with {File.append} =
    l.log()
\end{lstlisting}

\vspace{-7pt}
\caption{The implementation of $\kwa{logger.log}$ exceeds its specified authority.}
\label{fig:opercalc_motivating}
\end{figure}

$\kwa{client.run}$ and $\kwa{logger.log}$ are both annotated with $\{ \kwa{File.append} \}$, but the (potentially malicious) implementation of $\kwa{logger.log}$ incurs the $\kwa{File.read}$ effects.
By the end of this section, we will have developed rules for $\opercalc$ that can determine such mismatches between specification and implementation in annotated code.

$\opercalc$ makes some simplifying assumptions.
The semantics of particular operations are not modeled --- our only interest is in what operations could be invoked, and by whom.
Therefore, we assume all operations are null-ary and return a dummy $\unit$ value; $\kwa{File.write(``hello, world!'')}$ becomes $\kwa{File.write}$.
Primitive capabilities and operations are fixed throughout run time and cannot be created or destroyed.

\subsection{Grammar ($\opercalc$)}

A grammar for $\opercalc$ programs is given in Figure \ref{fig:opercalc_grammar}. In addition to those from the lambda calculus, there are two new forms. A resource literal $r$ is a variable drawn from a fixed set $R$. Resources model the primitive capabilities that the system passes into the program. $\kwa{File}$ and $\kwa{Socket}$ are examples of resource literals. An operation call $e.\pi$ is the invocation of an operation $\pi$ on $e$. For example, invoking the $\kwa{open}$ operation on the $\kwa{File}$ resource would be $\kwa{File.open}$. Operations are drawn from a fixed set $\Pi$.

\begin{figure}[h]
\vspace{-5pt}

\[
\begin{array}{lll}

\begin{array}{lllr}

e & ::= & ~ & exprs: \\
	& | & x & variable \\
	& | & v & value \\
	& | & e ~ e & application \\
	& | & e.\pi & operation~call \\
	&&\\

\end{array}

	\hspace{5ex}

\begin{array}{lllr}

v & ::= & ~ & values: \\ 
	& | & r & resource~literal \\
	& | & \lambda x: \tau.e & abstraction \\
	&&\\

\end{array}

\end{array}
\]

\vspace{-7pt}
\caption{Grammar for $\opercalc$ programs.}
\label{fig:opercalc_grammar}
\end{figure}

An effect is a pair $(r, \pi) \in R \times \Pi$. Sets of effects are denoted $\varepsilon$. As a shorthand, we write $r.\pi$ instead of $(r, \pi)$. Effects should be distinguished from operation calls: an operation call is the invocation of a particular operation on a particular resource in a program, while an effect is a mathematical object describing this behaviour. The notation $r.*$ is a short-hand for the set $\{ r.\pi \mid \pi \in \Pi \}$, which contains every effect on $r$. Sometimes we abuse notation by conflating the effect $r.\pi$ with the singleton $\{ r.\pi \}$. We may also write things like $\{ r_1.*, r_2.* \}$, which should be understood as the set of all operations on $r_1$ and $r_2$.

\subsection{Substitution ($\opercalc$)}

Figure \ref{fig:opercalc_sub_defn} defines substitution on the new forms of $\opercalc$.
We don't want substitution to introduce unexpected new effects into an expression, so we restrict substitution to replacing a variable with a  value.
Thus $\opercalc$ is a call-by-value language.

\begin{figure}[h]

$\kwa{substitution :: e \times v \times v \rightarrow e}$

\begin{itemize}
	\item[] $[v/y]r = r$
	\item[] $[v/y](e_1.\pi) = ([v/y]e_1).\pi$
\end{itemize}

\vspace{-7pt}
\caption{Extra cases for $\kwa{substitution}$ in $\opercalc$.}
\label{fig:opercalc_sub_defn}
\end{figure}


\subsection{Semantics ($\opercalc$)}

During reduction an operation call may be evaluated. When this happens we say that a run time effect has taken place. Reflecting this, the form of the single-step reduction judgement is now $e \longrightarrow e~|~\varepsilon$, which means that $e$ reduces to $e'$, incurring the set of effects $\varepsilon$ in the process. In the case of single-step reduction, $\varepsilon$ is at most a single effect. Judgements for single-step reductions are given in Figure \ref{fig:opercalc_singlestep}.

\begin{figure}[h]

\noindent
\fbox{$e \longrightarrow e~|~\varepsilon$}

\[
\begin{array}{c}

\infer[\textsc{(E-App1)}]
	{e_1 e_2 \longrightarrow e_1'~ e_2~|~\varepsilon}
	{e_1 \longrightarrow e_1'~|~\varepsilon}
    
	\hspace{5ex}
    
\infer[\textsc{(E-App2)}]
	{v_1 ~ e_2 \longrightarrow v_1 ~ e_2'~|~\varepsilon} 
	{e_2 \longrightarrow e_2'~|~\varepsilon}
    
	\hspace{5ex}
    
\infer[\textsc{(E-App3)}]
	{ (\lambda x: \tau. e) v_2 \longrightarrow [ v_2/x] e~|~\varnothing }
	{}\\[4ex]
	
\infer[\textsc{(E-OperCall1)}]
	{ e.\pi \longrightarrow  e'.\pi~|~\varepsilon }
	{ e \rightarrow  e'~|~\varepsilon}
		
	\hspace{5ex}
	
\infer[\textsc{(E-OperCall2)}]
	{r.\pi \longrightarrow \kwa{unit}~|~\{ r.\pi \} }
	{}
	 \\[4ex]
	 
\end{array}
\]


\vspace{-7pt}
\caption{Single-step reductions in $\opercalc$.}
\label{fig:opercalc_singlestep}
\end{figure}

The first three rules are analogous to reductions in the lambda calculus. \textsc{E-App1} and \textsc{E-App2} incur the effects of reducing their subexpressions. \textsc{E-App3} replaces the formal name $x$ with the actual value $v_2$ being passed as an argument, which incurs no effects. The first new rule is \textsc{E-OperCall1}, which reduces the receiver of an operation call; the effects incurred are the effects incurred by reducing the receiver. 
When an operation $\pi$ is invoked on a resource literal $r$, \textsc{E-OperCall2} will reduce it to $\unit$,
%\footnote{Our formal grammar does not include $\unit$, but it can be defined as \lambda x: {}. x}
incurring $\{ r.\pi \}$ as a result. For example, $\kwa{File.write} \longrightarrow \unit~|~\{ \kwa{File.write} \}$ by \textsc{E-OperCall2}. $\unit$ can be treated as a derived form; an explanation is given in section 4.

A multi-step reduction is a sequence of zero\footnote{We permit multi-step reductions of length zero to be consistent with Pierce, who defines multi-step reduction as a reflexive relation \cite[p. 39]{pierce02}.} or more single-step reductions. The resulting set of run time effects is the union of all the run time effects from the intermediate single-steps. Judgements for multi-step reductions are given in Figure \ref{fig:opercalc_multistep_defn}. By \textsc{E-MultiStep1}, any expression can ``reduce'' to itself with no run time effects. By \textsc{E-MultiStep2}, any single-step reduction is also a multi-step reduction. If $e \longrightarrow e'~|~\varepsilon_1$ and $e' \longrightarrow e''~|~\varepsilon_2$ are sequences of reductions, then so is $e \longrightarrow e''~|~\varepsilon_1 \cup \varepsilon_2$, by \textsc{E-MultiStep3}.

\begin{figure}[h]

\noindent
\fbox{$ e \longrightarrow^{*}  e~|~\varepsilon$}

\[
\begin{array}{c}

\infer[\textsc{(E-MultiStep1)}]
	{ e \rightarrow^{*}  e~|~\varnothing}
	{}
	\hspace{5ex}
\infer[\textsc{(E-MultiStep2)}]
	{ e \rightarrow^{*}  e'~|~\varepsilon}
	{ e \rightarrow  e'~|~\varepsilon}
	\hspace{5ex}
\infer[\textsc{(E-MultiStep3)}]
	{ e \rightarrow^{*}  e''~|~\varepsilon_1 \cup \varepsilon_2}
	{ e \rightarrow^{*}  e'~|~\varepsilon_1 &  e' \rightarrow^{*}  e''~|~\varepsilon_2}
\end{array}
\]

\vspace{-7pt}
\caption{Multi-step reductions in $\opercalc$.}
\label{fig:opercalc_multistep_defn}
\end{figure}

\subsection{Static Rules ($\opercalc$)}

A grammar for types and type contexts is given in Figure \ref{fig:opercalc_types}. The base types of $\opercalc$ are sets of resources, denoted $\{ \bar r\}$. If an expression $e$ is associated with type $\{ \bar r \}$, it means $e$ will reduce to one of the literals in $\bar r$ (assuming $e$ terminates). The set of empty resources (denoted $\varnothing$) is also a valid type, but has no inhabitants. There is a single type constructor $\rightarrow_{\varepsilon}$, where $\varepsilon$ is a concrete set of effects. $\tau_1 \rightarrow_{\varepsilon} \tau_2$ is the type of a function which takes a $\tau_1$ as input, returns a $\tau_2$ as output, and whose body incurs no more than those effects in $\varepsilon$. $\varepsilon$ is a conservative bound: if an effect $r.\pi \in \varepsilon$, it is not guaranteed to happen at run time, but if $r.\pi \notin \varepsilon$, it cannot happen at run time. A typing context $\Gamma$ maps variables to types. 

\begin{figure}[h]
\vspace{-5pt}

\[
\begin{array}{lll}

\begin{array}{lllr}

\tau & ::= & ~ & types: \\
		& | & \{ \bar r \} & resource~set \\
		& | & \tau \rightarrow_{\varepsilon} \tau & annotated~arrow \\ 
		&&\\
\end{array}

	\hspace{5ex}
    
\begin{array}{lllr}

\Gamma & ::= & ~ & type~ctx: \\
				& | & \varnothing & empty~ctx. \\
				& | & \Gamma, x: \tau & var.~binding \\
				&&\\
\end{array}

\end{array}
\]

\vspace{-7pt}
\caption{Grammar for types in $\opercalc$.}
\label{fig:opercalc_types}
\end{figure}

To illustrate the types of some functions, if $\kwa{log_1}$ has the type $\{ \kwa{File} \} \rightarrow_{\{\kwa{File.append}\}} \Unit$, then invoking $\kwa{log_1}$ will either incur $\kwa{File.append}$ or no effects. If $\kwa{log_2}$ has the type $\{ \kwa{File} \} \rightarrow_{\{\kwa{File.*}\}} \Unit$, then invoking $\kwa{log_2}$ could incur any effect on $\kwa{File}$.

Knowing approximately what effects a piece of code may incur helps a developer determine whether it can be trusted. For example, consider $\kwa{log_3} = \lambda f: \kwa{File}.~e$, which is a logging function that takes a $\kwa{File}$ as an argument and then executes $e$. Suppose this function were to typecheck as $\{ \File \} \rightarrow_{\{ \kwa{File.*} \}} \Unit$ --- seeing that invoking this function could incur any effect on $\kwa{File}$, and not just its expected least authority $\kwa{File.append}$, a developer may therefore decide this implementation cannot be trusted and choose not to execute it. In this spirit, the static rules of $\opercalc$ associate well-typed programs with a type and a set of effects: the judgement $\Gamma \vdash e: \tau~\kw{with} \varepsilon$, means $e$ will reduce to a term of type $\tau$ (assuming it terminates), incurring no more effects than those in $\varepsilon$. Judgements are given in Figure \ref{fig:opercalc_static_rules}.

\begin{figure}[h]

\noindent
\fbox{$\Gamma \vdash e: \tau~\kw{with} \varepsilon$}

\[
\begin{array}{c}

\infer[\textsc{($\varepsilon$-Var)}]
	{ \Gamma, x:\tau \vdash x: \tau~\kw{with} \varnothing }
	{}
	
	\hspace{5ex}
	
\infer[\textsc{($\varepsilon$-Resource)}]
 	{ \Gamma, r: \{ r \} \vdash r : \{ r \}~\kw{with} \varnothing }
 	{} \\[3ex]
 	
 	~~~
	\infer[\textsc{($\varepsilon$-Abs)}]
	{ \Gamma \vdash \lambda x:\tau_2 . e : \tau_2 \rightarrow_{\varepsilon_3} \tau_3~\kw{with} \varnothing }
	{ \Gamma, x: \tau_2 \vdash e: \tau_3~\kw{with} \varepsilon_3 }
	
	\hspace{5ex}
	
\infer[\textsc{($\varepsilon$-App)}]
	{ \Gamma \vdash e_1~e_2 : \tau_3~\kw{with} \varepsilon_1 \cup \varepsilon_2 \cup \varepsilon  }
	{ \Gamma \vdash e_1: \tau_2 \rightarrow_{\varepsilon} \tau_3~\kw{with} \varepsilon_1 & \Gamma \vdash e_2: \tau_2~\kw{with} \varepsilon_2 } \\[3ex]
	
\infer[\textsc{($\varepsilon$-OperCall)}]
	{ \Gamma \vdash e.\pi: \kw{Unit} \kw{with} \varepsilon \cup \{ \bar r.\pi \} }
	{ \Gamma \vdash e: \{ \bar r \}~ \kw{with} \varepsilon}

	\hspace{5ex}
    
\infer[\textsc{($\varepsilon$-Subsume)}]
	{ \Gamma \vdash e: \tau' ~\kw{with} \varepsilon'}
	{ \Gamma \vdash e: \tau ~\kw{with} \varepsilon & \tau <: \tau' & \varepsilon \subseteq \varepsilon'}\\[3ex]
	
\end{array}
\]


\vspace{-7pt}
\caption{Type-with-effect judgements in $\opercalc$.}
\label{fig:opercalc_static_rules}
\end{figure}



\textsc{$\varepsilon$-Var} approximates the run time effects of a variable as $\varnothing$. \textsc{$\varepsilon$-Resource} does the same for resource literals. Though a resource captures several effects (namely, every possible operation on itself), attempting to ``reduce'' a resource will incur no effects; something must be done with the resource, such as an operation call, in order to have an effect. For a similar reason, \textsc{$\varepsilon$-Abs} approximates the effects of a function literal as $\varnothing$, and ascribes an arrow type annotated with those effects captured by the function. \textsc{$\varepsilon$-App} approximates a lambda application as incurring those effects from evaluating the subexpressions and the effects incurred by executing the body of the function to which the left-hand side evaluates. The effects of the function body are taken from the function's arrow type. An operation call on a resource literal reduces to $\unit$, so \textsc{$\varepsilon$-OperCall} ascribes its type as $\Unit$.
The approximate effects of an operation call are: the effects of reducing the subexpression, and then the operation $\pi$ on every possible resource which that subexpression to which that subexpression might reduce. For example, consider $e.\pi$, where $\Gamma \vdash e: \{ \kwa{File, Socket} \}~\kw{with} \varnothing$. Then $e$ could evaluate to $\kwa{File}$, in which case the actual run time effect is $\kwa{File}.\pi$, or it could evaluate to $\kwa{Socket}$, in which case the actual run time effect is $\kwa{Socket.\pi}$. Determining which will happen is, in general, undecidable; the safe approximation is to treat them both as happening. The last rule \textsc{$\varepsilon$-Subsume} produces a new judgement by widening the type or approximate effects on an existing one. Subtyping judgements are given in Figure \ref{fig:opercalc_static_rules}.


\begin{figure}[h]
\vspace{-5pt}

\fbox{$\tau <: \tau$}

\[
\begin{array}{c}

\infer[\textsc{(S-Arrow)}]
	{ \tau_1 \rightarrow_{\varepsilon} \tau_2 <: \tau_1' \rightarrow_{\varepsilon'} \tau_2' }
	{ \tau_1' <: \tau_1 & \tau_2 <: \tau_2' & \varepsilon \subseteq \varepsilon' }
	\hspace{5ex}
\infer[\textsc{(S-Resource)}]
	{ \{ \bar r_1 \} <: \{ \bar r_2 \} }
	{ r \in r_1 \implies r \in r_2 }

\end{array}
\]

\vspace{-7pt}
\caption{Subtyping judgements of $\opercalc$.}
\label{fig:opercalc_static_rules}
\end{figure}

\textsc{S-Arrow} is the standard rule for arrow types, but also stipulates that the effects on the arrow of the subtype must be contained in the effects on the arrow of the supertype; a valid subtype should not invoke any effects the supertype does not already know about. \textsc{S-Resource} says that a subset of resourcse is a subtype: consider $\{ \bar r_1 \} <: \{ \bar r_2 \}$. Any value with type $\{ \bar r_1 \}$ can reduce to any resource literal in $\bar r_1$, so to be compatible with an interface $\{ \bar r_2 \}$, the resource literals in $\bar r_1$ must also be in $\bar r_2$.

These rules let us determine what sort of effects might be incurred when a piece of code is executed. For example, consider $rw = \lambda x: \{ \kwa{File, Socket} \}.~\kwa{x.write}$, which takes either a $\Socket$ or a $\File$ and writes to it. If $rw$ is applied, it could incur $\kwa{Socket.write}$ or $\kwa{File.write}$, depending on what had been passed. In general, there is no way to statically determine what this will be, so the safe approximation is $\{ \kwa{File.write, Socket.write} \}$. This is the approximation given in a judgement like $\vdash rw~\File: \Unit~\kw{with} \{ \kwa{File.write, Socket.write} \}$. A derivation of this judgement is given in Figure~\ref{fig:opercalc_tree}. To fit on the page, all resources and operations have been abbreviated to their first letter. A developer who only expects $rw$ to be incurring $\kwa{File.write}$ can typecheck $rw$, see that it could also be writing to $\kwa{Socket}$, and decide it should not be used. If client code was annotated with $\kwa{ \{ \kwa{File.write} \} }$ and tried to use this function, the type system would reject it.

\begin{figure}[h]


    \begin{prooftree*}

    		\Infer0[\textsc{($\varepsilon$-Var)}]{x: \{ \kwa{F}, \kwa{S} \} \vdash x: \{ \kwa{F}, \kwa{S} \}}
    		
    		\Infer1[\textsc{($\varepsilon$-OperCall)}]{x: \{ \kwa{F}, \kwa{S} \} \vdash \kwa{x.w} : \Unit~\kw{with} \{ \kwa{F.w, S.w} \} }
    		
    		\Infer1[\textsc{($\varepsilon$-Abs)}]{ \lambda x: \{ \kwa{F}, \kwa{S} \}.~\kwa{x.w} : \{ \kwa{F, S} \} \rightarrow_{\{\kwa{F.w, S.w}\}} \Unit~\kw{with} \varnothing }
    		
    
       \Infer0[\textsc{($\varepsilon$-Resource)}]{\vdash \kwa{F}: \{ \kwa{F} \}~\kw{with} \varnothing}
    
    		\Infer2[\textsc{($\varepsilon$-App)}]{ \vdash (\lambda x: \{ \kwa{F, S} \}. ~\kwa{x.w})~\kwa{F} : \Unit~\kw{with} \{ \kwa{F.w, S.w} \}  }
    		
 	\end{prooftree*}
 	
\vspace{-12pt}
\caption{Derivation tree for $\vdash rw~\File: \Unit~\kw{with} \{ \kwa{File.write, Socket.write} \}$.}
\label{fig:opercalc_tree}
\end{figure}

\subsection{Soundness ($\opercalc$)}

To show the rules of $\opercalc$ are sound requires an appropriate notion of static approximations being safe with respect to the reductions. If a static judgement like $\Gamma \vdash e: \tau~\kw{with} \varepsilon$ were correct, successive reductions on $e$ should never incur effects not in $\varepsilon$. Furthermore, as $e$ is reduced, we learn more about what it is, so approximations on the reduced forms can only get more specific; compare this with how the type of reduced expressions can only get more specific. Adding this to the standard definition of soundness yields the following theorem statement.

\begin{theorem}[$\opercalc$ Single-step Soundness]
If $ \Gamma \vdash  e_A:  \tau_A~\kw{with} \varepsilon_A$ and $ e_A$ is not a value, then $e_A \longrightarrow e_B~|~\varepsilon$, where $ \Gamma \vdash e_B:  \tau_B~\kw{with} \varepsilon_B$ and $ \tau_B <:  \tau_A$ and $\varepsilon_B \cup \varepsilon \subseteq \varepsilon_A$, for some $e_B, \varepsilon, \tau_B, \varepsilon_B$.
\end{theorem}

Our approach to proving soundness is to show progress and preservation. Noting that the rules for values give $\varnothing$ as their approximate effects, the proof of the progress theorem is routine.

\begin{theorem}[$\opercalc$ Progress]
If $ \Gamma \vdash  e:  \tau~\kw{with} \varepsilon$ and $ e$ is not a value or variable, then $ e \longrightarrow  e'~|~\varepsilon'$, for some $e', \varepsilon' \subseteq \varepsilon$.
\end{theorem}

\begin{proof} By induction on derivations of $ \Gamma \vdash  e:  \tau~\kw{with} \varepsilon$.
\end{proof}

To show preservation we need to know that effect safety is preserved by the substitution in \textsc{E-App3}. The semantics are call-by-value, so the name of a function argument is only ever replaced with a value, and we know that the approximate effects of values are $\varnothing$, so the substitution does not introduce more effects. Beyond this observation, the proof is routine.

\begin{theorem}[$\opercalc$ Preservation]
If $\Gamma \vdash e_A: \tau_A~\kw{with} \varepsilon_A$ and $e_A \longrightarrow e_B~|~\varepsilon$, then $\tau_B <: \tau_A$ and $\varepsilon_B \cup \varepsilon \subseteq \varepsilon_A$, for some $e_B, \varepsilon, \tau_B, \varepsilon_B$.
\end{theorem}

\begin{proof}  By induction on the derivations of $\Gamma \vdash e_A: \tau_A~\kw{with} \varepsilon_A$ and $e_A \longrightarrow e_B~|~\varepsilon$.
\end{proof}

Single-step soundness theorem now holds by combining progress and preservation. The soundness of multi-step reductions follows by inducting on the length of a multi-step and appealing to single-step soundness.

\begin{theorem}[$\opercalc$ Single-step Soundness]
If $ \Gamma \vdash  e_A:  \tau_A~\kw{with} \varepsilon_A$ and $ e_A$ is not a value, then $e_A \longrightarrow e_B~|~\varepsilon$, where $ \Gamma \vdash e_B:  \tau_B~\kw{with} \varepsilon_B$ and $ \tau_B <:  \tau_A$ and $\varepsilon_B \cup \varepsilon \subseteq \varepsilon_A$, for some $e_B, \varepsilon, \tau_B, \varepsilon_B$.
\end{theorem}
\begin{proof}
If $ e_A$ is not a value then the reduction exists by the progress theorem. The rest follows by the preservation theorem.
\end{proof}

\begin{theorem}[$\opercalc$ Multi-step Soundness]
If $ \Gamma \vdash  e_A:  \tau_A~\kw{with} \varepsilon_A$ and $e_A \longrightarrow^{*} e_B~|~\varepsilon$, where $\Gamma \vdash e_B: \tau_B~\kw{with} \varepsilon_B$ and $ \tau_B <: \tau_A$ and $\varepsilon_B \cup \varepsilon \subseteq \varepsilon_A$.
\end{theorem}

\begin{proof} By induction on the length of the multi-step reduction.
\end{proof}
\section{Capability Calculus ($\epscalc$)}

$\opercalc$ requires every function to be annotated. The verbosity of such effect systems has been given as a reason for why they have not seen widespread use \cite{rytz12} --- if we relax the requirement that all code be annotated, can a type system say anything useful about the parts which are not? Allowing a mix of annotated and unannotated code helps reduce the cognitive overhead on developers, allowing them to rapidly prototype in the unannotated sublanguage and incrementally add annotations as they are needed. However, reasoning about unannotated code is difficult in general. Figure \ref{fig:unannotated_reasoning} demonstrates the issue: $\kwa{someMethod}$ takes a function $f$ as input and executes it, but the effects of $f$ depends on its implementation. Without more information, there is no way to know what effects might be incurred by $\kwa{someMethod}$.


\begin{figure}[h]
\begin{lstlisting}
def someMethod(f: Unit $\rightarrow$ Unit):
   f()
\end{lstlisting}
\vspace{-7pt}
\caption{What effects can $\kwa{someMethod}$ incur?}
\label{fig:unannotated_reasoning}
\end{figure}

A capability-safe design can help us: because the only authority code can exercise is that which is explicitly given to it, the only capabilities that the unannotated code can use must be passed into it. If these capabilities are being passed in from an annotated environment, we know what effects they capture. These effects are therefore a conservative upper bound on what can happen in the unannotated code. To demonstrate, consider a developer who wants to decide whether to use the $\kwa{logger}$ functor in Figure \ref{fig:cc_motivation}. It must be instantiated with two capabilities, $\kwa{File}$ and $\kwa{Socket}$, and provides an unannotated function $\kwa{log}$.

\begin{figure}[h]
\begin{lstlisting}
module def logger(f:{File},s:{Socket}):Logger

def log(x: Unit): Unit
   ...
\end{lstlisting}
\vspace{-7pt}
\caption{In a capability-safe setting, $\kwa{logger}$ can only exercise authority over the $\kwa{File}$ and $\kwa{Socket}$ capabilities given to it.}
\label{fig:cc_motivation}
\end{figure}

What effects will be incurred if $\kwa{Logger.log}$ is invoked? One approach is to manually\footnote{or automatically---but if the automation produces an unexpected result we must fall back to manual reasoning to understand why.} examine its source code, but this is tedious and error-prone. In many real-world situations, the source code may be obfuscated or unavailable. A capability-based argument can do better: the only authority which $\kwa{Logger}$ can exercise is that which it has been explicitly given. Here, the $\kwa{Logger}$ requires a $\kwa{File}$ and a $\kwa{Socket}$, so $\kwa{ \{ \kwa{File.*, Socket.*} \} }$ is an upper bound on the effects of $\kwa{Logger}$. Knowing $\kwa{Logger}$ could be performing arbitrary reads and writes to $\kwa{File}$, or arbitrary communication with the $\kwa{Socket}$, the developer decides this implementation cannot be trusted and does not use it.

The reasoning we employed only required us to examine the interface of the unannotated code for the capabilities be passed into it. To model this situation in $\epscalc$, we add a new $\kwa{import}$ expression that selects what authority $\varepsilon$ the unannotated code may exercise. In the above example, the expected least authority of $\kwa{Logger}$ is $\{ \kwa{File.append} \}$, so that is what the corresponding $\kwa{import}$ would select. The type system can then check if the capabilities being passed into the unannotated code exceed its selected authority. If it accepts, then $\varepsilon$ safely approximates the effects of the unannotated code. This is the key result: when unannotated code is nested inside annotated code, capability-safety enables us to make a safe inference about its effects by examining what capabilities are being passed in by the annotated code.


\subsection{Grammar ($\epscalc$)}

The grammar of $\epscalc$ is split into rules for annotated code and analogous rules for unannotated code. To distinguish the two, we put a hat above annotated types, expressions, and contexts: $\hat e$, $\hat \tau$, and $\hat \Gamma$ are annotated, while $e$, $\tau$, and $\Gamma$ are unannotated. The rules for unannotated programs and their types are given in Figure \ref{fig:epscalc_unannotated_grammar}. They are much the same as in $\opercalc$, but the type constructor $\rightarrow$ is not annotated with a set of effects. The type $\tau_1 \rightarrow \tau_2$ says nothing about what effects a function may or may not incur. Unannotated types $\tau$ are built using $\rightarrow$ and sets of resources $\{ \bar r \}$. An unannotated context $\Gamma$ maps variables to unannotated types.

\begin{figure}[h]
\vspace{-5pt}

\[
\begin{array}{lll}

\begin{array}{lllr}
e & ::= & ~ & exprs: \\
	& | & x & variable \\
	& | & v & value \\
	& | & e ~ e & application \\
	& | & e.\pi & operation \\
	&&\\

v & ::= & ~ & values: \\
	& | & r & resource~literal \\
	& | & \lambda x: \tau.e & abstraction \\
	&&\\
\end{array}

\hspace{5ex}

\begin{array}{lllr}

\tau & ::= & ~ & types: \\
		& | & \{ \bar r \} \\
		& | & \tau \rightarrow \tau \\ 
		&&\\

\Gamma & ::= & ~ & type~ctx: \\
				& | & \varnothing \\
				& | & \Gamma, x: \tau \\
				&&\\
				
\end{array}

\end{array}
\]

\vspace{-7pt}
\caption{Unannotated programs and types in $\epscalc$.}
\label{fig:epscalc_unannotated_grammar}
\end{figure}


Rules for annotated programs and their types are given in Figure \ref{fig:epscalc_annotated_grammar}. Except for the new $\kwa{import}$ expression, the rules are identical to those in $\opercalc$, except now everything has a hat above it.

\begin{figure}[h]
\vspace{-5pt}

\[
\begin{array}{lll}

\begin{array}{lllr}

\hat e & ::= & ~ & labeled~exprs: \\
	& | & x \\
	& | & \hat v \\
	& | & \hat e ~ \hat e \\
	& | & \hat e.\pi \\
	& | & \kwa{import}(\varepsilon)~x = \hat e~\kwa{in}~e & import \\
	&&\\

\hat v & ::= & ~ & labeled~values: \\
	& | & r \\
	& | & \lambda x: \hat \tau.\hat e \\
	&&\\

\end{array}

& ~~~~~~~~&

\begin{array}{lllr}

\hat \tau & ::= & ~ & labeled ~types: \\
		& | & \{ \bar r \} \\
		& | & \hat \tau \rightarrow_{\varepsilon} \hat \tau \\
		&&\\

\hat \Gamma & ::= & ~ & labeled~type~ctx:\\
				& | & \varnothing \\
				& | & \hat \Gamma, x: \hat \tau \\
				&&\\

\end{array}

\end{array}
\]

\vspace{-7pt}
\caption{Annotated programs and types in $\epscalc$.}
\label{fig:epscalc_annotated_grammar}
\end{figure}

The new form is $\import{\varepsilon}{x}{\hat e}{e}$, which represents the point at which capabilities are passed from annotated code into unannotated code. $e$ is the unannotated code. $\hat e$ is the capability being given to it; we call $\hat e$ an import. For simplicity, we assume only one capability is being passed into $e$. $\hat e$ is associated with the name $x$ inside $e$. $\varepsilon$ is the maximum authority that $e$ is allowed to exercise (its ``selected authority''). As an example, suppose an unannotated $\kwa{Logger}$, which requires $\kwa{File}$, is expected to only $\kwa{append}$ to a file, but has an implementation that writes. This would be modelled by the expression $\import{\kwa{File.append}}{x}{\kwa{File}}{\lambda y: \Unit. \kwa{x.write}}$.

$\kwa{import}$ is the only way to mix annotated and unannotated code, because it is the only situation in which we can say something interesting about the unannotated code. For the rest of our discussion on $\epscalc$, we will only be interested in unannotated code when it is encapsulated by an $\kwa{import}$ expression.

One of the requirements of capability safety is there be no ambient authority.
This requirement is met by forbidding resource literals $r$ from being used directly inside an \kwat{import} statement (they can still be passed in as a capability via the \kwat{import}'s binding variable $x$).
We could enforce this syntactically, by removing $r$ from the language of unannotated expressions, but we choose to do it instead using the typing rule for \kwat{import}, given below.

\subsection{Semantics ($\epscalc$)}

Reductions are defined on annotated expressions. If unannotated code $e$ is wrapped inside annotated code $\import{\varepsilon}{x}{\hat e}{e}$, we transform it into annotated code by labeling its parts with $\varepsilon$. In practice, it is meaningful to execute purely unannotated code --- but our only interest is when that code is wrapped inside an $\kwa{import}$ expression, so we do not bother to give rules for it.

Excluding $\kwa{import}$, the annotated sublanguage of $\epscalc$ is the same as $\opercalc$, so we take every reduction rule of $\opercalc$ as a valid reduction rule in $\epscalc$. For brevity, they are not restated. There are two new rules in $\epscalc$ for reducing $\kwa{import}$ expressions, given in Figure \ref{fig:epscalc_reductions}. \textsc{E-Import1} reduces the capability being imported, while \textsc{E-Import2} first annotates $e$ with its selected authority $\varepsilon$ --- this is $\annot{e}{\varepsilon}$ --- and then substitutes the import $\hat v$ for its name $x$ in $e$ --- this is $[\hat v/x]\annot{e}{\varepsilon}$.

\begin{figure}[h]

\fbox{$\hat e \longrightarrow \hat e~|~\varepsilon$}

\[
\begin{array}{c}
\infer[\textsc{(E-Import1)}]
	{\kwa{import}(\varepsilon)~x = \hat e~\kw{in} e \longrightarrow \kwa{import}(\varepsilon)~x = \hat e'~\kw{in} e~|~\varepsilon'}
	{\hat e \longrightarrow \hat e'~|~\varepsilon'}\\[4ex]

\infer[\textsc{(E-Import2)}]
	{\kwa{import}(\varepsilon)~x = \hat v~\kw{in} e \longrightarrow [\hat v/x]\kwa{annot}(e, \varepsilon)~|~\varnothing}
	{}

\end{array}
\]


\vspace{-7pt}
\caption{New single-step reductions in $\epscalc$.}
\label{fig:epscalc_reductions}
\end{figure}

A definition for $\kwa{annot}$ is given in Figure \ref{fig:annot_defn}. $\annot{e}{\varepsilon}$ produces the expression obtained by recursively labeling the parts of $e$ with the set of effects $\varepsilon$. There are versions of $\kwa{annot}$ defined for expressions and types. Later we shall need to annotate contexts, so the definition is given here. It is worth mentioning that $\kwa{annot}$ operates on a purely syntatic level --- nothing prevents us from annotating a program with something unsafe, so any use of $\kwa{annot}$ must be justified.

\begin{figure}[h]
\vspace{-5pt}

$\kwa{annot} :: e \times \varepsilon \rightarrow \hat e$

\begin{itemize}
	\setlength\itemsep{-0.2em}
	\item[] $\annot{r}{\_} = r$
	\item[] $\annot{\lambda x: \tau_1 . e}{\varepsilon} = \lambda x: \annot{\tau_1}{\varepsilon} . \annot{e}{\varepsilon}$
	\item[] $\annot{e_1~e_2}{\varepsilon} = \kwa{annot}(e_1, \varepsilon)~\kwa{annot}(e_2, \varepsilon)$
	\item[] $\annot{e_1.\pi}{\varepsilon} = \annot{e_1}{\varepsilon}.\pi$
\end{itemize}
	
$\kwa{annot} :: \tau \times \varepsilon \rightarrow \hat \tau$

\begin{itemize}
	\setlength\itemsep{-0.2em}
	\item[] $\annot{\{ \bar r \}}{\_} = \{ \bar r \}$
	\item[] $\annot{\tau_1 \rightarrow \tau_2}{\varepsilon} = \annot{\tau_1}{\varepsilon} \rightarrow_{\varepsilon} \annot{\tau_2}{\varepsilon}$.	
\end{itemize}

$\kwa{annot} :: \Gamma \times \varepsilon \rightarrow \hat \Gamma$

\begin{itemize}
	\setlength\itemsep{-0.2em}
	\item[] $\annot{\varnothing}{\_} = \varnothing$
	\item[] $\annot{\Gamma, x: \tau}{\varepsilon} = \annot{\Gamma}{\varepsilon}, x: \annot{\tau}{\varepsilon}$
\end{itemize}

\vspace{-7pt}
\caption{Definition of $\kwa{annot}$.}
\label{fig:annot_defn}
\end{figure}

\subsection{Static Rules ($\epscalc$)}

A term can be annotated or unannotated, so we need to be able to recognise when either is well-typed. We do not reason about the effects of unannotated code directly, so judgements about them have the form $\Gamma \vdash e: \tau$. Subtyping judgements have the form $\tau <: \tau$. A summary of the rules for unannotated judgements is given in Figure \ref{fig:unannotated_static_rules}. Each is analogous to some rule in $\opercalc$, but the parts relating to effects have been removed.

\begin{figure}[h]
\vspace{-5pt}


\fbox{$\Gamma \vdash e: \tau$}

\[
\begin{array}{c}


\infer[\textsc{(T-Var)}]
	{\Gamma, x: \tau \vdash x: \tau}
	{}
\hspace{5ex}
\infer[\textsc{(T-Resource)}]
	{\Gamma, r: \{ r \} \vdash r : \{ r \}}
	{}

\hspace{5ex}
\infer[\textsc{(T-Abs)}]
	{\Gamma \vdash \lambda x: \tau_1.e : \tau_1 \rightarrow \tau_2}
	{\Gamma, x: \tau_1 \vdash e: \tau_2}\\[4ex]
	
\infer[\textsc{(T-App)}]
	{\Gamma \vdash e_1~e_2: \tau_3}
	{\Gamma \vdash e_1: \tau_2 \rightarrow \tau_3 & \Gamma \vdash e_2: \tau_2}
\hspace{5ex}
\infer[\textsc{(T-OperCall)}]
	{\Gamma \vdash e.\pi: \kwa{Unit}}
	{\Gamma \vdash e: \{ \bar r \}}

\end{array}
\]



\fbox{$\tau <: \tau$}

\[
\begin{array}{c}

\infer[\textsc{(S-Arrow)}]
	{ \tau_1 \rightarrow \tau_2 <: \tau_1' \rightarrow \tau_2' }
	{ \tau_1' <: \tau_1 & \tau_2 <: \tau_2' }
\hspace{5ex}
\infer[\textsc{(S-Resources)}]
	{ \{ \bar r_1 \} <: \{ \bar r_2 \} }
	{ \{ \bar r_1 \} \subseteq \{ \bar r_2 \} }

\end{array}
\]

\vspace{-7pt}
\caption{(Sub)typing judgements for the unannotated sublanguage of $\epscalc$}
\label{fig:unannotated_static_rules}
\end{figure}

Since the annotated subset of $\epscalc$ contains $\opercalc$, all the $\opercalc$ judgements apply, but now we put hats on everything to signify that a typing judgement is being made about annotated code inside an annotated context. This looks like $\hat \Gamma \vdash \hat e: \hat \tau~\kw{with} \varepsilon$. Except for notation the judgements are the same, so we shall not repeat them. The only new rule is \textsc{$\varepsilon$-Import}, given in Figure \ref{fig:import_rule}, which gives the type and approximate effects of an $\kwa{import}$ expression. This is the only way to reason about what effects might be incurred by some unannotated code. The rule is complicated, so to explain it we shall start with a simplified version and spend the rest of this section building up to the final version of \textsc{$\varepsilon$-Import}.

%\begin{figure}[h]
%\vspace{-5pt}
%
%\[
%\begin{array}{c}
%
%\infer[\textsc{($\varepsilon$-Import)}]
%	{ \hat \Gamma \vdash \kwa{import}(\varepsilon)~x = \hat e~\kw{in} e: \kwa{annot}(\tau, \varepsilon)~\kw{with} \varepsilon \cup \varepsilon_1 }
%{{\def\arraystretch{1.4}
%  \begin{array}{c}
%\kwa{effects}(\hat \tau) \cup \hofx{\annot{\tau}{\varnothing}}\subseteq \varepsilon \\
%\hat \Gamma \vdash \hat e: \hat \tau~\kw{with} \varepsilon_1 ~~~~~~ \kwa{ho \hyphen safe}(\hat \tau, \varepsilon) ~~~~~~ x: \kwa{erase}(\hat \tau) \vdash e: \tau
%  \end{array}}} 
% 
%\end{array}
%\]
%\vspace{-7pt}
%\caption{(Sub)typing judgements for the unannotated sublanguage of $\epscalc$}
%\label{fig:import_rule}
%\end{figure}

To begin, typing $\import{\varepsilon}{x}{\hat e}{e}$ in a context $\hat \Gamma$ requires us to know that the import $\hat e$ is well-typed, so we add the premise $\hat \Gamma \vdash \hat e: \hat \tau~\kw{with} \varepsilon_1$. Since $x = \hat e$ is an import, it can be used throughout $e$. However, we do not want $e$ to exercise authority it hasn't explicitly selected, so whatever capabilities are used must be selected by the $\kwa{import}$ expression; therefore, we require that $e$ can be typechecked using only the binding $x: \hat \tau$. There is a problem though: $e$ is unannotated and $\hat \tau$ is annotated, and there is no rule for typechecking unannotated code in an annotated context. To get around this, we define a function $\kwa{erase}$ in Figure \ref{fig:erase_defn} which removes the annotations from a type. We then add $x: \erase{\hat \tau} \vdash e: \tau$ as a premise.

\begin{figure}[h]
\vspace{-5pt}

$\kwa{erase} :: \hat \tau \rightarrow \tau$
\begin{itemize}
	\setlength\itemsep{-0.2em}
	\item[] $\erase{\{ \bar r \}}$
	\item[] $\erase{\hat \tau_1 \rightarrow_{\varepsilon} \hat \tau_2} = \erase{\hat \tau_1} \rightarrow \erase{\hat \tau_2}$
\end{itemize}

\vspace{-7pt}
\caption{Definition of $\kwa{erase}$.}
\label{fig:erase_defn}
\end{figure}

Note that, since the environment $\Gamma$ for $e$ has only one binding (for $x$), it cannot contain any bindings of resource literals---and the rule \textsc{T-Resource} requires a binding in the environment in order to type a resource literal in an expression.  We thus use the restricted environment given by \kwat{import} to prohibit ambient authority. 

The first version of \textsc{$\varepsilon$-Import} is given in Figure \ref{fig:import_rule_1}. Since $\import{\varepsilon}{x}{\hat v}{e} \longrightarrow [\hat v/x]\annot{e}{\varepsilon}$ by \textsc{E-Import2}, the ascribed type is $\annot{\tau}{\varepsilon}$, which is the type of the unannotated code, annotated with its selected authority $\varepsilon$. The effects of the $\kwa{import}$ are $\varepsilon_1 \cup \varepsilon$ --- the former comes from reducing the imported capability, which happens before the body of the $\kwa{import}$ is annotated and executed, and the latter contains all the effects which the unannotated code might incur.


\begin{figure}[h]

\[
\begin{array}{c}

\infer[\textsc{($\varepsilon$-Import1-Bad)}]
	{ \hat \Gamma \vdash \import{\varepsilon}{x}{\hat e}{e}: \kwa{annot}(\tau, \varepsilon)~\kw{with} \varepsilon \cup \varepsilon_1 }
	{ \hat \Gamma \vdash \hat e: \hat \tau~\kw{with} \varepsilon_1 & x: \kwa{erase}(\hat \tau) \vdash e: \tau }

\end{array}
\]
\vspace{-7pt}
\caption{A first (incorrect) rule for type-and-effect checking $\kwa{import}$ expressions.}
\label{fig:import_rule_1}
\end{figure}

At the moment there is no relation between the selected authority $\varepsilon$ and those effects captured by the imported capability $\hat e$. Consider $\hat e' = \import{\varnothing}{x}{\File}{\kwa{x.write}}$, which imports a $\File$ and writes to it, but declares its authority as $\varnothing$. According to \textsc{$\varepsilon$-Import1}, $\vdash \hat e': \Unit~\kw{with} \varnothing$, but this is clearly wrong since $\hat e'$ writes to $\kwa{File}$.
An $\kwa{import}$ should only be well-typed if the capability being imported only captures effects that are in the unannotated code's selected authority $\varepsilon$.
To this end we define a function $\kwa{effects}$, which collects the set of effects that an annotated type captures. A first (but not yet correct) definition is given in Figure \ref{fig:fx_defn}. We can then add the premise $\kwa{effects}(\hat \tau) \subseteq \varepsilon$ to require that any imported capability must not capture authority beyond that selected in $\varepsilon$. The updated rule is given in Figure \ref{fig:import_rule_2}.

\begin{figure}[h]

$\kwa{effects} :: \hat \tau \rightarrow \varepsilon$
\begin{itemize}
	\setlength\itemsep{-0.2em}
	\item[] $\fx{\{ \bar r \}} = \{ r.\pi \mid r \in \bar r, \pi \in \Pi \}$
	\item[] $\fx{\hat \tau_1 \rightarrow_{\varepsilon} \hat \tau_2} = \fx{\hat \tau_1} \cup \varepsilon \cup \fx{\hat \tau_2}$
\end{itemize}
\vspace{-7pt}
\caption{A first (incorrect) definition of $\kwa{effects}$.}
\label{fig:fx_defn}
\end{figure}

\begin{figure}[h]

\[
\begin{array}{c}

\infer[\textsc{($\varepsilon$-Import2-Bad)}]
	{ \hat \Gamma \vdash \import{\varepsilon}{x}{\hat e}{e}: \kwa{annot}(\tau, \varepsilon)~\kw{with} \varepsilon \cup \varepsilon_1 }
	{ \hat \Gamma \vdash \hat e: \hat \tau~\kw{with} \varepsilon_1 & x: \kwa{erase}(\hat \tau) \vdash e: \tau & \kwa{effects}(\hat \tau) \subseteq \varepsilon}

\end{array}
\]
\vspace{-7pt}
\caption{A second (still incorrect) rule for type-and-effect checking $\kwa{import}$ expressions.}
\label{fig:import_rule_2}
\end{figure}

The counterexample from before is now rejected by \textsc{$\varepsilon$-Import2}, but there are still issues: the annotations on one import can be broken by another import. To illustrate, consider Figure \ref{fig:rule_import2_counterexample} where two\footnote{Our formalisation only permits a single capability to be imported, but this discussion leads to a generalisation needed for the rules be safe when multiple capabilities can be imported.  In any case importing multiple capabilities can be handled with an encoding of pairs.} capabilities are imported. This program imports a function $\kwa{go}$ which, when given a $\Unit \rightarrow_{\varnothing} \Unit$ function with no effects, will execute it. The other import is $\kwa{File}$. The unannotated code creates a $\Unit \rightarrow \Unit$ function which writes to $\kwa{File}$ and passes it to $\kwa{go}$, which subsequently incurs $\kwa{File.write}$.

\begin{figure}[h]

\begin{lstlisting}
import({File.*})
   go = $\lambda$x: Unit $\rightarrow_{\varnothing}$ Unit. x unit
   f = File
in
   go ($\lambda$y: Unit. f.write)

\end{lstlisting}

\vspace{-7pt}
\caption{Permitting multiple imports will break \textsc{$\varepsilon$-Import2}.}
\label{fig:rule_import2_counterexample}
\end{figure}

In the world of annotated code it is not possible to pass a file-writing function to $\kwa{go}$, but because the judgement $x: \erase{\hat \tau} \vdash e: \tau$ discards the annotations on $\kwa{go}$, and since the file-writing function has type $\unit \rightarrow \unit$, the unannotated world accepts it. The approximation is actually safe at the top-level, because the $\kwa{import}$ selects $\{ \kwa{File.*} \}$, which contains $\kwa{File.write}$ --- but it contains code that violates the type signature of $\kwa{go}$. We want to prevent this.

If $\kwa{go}$ had the type $\Unit \rightarrow_{\{ \kwa{File.write} \}} \Unit$ the above example would be safe, but a modified version where a file-reading function is passed to $\kwa{go}$ would have the same issue. $\kwa{go}$ is only safe when it expects every effect that the unannotated code might pass to it: if $\kwa{go}$ had the type $\Unit \rightarrow_{\{ \kwa{File.*} \}} \Unit$, then the unannotated code cannot pass it a capability with an effect it isn't already expecting, so the annotation on $\kwa{go}$ cannot be violated. Therefore we require imported capabilities to have authority to incur the effects in $\varepsilon$. To achieve greater control in how we say this, the definition of $\kwa{effects}$ is split into two separate functions called $\kwa{effects}$ and $\kwa{ho \hyphen effects}$. The latter is for higher-order effects, i.e. the effects that are not captured within a function, but rather are possible because of what it is passed as an argument. If values of $\hat \tau$ possess a capability that can be used to incur the effect $r.\pi$, then $r.\pi \in \fx{\hat \tau}$. If values of $\hat \tau$ can incur an effect $r.\pi$, but need to be given the capability by someone else in order to do it, then $r.\pi \in \hofx{\hat \tau}$. Definitions are given in Figure \ref{fig:fx_defns}.


\begin{figure}[h]

$\kwa{effects} :: \hat \tau \rightarrow \varepsilon$

\begin{itemize}
	\setlength\itemsep{-0.2em}
	\item[] $\fx{\{ \bar r \}} = \{ r.\pi \mid r \in \bar r, \pi \in \Pi \}$
	\item[] $\fx{\hat \tau_1 \rightarrow_{\varepsilon} \hat \tau_2} = \hofx{\hat \tau_1} \cup \varepsilon \cup \fx{\hat \tau_2}$
\end{itemize}

$\kwa{ho \hyphen effects} :: \hat \tau \rightarrow \varepsilon$

\begin{itemize}
	\setlength\itemsep{-0.2em}
	\item[] $\hofx{\{ \bar r \}} = \varnothing$
	\item[] $\hofx{\hat \tau_1 \rightarrow_{\varepsilon} \hat \tau_2} = \fx{\hat \tau_1} \cup \hofx{\hat \tau_2}$
\end{itemize}

\vspace{-7pt}
\caption{Effect functions (corrected).}
\label{fig:fx_defns}
\end{figure}

$\kwa{effects}$ and $\kwa{ho \hyphen effects}$ are mutually recursive, with base cases for resource types. Any effect can be directly incurred by a resource on itself, hence $\fx{\{ \bar r \}} = \{ r.\pi \mid r \in \bar r, \pi \in \Pi \}$. A resource cannot be used to indirectly invoke some other effect, so $\hofx{\{ \bar r \}} = \varnothing$. The mutual recursion echoes the subtyping rule for functions. Recall that functions are contravariant in their input type and covariant in their output type. The mutual recursion here is similar: both functions recurse on the input-type using the other function, and recurse on the output-type using the same function.

In light of these new definitions, we still require $\fx{\hat \tau} \subseteq \varepsilon$ --- unannotated code must select any effect its capabilities can incur --- but we add a new premise $\varepsilon \subseteq \hofx{\hat \tau}$, stipulating that imported capabilities must select every effect they could be given by unannotated code. The counterexample from Figure \ref{fig:rule_import2_counterexample} is now rejected, because $\hofx{\Unit \rightarrow_{\varnothing} \Unit) \rightarrow_{\varnothing} \Unit} = \varnothing$, but $\{ \kwa{File.*} \} \not\subseteq \varnothing$.  However, this is \textit{still} not sufficient! Consider $\varepsilon \subseteq \hofx{ \hat \tau_1 \rightarrow_{\varepsilon'} \hat \tau_2 }$. We want \textit{every} higher-order capability involved to be expecting $\varepsilon$. Expanding the definition of $\kwa{ho \hyphen effects}$, this is the same as $\varepsilon \subseteq \fx{\hat \tau_1} \cup \hofx{\hat \tau_2}$. Let $r.\pi \in \varepsilon$ and suppose $r.\pi \in \fx{\hat \tau_1}$, but $r.\pi \notin \hofx{\hat \tau_2}$. Then $\varepsilon \subseteq \fx{\hat \tau_1} \cup \hofx{\hat \tau_2}$ is still true, but $\hat \tau_2$ is not expecting $r.\pi$. Unannotated code could then violate the annotations on $\hat \tau_2$ by passing it a capability for $r.\pi$, using the same trickery as before. The cause of the issue is that $\subseteq$ does not distribute over $\cup$. We want a relation like $\varepsilon \subseteq \fx{\hat \tau_1} \cup \hofx{\hat \tau_2}$, but which also implies $\varepsilon \subseteq \fx{\hat \tau_1}$ and $\varepsilon \subseteq \fx{\hat \tau_2}$. Figure \ref{fig:safe_defns} defines this: $\kwa{safe}$ is a distributive version of $\varepsilon \subseteq \fx{\hat \tau}$ and $\kwa{ho \hyphen safe}$ is a distributive version of $\varepsilon \subseteq \hofx{\hat \tau}$.


\begin{figure}[h]

\noindent
$\fbox{$\safe{\hat \tau}{\varepsilon}$}$

\[
\begin{array}{c}

\infer[\textsc{(Safe-Resource)}]
	{ \kwa{safe}(\{ \bar r \}, \varepsilon) }
	{} 
\hspace{5ex}
	
\infer[\textsc{(Safe-Arrow)}]
	{\kwa{safe}(\hat \tau_1 \rightarrow_{\varepsilon'} \hat \tau_2, \varepsilon)}
	{\varepsilon \subseteq \varepsilon' & \kwa{ho \hyphen safe}(\hat \tau_1, \varepsilon) & \kwa{safe}(\hat \tau_2, \varepsilon)} \\[3ex]

\end{array}
\]

\noindent
$\fbox{$\hosafe{\hat \tau}{\varepsilon}$}$

\[
\begin{array}{c}

\infer[\textsc{(HOSafe-Resource)}]
	{ \kwa{ho \hyphen safe}( \{ \bar r \}, \varepsilon)} 
	{}
\hspace{5ex}

\infer[\textsc{(HOSafe-Arrow)}]
	{ \kwa{ho \hyphen safe}( \hat \tau_1 \rightarrow_{\varepsilon'} \hat \tau_2, \varepsilon ) }
	{ \kwa{safe}(\hat \tau_1, \varepsilon)  & \kwa{ho \hyphen safe}(\hat \tau_2, \varepsilon) }\\[3ex]

\end{array}
\]

\vspace{-7pt}
\caption{Safety judgements in $\epscalc$.}
\label{fig:safe_defns}
\end{figure}

An amended version of \textsc{$\varepsilon$-Import} is given in Figure \ref{fig:import_rule3}. It contains a new premise $\hosafe{\hat \tau}{\varepsilon}$ which formalises the notion that every capability which could given to a value of $\hat \tau$ --- or any of its constituent pieces --- must be expecting the effects $\varepsilon$ it might be given by the unannotated code.

\begin{figure}[h]

\[
\begin{array}{c}

\infer[\textsc{($\varepsilon$-Import3-Bad)}]
	{ \hat \Gamma \vdash \kwa{import}(\varepsilon)~x = \hat e~\kw{in} e: \kwa{annot}(\tau, \varepsilon)~\kw{with} \varepsilon \cup \varepsilon_1 }
{{\def\arraystretch{1.4}
  \begin{array}{c}
\hat \Gamma \vdash \hat e: \hat \tau~\kw{with} \varepsilon_1
~~~~~~
\kwa{effects}(\hat \tau) \subseteq \varepsilon \\
\hosafe{\hat \tau}{\varepsilon} ~~~~~~ x: \kwa{erase}(\hat \tau) \vdash e: \tau
  \end{array}}} 
 
\end{array}
\]

\vspace{-7pt}
\caption{A third (still incorrect) rule for type-and-effect checking $\kwa{import}$ expressions.}
\label{fig:import_rule3}
\end{figure}

The premises so far restrict what authority can be selected by unannotated code, but what about authority passed as a function argument? Consider the example $\hat e = \import{\varnothing}{x}{\unit}{\lambda f: { \File }.~\kwa{f.write}}$. The unannotated code selects no capabilities and returns a function which, when given $\kwa{File}$, incurs $\kwa{File.write}$. This satisfies the premises in \textsc{$\varepsilon$-Import3}, but its annotated type is $\{ \File \} \rightarrow_{\varnothing} \Unit$ --- not good!

Suppose the unannotated code defines a function $f$, which gets annotated with $\varepsilon$ to produce $\annot{f}{\varepsilon}$. Suppose $\annot{f}{\varepsilon}$ is invoked at a later point in the annotated world and incurs the effect $r.\pi$. What is the source of $r.\pi$? If $r.\pi$ was selected by the $\kwa{import}$ expression surrounding $f$, it is safe for $\annot{f}{\varepsilon}$ to incur this effect. Otherwise, $\annot{f}{\varepsilon}$ may have been passed an argument which can be used to incur $r.\pi$, in which case $r.\pi$ is a higher-order effect of $\annot{f}{\varepsilon}$. If the argument is a function, then by the soundness of $\opercalc$, it must be that $r.\pi \in \varepsilon$, or it will not typecheck. If the argument is a resource $r$ then $\annot{f}{\varepsilon}$ may exercise $r.\pi$ without declaring it --- this is the case we do not yet account for.

We want $\varepsilon$ to contain every effect captured by resources passed into $\annot{f}{\varepsilon}$ as arguments. We can do this by inspecting its (unannotated type) for resource sets. For example, if the unannotated code has the type $\kwa{ \{ File \} \rightarrow \Unit}$, then we need $\kwa{ \{ File.* \} }$ in $\varepsilon$. To do this, we add a new premise $\hofx{\annot{\tau}{\varnothing}} \subseteq \varepsilon$. $\kwa{ho \hyphen effects}$ is only defined on annotated types, so we first annotate $\tau$ with $\varnothing$. We are only inspecting the resources passed into $f$ as arguments, so the annotations are not relevant -- annotating $\tau$ with $\varnothing$ is therefore a good choice. We can now handle the example from before. The unannotated code types via the judgement $x: \Unit \vdash \lambda f: \{ \File \}.~\kwa{f.write}: \{ \File \} \rightarrow \Unit$. Its higher-order effects are $\hofx{\annot{ \{ \File \} \rightarrow \Unit}{\varnothing}} = \{ \kwa{File.*} \}$, but $\{ \kwa{File.*} \} \not\subseteq \varnothing$, so the example is safely rejected.

The final version of \textsc{$\varepsilon$-Import} is given in Figure \ref{fig:import_rule}.

\begin{figure}[h]

\[
\begin{array}{c}

\infer[\textsc{($\varepsilon$-Import)}]
	{ \hat \Gamma \vdash \kwa{import}(\varepsilon)~x = \hat e~\kw{in} e: \kwa{annot}(\tau, \varepsilon)~\kw{with} \varepsilon \cup \varepsilon_1 }
{{\def\arraystretch{1.4}
  \begin{array}{c}
\kwa{effects}(\hat \tau) \cup \hofx{\annot{\tau}{\varnothing}}\subseteq \varepsilon \\
\hat \Gamma \vdash \hat e: \hat \tau~\kw{with} \varepsilon_1 ~~~~~~ \kwa{ho \hyphen safe}(\hat \tau, \varepsilon) ~~~~~~ x: \kwa{erase}(\hat \tau) \vdash e: \tau
  \end{array}}} 
 
\end{array}
\]


\vspace{-7pt}
\caption{The final rule for typing imports.}
\label{fig:import_rule}
\end{figure}

We can now model the example from the beginning of this section, where the $\kwa{Logger}$ selects the $\kwa{File}$ capability and exposes an unannotated function $\kwa{log}$ with type $\Unit \rightarrow \Unit$ and implementation $e$. The expected least authority of $\kwa{Logger}$ is $\{ \kwa{File.append} \}$, so its corresponding $\kwa{import}$ expression would be $\import{\kwa{File.append}}{f}{\kwa{File}}{\lambda x: \Unit.~e}$. The imported capability is $ f = \kwa{File}$, and $\fx{\{\File\}} = \{ \kwa{File.*} \} \not\subseteq \{ \kwa{File.append} \}$, so this example is safely rejected: $\kwa{Logger.log}$ has authority to do anything with $\kwa{File}$, and its implementation $e$ might be violating its stipulated least authority $\{ \kwa{File.append} \}$.

\subsection{Soundness ($\epscalc$)} 

For this section we adopt a different notational convention to avoid name clashes: the selected authority of an $\kwa{import}$ will be written $\varepsilon_{s}$ (``epsilon select''). An $\kwa{import}$ expression will look like this: $\import{\varepsilon_s}{x}{\hat e_i}{e}$.

Only annotated programs can be reduced and have their effects approximated, so the soundness theorem only applies to annotated judgements. Its statement is given below.

\begin{theorem}[$\epscalc$ Single-step Soundness]
If $\hat \Gamma \vdash \hat e_A: \hat \tau_A~\kw{with} \varepsilon_A$ and $\hat e_A$ is not a value, then $\hat e_A \longrightarrow \hat e_B~|~\varepsilon$, where $\hat \Gamma \vdash \hat e_B: \hat \tau_B~\kw{with} \varepsilon_B$ and $\hat \tau_B <: \hat \tau_A$ and $\varepsilon_B \cup \varepsilon \subseteq \varepsilon_A$, for some $\hat e_B, \varepsilon, \hat \tau_B, \varepsilon_B$.
\end{theorem}

Because the rules of $\opercalc$, proven sound in section 2, are also rules of $\epscalc$, we do not repeat them here. The progress theorem has a new case for when the typing rule used is \textsc{$\varepsilon$-Import}, but the proof is routine.

\begin{theorem}[$\epscalc$ Progress]
If $\hat \Gamma \vdash \hat e: \hat \tau~\kw{with} \varepsilon$ and $\hat e$ is not a value, then $\hat e \longrightarrow \hat e'~|~\varepsilon'$, for some $\hat e', \varepsilon' \subseteq \varepsilon$.
\end{theorem}

\begin{proof} By induction on derivations of $\hat \Gamma \vdash \hat e: \hat \tau~\kw{with} \varepsilon$.
\end{proof}

The preservation theorem also has an extra case for when the typing rule used is \textsc{$\varepsilon$-Import}. This has two subcases, depending on whether the reduction rule used was \textsc{E-Import1} and \textsc{E-Import2}. The former is straightforward, but the latter is tricky; we need several lemmas to do it. Firstly, since $\varepsilon_s$ is an upper bound on what effects can be incurred by the unannotated code, it should also be an upper bound on what effects can be incurred by the capabilities passed into the unannotated code; therefore, if we take $\hat \tau_i$ and replace its annotations with $\varepsilon_s$, we should get a more general function type $\annot{\erase{\hat \tau_i}}{\varepsilon}$. This result is given as the pair of lemmas below.

\begin{lemma}[$\epscalc$ Approximation 1]
If $\kwa{effects}(\hat \tau) \subseteq \varepsilon$ and $\kwa{ho \hyphen safe}(\hat \tau, \varepsilon)$ then $\hat \tau <: \kwa{annot}(\kwa{erase}(\hat \tau), \varepsilon)$.
\end{lemma}

\begin{lemma}[$\epscalc$ Approximation 2]
If $\kwa{ho \hyphen effects}(\hat \tau) \subseteq \varepsilon$ and $\safe{\hat \tau}{\varepsilon}$ then $\kwa{annot(erase}(\hat \tau), \varepsilon) <: \hat \tau$.
\end{lemma}

\begin{proof}
By simultaneous induction on derivations of $\hosafe{\hat \tau}{\varepsilon}$ and $\safe{\hat \tau}{\varepsilon}$.
\end{proof}

Recall that function types are contravariant in their input, so the subtyping and subsetting relations flip direction when considering the input type of a function. This is why there are two lemmas: one for each direction.

Now, if \textsc{E-Import2} is applied, the reduction has the form $\import{\varepsilon_{s}}{x}{\hat v_i}{e} \longrightarrow [\hat v_i/x]\annot{e}{\varepsilon_s}~|~\varnothing$. Since $x: \erase{\hat \tau} \vdash e: \tau$, it is reasonable to expect (1) $\hat \Gamma \vdash \annot{e}{\varepsilon_s}: \annot{\tau}{\varepsilon_s}~\kw{with} \varepsilon_s$ is true --- the reduction annotates $e$ with $\varepsilon_s$, so the type after annotation ought to be the type of $e$, annotated with $\varepsilon_s$, i.e. $\annot{\tau}{\varepsilon_s}$. Furthermore, $\annot{e}{\varepsilon_s}$ is really the same program as $e$, but with extra labels. These labels do not change what capabilities can be used by the code --- the bound $\varepsilon_s$ on the authority of $e$ is therefore also a bound on the authority of $\annot{e}{\varepsilon_s}$. Now, if judgement (1) holds, then $\hat \Gamma \vdash [\hat v_i/x]\annot{e}{\varepsilon_s}: \annot{\tau}{\varepsilon_s}~\kw{with} \varepsilon_s$ would hold by the substitution lemma. That judgement (1) does hold is the subject of the following lemma.

\begin{lemma}[$\epscalc$ Annotation]
If the following are true:

\begin{enumerate}
	\item $\hat \Gamma \vdash \hat v_i : \hat \tau_i~\kw{with} \varnothing$
	\item $\Gamma, y: \kwa{erase}(\hat \tau_i) \vdash e: \tau$
	\item $\kwa{effects}(\hat \tau_i) \cup \hofx{\annot{\tau}{\varnothing}} \cup \fx{\annot{\Gamma}{\varnothing}} \subseteq \varepsilon_{s}$
	\item $\kwa{ho \hyphen safe}(\hat \tau_i, \varepsilon_s)$
\end{enumerate}

Then $\hat \Gamma, \kwa{annot}(\Gamma, \varepsilon_s), y: \hat \tau_i \vdash \kwa{annot}(e, \varepsilon_s) : \kwa{annot}(\tau, \varepsilon_s)~\kw{with} \varepsilon_s$.
\end{lemma}


The premises of the lemma are very specific to the premises of \textsc{$\varepsilon$-Import}, but generalised to accommodate a proof by induction: $e$ is allowed to typecheck with bindings in $\Gamma$, so long as $\Gamma$ does not introduce any resources whose authority is not already in $\varepsilon_s$. We need $\Gamma$ to keep track of effects introduced by function arguments. For example, typechecking $\kwa{f.write}$ requires a binding for $f$, but $\lambda f: \{\File\}.~\kwa{f.write}$ does not. Proving the lemma requires us to inductively step into the bodies of functions, at which point we need to keep track of what has been bound at that point --- to do this, we permit $e$ to typecheck in a larger environment $\Gamma$. We stipulate $\fx{\annot{\Gamma}{\varnothing}} \subseteq \varepsilon_s$ so that any effects captured by $\Gamma$ are not ambient. Note that when $\Gamma = \varnothing$ we have exactly the premises of \textsc{$\varepsilon$-Import}. When we apply the annotation lemma in the proof of preservation, we shall choose $\Gamma = \varnothing$. A proof-sketch of the annotation lemma is given below.

\begin{proof}
By induction on derivations of $\Gamma, y: \kwa{erase}(\hat \tau_i) \vdash e: \tau$.\\

\textit{Case:} \textsc{T-Var}. Then $e = x$. If $x \neq y$ use \textsc{$\varepsilon$-Var} and \textsc{$\varepsilon$-Subsume}. Otherwise $x = y$. Then $y: \erase{\hat \tau_i} \vdash x: \tau$ implies that $\hat \tau_i = \tau$. Apply the approximation lemma and simplify to obtain $\hat \tau_i <: \annot{\tau_i}{\varepsilon_s}$, then use \textsc{$\varepsilon$-Subsume} to get the result.\\

\textit{Case:} \textsc{T-Resource}. Use \textsc{$\varepsilon$-Resource} and \textsc{$\varepsilon$-Subsume}.\\

\textit{Case:} \textsc{T-Abs}. Use inversion to get a judgement for the body of the function $\Gamma, y: \erase{\hat \tau_i}, x: \tau_2 \vdash e_{body}: \tau_3~\kw{with} \varepsilon_s$. Apply the inductive hypothesis to $e_{body}$ with $\Gamma, x: \tau_2$ as the context in which $e_{body}$ typechecks, noting the premises for the inductive application are satisfied because $\hofx{\annot{\tau}{\varnothing}} \subseteq \varepsilon_s$ implies $\kwa{effects}(\annot{\tau_1}{\varnothing} \subseteq \varepsilon_s$. Then use \textsc{$\varepsilon$-Abs} and \textsc{$\varepsilon$-Subsume}.	\\

\textsc{Case:} \textsc{T-App}. Apply the inductive assumption to the subexpressions, then use \textsc{$\varepsilon$-App} and simplify.\\

\textsc{Case:} \textsc{T-OperCall}. Apply the inductive hypothesis to the receiver and use \textsc{$\varepsilon$-OperCall}. This gives the approximate effects $\varepsilon_s \cup \{ \bar r.\pi \}$. Consider where the binding for $\{ \bar r \}$ is in $\hat \Gamma, \annot{\Gamma}{\varepsilon_s}, y: \hat \tau$ and conclude that $\{ \bar r.\pi \} \subseteq \varepsilon_s$.
\end{proof}

Armed with the annotation lemma, we can now prove preservation.


\begin{theorem}[$\epscalc$ Preservation]
If $\hat \Gamma \vdash \hat e_A: \hat \tau_A~\kw{with} \varepsilon_A$ and $\hat e_A \longrightarrow \hat e_B~|~\varepsilon$, then $\hat \Gamma \vdash \hat e_B: \hat \tau_B~\kw{with} \varepsilon_B$, where $\hat e_B <: \hat e_A$ and $\varepsilon \cup \varepsilon_B \subseteq \varepsilon_A$, for some $\hat e_B, \varepsilon, \hat \tau_B, \varepsilon_B$.
\end{theorem}

\begin{proof} By induction on derivations of $\hat \Gamma \vdash \hat e_A: \hat \tau_A~\kw{with} \varepsilon_A$ and $\hat e_A \longrightarrow \hat e_B~|~\varepsilon$. \\

\textit{Case:} \textsc{$\varepsilon$-Import}. Then $e_A = \import{\varepsilon_s}{x}{\hat e}{e}$. If the reduction rule used was \textsc{E-Import1} then the result follows by applying the inductive hypothesis to $\hat e$. Otherwise $\hat e$ is a value and the reduction used was \textsc{E-Import2}. Apply the annotation lemma with $\Gamma = \varnothing$ to get the judgement $\hat \Gamma, x: \hat \tau \vdash \kwa{annot}(e, \varepsilon_s): \kwa{annot}(\tau, \varepsilon_s)~\kw{with} \varepsilon_s$. By assumption, $\hat \Gamma \vdash \hat v: \hat \tau~\kw{with} \varnothing$, so the substitution lemma applies, giving $\hat \Gamma \vdash [\hat v/x]\kwa{annot}(e, \varepsilon): \annot{\tau}{\varepsilon_s}$. Then $\varepsilon_B = \varepsilon_s = \varepsilon_A \cup \varepsilon$ and $\tau_A = \tau_B = \annot{\tau}{\varepsilon_s}$.
\end{proof}

From progress and preservation we can prove the single-step and multi-step soundness theorems for $\epscalc$. Their proofs are identical to the ones in $\opercalc$.

\begin{theorem}[$\epscalc$ Single-step Soundness]
If $\hat \Gamma \vdash \hat e_A: \hat \tau_A~\kw{with} \varepsilon_A$ and $\hat e_A$ is not a value, then $\hat e_A \longrightarrow \hat e_B~|~\varepsilon$, where $\hat \Gamma \vdash \hat e_B: \hat \tau_B~\kw{with} \varepsilon_B$ and $\hat \tau_B <: \hat \tau_A$ and $\varepsilon_B \cup \varepsilon \subseteq \varepsilon_A$, for some $\hat e_B$, $\varepsilon$, $\hat \tau_B$, and $\varepsilon_B$.
\end{theorem}

\begin{theorem}[$\epscalc$ Multi-step Soundness]
If $\hat \Gamma \vdash \hat e_A: \hat \tau_A~\kw{with} \varepsilon_A$ and $\hat e_A \longrightarrow^{*} e_B~|~\varepsilon$, then $\hat \Gamma \vdash \hat e_B: \hat \tau_B~\kw{with} \varepsilon_B$, where $\hat \tau_B <: \hat \tau_A$ and $\varepsilon_B \cup \varepsilon \subseteq \varepsilon_A$, for some $\hat \tau_B$, $\varepsilon_B$.
\end{theorem}
\section{Desugaring}

Our aim in this section is to develop the techniques for our calculi to express practical examples. To do this we introduce two derived forms, $\unit$ and $\kwa{let}$, make some simplifying assumptions, and explain how Wyvern-like programs can be expressed in $\epscalc$.

\subsection{Unit}

The $\unit$ literal is defined as $\unit \defn \lambda x: \varnothing.~x$. It is the same in both annotated and unannotated code. In annotated code, it has the type $\Unit \defn \varnothing \rightarrow_{\varnothing} \varnothing$, while in unannotated code it has the type $\Unit \defn \varnothing \rightarrow \varnothing$. These are technically two separate types, but we will not distinguish between them. Note that $\unit$ is a value, and because $\varnothing$ is uninhabited (there is no empty resource literal), $\unit$ cannot be applied to anything. Furthermore, $\vdash \unit: \Unit~\kw{with} \varnothing$ by \textsc{$\varepsilon$-Abs}, and $\vdash \unit: \Unit$ by \textsc{T-Abs}. This leads to the derived rules in Figure \ref{fig:unit_rules}.

\begin{figure}[h]


\fbox{$\Gamma \vdash e: \tau$} 
\fbox{$\hat \Gamma \vdash \hat e: \hat \tau~\kw{with} \varepsilon$}


\[
\begin{array}{c}

\infer[\textsc{(T-Unit)}]
	{\Gamma \vdash \unit : \Unit}
	{} ~~~~

\infer[(\textsc{$\varepsilon$-Unit})]
	{\hat \Gamma \vdash \unit : \Unit~\kw{with} \varnothing}
	{}

\end{array}
\]

\caption{Derived $\kwa{Unit}$ rules.}
\label{fig:unit_rules}
\end{figure}

Since $\unit$ represents the absence of information, we use $\Unit$ when a function takes no input or returns no value. Figure \ref{fig:unit_sugaring} shows the definition of a Wyvern function which takes no argument and returns nothing, and its corresponding representation in $\epscalc$.

\begin{figure}[h]

\begin{lstlisting}
def method():Unit
   unit
\end{lstlisting}

\begin{lstlisting}
$\lambda$x:Unit. unit
\end{lstlisting}

\caption{Desugaring of functions which take no arguments or return nothing.}
\label{fig:unit_sugaring}
\end{figure}

~

\subsection{Let}

\noindent
The expression $\letxpr{x}{\hat e_1}{\hat e_2}$ reduces $\hat e_1$ to a value $\hat v_1$, binds it to the name $x$ in $\hat e_2$, and then executes $[\hat v_1/x]\hat e_2$. If $\hat \Gamma \vdash \hat e_1: \hat \tau_1~\kw{with} \varepsilon_1$, then $\letxpr{x}{\hat e_1}{\hat e_2} \defn (\lambda x: \hat \tau_1 . \hat e_2) \hat e_1$\footnote{You can also define an unannotated version of $\kwa{let}$, but we only need the annotated version}. If $\hat e_1$ is a non-value, we can reduce the $\kwa{let}$ by \textsc{E-App2}. If $\hat e_1$ is a value, we may apply \textsc{E-App3}, which binds $\hat e_1$ to $x$ in $\hat e_2$. $\kwa{let}$ expressions can be typed using \textsc{$\varepsilon$-App}. The new rules in \ref{fig:let_rules} capture these derivations.

\begin{figure}[h]

\fbox{$\Gamma \vdash e: \tau$}
\fbox{$\hat \Gamma \vdash \hat e: \hat \tau~\kw{with} \varepsilon$}
\fbox{$\hat e \rightarrow \hat e ~|~ \varepsilon$}

\[
\begin{array}{c}

	~~~
	
	\infer[\textsc{(T-Let)}]
	{\Gamma \vdash \letxpr{x}{e_1}{e_2}: \tau_2}
	{\Gamma \vdash e_1: \tau_1 & \Gamma, x: \tau_1 \vdash e_2: \tau_2}

\infer[\textsc{($\varepsilon$-Let)}]
	{\hat \Gamma \vdash \letxpr{x}{\hat e_1}{\hat e_2} : \hat \tau_2~\kw{with} \varepsilon_1 \cup \varepsilon_2}
	{\hat \Gamma \vdash \hat e_1 : \hat \tau_1~\kw{with} \varepsilon_1 & \hat \Gamma, x: \hat \tau_1 \vdash \hat e_2: \hat \tau_2~\kw{with} \varepsilon_2} \\[2ex]
	
\infer[\textsc{(E-Let1)}]
	{\letxpr{x}{\hat e_1}{\hat e_2} \longrightarrow \letxpr{x}{\hat e_1'}{\hat e_2}~|~\varepsilon_1}
	{\hat e_1 \longrightarrow \hat e_1'~|~\varepsilon_1} 
	
\infer[\textsc{(E-Let2)}]
	{\letxpr{x}{\hat v}{\hat e} \longrightarrow [\hat v/x]\hat e~|~\varnothing}
	{} 

\end{array}
\]

\caption{Derived $\kwa{let}$ rules.}
\label{fig:let_rules}
\end{figure}


\subsection{Modules and Objects}

Wyvern's modules are first-class, desugaring into objects --- invoking a module's function is no different from invoking an object's method. There are two kinds of modules: pure and resourceful. For our purposes, a pure module is one with no (transitive) authority over any resources, while a resource module has (transitive) authority over some resource. A pure module may still be given a capability, for example by requesting it in a function signature, but it may not possess or capture the capability for longer than the duration of the method call. \ref{fig:wyv_modules} shows an example of two modules, one pure and one resourceful, each declared in a separate file. Pure modules are declared with the $\kwa{module}$ keyword, while resource modules are declared with $\kwa{resource~module}$.

\begin{figure}[h]

\begin{lstlisting}
module PureMod

def tick(f: {File}):Unit with {File.append}
   f.append

\end{lstlisting}

\begin{lstlisting}
resource module ResourceMod
require File

def tick():Unit with {File.append}
   File.append
\end{lstlisting}

\caption{Definition of two modules, one pure and the other resourceful.}
\label{fig:wyv_modules}
\end{figure}

Resource modules, like objects, must be instantiated. When they are instantiated they must be given the capabilities they require. In Figure \ref{fig:wyv_modules}, $\kwa{ResourceMod}$ requests the use of a $\kwa{File}$ capability. Figure \ref{fig:wyv_module_instantiation} demonstrates how the two modules above would be instantiated and used. To prevent infinite regress the $\kwa{File}$ must, at some point, be introduced into the program. This happens in a special main module. When the program begins execution, the $\kwa{File}$ capability is passed into the program from the system environment. $\kwa{Main}$ then instantiates all the other modules in the program with their capabilities.

If a module is annotated, its function signatures will have effect annoations. For example, in Figure \ref{fig:wyv_modules}, $\kwa{PureMod.tick}$ has the $\kwa{File.append}$ annotation, meaning it should typecheck as $\kwa{ \{ File \} \rightarrow_{\{\kwa{File.append}\}} \Unit }$. 


\begin{figure}[h]

\begin{lstlisting}
resource module Main
require File
instantiate PureMod
instantiate ResourceMod(File)

PureMod.tick(File)
\end{lstlisting}

\caption{The $\kwa{Main}$ module which instantiates $\kwa{PureMod}$ and $\kwa{ResourceMod}$ and then invokes $\kwa{PureMod.tick}$.}
\label{fig:wyv_module_instantiation}
\end{figure}

Several simplifications make our desugaring possible. The only objects we use in the Wyvern examples are modules which only contain one function and the capabilities they require; they have no mutable fields. There are no self-referencing modules or recursive function definitions. Modules will not reference each other cyclically. This enables us to model each module as a function. Applying this function will be equivalent to applying the single function defined by the module. A collection of modules is desugared into $\epscalc$ as follows. First, a sequence of let-bindings are used to name constructor functions which, when given the capabilities requested by a module, will return an instance of the module. If the module does not require any capabilities it will take $\Unit$ as its argument. The constructor for $\kwa{M}$ is called $\kwa{MakeM}$. A function is then defined which represents the body of code in the $\kwa{Main}$ module. When invoked, this function will instantiate all the modules by invoking their constructors, and then execute the code in main. Finally, this function is invoked with the primitive capabilities passed into $\kwa{Main}$.

To demonstrate, Figure \ref{fig:wyv_tutorial_desugaring} shows how the examples above desugar. Lines 1-3 define the constructor for $\kwa{PureMod}$. Since $\kwa{PureMod}$ requires no capabilities, the constructor takes $\Unit$ as an argument on line 2. Lines 6-8 define the constructor for $\kwa{ResourceMod}$. It requires a $\kwa{File}$ capability, so the constructor takes $\kwa{\{File\}}$ as its input type on line 7. The constructor for $\kwa{Main}$ is defined on lines 11-16, which instantiates the other modules and then runs the code inside $\kwa{Main}$. Line 17 starts everything off by invoking $\kwa{Main}$ with the initial set of capabilities, which in this case is just $\kwa{File}$.

\begin{figure}[h]

\begin{lstlisting}
let MakePureMod =
   $\lambda$x:Unit.
      $\lambda$f:{File}. f.append
in

let MakeResourceMod =
   $\lambda$f:{File}.
      $\lambda$x:Unit. f.append
in

let MakeMain =
   $\lambda$f:{File}.
      $\lambda$x: Unit.
         let PureMod = (MakePureMod unit) in
         let ResourceMod = (MakeResourceMod f) in
         (ResourceMod unit)

(MakeMain File) unit
\end{lstlisting}

\caption{Desugaring of $\kwa{PureMod}$ and $\kwa{ResourceMod}$ into $\epscalc$.}
\label{fig:wyv_tutorial_desugaring}
\end{figure}

When an unannotated module is translated into $\epscalc$, the desugared contents will be encapsulated with an $\kwa{import}$ expression. The selected authority on the $\kwa{import}$ expression will be that we expect of the unannotated code according to the principle of least authority in the particular example under consideration. For example, if the client only expects the unannotated code to have the $\kwa{File.append}$ effect, its corresponding $\kwa{import}$ expression will select $\kwa{\{File.append\}}$.

\vspace{-0.5cm}
\section{Applications}
\vspace{-0.3cm}
\label{s:app}

In this section, we examine a number of scenarios to show how capabilities can help
developers reason about the effects and behaviour of code. In each story we will
discuss some Wyvern code before translating it to $\epscalc$ and explaining how its
rules apply. In doing this, we hope to convince the reader of the benefits of
capability-based reasoning, and that $\epscalc$ captures the intuitive properties of
capability-safe languages like Wyvern.

\vspace{-0.5cm}
\subsection{Unannotated Client}
\vspace{-0.2cm}

A \kwat{logger} module, when given \kwat{File}, exposes a \kwat{log} function
which incurs the effect \kwat{File.append}. The \kwat{client} module, possessing the
\kwat{logger} module, exposes an unannotated function \kwat{run}. While
\kwat{logger} has been annotated, \kwat{client} has not. If \kwat{client.run} is
executed, what effects might it have? Code for this example is given below. 

\begin{lstlisting}
module def logger(f: {File}):Logger

def log(): Unit with {File.append} =
    f.append(``message logged'')
\end{lstlisting}

\begin{lstlisting}
module def client(logger: Logger)

def run(): Unit =
   logger.log()
\end{lstlisting}

\begin{lstlisting}
require File

instantiate logger(File)
instantiate client(logger)

client.run()
\end{lstlisting}

A translation into $\epscalc$ is given below. Lines 1-3 and 5-8 define
\kwat{MakeLogger} and \kwat{MakeClient}, which instantiate the \kwat{logger} and
\kwat{client} modules respectively (represented as functions). Lines 10-14 define
\kwat{MakeMain}, which returns a function which, when executed, instantiates all
other modules and invokes the code in the body of \kwat{main}. Program execution
begins on line 16, where \kwat{main} is given the initial capabilities (just \kwat{File}
in this case).

\begin{lstlisting}
let MakeLogger =
   ($\lambda$f: File.
      $\lambda$x: Unit. f.append) in
          
let MakeClient =
   ($\lambda$logger: Unit $\rightarrow_{ \{ \kwa{File.append} \}}$ Unit.
      import(File.append) l = logger in
         $\lambda$x: Unit. l unit) in
          
let MakeMain =
   ($\lambda$f: File.
         let loggerModule = MakeLogger f in
         let clientModule = MakeClient loggerModule in
         clientModule unit) in

MakeMain File
\end{lstlisting}

The interesting part  is on line $7$, where the unannotated code selects $\{ \kwa{File.append} \}$ as its authority. This matches the effects of \kwat{logger}, i.e.
 $\kwa{effects}(\Unit \rightarrow_{\{\kwa{File.append}\}} \Unit) = \{
 \kwa{File.append} \}$. The unannotated code typechecks by \textsc{$\varepsilon$-Import}, approximating its effects as $\kwa{\{ \kwa{File.append} \}}$.

\vspace{-0.5cm}
\subsection{Unannotated Library}
\vspace{-0.2cm}

The next example inverts the roles of the last scenario. Now, the annotated 
\kwat{client} wants to use the unannotated \kwat{logger}, which captures 
\kwat{File} and exposes a single function \kwat{log}, which incurs the
 \kwat{File.append} effect. The implementation of \kwat{client.run} executes
 \kwat{logger.log}; it is annotated with $\varnothing$, so this violates its interface.

\begin{lstlisting}
module def logger(f: {File}): Logger

def log(): Unit =
    f.append(``message logged'')
\end{lstlisting}

\begin{lstlisting}
module def client(logger: Logger)

def run(): Unit with {File.append} =
   logger.log()
\end{lstlisting}

\begin{lstlisting}
require File

instantiate logger(File)
instantiate client(logger)

client.run()
\end{lstlisting}

The translation is given below. On lines 3-4, the unannotated code is wrapped in an $\kwa{import}$ expression selecting $\{ \kwa{File.append} \}$ as its authority. The implementation of $\kwa{logger}$ actually abides by this, but since it captures
\kwat{File} it could, in general, do anything to \kwat{File}; therefore,
\textsc{$\varepsilon$-Import} rejects this example. Formally, the imported capability
has the type \kwat{ \{ File \} }, but $\fx{\{ \File \}} = \{ \kwa{File}.* \}
\not\subseteq \{ \kwa{ File.append } \}$.

\begin{lstlisting}
let MakeLogger =
   ($\lambda$f: File.
      import(File.append) f = f in
         $\lambda$x: Unit. f.append) in

let MakeClient =
	($\lambda$logger: Logger.
	   $\lambda$x: Unit. logger unit) in

let MakeMain =
   ($\lambda$f: File.
      let loggerModule = MakeLogger f in
      let clientModule = MakeClient loggerModule in
      clientModule unit) in

MakeMain File
\end{lstlisting}

The only way for this to typecheck would be to annotate $\kwa{client.run}$ as having every effect on $\File$. This demonstrates how the effect-system of $\epscalc$ approximates unannotated code: it simply considers it as having every effect which could be incurred on those resources in scope, which here is $\kwa{File}.*$.

\vspace{-0.5cm}
\subsection{Higher-Order Effects}
\vspace{-0.2cm}

Here, $\kwa{Main}$ gains its functionality from a plugin. Plugins might be written by
third-parties, so we may not be able to view their source code, but still want to reason
about the authority they exercise. In this example, \kwat{plugin} has access to
\kwat{File}, but its interface does not permit it to perform any operations on
\kwat{File}. It tries to subvert this by wrapping \kwat{File} inside a function and
passing it to \kwat{malicious}, which invokes \kwat{File.read} in a higher-order
manner in an unannotated context.

\begin{lstlisting}
module malicious

def log(f: Unit $\rightarrow$ Unit): Unit
   f()
\end{lstlisting}

\begin{lstlisting}
module plugin
import malicious

def run(f: {File}): Unit with $\varnothing$
   malicious.log($\lambda$x:Unit. f.read)
\end{lstlisting}

\begin{lstlisting}
require File
import plugin

plugin.run(File)
\end{lstlisting}

This example shows how higher-order effects can obfuscate potential security risks. On line 3 of $\kwa{malicious}$, the argument to $\kwa{log}$ has type $\Unit \rightarrow \Unit$. The body of $\kwa{log}$ types with the \textsc{T-}rules, which do not approximate effects. It is not clear from inspecting the unannotated code that a $\kwa{File.read}$ will be incurred. To realise this requires one to examine the source code of both $\kwa{plugin}$ and $\kwa{malicious}$.

A translation is given below. On lines 2-3, the $\kwa{malicious}$ code selects its authority as $\varnothing$, to be consistent with the annotation on $\kwa{plugin.run}$. \textsc{$\varepsilon$-Import} safely rejects this: when the unannotated
 code is annotated with $\varnothing$, it has type $\{ \File \} \rightarrow_{\varnothing} \Unit$, but the higher-order effects of this type are
\kwat{ \{ File.* \} }, which are not contained in the selected authority $\varnothing$.

\begin{lstlisting}
let malicious =
   (import($\varnothing$) y=unit in
      $\lambda$f: Unit $\rightarrow$ Unit. f()) in

let plugin =
   ($\lambda$f: {File}.
      malicious($\lambda$x:Unit. f.read)) in

let MakeMain =
   ($\lambda$f: {File}.
      plugin f) in

MakeMain File
\end{lstlisting}

To get this example to typecheck, the program would have to be rewritten to explicitly
say that plugins can exercise arbitrary authority over \kwat{File}, by changing the
selected authority of \kwat{import} and the annotation on \kwat{plugin.run}.

\vspace{-0.5cm}
\subsection{Resource Leak}
\vspace{-0.2cm}

This is another example which obfuscates an unsafe effect by invoking it in a higher-order manner. The setup is the same, except the function which $\kwa{plugin}$ passes to $\kwa{malicious}$ now returns $\kwa{File}$ when invoked. $\kwa{malicious}$ uses this function to obtain $\kwa{File}$ and directly invokes $\kwa{read}$ upon it, violating the declared purity of $\kwa{plugin}$.

\begin{lstlisting}
module malicious

def log(f: Unit $\rightarrow$ File):Unit
   f().read
\end{lstlisting}

\begin{lstlisting}
module plugin
import malicious

def run(f: {File}): Unit with $\varnothing$
   malicious.log($\lambda$x:Unit. f)
\end{lstlisting}

\begin{lstlisting}
require File

import plugin

plugin.run(File)
\end{lstlisting}

The translation is given below. The unannotated code in $\kwa{malicious}$ is on lines
5-6. It has selected authority is $\varnothing$, to be consistent with the annotation on
$\kwa{plugin}$. Nothing is being imported, so the $\kwa{import}$ binds $\kwa{y}$ to
 $\unit$. This example is rejected by \textsc{$\varepsilon$-Import} because the
  premise $\varepsilon = \fx{\hat \tau} \cup \hofx{\annot{\tau}{\varepsilon}}$ is not satisfied. In this case, $\varepsilon = \varnothing$ and $\tau = \kwa{ (Unit \rightarrow
\{ File \}) \rightarrow Unit }$. Then $\annot{\tau}{\varepsilon} = \kwa{ (Unit
\rightarrow_{\varnothing} \{ File \}) \rightarrow_{\varnothing} Unit }$ and
$\hofx{\annot{\tau}{\varepsilon}} = \{ \kwa{File.*} \}$. Thus, the premise cannot
be satisfied and the example is safely rejected.


\begin{lstlisting}
let malicious =
   (import($\varnothing$) y=unit in
      $\lambda$f: Unit $\rightarrow$ {File}. f().read) in

let plugin =
   ($\lambda$f: {File}.
      malicious($\lambda$x:Unit. f)) in

let MakeMain =
   ($\lambda$f: {File}.
      plugin f) in

MakeMain File
\end{lstlisting}
\section{Conclusions}

We introduced $\opercalc$, a lambda calculus with primitive capabilities and their effects. $\opercalc$ programs are fully annotated with their effects. Relaxing this requirement, we obtained $\epscalc$, which allows unannotated code to be nested inside annotated code with a new $\kwa{import}$ construct. The capability-safe design of $\epscalc$ allows us to safely infer the effects of unannotated code by inspecting what capabilities are passed into it by its annotated surroundings. Such an approach allows code to be incrementally annotated, giving developers a balance between safety and convenience and alleviating the verbosity that has discouraged widespread adoption of effect systems \cite{rytz12}.

\subsection{Related Work}

Capabilities were introduced by Dennis and Van Horn \cite{dennis66} to control which processes in an operating system had permission to access certain parts of memory, perhaps via an \textit{access control list}. These early ideas are considerably different to the object capability model exemplified in the work of Mark Miller \cite{miller06}, which constrains how permissions may proliferate. Maffeis et. al. formalised the notion of a capability-safe language and showed that a subset of Caja (a Javascript implementation) is capability-safe \cite{maffeis10}. Other capability-safe languages include Wyvern \cite{nistor13} and Newspeak \cite{bracha10}. Miller's model has been applied to more heavyweight systems: Drossopoulou et. al. combined Hoare logic with capabilities to formalise the notion of trust \cite{drossopoulou07}.

The original effect system by Lucassen and Gifford was used to determine what expressions could safely execute in parallel \cite{lucassen88}. Subsequent applications include determining what functions a program might invoke \cite{tang94} and what regions in memory might be accessed or updated during execution \cite{talpin94}. In these systems, ``effects'' are performed upon ``regions''; in ours, ``operations'' are performed upon ``resources'''. $\epscalc$ also distinguishes between unannotated and annotated code: only the latter will type-and-effect-check. Another capability-based effect system is the one by Devriese et. al \cite{devriese16}, who use effect polymorphism and possible world semantics to express behavioural invariants on data structures. $\epscalc$ is not as expressive, since it only topographically analyses how capabilities can be passed around a program, but the resulting formalism and theory is much more lightweight.

\subsection{Future Work}

Our effects model only the use of capabilities which manipulate system resources. This definition could be generalised to track other sorts of effects, such as stateful updates. Resources and operations are fixed throughout runtime; it would be interesting to consider the theory when they can be created and destroyed at runtime.

The current theory contains no effect polymorphism, whereby a function's type is parameterised by a set of effects. The only way for such a function to typecheck in $\epscalc$ would be to approximate it as having every effect, in which case all precision has been lost. A polymorphic effect system which considers such a function as having an effect parameterised type could give more meaningful approximations.

Many believe in the value of the object capability model, but we do not fully understand its formal benefits. We hope to extend the ideas in this paper to the point where they might be used in capability-safe languages to help authority-safe design and development. Implementing these ideas in a general-purpose, capability-safe language would do much towards that end.

%Many believe in the real and practical value of the object capability model, but we do not fully understand its formal benefits. 

%%% Acknowledgments
%\begin{acks}                            %% acks environment is optional
%                                        %% contents suppressed with 'anonymous'
%  %% Commands \grantsponsor{<sponsorID>}{<name>}{<url>} and
%  %% \grantnum[<url>]{<sponsorID>}{<number>} should be used to
%  %% acknowledge financial support and will be used by metadata
%  %% extraction tools.
%  This material is based upon work supported by the
%  \grantsponsor{GS100000001}{National Science
%    Foundation}{http://dx.doi.org/10.13039/100000001} under Grant
%  No.~\grantnum{GS100000001}{nnnnnnn} and Grant
%  No.~\grantnum{GS100000001}{mmmmmmm}.  Any opinions, findings, and
%  conclusions or recommendations expressed in this material are those
%  of the author and do not necessarily reflect the views of the
%  National Science Foundation.
%\end{acks}

%% Bibliography
\bibliography{biblio}

%% Appendix
\appendix
% Uncomment to put the proofs at the end as an appendix.
% \documentclass{llncs}

\usepackage{listings}
\usepackage{amssymb}
\usepackage[margin=.9in]{geometry}
\usepackage{amsmath}
%\usepackage{amsthm}
\usepackage{mathpartir}
\usepackage{color,soul}
\usepackage{graphicx}
\usepackage[framemethod=tikz]{mdframed}

%\theoremstyle{definition}
%%\newtheorem{case1}{Case1}
\spnewtheorem{casethm}{Case}[theorem]{\itshape}{\rmfamily}
\spnewtheorem{subcase}{Subcase}{\itshape}{\rmfamily}
\numberwithin{subcase}{casethm}
\numberwithin{casethm}{theorem}
\numberwithin{casethm}{lemma}




\lstdefinestyle{custom_lang}{
  xleftmargin=\parindent,
  showstringspaces=false,
  basicstyle=\ttfamily,
  keywordstyle=\bfseries
}

\lstset{emph={%  
    val, def, type, new, z%
    },emphstyle={\bfseries \tt}%
}

\begin{document}
\hl{*Note highlighted text implies more work is needed.}
\section{Type Safety}

\subsection{Subtype Transitivity and Environment Narrowing}

%---------------------- Narrowing ----------------------%



Before we can prove \emph{Preservation}, we need to 
prove \emph{Subtype Transitivity}. \emph{Transitivity} 
is mutually dependent on \emph{Environment Narrowing}.
\begin{mathpar}
\inferrule
  {\Gamma, (x : U); \Sigma \vdash T <: T' \\
  	\Gamma; \Sigma \vdash S <: U}
  {\Gamma, (x : S); \Sigma \vdash T <: T'}
\end{mathpar}
Instead of proving them at the same time, we weaken the 
\emph{Environment Narrowing} proof by admitting transitivity.
This is the the same tactic taken by Amin et. al. \cite{Amin:2014}.
To do this we define the following \emph{Subtype Transitivity} 
judgement 
\begin{mathpar}
\inferrule
  {}
  {\Gamma \vdash S \; <:^* \; S}
  \quad (\textsc {S\textsuperscript{*}-Refl})
	\and
\inferrule
  {\Gamma \vdash S \; <:^* \; T \\
	\Gamma \vdash T \; <: \; U}
  {\Gamma \vdash S \; <:^* \; U}
  \quad (\textsc {S\textsuperscript{*}-Trans})
\end{mathpar}
\emph{Environment Narrowing} can now be rewritten as 
\begin{mathpar}
\inferrule
  {\Gamma, (x : U); \Sigma \vdash T <: T' \\
  	\Gamma; \Sigma \vdash S <: U}
  {\Gamma, (x : S); \Sigma \vdash T <:^* T'}
\end{mathpar}

We also make a change to the subsumption typing rule 
\textsc{T-Sub} to also admit transitivity. This is done to 
assist in the induction case for type parametrised expressions 
during the \emph{Narrowing} proof. Our new rule, \textsc{T-Sub*} 
replaces the \textsc{T-Sub} rule in the expression typing 
judgement. 
\begin{mathpar}
\inferrule
  {	\Gamma; \Sigma \vdash e : S \\
  	\Gamma; \Sigma \vdash S <:^* T}
  {	\Gamma; \Sigma \vdash e : T}
  \quad (\textsc {T-Sub*})
\end{mathpar}
With these two additions, we are able to prove the relaxed 
\emph{Narrowing} proof in Theorem \ref{thm:narrowing}. We then go 
on to prove \emph{Subtype Transitivity} in Theorem \ref{thm:trans}. 
Once we have \emph{Subtype Transitivity} it follows that 
the $<:^*$ judgement is equivalent to the subtyping judgement, 
and \textsc{T-Sub*} is equivalent to \textsc{T-Sub}.



Induction principles need to be constructed for each of our 
judgements. We can do this by treating judgements as either 
base cases or complex cases derived from simpler judgements. 
As an example, the induction principle for the subtyping 
judgement can be described below.
For any theorem $\mathcal{H}_{<:}$ on the subtyping judgement, if 
the base cases
\begin{mathpar}
\inferrule
  {}
  {\mathcal{H}_{<:}(\textsc{S-Refl}) \\
	\mathcal{H}_{<:}(\textsc{S-Top}) \\
	\mathcal{H}_{<:}(\textsc{S-Bottom})}
\end{mathpar}
hold, and if given $\mathcal{H}_{<:}$ holds for any simpler 
derivation of \textsc{S-Rec}, \textsc{S-Select-Upper} and
\textsc{S-Select-Lower} we can show that 
\begin{mathpar}
\inferrule
  {}
  {\mathcal{H}_{<:}(\textsc{S-Rec}) \\
	\mathcal{H}_{<:}(\textsc{S-Select-Upper}) \\
	\mathcal{H}_{<:}(\textsc{S-Select-Lower})}
\end{mathpar}
hold, then it follows that $\mathcal{H}_{<:}$ holds for all 
derived instances of the subtyping judgement. Similar inductive
schemes can be constructed for the \emph{Expansion}, \emph{Membership}, 
\emph{Typing} and \emph{Reduction} judgements.

For the following lemmas, we define 
$\Gamma; \Sigma \vdash \overline{\sigma} <:^* \overline{\sigma}'$
as $\forall \sigma_i' \in \overline{\sigma}', \exists 
\sigma_i \in \overline{\sigma}: 
\Gamma; \Sigma \vdash \sigma_i <:^* \sigma_i'$.

\newpage

\begin{lemma}\label{lem:subtype:decl} 
If 	$\Gamma; \Sigma \vdash S <: U, S \prec_z \overline{\sigma}, 
	U \prec_z \overline{\sigma}'$ then
	$\Gamma, (z:S); \Sigma \vdash \overline{\sigma} <:^* \overline{\sigma}'$.
\end{lemma}
\begin{proof}
By induction on the derivation of $\Gamma; \Sigma \vdash S <: U$.
\begin{casethm}[\textsc{S-Refl}]
\begin{mathpar}
\inferrule
  {S = T}
  {}
\end{mathpar}
Trivial.
\end{casethm}
\begin{casethm}[\textsc{S-Rec}]
\begin{mathpar}
\inferrule
  {S = \{z \Rightarrow \overline{\sigma}\} \\
  	T = \{z \Rightarrow \overline{\sigma}'\}}
  {}
  \and
\inferrule
  {\forall \sigma_i' \in \overline{\sigma}', \; \exists \; \sigma_i \in \overline{\sigma} \; st \; \Gamma, z : \{z \Rightarrow \overline{\sigma}\}; \Sigma \vdash \sigma_i <:\; \sigma_i'}
  {\Gamma; \Sigma \vdash \{z \Rightarrow \overline{\sigma}\}\; <:\; \{z \Rightarrow \overline{\sigma}'\}}
\end{mathpar}
Follows immediately from our definition.
\end{casethm}
\begin{casethm}[\textsc{S-Select-Upper}]
\begin{mathpar}
\inferrule
  {S = p.L}
  {}
  \and
\inferrule
  {\Gamma; \Sigma \vdash p \ni \texttt{type} \; L : S' .. U'\\
  	\Gamma; \Sigma \vdash S' <: U' \\
  	\Gamma; \Sigma \vdash U' <: T}
  {\Gamma; \Sigma \vdash p.L\; <:\; T}
\end{mathpar}
\end{casethm}
\begin{casethm}[\textsc{S-Select-Lower}]
\begin{mathpar}
\inferrule
  {T = p.L}
  {}
  \and
\inferrule
  {\Gamma; \Sigma \vdash p \ni \texttt{type} \; L : S' .. U' \\
  	\Gamma; \Sigma \vdash S' <: U' \\
  	\Gamma; \Sigma \vdash S <: S'}
  {\Gamma; \Sigma \vdash S \; <:\; p.L}
\end{mathpar}
\end{casethm}
\begin{casethm}[\textsc{S-Top}]
\begin{mathpar}
\inferrule
  {T = \top}
  {}
  \and
\inferrule
  {}
  {\Gamma; \Sigma \vdash T\; \texttt{<:}\; \top}
\end{mathpar}
\end{casethm}
\begin{casethm}[\textsc{S-Bottom}]
\begin{mathpar}
\inferrule
  {S = \bot}
  {}
\end{mathpar}
\end{casethm}
Since there is no expansion to $\bot$, we end up with a contradiction.
\end{proof}
\qed

%\begin{lemma} \label{lem:subst_type}
%\begin{mathpar}
%\inferrule
%  {\Gamma; \Sigma \vdash p : \Gamma(x)}
%  {	\Gamma; \Sigma \vdash [p/x]T <: T}
%\end{mathpar}
%\end{lemma}
%\begin{proof}
%By induction on $T$.
%\begin{casethm}[T-Rec]
%\begin{mathpar}
%\inferrule
%  {T = \{z \Rightarrow \overline{\sigma}\}}
%  {}
%\end{mathpar}
%By our induction hypothesis we assume that 
%$\Gamma$.
%\end{casethm}
%\begin{casethm}[T-Sel]
%\begin{mathpar}
%\inferrule
%  {T = p.L}
%  {}
%\end{mathpar}
%\end{casethm}
%\begin{casethm}[T-Top]
%\begin{mathpar}
%\inferrule
%  {T = \top}
%  {}
%\end{mathpar}
%\end{casethm}
%\begin{casethm}[T-Bottom]
%\begin{mathpar}
%\inferrule
%  {T = \bot}
%  {}
%\end{mathpar}
%\end{casethm}
%\end{proof}
%\qed

\newpage

We restrict expression substition to path substitution as 
the substitution of general expressions can lead to a loss 
of soundness.
\begin{lemma} \label{lem:subst}
\begin{mathpar}
\inferrule
  {\Gamma, (x : U); \Sigma \vdash e : T \\
  	\Gamma; \Sigma \vdash p : S \\
  	\Gamma; \Sigma \vdash S <: U}
  {[p \unlhd U/x]\Gamma; \Sigma \vdash [p \unlhd U/x]e : [p \unlhd U/x]T}
\end{mathpar}
\end{lemma}
\begin{proof}
By induction on the derivation of $\Gamma; \Sigma \vdash e : T$.
\begin{casethm}[T-Var]
\begin{mathpar}
\inferrule
  {e = y}
  {}
  \and
\inferrule
  {y \in dom(\Gamma, (x : U))}
  {	\Gamma, (x : U); \Sigma \vdash y : \Gamma(y)}
\end{mathpar}
A base case for our induction, we approach \textsc{T-Var} by case analysis
on the equality of $x$ and $y$.
\begin{subcase}[$y = x$]
If $y = x$, then $	\Gamma, (x : U); \Sigma \vdash y : U$,
$[p \unlhd U/x]y = p \unlhd U$ and $T = U$. $[p \unlhd U/x]U = U$ 
because the type of a variable in a well-formed environment 
cannot contain the variable itself. Thus by \textsc{T-Type} 
we have $\Gamma; \Sigma \vdash p \unlhd U : U$.
\end{subcase}
\begin{subcase}[$y \neq x$]
If $y \neq x$, then clearly $y \in dom(\Gamma)$ and by 
\textsc{T-Var}, $\Gamma; \Sigma \vdash y : \Gamma(y)$.
\end{subcase}
\end{casethm}

\begin{casethm}[T-Loc]
\begin{mathpar}
\inferrule
  {	l \in dom(\Sigma)}
  {	\Gamma; \Sigma \vdash l : \Sigma(l)}
\end{mathpar}
\textsc{T-Loc} is another but base case, and resolves trivially since 
$[p \unlhd U/x]l = l$ and $[p \unlhd U/x]\Sigma(l) = \Sigma(l)$.
\end{casethm}

\begin{casethm}[T-New]
\begin{mathpar}
\inferrule
  {\Gamma, z : \{z \Rightarrow \overline{\sigma}\}; \Sigma 
  \vdash \overline{d} : \overline{\sigma}}
  {	\Gamma; \Sigma\vdash \texttt{new} \; \{z \Rightarrow \overline{d}\} : 
  \{z \Rightarrow \overline{\sigma}\}}
\end{mathpar}
Our induction hypothesis assumes 
$\Gamma, z : \{z \Rightarrow \overline{\sigma}\}; \Sigma \vdash [p \unlhd U/x]\overline{d} : [p \unlhd U/x]\overline{\sigma}$
holds. Since 
$[p \unlhd U/x]\texttt{new} \; \{z \Rightarrow \overline{d}\} = 
\texttt{new} \; \{z \Rightarrow [p \unlhd U/x]\overline{d}\}$ it follows 
from \textsc{T-New} that 
$\Gamma; \Sigma\vdash [p \unlhd U/x]\texttt{new} \; \{z \Rightarrow \overline{d}\} : 
  [p \unlhd U/x]\{z \Rightarrow \overline{\sigma}\}$.
\end{casethm}

\begin{casethm}[T-Meth]
\begin{mathpar}
\inferrule
  {\Gamma; \Sigma \vdash e_0 \ni \texttt{def} \; m:S \rightarrow T \\
  	\Gamma; \Sigma \vdash e_1 : S \\
  	\Gamma; \Sigma \vdash T <: U}
  {	\Gamma; \Sigma \vdash e_0.m_U(e_1) : U}
\end{mathpar}
We assume by our induction hypothesis that 
\begin{mathpar}
\inferrule
  {\Gamma; \Sigma \vdash [p \unlhd U/x]e_0 \ni [p \unlhd U/x]\texttt{def} \; m:S' \rightarrow T' \\
  	\Gamma; \Sigma \vdash S <: [p \unlhd U/x]S', [p \unlhd U/x]T' <: T \\
  	\Gamma; \Sigma \vdash [p \unlhd U/x]e_1 : [p \unlhd U/x]S'}
  {}
\end{mathpar}
Thus by \emph{Subtype Transitivity} 
we get $\Gamma; \Sigma \vdash T' <: U$, and it follows by \textsc{T-Meth} 
that $	\Gamma; \Sigma \vdash [e'/x](e_0.m_U(e_1)) : U$.

\hl{TODO: include mutual induction scheme}
\end{casethm}

\begin{casethm}[T-Acc]
\begin{mathpar}
\inferrule
  {%	\Gamma; \Sigma \vdash e : S \\
  	\Gamma; \Sigma \vdash e \ni \texttt{val} \; f:T}
  {	\Gamma; \Sigma \vdash e.f : T}
\end{mathpar}
We assume by our mutual induction hypothesis that
\begin{mathpar}
\inferrule
  {\Gamma; \Sigma \vdash [e'/x]e \ni \texttt{val} \; f:T' \\
  	\Gamma; \Sigma \vdash T' <: T}
  {}
\end{mathpar}
Therefore, by \textsc{T-Acc} we have 
$\Gamma; \Sigma \vdash e.f : T'$, and by \textsc{T-Sub} 
we get $\Gamma; \Sigma \vdash e.f : T$.
\end{casethm}

\begin{casethm}[T-Type]
\begin{mathpar}
\inferrule
  {e = e_T \unlhd T}
  {}
  \and
\inferrule
  {	\Gamma; \Sigma \vdash e : T}
  {	\Gamma; \Sigma \vdash e \unlhd T : T}
\end{mathpar}
By our mutual induction hypothesis we assume that
\begin{mathpar}
\inferrule
  {\Gamma; \Sigma \vdash [e'/x]e : T}
  {}
\end{mathpar}
\textsc{T-Type} gives us $\Gamma; \Sigma \vdash [e'/x]e \unlhd T : T$.
\end{casethm}

\begin{casethm}[T-Sub]
\begin{mathpar}
\inferrule
  {	\Gamma; \Sigma \vdash e : S \\
  	\Gamma; \Sigma \vdash S <: T}
  {	\Gamma; \Sigma \vdash e : T}
\end{mathpar}
By our mutual induction hypothesis we assume that
\begin{mathpar}
\inferrule
  {\Gamma; \Sigma \vdash [e'/x]e : S}
  {}
\end{mathpar}
Then, by \textsc{T-Sub} we get $\Gamma; \Sigma \vdash [e'/x]e : T$.
\end{casethm}
\end{proof}
\qed

\newpage

\begin{lemma} \label{lem:subst}
If $\Gamma; \Sigma \vdash p : \Gamma(x)$ then 
$\Gamma; \Sigma \vdash [p/x]T <: T$.
\end{lemma}
\begin{proof}
\end{proof}
\qed

\newpage

There are four judgments mutually defined in our type system. 
Subtyping, Membership, Expansion and Expression Typing are all 
interdependent. Subtyping is dependent on Membership, which is 
dependent on both Expansion and Typing. Expansion is dependent on 
Membership and Typing is dependent on Subtyping. This means that 
when performing induction on the derivation of one, we need to be 
able extend the induction hypothesis to the others. Therefore, when 
proving environmental narrowing, we need to prove the following 
theorems mutually.
\begin{lemma}[Environment Narrowing*]\label{thm:narrowing}
\begin{mathpar}
\inferrule
  {\Gamma, (x : U); \Sigma \vdash T <: T' \\
  	\Gamma; \Sigma \vdash S <: U}
  {\Gamma, (x : S); \Sigma \vdash T <:^* T'}
  \quad (\textsc {<:-Narrowing*})
	\and
\inferrule
  {\Gamma, (x : U); \Sigma \vdash T <:^* T' \\
  	\Gamma; \Sigma \vdash S <: U}
  {\Gamma, (x : S); \Sigma \vdash T <:^* T'}
  \quad (\textsc {<:\textsuperscript{*}-Narrowing*})
	\and
\inferrule
  {\Gamma, (x : U); \Sigma \vdash T \prec \overline{\sigma} \\
  	\Gamma; \Sigma \vdash S <: U}
  {\exists \overline{\sigma}':
  	\Gamma, (x : S); \Sigma \vdash T \prec \overline{\sigma}' \\
  	\Gamma, (x : S); \Sigma \vdash \overline{\sigma}' <:^* \overline{\sigma}}
  \quad (\textsc {$\prec$-Narrowing*})
	\and
\inferrule
  {\Gamma, (x : U); \Sigma \vdash p \ni \sigma \\
  	\Gamma; \Sigma \vdash S <: U}
  {\exists \sigma':
  	\Gamma, (x : S); \Sigma \vdash p \ni \sigma' \\
  	\Gamma, (x : S); \Sigma \vdash \sigma' <:^* \sigma}
  \quad (\textsc {$\ni$-Narrowing*})
	\and
\inferrule
  {\Gamma, (x : U); \Sigma \vdash p : T \\
  	\Gamma; \Sigma \vdash S <: U}
  {\exists T':
  	\Gamma, (x : S); \Sigma \vdash p : T' \\
  	\Gamma, (x : S); \Sigma \vdash T' <:^* T}
  \quad (\textsc {:-Narrowing*})
\end{mathpar}
\end{lemma}
\begin{proof}
We proceed by mutual structural induction on the of each of the judgements:
\begin{mathpar}
\inferrule
  {\Gamma, (x : U); \Sigma \vdash T <: T' \\
  	\Gamma, (x : U); \Sigma \vdash T \prec \overline{\sigma} \\
  	\Gamma, (x : U); \Sigma \vdash p \ni \sigma \\
  	\Gamma, (x : U); \Sigma \vdash p : T}
  {}
\end{mathpar}
That is, if for each judgment the base cases (those cases that 
do not require the derivation of simpler judgments) are satisfied, 
and assuming the result for all simpler sub-derivations the we can 
show it holds for more complex cases, we can conclude it holds in 
general. While not explicitly stated above, we also include the 
relevant declaration forms of the above judgments.

\subsubsection*{\textsc {$<:$-Narrowing*}:}

\begin{casethm}[\textsc{S-Refl}]
\begin{mathpar}
\inferrule
  {}
  {T' = T \\
  	\Gamma, (x : U); \Sigma \vdash T\; \texttt{<:}\; T}
\end{mathpar}
In the reflexive base case \textsc{S-Refl}, $T' = T$. 
It follows trivially from \textsc{S-Refl} that 
$\Gamma, (x : S); \Sigma \vdash T\; \texttt{<:}\; T$.
\end{casethm}
\begin{casethm}[\textsc{S-Rec}]
\begin{mathpar}
\inferrule
  {}
  {T = \{z \Rightarrow \overline{\sigma}\} \\
  	T' = \{z \Rightarrow \overline{\sigma}'\}}
  	\and
\inferrule
  {\forall \sigma_i' \in \overline{\sigma}', \; \exists \; \sigma_i \in \overline{\sigma} \; st \; \Gamma, z : \{z \Rightarrow \overline{\sigma}\}, (x : U); \Sigma \vdash \sigma_i <:\; \sigma_i'}
  {\Gamma, (x : U); \Sigma \vdash \{z \Rightarrow \overline{\sigma}\}\; <:\; \{z \Rightarrow \overline{\sigma}'\}}
\end{mathpar}
From our induction hypothesis we need to show that if our result holds for each 
smaller subtype derivation, it holds for the larger derivation. Thus, if it
holds for each individual 
declaration type, it holds for the larger record type. 
That is, for each $\sigma'_i$ and $\sigma_i$ such 
that 
$\Gamma, z : \{z \Rightarrow \overline{\sigma}\}, (x : U); \Sigma 
\vdash \sigma_i <:\; \sigma_i'$,
the induction hypothesis for subtyping gives us 
$\Gamma, z : \{z \Rightarrow \overline{\sigma}\}, (x : S); \Sigma 
\vdash \sigma_i <:^*\; \sigma_i'$. From here is is simple to construct 
a chain of record types such that 
$\Gamma, (x : S); \Sigma \vdash \{z \Rightarrow \overline{\sigma}\} <:^*\; 
\{z \Rightarrow \overline{\sigma}'\}$.
\end{casethm}
\begin{casethm}[\textsc{S-Select-upper}]
\begin{mathpar}
\inferrule
  {}
  {T = p.L}
  	\and
\inferrule
  {\Gamma, (x : U); \Sigma \vdash p \ni \texttt{type} \; L : S' .. U'\\
%  	\Gamma, (x : U); \Sigma \vdash S' <: U' \\
  	\Gamma, (x : U); \Sigma \vdash U' <: T'}
  {\Gamma, (x : U); \Sigma \vdash x.L\; <:\; T'}
\end{mathpar}
Firstly, we need to use our mutually defined induction hypothesis for 
\textsc {$\ni$-Narrowing*} to derive 
$\exists \sigma':
\Gamma, (x : S); \Sigma \vdash p \ni \sigma'$ and
$\Gamma, (x : S); \Sigma \vdash \sigma' <:^* L : S' .. U'$.
i.e. $\exists S'' \, U'' : \Gamma, (x : S); \Sigma \vdash p \ni L : S'' .. U'', \;
U'' <:^* U'$
Now using the induction hypothesis for subtyping the Narrowing result holds for 
sub-derivations of the subtype relation. In this case the sub-derivation is 
$\Gamma, (x : U); \Sigma \vdash U' <: T'$, and it follows that 
$\Gamma, (x : S); \Sigma \vdash U' <:^* T'$ holds. 
We can now combine the two subtype chains to show that 
$\Gamma, (x : S); \Sigma \vdash U'' <:^* U', \; U' <:^* T' \Rightarrow  U'' <:^* T'$. 
It then follows that there exists some type $T''$ such that 
$\Gamma, (x : S); \Sigma \vdash p.L <: T'', \; T'' <:^* T'$ 
which gives us $\Gamma, (x : S); \Sigma \vdash p.L <:^* T'$.

\hl{ToDo: Show that $\Gamma \vdash S <: T, T <:^* U \Rightarrow S <:^* U$}
\end{casethm}
\begin{casethm}[\textsc{S-Select-Lower}]
\begin{mathpar}
\inferrule
  {}
  {T' = p.L}
  	\and
\inferrule
  {\Gamma, (x : U); \Sigma \vdash p \ni \texttt{type} \; L : S' .. U' \\
%  	\Gamma, (x : U); \Sigma \vdash S' <: U' \\
  	\Gamma, (x : U); \Sigma \vdash T <: S'}
  {\Gamma, (x : U); \Sigma \vdash T\; <:\; p.L}
\end{mathpar}
As in the previous case, we can use the \textsc {$\ni$-Narrowing*}
induction hypothesis to show that 
$\exists S'' \, U'' : \Gamma, (x : S); \Sigma \vdash p \ni L : S'' .. U'', \;
S' <:^* S''$.
We can also apply the \textsc {<:-Narrowing*} induction hypothesis 
to show that $\Gamma, (x : S); \Sigma \vdash T <:^* S'$, 
and subsequently that 
$\Gamma, (x : S); \Sigma \vdash T <:^* S''$. It  should then be easy 
to construct a subtyping chain such that 
$\Gamma, (x : S); \Sigma \vdash T <:^* p.L$.

\end{casethm}
\begin{casethm}[\textsc{S-Top}]
\begin{mathpar}
  {}
  {T' = \top \\ 
  	\Gamma, (x : U); \Sigma \vdash T\; \texttt{<:}\; \top}
\end{mathpar}
The base case dealing with subtyping of the top type is trivial 
since by \textsc{S-Top} 
$\Gamma, (x : S); \Sigma \vdash T\; \texttt{<:}\; \top$.
\end{casethm}
\begin{casethm}[\textsc{S-Bottom}]
\begin{mathpar}
\inferrule
  {}
  {T = \bot \\
  	\Gamma, (x : U); \Sigma \vdash \bot\; \texttt{<:}\; T'}
\end{mathpar}
The base case \textsc{S-Bottom} resolves trivially since 
by \textsc{S-Bottom} $\Gamma, (x : S); \Sigma \vdash \bot\; \texttt{<:}\; T'$
\end{casethm}
\begin{casethm}[\textsc{S-Decl-Val}]
\begin{mathpar}
\inferrule
  {}
  {\sigma = \texttt{val} \; f:T \\
  	\sigma' = \texttt{val} \; f:T}
  	\and
\inferrule
  {}
  {\Gamma, (x : U); \Sigma \vdash \texttt{val} \; f:T <: \texttt{val} \; f:T}
\end{mathpar}
The base case for declaration subtyping is resolves trivially by 
\textsc{S-Decl-Val}: 
$\Gamma, (x : S); \Sigma \vdash \texttt{val} \; f:T <: \texttt{val} \; f:T$
\end{casethm}
\begin{casethm}[\textsc{S-Decl-Def}]
\begin{mathpar}
\inferrule
  {}
  {\sigma = \texttt{def} \; m:S' \rightarrow T \\
  	\sigma' = \texttt{def} \; m:S'' \rightarrow T'}
  	\and
\inferrule
  {\Gamma, (x : U); \Sigma \vdash T <: T' \\
  	\Gamma, (x : U); \Sigma \vdash S'' <: S'}
  {\Gamma, (x : U); \Sigma \vdash \texttt{def} \; m:S' \rightarrow T <: \texttt{def} \; m:S'' \rightarrow T'}
\end{mathpar}
Using the \textsc {<:-Narrowing*} induction hypothesis 
for argument and return type subtyping we 
get $\Gamma, (x : S); \Sigma \vdash T <:^* T'$ and
$\Gamma, (x : S); \Sigma \vdash S'' <:^* S'$. We can then 
create a chain of declaration subtype judgments to show that 
$\Gamma, (x : S); \Sigma \vdash \texttt{def} \; m:S' \rightarrow T <:^* \texttt{def} \; m:S'' \rightarrow T'$.
\end{casethm}

\begin{casethm}[\textsc{S-Decl-Type}]
\begin{mathpar}
\inferrule
  {}
  {\sigma = \texttt{type} \; L : S' .. U' \\
  	\sigma' = \texttt{type} \; L : S'' .. U''}
  	\and
\inferrule
  {\Gamma, (x : U); \Sigma \vdash S'' <: S' \\
  	\Gamma, (x : U); \Sigma \vdash U' <: U''}
  {\Gamma, (x : U); \Sigma \vdash \texttt{type} \; L : S' .. U' \; <:\; \texttt{type} \; L : S'' .. U''}
\end{mathpar}
Similar to the \textsc{S-Decl-Def} case, by the induction hypothesis, 
if $\Gamma, (x : S); \Sigma \vdash S'' <:^* S'$ and
$\Gamma, (x : S); \Sigma \vdash U' <:^* U''$ hold, it can be shown that 
$\Gamma, (x : S); \Sigma \vdash \texttt{type} \; L : S' .. U' \; <:^* \; \texttt{type} \; L : S'' .. U''$ holds too.

\end{casethm}

\subsubsection*{\textsc {$\prec$-Narrowing*}:}

\begin{casethm}[\textsc{E-Rec}]
\begin{mathpar}
\inferrule
  {}
  {T = \{z \Rightarrow \overline{\sigma}\}}
  	\and
\inferrule
  {}
  {\Gamma, (x : U); \Sigma \vdash 
  		\{z \Rightarrow \overline{\sigma}\} \prec_z \overline{\sigma}}
\end{mathpar}
The base case for expansion follows immediately from \textsc{E-Rec}:
$\Gamma, (x : S); \Sigma \vdash \{z \Rightarrow \overline{\sigma}\} \prec_z \overline{\sigma}$.
\end{casethm}
\begin{casethm}[\textsc{E-Select}]
\begin{mathpar}
\inferrule
  {}
  {T = p.L}
  	\and
\inferrule
  {\Gamma, (x : U); \Sigma \vdash p \ni \texttt{type} \; L : S'..U' \\
  	\Gamma, (x : U); \Sigma \vdash U' \prec_z \overline{\sigma}}
  {\Gamma, (x : U); \Sigma \vdash p.L \prec_z \overline{\sigma}}
\end{mathpar}
By our mutual induction hypothesis we assume that
\begin{mathpar}
\inferrule
  {\Gamma, (x : S); \Sigma \vdash p \ni \texttt{type} \; L : S''..U'', \;
  	\texttt{type} \; L : S''..U'' <:^* \texttt{type} \; L : S'..U', \;
  	U' \prec_z \overline{\sigma}', \;
  	\overline{\sigma}' <:^* \overline{\sigma}}
  {}
\end{mathpar}
holds. By \textsc{S-Decl-Type}, we can infer that 
$\Gamma, (x : S); \Sigma \vdash U'' <:^* U'$. Using 
Lemma \ref{lem:subtype:decl} (\hl{TODO}), we can show that 
$\exists \overline{\sigma}'': \Gamma, (x : S); \Sigma \vdash U'' \prec 
\overline{\sigma}'', \; \overline{\sigma}'' <:^* \overline{\sigma}'$.
Thus, we can show that 
$\Gamma, (x : S); \Sigma \vdash \overline{\sigma}'' <:^* \overline{\sigma}$ 
which completes the case.

\hl{ToDo: U = $\bot$?}
\end{casethm}
\begin{casethm}[\textsc{E-Top}]
\begin{mathpar}
\inferrule
  {}
  {T = \top}
  	\and
\inferrule
  {}
  {\Gamma, (x : U); \Sigma \vdash \top \prec_z \varnothing}
\end{mathpar}
The \textsc{E-Top} base case resolves trivially. By 
\textsc{E-Top}, $\Gamma, (x : S); \Sigma \vdash \top \prec_z \varnothing$.
\end{casethm}

\subsubsection*{\textsc {$\ni$-Narrowing*}:}

\begin{casethm}[\textsc{M-Path}]
\begin{mathpar}
\inferrule
  {}
  {e = p \\
  	\sigma = [p/z]\sigma_i}
  	\and
\inferrule
  {\Gamma, (x : U); \Sigma \vdash p : T \\
  	\Gamma, (x : U); \Sigma \vdash T \prec_z \overline{\sigma}\\
  	\sigma_i \in \overline{\sigma}}
  {\Gamma, (x : U); \Sigma \vdash p \ni [p/z]\sigma_i}
\end{mathpar}
From the mutually defined induction hypothesis, we assume 
\begin{mathpar}
\inferrule
  {\Gamma, (x : S); \Sigma \vdash p : T', \; T' <:^* T \\
  	\Gamma, (x : S); \Sigma \vdash T \prec_z \overline{\sigma}', \;
  	\overline{\sigma}' <:^* \overline{\sigma}}
  {}
\end{mathpar}
By Lemma \ref{lem:subtype:decl} (\hl{TODO}), $\Gamma, (x : S); \Sigma \vdash T' \prec \overline{\sigma}''$ 
and $\Gamma, (x : S); \Sigma \vdash \overline{\sigma}'' <:^* \overline{\sigma}'$. 
We can now show that $\exists \sigma_i'' \in \overline{\sigma}'':
\Gamma, (x : S); \Sigma \vdash \sigma_i'' <:^* \sigma_i$, and thus that 
$\Gamma, (x : S); \Sigma \vdash [p/z]\sigma_i'' <:^* [p/z]\sigma_i$.

\hl{ToDo: Show T' expanding \\ Substitution preserves subtype?}
\end{casethm}
\begin{casethm}[\textsc{M-Exp}]
\begin{mathpar}
\inferrule
  {\Gamma, (x : U); \Sigma \vdash p : T \\
  	\Gamma, (x : U); \Sigma \vdash T \prec_z \overline{\sigma}\\
  	\sigma \in \overline{\sigma} \\
  	z \notin \sigma}
  {\Gamma, (x : U); \Sigma \vdash p \ni \sigma}
\end{mathpar}
We explicitly restrict the statement of the theorem to only include 
membership of paths, as a result, this cases resolves as in \textsc{M-Path}.
\end{casethm}

\subsubsection*{\textsc {:-Narrowing*}:}
We restrict Narrowing on expression typing to paths. This is done since for 
a generic expression narrowing does not hold. For this reason the below cases only deal 
with path expressions.
\begin{casethm}[\textsc{T-Var}]
\begin{mathpar}
\inferrule
  {}
  {p = y \\
  	T = \Gamma, (x : U)(y)}
  	\and
\inferrule
  {y \in dom(\Gamma, (x : U))}
  {	\Gamma, (y : U); \Sigma \vdash y : \Gamma(y)}
\end{mathpar}
\begin{itemize}
\item[]  \textit{Subcase 1} ($y = x$):
If $y = x$, then the result is immediate since by \textsc{T-Var} 
$\Gamma,(x:S); \Sigma \vdash x : S$, and by assumption 
$\Gamma; \Sigma \vdash S <: U$.
\item[]  \textit{Subcase 2} ($y \neq x$):
If $y \neq x$, then the type of $y$ remains the same and the result 
is achieved by reflexive.
\end{itemize}
\end{casethm}
\begin{casethm}[\textsc{T-Loc}]
\begin{mathpar}
\inferrule
  {}
  {p = l}
  	\and
\inferrule
  {	l \in dom(\Sigma)}
  {	\Gamma, (x : U); \Sigma \vdash l : \Sigma(l)}
\end{mathpar}
$\Gamma, (x : S); \Sigma \vdash l : \Sigma(l)$ follows immediately from \textsc{T-Loc}.
\end{casethm}
\begin{casethm}[\textsc{T-Acc}]
\begin{mathpar}
\inferrule
  {\Gamma, (x : U); \Sigma \vdash p : S \\
  	\Gamma, (x : U); \Sigma \vdash p \ni \texttt{val} \; f:T}
  {	\Gamma, (x : U); \Sigma \vdash p.f : T}
\end{mathpar}
By our mutual induction hypothesis we assume narrowing holds for the 
smaller premises.
\begin{mathpar}
\inferrule
  {\Gamma, (x : S); \Sigma \vdash p : S' \\
  	\Gamma, (x : S); \Sigma \vdash S' <:^* S}
  {}
  	\and
\inferrule
  {\Gamma, (x : S); \Sigma \vdash p \ni \texttt{val} \; f:T' \\
  	\Gamma, (x : S); \Sigma \vdash T' <:^* T}
  {}
\end{mathpar}
By \textsc{T-Acc}, we get
\begin{mathpar}
\inferrule
  {\Gamma, (x : S); \Sigma \vdash p.f : T'}
  {}
\end{mathpar}
completing the case.
\end{casethm}
\begin{casethm}[\textsc{T-Type}]
\begin{mathpar}
\inferrule
  {\Gamma, (x : U); \Sigma \vdash p : T}
  {	\Gamma, (x : U); \Sigma \vdash p \unlhd T : T}
\end{mathpar}
By the mutual induction hypothesis we assume narrowing holds 
for the simpler expression typing premise.
\begin{mathpar}
\inferrule
  {\Gamma, (x : S); \Sigma \vdash p : T' \\
  	\Gamma, (x : S); \Sigma \vdash T' <:^* T}
  {}
\end{mathpar}
By \textsc{T-Sub*}, we get $\Gamma, (x : S); \Sigma \vdash p : T$, 
and subsequently by \textsc{T-Type} we get $\Gamma, (x : S); \Sigma \vdash p \unlhd T : T$.
\end{casethm}

\begin{casethm}[\textsc{T-Sub*}]
\begin{mathpar}
\inferrule
  {	\Gamma, (x : U); \Sigma \vdash e : T' \\
  	\Gamma, (x : U); \Sigma \vdash T' <:^* T}
  {	\Gamma, (x : U); \Sigma \vdash e : T}
\end{mathpar}
From our mutual induction hypothesis, we assume 
$\Gamma, (x : S); \Sigma \vdash e : T'', T'' <:^* T'$ and 
$\Gamma, (x : S); \Sigma \vdash T' <:^* T$. This gives us 
$\Gamma, (x : S); \Sigma \vdash T'' <:^* T$ and by \textsc{T-Sub} we
get the desired result.
\end{casethm}

\subsubsection*{\textsc {$<:^*$-Narrowing*}:}
\begin{casethm}[\textsc {S\textsuperscript{*}-Refl}]
\begin{mathpar}
\inferrule
  {}
  {\Gamma, (x : U); \Sigma \vdash S \; <:^* \; S}
\end{mathpar}
The base case for transitive subtyping is simple by \textsc{S\textsuperscript{*}-Refl}.
\end{casethm}
\begin{casethm}[\textsc {S\textsuperscript{*}-Trans}]
\begin{mathpar}
\inferrule
  {\Gamma, (x : U); \Sigma \vdash T \; <:^* \; T' \\
	\Gamma, (x : U); \Sigma \vdash T' \; <: \; T''}
  {\Gamma, (x : U); \Sigma \vdash T \; <:^* \; T''}
\end{mathpar}
From our mutual induction hypothesis we get 
\begin{mathpar}
\inferrule
  {\Gamma, (x : S); \Sigma \vdash S \; <:^* \; T \\
	\Gamma, (x : S); \Sigma \vdash T \; <:^* \; U}
  {}
\end{mathpar}
This immediately gives us the desired result 
$\Gamma, (x : S); \Sigma \vdash T \; <:^* \; T''$.
\end{casethm}
\end{proof}
\qed

%First we define the following \emph{Subtype Transitivity} judgement 
%\begin{mathpar}
%\inferrule
%  {}
%  {\Gamma \vdash S \; <:^* \; S}
%  \quad (\textsc {S\textsuperscript{*}-Refl})
%	\and
%\inferrule
%  {\Gamma \vdash S \; <:^* \; T \\
%	\Gamma \vdash T \; <: \; U}
%  {\Gamma \vdash S \; <:^* \; U}
%  \quad (\textsc {S\textsuperscript{*}-Trans})
%\end{mathpar}
%This is used to admit transitivity when proving a relaxed version of
%Environment Narrowing below, making Narrowing independent of 
%Subtype Transitivity.
%
%
%Induction principles need to be constructed for each of our 
%judgements. We can do this by treating judgements as either 
%base cases or complex cases derived from simpler judgements. 
%As an example, the induction principle for the subtyping 
%judgement can be described below.
%For any theorem $\mathcal{H}_{<:}$ on the subtyping judgement, if 
%the base cases
%\begin{mathpar}
%\inferrule
%  {}
%  {\mathcal{H}_{<:}(\textsc{S-Refl}) \\
%	\mathcal{H}_{<:}(\textsc{S-Top}) \\
%	\mathcal{H}_{<:}(\textsc{S-Bottom})}
%\end{mathpar}
%hold, and if given $\mathcal{H}_{<:}$ holds for any simpler 
%derivation of \textsc{S-Rec}, \textsc{S-Select-Upper} and
%\textsc{S-Select-Lower} we can show that 
%\begin{mathpar}
%\inferrule
%  {}
%  {\mathcal{H}_{<:}(\textsc{S-Rec}) \\
%	\mathcal{H}_{<:}(\textsc{S-Select-Upper}) \\
%	\mathcal{H}_{<:}(\textsc{S-Select-Lower})}
%\end{mathpar}
%hold, then it follows that $\mathcal{H}_{<:}$ holds for all 
%derived instances of the subtyping judgement. Similar inductive
%schemes can be constructed for the \emph{Expansion}, \emph{Membership}, 
%\emph{Typing} and \emph{Reduction} judgements.
%
%\begin{lemma}\label{lem:subtype:decl} 
%If 	$\Gamma; \Sigma \vdash S <: U, S \prec \overline{\sigma}, 
%	U \prec \overline{\sigma}'$ then
%	$\Gamma; \Sigma \vdash \overline{\sigma} <:^* \overline{\sigma}'$.
%\end{lemma}
%\begin{proof}
%\hl{ToDo}
%\end{proof}
%\qed
%
%There are four judgments mutually defined in our type system. 
%Subtyping, Membership, Expansion and Expression Typing are all 
%interdependent. Subtyping is dependent on Membership, which is 
%dependent on both Expansion and Typing. Expansion is dependent on 
%Membership and Typing is dependent on Subtyping. This means that 
%when performing induction on the derivation of one, we need to be 
%able extend the induction hypothesis to the others. Therefore, when 
%proving environmental narrowing, we need to prove the following 
%theorems mutually.
%\begin{theorem}[Environment Narrowing*] \label{thm:narrow}
%\begin{mathpar}
%\inferrule
%  {\Gamma, (x : U); \Sigma \vdash T <: T' \\
%  	\Gamma; \Sigma \vdash S <: U}
%  {\Gamma, (x : S); \Sigma \vdash T <:^* T'}
%  \quad (\textsc {<:-Narrowing*})
%	\and
%\inferrule
%  {\Gamma, (x : U); \Sigma \vdash T \prec \overline{\sigma} \\
%  	\Gamma; \Sigma \vdash S <: U}
%  {\exists \overline{\sigma}':
%  	\Gamma, (x : S); \Sigma \vdash T \prec \overline{\sigma}' \\
%  	\Gamma, (x : S); \Sigma \vdash \overline{\sigma}' <:^* \overline{\sigma}}
%  \quad (\textsc {$\prec$-Narrowing*})
%	\and
%\inferrule
%  {\Gamma, (x : U); \Sigma \vdash p \ni \sigma \\
%  	\Gamma; \Sigma \vdash S <: U}
%  {\exists \sigma':
%  	\Gamma, (x : S); \Sigma \vdash p \ni \sigma' \\
%  	\Gamma, (x : S); \Sigma \vdash \sigma' <:^* \sigma}
%  \quad (\textsc {$\ni$-Narrowing*})
%	\and
%\inferrule
%  {\Gamma, (x : U); \Sigma \vdash p : T \\
%  	\Gamma; \Sigma \vdash S <: U}
%  {\exists T':
%  	\Gamma, (x : S); \Sigma \vdash p : T' \\
%  	\Gamma, (x : S); \Sigma \vdash T' <:^* T}
%  \quad (\textsc {:-Narrowing*})
%\end{mathpar}
%\end{theorem}
%\begin{proof}
%We proceed by mutual structural induction on the of each of the judgements:
%\begin{mathpar}
%\inferrule
%  {\Gamma, (x : U); \Sigma \vdash T <: T' \\
%  	\Gamma, (x : U); \Sigma \vdash T \prec \overline{\sigma} \\
%  	\Gamma, (x : U); \Sigma \vdash p \ni \sigma \\
%  	\Gamma, (x : U); \Sigma \vdash p : T}
%  {}
%\end{mathpar}
%That is, if for each judgment the base cases (those cases that 
%do not require the derivation of simpler judgments) are satisfied, 
%and assuming the result for all simpler sub-derivations the we can 
%show it holds for more complex cases, we can conclude it holds in 
%general. While not explicitly stated above, we also include the 
%relevant declaration forms of the above judgments.
%
%\subsubsection*{\textsc {$<:$-Narrowing*}:}
%
%\begin{casethm}[\textsc{S-Refl}]
%\begin{mathpar}
%\inferrule
%  {}
%  {T' = T \\
%  	\Gamma, (x : U); \Sigma \vdash T\; \texttt{<:}\; T}
%\end{mathpar}
%In the reflexive base case \textsc{S-Refl}, $T' = T$. 
%It follows trivially from \textsc{S-Refl} that 
%$\Gamma, (x : S); \Sigma \vdash T\; \texttt{<:}\; T$.
%\end{casethm}
%\begin{casethm}[\textsc{S-Rec}]
%\begin{mathpar}
%\inferrule
%  {}
%  {T = \{z \Rightarrow \overline{\sigma}\} \\
%  	T' = \{z \Rightarrow \overline{\sigma}'\}}
%  	\and
%\inferrule
%  {\forall \sigma_i' \in \overline{\sigma}', \; \exists \; \sigma_i \in \overline{\sigma} \; st \; \Gamma, z : \{z \Rightarrow \overline{\sigma}\}, (x : U); \Sigma \vdash \sigma_i <:\; \sigma_i'}
%  {\Gamma, (x : U); \Sigma \vdash \{z \Rightarrow \overline{\sigma}\}\; <:\; \{z \Rightarrow \overline{\sigma}'\}}
%\end{mathpar}
%From our induction hypothesis we need to show that if our result holds for each 
%smaller subtype derivation, it holds for the larger derivation. Thus, if it
%holds for each individual 
%declaration type, it holds for the larger record type. 
%That is, for each $\sigma'_i$ and $\sigma_i$ such 
%that 
%$\Gamma, z : \{z \Rightarrow \overline{\sigma}\}, (x : U); \Sigma 
%\vdash \sigma_i <:\; \sigma_i'$,
%the induction hypothesis for subtyping gives us 
%$\Gamma, z : \{z \Rightarrow \overline{\sigma}\}, (x : S); \Sigma 
%\vdash \sigma_i <:^*\; \sigma_i'$. From here is is simple to construct 
%a chain of record types such that 
%$\Gamma, (x : S); \Sigma \vdash \{z \Rightarrow \overline{\sigma}\} <:^*\; 
%\{z \Rightarrow \overline{\sigma}'\}$.
%\end{casethm}
%\begin{casethm}[\textsc{S-Select-upper}]
%\begin{mathpar}
%\inferrule
%  {}
%  {T = p.L}
%  	\and
%\inferrule
%  {\Gamma, (x : U); \Sigma \vdash p \ni \texttt{type} \; L : S' .. U'\\
%  	\Gamma, (x : U); \Sigma \vdash S' <: U' \\
%  	\Gamma, (x : U); \Sigma \vdash U' <: T'}
%  {\Gamma, (x : U); \Sigma \vdash x.L\; <:\; T'}
%\end{mathpar}
%Firstly, we need to use our mutually defined induction hypothesis for 
%\textsc {$\ni$-Narrowing*} to derive 
%$\exists \sigma':
%\Gamma, (x : S); \Sigma \vdash p \ni \sigma'$ and
%$\Gamma, (x : S); \Sigma \vdash \sigma' <:^* L : S' .. U'$.
%i.e. $\exists S'' \, U'' : \Gamma, (x : S); \Sigma \vdash p \ni L : S'' .. U'', \;
%U'' <:^* U'$
%Now using the induction hypothesis for subtyping the Narrowing result holds for 
%sub-derivations of the subtype relation. In this case the sub-derivation is 
%$\Gamma, (x : U); \Sigma \vdash U' <: T'$, and it follows that 
%$\Gamma, (x : S); \Sigma \vdash U' <:^* T'$ holds. 
%We can now combine the two subtype chains to show that 
%$\Gamma, (x : S); \Sigma \vdash U'' <:^* U', \; U' <:^* T' \Rightarrow  U'' <:^* T'$. 
%It then follows that there exists some type $T''$ such that 
%$\Gamma, (x : S); \Sigma \vdash p.L <: T'', \; T'' <:^* T'$ 
%which gives us $\Gamma, (x : S); \Sigma \vdash p.L <:^* T'$.
%
%\hl{ToDo: Show that $\Gamma \vdash S <: T, T <:^* U \Rightarrow S <:^* U$}
%\end{casethm}
%\begin{casethm}[\textsc{S-Select-Lower}]
%\begin{mathpar}
%\inferrule
%  {}
%  {T' = p.L}
%  	\and
%\inferrule
%  {\Gamma, (x : U); \Sigma \vdash p \ni \texttt{type} \; L : S' .. U' \\
%  	\Gamma, (x : U); \Sigma \vdash S' <: U' \\
%  	\Gamma, (x : U); \Sigma \vdash T <: S'}
%  {\Gamma, (x : U); \Sigma \vdash T\; <:\; p.L}
%\end{mathpar}
%As in the previous case, we can use the \textsc {$\ni$-Narrowing*}
%induction hypothesis to show that 
%$\exists S'' \, U'' : \Gamma, (x : S); \Sigma \vdash p \ni L : S'' .. U'', \;
%S' <:^* S''$.
%We can also apply the \textsc {<:-Narrowing*} induction hypothesis 
%to show that $\Gamma, (x : S); \Sigma \vdash T <:^* S'$, 
%and subsequently that 
%$\Gamma, (x : S); \Sigma \vdash T <:^* S''$. It  should then be easy 
%to construct a subtyping chain such that 
%$\Gamma, (x : S); \Sigma \vdash T <:^* p.L$.
%
%\end{casethm}
%\begin{casethm}[\textsc{S-Top}]
%\begin{mathpar}
%  {}
%  {T' = \top \\ 
%  	\Gamma, (x : U); \Sigma \vdash T\; \texttt{<:}\; \top}
%\end{mathpar}
%The base case dealing with subtyping of the top type is trivial 
%since by \textsc{S-Top} 
%$\Gamma, (x : S); \Sigma \vdash T\; \texttt{<:}\; \top$.
%\end{casethm}
%\begin{casethm}[\textsc{S-Bottom}]
%\begin{mathpar}
%\inferrule
%  {}
%  {T = \bot \\
%  	\Gamma, (x : U); \Sigma \vdash \bot\; \texttt{<:}\; T'}
%\end{mathpar}
%The base case \textsc{S-Bottom} resolves trivially since 
%by \textsc{S-Bottom} $\Gamma, (x : S); \Sigma \vdash \bot\; \texttt{<:}\; T'$
%\end{casethm}
%\begin{casethm}[\textsc{S-Decl-Val}]
%\begin{mathpar}
%\inferrule
%  {}
%  {\sigma = \texttt{val} \; f:T \\
%  	\sigma' = \texttt{val} \; f:T}
%  	\and
%\inferrule
%  {}
%  {\Gamma, (x : U); \Sigma \vdash \texttt{val} \; f:T <: \texttt{val} \; f:T}
%\end{mathpar}
%The base case for declaration subtyping is resolves trivially by 
%\textsc{S-Decl-Val}: 
%$\Gamma, (x : S); \Sigma \vdash \texttt{val} \; f:T <: \texttt{val} \; f:T$
%\end{casethm}
%\begin{casethm}[\textsc{S-Decl-Def}]
%\begin{mathpar}
%\inferrule
%  {}
%  {\sigma = \texttt{def} \; m:S' \rightarrow T \\
%  	\sigma' = \texttt{def} \; m:S'' \rightarrow T'}
%  	\and
%\inferrule
%  {\Gamma, (x : U); \Sigma \vdash T <: T' \\
%  	\Gamma, (x : U); \Sigma \vdash S'' <: S'}
%  {\Gamma, (x : U); \Sigma \vdash \texttt{def} \; m:S' \rightarrow T <: \texttt{def} \; m:S'' \rightarrow T'}
%\end{mathpar}
%Using the \textsc {<:-Narrowing*} induction hypothesis 
%for argument and return type subtyping we 
%get $\Gamma, (x : S); \Sigma \vdash T <:^* T'$ and
%$\Gamma, (x : S); \Sigma \vdash S'' <:^* S'$. We can then 
%create a chain of declaration subtype judgments to show that 
%$\Gamma, (x : S); \Sigma \vdash \texttt{def} \; m:S' \rightarrow T <:^* \texttt{def} \; m:S'' \rightarrow T'$.
%\end{casethm}
%
%\begin{casethm}[\textsc{S-Decl-Type}]
%\begin{mathpar}
%\inferrule
%  {}
%  {\sigma = \texttt{type} \; L : S' .. U' \\
%  	\sigma' = \texttt{type} \; L : S'' .. U''}
%  	\and
%\inferrule
%  {\Gamma, (x : U); \Sigma \vdash S'' <: S' \\
%  	\Gamma, (x : U); \Sigma \vdash U' <: U''}
%  {\Gamma, (x : U); \Sigma \vdash \texttt{type} \; L : S' .. U' \; <:\; \texttt{type} \; L : S'' .. U''}
%\end{mathpar}
%Similar to the \textsc{S-Decl-Def} case, by the induction hypothesis, 
%if $\Gamma, (x : S); \Sigma \vdash S'' <:^* S'$ and
%$\Gamma, (x : S); \Sigma \vdash U' <:^* U''$ hold, it can be shown that 
%$\Gamma, (x : S); \Sigma \vdash \texttt{type} \; L : S' .. U' \; <:^* \; \texttt{type} \; L : S'' .. U''$ holds too.
%
%\end{casethm}
%
%\subsubsection*{\textsc {$\prec$-Narrowing*}:}
%
%\begin{casethm}[\textsc{E-Rec}]
%\begin{mathpar}
%\inferrule
%  {}
%  {T = \{z \Rightarrow \overline{\sigma}\}}
%  	\and
%\inferrule
%  {}
%  {\Gamma, (x : U); \Sigma \vdash 
%  		\{z \Rightarrow \overline{\sigma}\} \prec_z \overline{\sigma}}
%\end{mathpar}
%The base case for expansion follows immediately from \textsc{E-Rec}:
%$\Gamma, (x : S); \Sigma \vdash \{z \Rightarrow \overline{\sigma}\} \prec_z \overline{\sigma}$.
%\end{casethm}
%\begin{casethm}[\textsc{E-Select}]
%\begin{mathpar}
%\inferrule
%  {}
%  {T = p.L}
%  	\and
%\inferrule
%  {\Gamma, (x : U); \Sigma \vdash p \ni \texttt{type} \; L : S'..U' \\
%  	\Gamma, (x : U); \Sigma \vdash U' \prec_z \overline{\sigma}}
%  {\Gamma, (x : U); \Sigma \vdash p.L \prec_z \overline{\sigma}}
%\end{mathpar}
%By our mutual induction hypothesis we assume that
%\begin{mathpar}
%\inferrule
%  {\Gamma, (x : S); \Sigma \vdash p \ni \texttt{type} \; L : S''..U'', \;
%  	\texttt{type} \; L : S''..U'' <:^* \texttt{type} \; L : S'..U', \;
%  	U' \prec_z \overline{\sigma}', \;
%  	\overline{\sigma}' <:^* \overline{\sigma}}
%  {}
%\end{mathpar}
%holds. By \textsc{S-Decl-Type}, we can infer that 
%$\Gamma, (x : S); \Sigma \vdash U'' <:^* U'$. Using 
%Lemma \ref{lem:subtype:decl}, we can show that 
%$\exists \overline{\sigma}'': \Gamma, (x : S); \Sigma \vdash U'' \prec 
%\overline{\sigma}'', \; \overline{\sigma}'' <:^* \overline{\sigma}'$.
%Thus, we can show that 
%$\Gamma, (x : S); \Sigma \vdash \overline{\sigma}'' <:^* \overline{\sigma}$ 
%which completes the case.
%
%\hl{ToDo: U = $\bot$?}
%\end{casethm}
%\begin{casethm}[\textsc{E-Top}]
%\begin{mathpar}
%\inferrule
%  {}
%  {T = \top}
%  	\and
%\inferrule
%  {}
%  {\Gamma, (x : U); \Sigma \vdash \top \prec_z \varnothing}
%\end{mathpar}
%The \textsc{E-Top} base case resolves trivially. By 
%\textsc{E-Top}, $\Gamma, (x : S); \Sigma \vdash \top \prec_z \varnothing$.
%\end{casethm}
%
%\subsubsection*{\textsc {$\ni$-Narrowing*}:}
%
%\begin{casethm}[\textsc{M-Path}]
%\begin{mathpar}
%\inferrule
%  {}
%  {e = p \\
%  	\sigma = [p/z]\sigma_i}
%  	\and
%\inferrule
%  {\Gamma, (x : U); \Sigma \vdash p : T \\
%  	\Gamma, (x : U); \Sigma \vdash T \prec_z \overline{\sigma}\\
%  	\sigma_i \in \overline{\sigma}}
%  {\Gamma, (x : U); \Sigma \vdash p \ni [p/z]\sigma_i}
%\end{mathpar}
%From the mutually defined induction hypothesis, we assume 
%\begin{mathpar}
%\inferrule
%  {\Gamma, (x : S); \Sigma \vdash p : T', \; T' <:^* T \\
%  	\Gamma, (x : S); \Sigma \vdash T \prec_z \overline{\sigma}', \;
%  	\overline{\sigma}' <:^* \overline{\sigma}}
%  {}
%\end{mathpar}
%By Lemma \ref{lem:subtype:decl}, $\Gamma, (x : S); \Sigma \vdash T' \prec \overline{\sigma}''$ 
%and $\Gamma, (x : S); \Sigma \vdash \overline{\sigma}'' <:^* \overline{\sigma}'$. 
%We can now show that $\exists \sigma_i'' \in \overline{\sigma}'':
%\Gamma, (x : S); \Sigma \vdash \sigma_i'' <:^* \sigma_i$, and thus that 
%$\Gamma, (x : S); \Sigma \vdash [p/z]\sigma_i'' <:^* [p/z]\sigma_i$.
%
%\hl{ToDo: Show T' expanding \\ Substitution preserves subtype?}
%\end{casethm}
%\begin{casethm}[\textsc{M-Exp}]
%\begin{mathpar}
%\inferrule
%  {\Gamma, (x : U); \Sigma \vdash p : T \\
%  	\Gamma, (x : U); \Sigma \vdash T \prec_z \overline{\sigma}\\
%  	\sigma \in \overline{\sigma} \\
%  	z \notin \sigma}
%  {\Gamma, (x : U); \Sigma \vdash p \ni \sigma}
%\end{mathpar}
%We explicitly restrict the statement of the theorem to only include 
%membership of paths, as a result, this cases resolves as in \textsc{M-Path}.
%\end{casethm}
%
%\subsubsection*{\textsc {:-Narrowing*}:}
%
%\begin{casethm}[\textsc{T-Var}]
%\begin{mathpar}
%\inferrule
%  {}
%  {p = y \\
%  	T = \Gamma, (x : U)(y)}
%  	\and
%\inferrule
%  {y \in dom(\Gamma, (x : U))}
%  {	\Gamma, (y : U); \Sigma \vdash y : \Gamma(y)}
%\end{mathpar}
%\begin{itemize}
%\item[]  \textit{Subcase 1} ($y = x$):
%If $y = x$, then the result is immediate since by \textsc{T-Var} 
%$\Gamma,(x:S); \Sigma \vdash x : S$, and by assumption 
%$\Gamma; \Sigma \vdash S <: U$.
%\item[]  \textit{Subcase 2} ($y \neq x$):
%If $y \neq x$, then the type of $y$ remains the same and the result 
%is achieved by reflexivity.
%\end{itemize}
%\end{casethm}
%\begin{casethm}[\textsc{T-Loc}]
%\begin{mathpar}
%\inferrule
%  {}
%  {\Gamma; \Sigma \vdash T\; \texttt{<:}\; T}
%\end{mathpar}
%\hl{ToDo}
%\end{casethm}
%\begin{casethm}[\textsc{T-New}]
%\begin{mathpar}
%\inferrule
%  {}
%  {\Gamma; \Sigma \vdash T\; \texttt{<:}\; T}
%\end{mathpar}
%\hl{ToDo}
%\end{casethm}
%\begin{casethm}[\textsc{T-Meth}]
%\begin{mathpar}
%\inferrule
%  {}
%  {\Gamma; \Sigma \vdash T\; \texttt{<:}\; T}
%\end{mathpar}
%\hl{ToDo}
%\end{casethm}
%\begin{casethm}[\textsc{T-Acc}]
%\begin{mathpar}
%\inferrule
%  {}
%  {\Gamma; \Sigma \vdash T\; \texttt{<:}\; T}
%\end{mathpar}
%\hl{ToDo}
%\end{casethm}
%\begin{casethm}[\textsc{T-Type}]
%\begin{mathpar}
%\inferrule
%  {}
%  {\Gamma; \Sigma \vdash T\; \texttt{<:}\; T}
%\end{mathpar}
%\hl{ToDo}
%\end{casethm}
%\begin{casethm}[\textsc{T-Sub}]
%\begin{mathpar}
%\inferrule
%  {}
%  {\Gamma; \Sigma \vdash T\; \texttt{<:}\; T}
%\end{mathpar}
%\hl{ToDo}
%\end{casethm}
%\begin{casethm}[\textsc{T-Decl-Var}]
%\begin{mathpar}
%\inferrule
%  {}
%  {\Gamma; \Sigma \vdash T\; \texttt{<:}\; T}
%\end{mathpar}
%\hl{ToDo}
%\end{casethm}
%\begin{casethm}[\textsc{T-Decl-Def}]
%\begin{mathpar}
%\inferrule
%  {}
%  {\Gamma; \Sigma \vdash T\; \texttt{<:}\; T}
%\end{mathpar}
%\hl{ToDo}
%\end{casethm}
%\begin{casethm}[\textsc{T-Decl-Def}]
%\begin{mathpar}
%\inferrule
%  {}
%  {\Gamma; \Sigma \vdash T\; \texttt{<:}\; T}
%\end{mathpar}
%\hl{ToDo}
%\end{casethm}
%\end{proof}
%\qed
%
%\begin{theorem}[Environment Narrowing*]
%If $\Gamma_a, (x : U), \Gamma_b \vdash T <: T'$ and 
%   	$\Gamma_a \vdash S <: U$ then
%	$\Gamma_a, (x : S), \Gamma_b \vdash T <:^* T'$
%\end{theorem}
%\begin{proof}
%By induction on the derivation of $\Gamma_a, (z : U), \Gamma_b \vdash T <: T'$.
%\begin{casethm}[\textsc{S-Refl}]
%\begin{mathpar}
%\inferrule
%  {T = T'}
%  {}
%\end{mathpar}
%Trivial.
%\end{casethm}
%\begin{casethm}[\textsc{S-Rec}]
%\begin{mathpar}
%\inferrule
%  {T = \{z \Rightarrow \overline{\sigma}\} \\
%  	T' = \{z \Rightarrow \overline{\sigma}'\} \\
%  	\forall \sigma_i' \in \overline{\sigma}', \; \exists \; \sigma_i \in \overline{\sigma} \; st \; \Gamma_a, (x : U), \Gamma_b,(z : \{z \Rightarrow \overline{\sigma}\}) \vdash \sigma_i <:\; \sigma_i'}
%  {}
%\end{mathpar}
%Applying our induction hypotheses to the smaller derivation of 
%$\Gamma_a, (x : U), \Gamma_b,(z : \{z \Rightarrow \overline{\sigma}\}) 
%\vdash \sigma_i <:\; \sigma_i'$ for each $\sigma_i$ and $\sigma_i'$, 
%we can show that 
%$\Gamma_a, (x : S), \Gamma_b,(z : \{z \Rightarrow \overline{\sigma}\}) 
%\vdash \sigma_i <:^*\; \sigma_i'$.
%We can use this and \textsc{S-Rec} to construct a series of record types 
%such that $\Gamma_a, (x : S), \Gamma_b \vdash 
%\{z \Rightarrow \overline{\sigma}\} <: 
%\{z \Rightarrow \overline{\sigma}_0\} <: ...
%<: \{z \Rightarrow \overline{\sigma}_n\} <:
%\{z \Rightarrow \overline{\sigma}'\}$.
%i.e that 
%$\Gamma_a, (x : S), \Gamma_b \vdash 
%\{z \Rightarrow \overline{\sigma}\}\; <:^*\; 
%\{z \Rightarrow \overline{\sigma}'\}$.
%\end{casethm}
%\begin{casethm}[\textsc{S-Select-Upper}]
%\begin{mathpar}
%\inferrule
%  {T = p.L \\
%  	\Gamma_a, (x : U), \Gamma_b \vdash p \ni \texttt{type} \; L : S' .. U' \\
%  	\Gamma_a, (x : U), \Gamma_b \vdash S' <: U' \\
%  	\Gamma_a, (x : U), \Gamma_b \vdash U' <: T'}
%  {}
%\end{mathpar}
%\hl{It can be shown that 
%$\Gamma_a, (x : S), \Gamma_b \vdash p \ni \texttt{type} \; L : S'' .. U''$
%where 
%$\Gamma_a, (x : S), \Gamma_b \vdash S' <: S'' <: U'' <: U'$}.
%Further, by applying our induction hypothesis to the smaller 
%derivation of $\Gamma_a, (x : U), \Gamma_b \vdash U' <: T'$ 
%to get $\Gamma_a, (x : S), \Gamma_b \vdash U' <: T'$ we 
%we can construct the chain 
%$\Gamma_a, (x : S), \Gamma_b \vdash U'' <: U' <: T'$ 
%and thus by \textsc{S-Select-Upper} and 
%\textsc {S\textsuperscript{*}-Trans} that 
%$\Gamma_a, (x : S), \Gamma_b \vdash p.L <:^* T'$ completing the case.
%\end{casethm}
%\begin{casethm}[\textsc{S-Select-Lower}]
%\begin{mathpar}
%\inferrule
%  {T' = p.L \\
%  	\Gamma_a, (x : U), \Gamma_b \vdash p \ni \texttt{type} \; L : S' .. U' \\
%  	\Gamma_a, (x : U), \Gamma_b \vdash S' <: U' \\
%  	\Gamma_a, (x : U), \Gamma_b \vdash T <: S'}
%  {}
%\end{mathpar}
%\hl{TODO: Complete reasoning in similar manner as \textsc{S-Select-Upper}}
%\end{casethm}
%\begin{casethm}[\textsc{S-Top}]
%\begin{mathpar}
%\inferrule
%  {T' = \top}
%  {}
%\end{mathpar}
%Trivial.
%\end{casethm}
%\begin{casethm}[\textsc{S-Bottom}]
%\begin{mathpar}
%\inferrule
%  {T = \bot}
%  {}
%\end{mathpar}
%Trivial.
%\end{casethm}
%\begin{casethm}[\textsc{S-Decl-Var}]
%\begin{mathpar}
%\inferrule
%  {\sigma = \texttt{val} \; f:T \\
%  	\sigma' = \texttt{val} \; f:T}
%  {}
%\end{mathpar}
%Trivial.
%\end{casethm}
%\begin{casethm}[\textsc{S-Decl-Meth}]
%\begin{mathpar}
%\inferrule
%  {\sigma = \texttt{def} \; m:S \rightarrow T \\
%  	\sigma' = \texttt{def} \; m:S' \rightarrow T' \\
%  	\Gamma, (x : U), \Gamma_b \vdash T <: T' \\
%  	\Gamma, (x : U), \Gamma_b \vdash S' <: S}
%  {}
%\end{mathpar}
%Applying the induction hypothesis to the smaller derivations of 
%$\Gamma, (x : U), \Gamma_b \vdash T <: T'$ and
%$\Gamma, (x : U), \Gamma_b \vdash S' <: S$ we get 
%\begin{mathpar}
%\inferrule
%  {\Gamma, (x : S), \Gamma_b \vdash T <:^* T' \\
%  	\Gamma, (x : S), \Gamma_b \vdash S' <:^* S}
%  {}
%\end{mathpar}
%This means we can construct two subtype chains: 
%\begin{mathpar}
%\inferrule
%  {\Gamma, (x : S), \Gamma_b \vdash T <: T_0 <: ... <: T_m <: T' \\
%  	\Gamma, (x : S), \Gamma_b \vdash S' <: S_0 <: ... <: S_n <: S}
%  {}
%\end{mathpar}
%Using these we can construct a similar subtype chain 
%\begin{mathpar}
%\inferrule
%  {\Gamma, (x : S), \Gamma_b \vdash 
%\texttt{def} \; m:S \rightarrow T <: \texttt{def} \; m:S_n \rightarrow T_0 
%<: ... <: \texttt{def} \; m:S_0 \rightarrow T_m <: 
%\texttt{def} \; m:S' \rightarrow T'}
%  {}
%\end{mathpar}
%i.e. $\Gamma, (x : S), \Gamma_b \vdash 
%\texttt{def} \; m:S \rightarrow T <:^* 
%\texttt{def} \; m:S' \rightarrow T'$
%\end{casethm}
%\begin{casethm}[\textsc{S-Decl-Type}]
%\begin{mathpar}
%\inferrule
%  {\sigma = \texttt{type} \; L : S .. U \\
%  	\sigma' = \texttt{type} \; L : S' .. U' \\
%  	\Gamma, (x : U), \Gamma_b \vdash S' <: S \\
%  	\Gamma, (x : U), \Gamma_b \vdash U <: U'}
%  {}
%\end{mathpar}
%In a similar manner to the \emph{casethm} for \textsc{S-Decl-Meth}, we can use 
%the induction hypothesis to derive the following.
%\begin{mathpar}
%\inferrule
%  {\Gamma, (x : S), \Gamma_b \vdash S' <:^* S \\
%  	\Gamma, (x : S), \Gamma_b \vdash U <:^* U'}
%  {}
%\end{mathpar}
%Similarly we can derive the following subtype chains,
%\begin{mathpar}
%\inferrule
%  {\Gamma, (x : S), \Gamma_b \vdash S' <: S_0 <: ... <: S_m <: S \\
%  	\Gamma, (x : S), \Gamma_b \vdash U <: U_0 <: ... <: U_n <: U'}
%  {}
%\end{mathpar}
%and subsequently the following chain of declaration subtypes.
%\begin{mathpar}
%\inferrule
%  {\Gamma, (x : S), \Gamma_b \vdash 
%\texttt{type} \; L : S .. U <: \texttt{type} \; L : S_m .. U_0 
%<: ... <: \texttt{type} \; L : S_0 .. U_n <: 
%\texttt{type} \; L : S' .. U'}
%  {}
%\end{mathpar}
%This gives us the desired result: 
%$\Gamma, (x : S), \Gamma_b \vdash 
%\texttt{type} \; L : S .. U <:^* \texttt{type} \; L : S' .. U'$.
%\end{casethm}
%\end{proof}
%\qed

%---------------- Subtype Chain Construction ----------------%
\begin{lemma} \label{lem:subtype_chain}
If $\Gamma; \Sigma \vdash T \; \textbf{\tt{wf}}$ and 
	$\Gamma; \Sigma \vdash S <: T <: U$
	then we can construct a subtype sequence
   $\Gamma; \Sigma \vdash S <: T'_0 <: ... <: T'_m <: U$ such that
	$\forall i \in [0,m], T'_i \neq p.L$.
\end{lemma}
\begin{proof}
Proceed by case analysis on $T$.
\begin{casethm}[$T = \{z \Rightarrow \overline{\sigma}\}$]
%\begin{mathpar}
%\inferrule
%  {T = \{z \Rightarrow \overline{\sigma}\}}
%  {}
%  \and
%\inferrule
%  {\forall \sigma_i \in \overline{\sigma}, \; \Gamma,z:\{z \Rightarrow \overline{\sigma}\}; \Sigma \vdash \sigma_i \; \textbf{wf} \\
%  	\forall j \neq i, \; dom(\sigma_j) \neq dom(\sigma_i)}
%  {\Gamma; \Sigma \vdash \{z \Rightarrow \overline{\sigma}\} \; \textbf{wf}}
%\end{mathpar}
Records present a base case, and the result is immediate since 
by assumption we have 
$\Gamma; \Sigma \vdash S <: \{z \Rightarrow \overline{\sigma}\} <: U$ and 
$\{z \Rightarrow \overline{\sigma}\} \neq p.L$.
\end{casethm}
\begin{casethm}[$T = p.L$]
Since $p.L$ is \textbf{\texttt{wf}}, we have
\begin{mathpar}
%\inferrule
%  {T = p.L}
%  {}
%  \and
\inferrule
  {\Gamma; \Sigma \vdash p \ni \texttt{type} \; L : S' .. U' \\
  	\Gamma; \Sigma \vdash S' <: U'\\
  	\Gamma; \Sigma \vdash S', U' \; \textbf{wfe}}
  {\Gamma; \Sigma \vdash p.L \; \textbf{wf}}
\end{mathpar}
We begin by replacing $T$ with $S' <: U'$ to get the subtype sequence 
\begin{mathpar}
%\inferrule
%  {T = p.L}
%  {}
%  \and
\inferrule
  {\Gamma; \Sigma \vdash S <: S' <: U' <: U}
  {}
\end{mathpar}
Next we look at $S'$ and $U'$. If they are not selection types, we 
are done, otherwise we similarly replace them in the sequence with 
their bounds. We now need to show that the bounds themselves are 
terminating. Since we restrict $T$ to be both well-formed and 
expanding, we know that the bounds are by definition also well-formed 
and expanding. An expanding type implies a finite expansion, and thus 
each bound produces a finite sequence.

\hl{TODO: Explicit proof of finite expansion?}

\end{casethm}
\begin{casethm}[$T = \top$]
%\begin{mathpar}
%\inferrule
%  {T = \top}
%  {}
%  \and
%\inferrule
%  {\Gamma; \Sigma \vdash \top \;  \textbf{wf}}
%  {}
%\end{mathpar}
The top type represents a base case that follows immediately by assumption.
\end{casethm}
\begin{casethm}[$T = \bot$]
%\begin{mathpar}
%\inferrule
%  {T = \bot}
%  {}
%  \and
%\inferrule
%  {\Gamma; \Sigma \vdash \bot \;  \textbf{wf}}
%  {}
%\end{mathpar}
The bottom type represents a base case that follows immediately by assumption.
\end{casethm}
%\begin{casethm}[\textsc {WF-Val}]
%\begin{mathpar}
%\inferrule
%  {\Gamma; \Sigma \vdash T : \textbf{wf}}
%  {\Gamma; \Sigma \vdash \texttt{val} \; f:T \;  \textbf{wf}}
%\end{mathpar}
%\end{casethm}
%\begin{casethm}[\textsc {WF-Def}]
%\begin{mathpar}
%\inferrule
%  {\Gamma; \Sigma \vdash T : \textbf{wf} \\
%  	\Gamma; \Sigma \vdash S : \textbf{wf}}
%  {\Gamma; \Sigma \vdash \texttt{def} \; m:S \rightarrow T \;  \textbf{wf}}
%\end{mathpar}
%\end{casethm}
%\begin{casethm}[\textsc {WF-Type}]
%\begin{mathpar}
%\inferrule
%  {\Gamma; \Sigma \vdash S : \textbf{wfe} \; \vee \; S = \bot\\
%  	\Gamma; \Sigma \vdash U : \textbf{wfe} \\
%  	\Gamma; \Sigma \vdash S <: U}
%  {\Gamma; \Sigma \vdash \texttt{type} \; L : S .. U \; \textbf{wf}}
%\end{mathpar}
%\end{casethm}
\end{proof}
\qed

%We now show that for any chain of \texttt{wfe} types, $S <:^* U$,
%it's possible to construct a chain that contains no selection types.
%\begin{lemma} \label{lem:subtype_chain}
%If $\Gamma \vdash S, U \; \textbf{\tt{wfe}}$ where
%	$\Gamma \vdash S <:^* U$ then we can construct a subtype sequence
%	$\Gamma \vdash S <: T'_0 <: ... <: T'_m <: U$ such that 
%	$\forall i \in [0,m], T'_i \neq p.L$.
%\end{lemma}
%\begin{proof}
%By induction on the derivation of $\Gamma \vdash S <:^* U$.
%\begin{casethm}[\textsc {S\textsuperscript{*}-Refl}]
%\begin{mathpar}
%\inferrule
%  {U = S}
%  {}
%\end{mathpar}
%Trivial.
%\end{casethm}
%\begin{casethm}[\textsc {S\textsuperscript{*}-Trans}]
%\begin{mathpar}
%\inferrule
%  {\Gamma \vdash S \; <:^* \; T \\
%	\Gamma \vdash T \; <: \; U}
%  {}
%\end{mathpar}
%Applying the induction hypothesis to the smaller 
%derivation of $\Gamma \vdash S \; <:^* \; T$ we 
%can show that there exists a subtype sequence 
%$\Gamma \vdash S <: T_0 <: ... <: T_n <: T$ where 
%for all $i \in [0,n]$, $T_i \neq p.L$ for any $p$ and $L$.
%
%Now we need to show that we can derive a similar 
%sequence $\Gamma \vdash T_n <: T_{n+1} <: ... <: T_m <: U$ 
%to complete the sequence and replace $T$. To do this we 
%do a case analysis on the structure of $T$.
%\begin{itemize}
%\item[]  \textit{Subcase 1} ($\{z \Rightarrow \overline{\sigma}\}$):
%\begin{mathpar}
%\inferrule
%  {T = \{z \Rightarrow \overline{\sigma}\}}
%  {}
%\end{mathpar}
%Trivial.
%\item[]  \textit{Subcase 2} ($p.L$):
%\begin{mathpar}
%\inferrule
%  {T = p.L}
%  {}
%\end{mathpar}
%\hl{TODO}
%\item[]  \textit{Subcase 3} ($\top$):
%\begin{mathpar}
%\inferrule
%  {T = \top}
%  {}
%\end{mathpar}
%Trivial.
%\item[]  \textit{Subcase 4} ($\bot$):
%\begin{mathpar}
%\inferrule
%  {T = \bot}
%  {}
%\end{mathpar}
%\hl{TODO: Solve issue with expanding types and $\bot$}
%\end{itemize} 
%\end{casethm}
%\end{proof}
%\qed

%---------------------- Transitivity ----------------------%
Subtype Transitivity 

\begin{lemma}[Subtype Transitivity]\label{thm:trans}
If $\Gamma \vdash S <:^* U$ then
	$\Gamma \vdash S <: U$
\end{lemma}
\begin{proof}
Expand $\Gamma \vdash S <:^* U$ to the following chain:
\begin{mathpar}
\inferrule
  {\Gamma \vdash S <: T_0 <: ... <: T_n <: U}
  {}
\end{mathpar}
Now proceed by breaking the proof into two parts. First the case where 
$\forall i \in [0,n], T_i \neq p.L$. Secondly we show that if the chain 
does contain a selection type, we can construct another chain that contains 
no such selection type.
\begin{casethm}
Begin by induction on the derivation of $\Gamma \vdash S <:^* U$.
\begin{itemize}
\item[]  \textit{Subcase 1} (\textsc {S\textsuperscript{*}-Refl}):
Trivial.
\item[]  \textit{Subcase 2} (\textsc {S\textsuperscript{*}-Trans}):
\begin{mathpar}
\inferrule
  {\Gamma \vdash S <:^* T_n \\
  	\Gamma \vdash T_n <: U}
  {}
\end{mathpar}
Applying the induction hypothesis to the smaller derivation 
of $\Gamma \vdash S <:^* T$ we get $\Gamma \vdash S <: T$. 
This gives us the more traditional form of \emph{Subtype Transitivity}.
\begin{mathpar}
\inferrule
  {\Gamma \vdash S <: T \\
   	\Gamma \vdash T <: U}
  {\Gamma \vdash S <: U}
\end{mathpar}
We can now proceed by induction on the derivation of 
$\Gamma \vdash S <: T$ keeping in mind that we have restricted 
$T$ to only non-selection types (meaning \textsc{S-Select-Lower} 
is not considered).
\begin{itemize}
\item[]  \textit{Subsubcase 1} (\textsc {S-Refl}):
\begin{mathpar}
\inferrule
  {S = T}
  {}
\end{mathpar}
Trivial.
\item[]  \textit{Subsubcase 2} (\textsc {S-Rec}):
\begin{mathpar}
\inferrule
  {S =  \{z \Rightarrow \overline{\sigma}\} \\
  	T =  \{z \Rightarrow \overline{\sigma}'\}}
  {}
	\and
\inferrule
  {\forall \sigma_i' \in \overline{\sigma}', \; \exists \; \sigma_i \in \overline{\sigma} \; st \; \Gamma, z : \{z \Rightarrow \overline{\sigma}\}; \Sigma \vdash \sigma_i <:\; \sigma_i'}
  {\Gamma; \Sigma \vdash \{z \Rightarrow \overline{\sigma}\}\; <:\; \{z \Rightarrow \overline{\sigma}'\}}
\end{mathpar}
\hl{TODO: complete}
\item[]  \textit{Subsubcase 3} (\textsc {S-Select-Upper}):
\begin{mathpar}
\inferrule
  {S =  p.L \\
  	\Gamma \vdash p \ni \texttt{type} \; L : S' .. U'\\
  	\Gamma \vdash S' <: U' \\
  	\Gamma \vdash U' <: T}
  {}
\end{mathpar}
Applying the induction hypothesis to the smaller derivations 
$\Gamma \vdash U' <: T$ and $\Gamma \vdash T <: U$ we get $\Gamma \vdash U' <: U$. 
By \textsc{S-Select-Upper} is follows that $\Gamma \vdash S <: U$.
\item[]  \textit{Subsubcase 4} (\textsc {S-Top}):
\begin{mathpar}
\inferrule
  {T = \top}
  {}
\end{mathpar}
\hl{TODO: Show that if $\Gamma \vdash S <: T, S \prec \overline{\sigma}, 
T \prec \overline{\sigma}'$ then 
$\Gamma \vdash \overline{\sigma} <: \overline{\sigma}'$}\\
$\Gamma \vdash U \; \texttt{wfe} \Rightarrow
\exists \overline{\sigma}: \; \Gamma \vdash U \prec \overline{\sigma}$. 
Since $\Gamma \vdash \top \prec \varnothing$, it follows that 
for every declaration $\sigma \in \overline{\sigma}$, 
$\exists \sigma' \in \varnothing$. This implies that 
$\overline{\sigma} = \varnothing$, and subsequently that 
$U = \top$. The desired result follows immediately from \textsc{S-Top}.
\item[]  \textit{Subsubcase 5} (\textsc {S-Bottom}):
\begin{mathpar}
\inferrule
  {S = \bot}
  {}
\end{mathpar}
Trivial.
\end{itemize}
\end{itemize}
\end{casethm}
\begin{casethm}
In the case where our subtype chain $\Gamma \vdash S <:^* U$ 
contains a selection type, we can use Lemma \ref{lem:subtype_chain}
to construct one that does not. We can now apply our reasoning from the 
previous case to the new subtype chain to show that $\Gamma \vdash S <: U$.
\end{casethm}
\end{proof}
\qed

\newpage

\subsection{Subject Reduction}

\begin{lemma} \label{lem:path_type_preservation}
If $\Gamma; \Sigma \vdash v : T$, 
$\Sigma \vdash  \mu$ and $\mu \vdash v \leadsto l$ then 
$\Gamma; \Sigma \vdash l : T$.
\end{lemma}
\begin{proof}
By induction on the derivation of $\mu \vdash v \leadsto l$.
\begin{casethm}[\textsc{L-Loc}]
\begin{mathpar}
\inferrule
  {v = l}
  {}
  \and
\inferrule
  {}
  {\mu \vdash l \leadsto l}
\end{mathpar}
Trivial.
\end{casethm}
\begin{casethm}[\textsc{L-Type}]
\begin{mathpar}
  {v = v' \unlhd T}
  {}
  \and
\inferrule
  {\mu \vdash v' \leadsto l}
  {\mu \vdash v' \unlhd T \leadsto l}
\end{mathpar}
Since we know that $\Gamma; \Sigma \vdash v' \unlhd T: T$, 
by \textsc{T-Type} we can infer $\Gamma; \Sigma \vdash v' : T$.
By our induction hypothesis, we assume that if 
$\Gamma; \Sigma \vdash v' : T$ then 
$\Gamma; \Sigma \vdash l : T$. This completes the case.
\end{casethm}
\begin{casethm}[\textsc{L-Path}]
\begin{mathpar}
\inferrule
  {\mu \vdash v' \leadsto l' \\
	\mu(l') = \{z \Rightarrow ..., \texttt{val} f : T = l, ...\}}
  {\mu \vdash v'.f \leadsto l}
\end{mathpar}
By inversion on the derivation of $\Gamma; \Sigma \vdash v'.f : T$ we 
have 
\begin{mathpar}
\inferrule
  {	\Gamma; \Sigma \vdash v' : S \\
  	\Gamma; \Sigma \vdash v' \ni \texttt{val} \; f:T}
  {	\Gamma; \Sigma \vdash v'.f : T}
\end{mathpar}
By assumption $\Sigma \vdash \mu$, which implies $\varnothing; \Sigma 
\vdash l : T$. Environment weakening gives us 
$\Gamma; \Sigma \vdash l : T$
%And by our induction hypothesis we assume $\Gamma; \Sigma \vdash l' : S$.
\end{casethm}
\qed
\end{proof}

\newpage

%---------------------- Preservation ----------------------%
\begin{theorem}[Preservation]
If $\Gamma; \Sigma \vdash e : T$, 
   	$\mu \; | \; e \; \rightarrow \mu' \; | \; e'$ where
	$\Sigma \vdash \mu \; \tt{\bf{wf}}$ then 
 	$\exists \Sigma'$ s.t. 
	$\Sigma'$ extends $\Sigma$, 
	$\Sigma' \vdash \mu' \; \tt{\bf{wf}}$, 
	$\Gamma; \Sigma' \vdash e' : T$.
\end{theorem}
\begin{proof}
By induction on the derivation of $\mu \; | \; e \; \rightarrow \mu' \; | \; e'$.
\begin{casethm}[\textsc{R-New}]
 \begin{mathpar}
\inferrule
  {\mu \vdash \overline{d_v} \leadsto \overline{d} \\
  	l \notin dom(\mu) \\
  	\mu' = \mu, l \mapsto \{\texttt{z} \Rightarrow \overline{d}\}}
  {\mu \; | \; \texttt{new} \; \{\texttt{z} \Rightarrow \overline{d_v}\} \; \rightarrow \mu' \; | \; l}
	\and
\inferrule
	{\Gamma, z : \{z \Rightarrow \overline{\sigma}\}; \Sigma 
	\vdash \overline{d_v} : \overline{\sigma} \\
	y \notin dom(\Gamma)}
	{\Gamma; \Sigma\vdash \texttt{new} \; \{z \Rightarrow \overline{d_v}\} : 
	\{z \Rightarrow \overline{\sigma}\}}
\end{mathpar}
Let $\Sigma' = \Sigma, (l:\{z \Rightarrow \overline{\sigma}\})$.
By Lemma \ref{lem:path_type_preservation}, we can show that 
$\Gamma; \Sigma' \vdash \{\texttt{z} \Rightarrow \overline{d}\} :
\{z \Rightarrow \overline{\sigma}\}$. Thus by \textsc{T-Loc} we get the result 
$\Gamma; \Sigma' \vdash l : \{z \Rightarrow \overline{\sigma}\}$.
\end{casethm}

\begin{casethm}[\textsc{R-Meth}]
\begin{mathpar}
\inferrule
  {\mu \vdash v_1 \leadsto l \\
  	\mu(l) = \{\texttt{z} \Rightarrow ...,m:T(x:S)=e,...\}}
  {\mu \; | \; v_1.m_U(v_2) \;\rightarrow \mu \; | \; [v_1/z,v_2 \unlhd S/x]e \unlhd U}
	\and
\inferrule
  {\Gamma; \Sigma \vdash v_1 \ni \texttt{def} \; m:S' \rightarrow T' \\
  	\Gamma; \Sigma \vdash v_1 : T_1 \\
  	\Gamma; \Sigma \vdash v_2 : S' \\
  	\Gamma; \Sigma \vdash T' <: U}
  {	\Gamma; \Sigma \vdash v_1.m_U(v_2) : U}
\end{mathpar}
By Lemma \ref{lem:path_type_preservation} we know that 
$\Gamma; \Sigma \vdash l : T_1$. From Lemma \ref{lem:subtype:decl} 
we can show that $\Gamma; \Sigma \vdash [v_1/z](\texttt{def} \; m:S \rightarrow T) <:
\texttt{def} \; m:S' \rightarrow T'$ which implies 
$\Gamma; \Sigma \vdash [v_1/z] T <: T'$. Since 
$\Gamma; \Sigma \vdash e : T$, by Lemma \ref{lem:subst} 
we get $\Gamma; \Sigma \vdash [v_1/z,v_2 \unlhd S/x]e : [v_1/z,v_2 \unlhd S/x]T$.
Since $T$ is well-formed in the absence of $x$, 
$[v_1/z,v_2 \unlhd S/x]T = [v_1/z]T$. Thus we get 
$\Gamma; \Sigma \vdash [v_1/z,v_2 \unlhd S/x]e : [v_1/z]T$, and by 
\textsc{T-Sub} we have 
$\Gamma; \Sigma \vdash [v_1/z,v_2 \unlhd S/x]e : T'$ and subsequently 
$\Gamma; \Sigma \vdash [v_1/z,v_2 \unlhd S/x]e : U$. 
Now, let $\Sigma' = \Sigma$, and by \textsc{T-Type} 
we have $\Gamma; \Sigma' \vdash [v_1/z,v_2 \unlhd S/x]e \unlhd U : U$

\end{casethm}
\begin{casethm}[\textsc{R-Context}]
\begin{mathpar}
\inferrule
  {	\mu \; | \; e \; \rightarrow \; \mu' \; | \; e'}
  {\mu \; | \; E[e] \; \rightarrow \mu' \; | \; E[e']}
\end{mathpar}
By case analysis on $E[e]$.
\begin{subcase}[$\mu \; | \; e.m_U(e_S) \; \rightarrow \mu' \; | \; e'.m_U(e_S)$]
\begin{mathpar}
\inferrule
  {\Gamma; \Sigma \vdash e \ni \texttt{def} \; m:S' \rightarrow T' \\
  	\Gamma; \Sigma \vdash e : T \\
  	\Gamma; \Sigma \vdash e_S : S' \\
  	\Gamma; \Sigma \vdash T' <: U}
  {	\Gamma; \Sigma \vdash e.m_U(e_S) : U}
\end{mathpar}
By our induction hypothesis we assume 
$\Gamma; \Sigma' \vdash e' : T$. We now need to investigate 
method membership for $e'$. Since $e$ reduces to $e'$, we can be
sure that $e$ is not a path. By inversion on 
$\Gamma; \Sigma \vdash e \ni \texttt{def} \; m:S' \rightarrow T'$ it
follows that since $e$ is not a path, $z \notin \texttt{def} \; m:S' \rightarrow T'$. 
Thus $\Gamma; \Sigma' \vdash e' \ni \texttt{def} \; m:S' \rightarrow T'$, since 
no substitution is required. By simple weakening of the store type (\hl{TODO?})
the following hold.
\begin{mathpar}
\inferrule
  {\Gamma; \Sigma' \vdash e_S : S' \\
  	\Gamma; \Sigma' \vdash T' <: U}
  {}
\end{mathpar}
By \textsc{T-Meth} we have $\Gamma; \Sigma' \vdash e.m_U(e_S) : U$
\end{subcase}

\begin{subcase}[$\mu \; | \; p.m_U(e) \; \rightarrow \mu' \; | \; p.m_U(e')$]
\begin{mathpar}
\inferrule
  {\Gamma; \Sigma \vdash p \ni \texttt{def} \; m:S \rightarrow T \\
  	\Gamma; \Sigma \vdash p : T_p \\
  	\Gamma; \Sigma \vdash e : S \\
  	\Gamma; \Sigma \vdash T <: U}
  {	\Gamma; \Sigma \vdash p.m_U(e) : U}
\end{mathpar}
By our induction hypothesis we assume 
$\exists \Sigma' : \Gamma; \Sigma \vdash e : S$, where 
$\Sigma'$ extends $\Sigma$. By simple weakening of the store type (\hl{TODO?})
we get
\begin{mathpar}
\inferrule
  {\Gamma; \Sigma' \vdash p \ni \texttt{def} \; m:S \rightarrow T \\
  	\Gamma; \Sigma' \vdash p : T_p \\
  	\Gamma; \Sigma' \vdash T <: U}
  {}
\end{mathpar}
Thus by 
\textsc{T-Meth}, $\Gamma; \Sigma' \vdash p.m_U(e') : U$
\end{subcase}

\begin{subcase}[$\mu \; | \; e.f \; \rightarrow \mu' \; | \; e'.f$]
\begin{mathpar}
\inferrule
  {	\Gamma; \Sigma \vdash e : S \\
  	\Gamma; \Sigma \vdash e \ni \texttt{val} \; f:T}
  {	\Gamma; \Sigma \vdash e.f : T}
\end{mathpar}
By our induction hypothesis we assume $\Gamma; \Sigma \vdash e' : S$. 
Since $e$ reduces to $e'$ it follows that $e$ is not a path. By inversion 
on the derivation of $\Gamma; \Sigma \vdash e \ni \texttt{val} \; f:T$, 
$z \notin \texttt{val} \; f:T$, and it follows that 
$\Gamma; \Sigma' \vdash e' \ni \texttt{val} \; f:T$. Now, by \textsc{T-Field} 
we have $\Gamma; \Sigma' \vdash e'.f : S$
\end{subcase}

\begin{subcase}[$\mu \; | \; e \unlhd T \; \rightarrow \mu' \; | \; e' \unlhd T$]
\begin{mathpar}
\inferrule
  {	\Gamma; \Sigma \vdash e : T}
  {	\Gamma; \Sigma \vdash e \unlhd T : T}
\end{mathpar}
By our induction hypothesis we assume $\Gamma; \Sigma' \vdash e' : T$. 
Now by the simple application of \textsc{T-Type} we have 
$\Gamma; \Sigma' \vdash e' \unlhd : T$
\end{subcase}

\end{casethm}
\end{proof}
\qed

\newpage

%---------------------- Progress ----------------------%
\subsection{Progress}
\begin{theorem}[Progress]
If $\Gamma \vdash e : T$, then either
\begin{enumerate}
\item e is a value, or
\item $\forall \mu$ s.t.
		   $\Gamma \vdash \mu$,
         $\exists e'$ and $\mu'$ s.t. 
         $\mu \; | \; e \; \rightarrow \mu' \; | \; e'$
\end{enumerate}
\end{theorem}
\qed 

\newpage

%---------------------- Preservation ----------------------%
\begin{theorem}[Preservation]
If $\Gamma; \Sigma \vdash e : T$, 
   	$\mu \; | \; e \; \rightarrow \mu' \; | \; e'$ where
	$\Sigma \vdash \mu \; \tt{\bf{wf}}$ then 
 	$\exists \Sigma'$ s.t. 
	$\Sigma'$ extends $\Sigma$, 
	$\Sigma' \vdash \mu' \; \tt{\bf{wf}}$, 
	$\Gamma; \Sigma' \vdash e' : T$.
\end{theorem}
\begin{proof}
By induction on the derivation of $\mu \; | \; e \; \rightarrow \mu' \; | \; e'$.
\begin{casethm}[\textsc{R-New}]
 \begin{mathpar}
\inferrule
  {\mu \vdash \overline{d_v} \leadsto \overline{d} \\
  	l \notin dom(\mu) \\
  	\mu' = \mu, l \mapsto \{\texttt{z} \Rightarrow \overline{d}\}}
  {\mu \; | \; \texttt{new} \; \{\texttt{z} \Rightarrow \overline{d_v}\} \; \rightarrow \mu' \; | \; l}
	\and
\inferrule
	{\Gamma, z : \{z \Rightarrow \overline{\sigma}\}; \Sigma 
	\vdash \overline{d_v} : \overline{\sigma} \\
	y \notin dom(\Gamma)}
	{\Gamma; \Sigma\vdash \texttt{new} \; \{z \Rightarrow \overline{d_v}\} : 
	\{z \Rightarrow \overline{\sigma}\}}
\end{mathpar}
Let $\Sigma' = \Sigma, (l:\{z \Rightarrow \overline{\sigma}\})$.
By Lemma \ref{lem:path_type_preservation}, we can show that 
$\Gamma; \Sigma' \vdash \{\texttt{z} \Rightarrow \overline{d}\} :
\{z \Rightarrow \overline{\sigma}\}$. Thus by \textsc{T-Loc} we get the result 
$\Gamma; \Sigma' \vdash l : \{z \Rightarrow \overline{\sigma}\}$.
\end{casethm}

\begin{casethm}[\textsc{R-Meth}]
\begin{mathpar}
\inferrule
  {\mu \vdash v_1 \leadsto l \\
  	\mu(l) = \{\texttt{z} \Rightarrow ...,m:T(x:S)=e,...\}}
  {\mu \; | \; v_1.m_U(v_2) \;\rightarrow \mu \; | \; [v_1/z,v_2 \unlhd S/x]e \unlhd U}
	\and
\inferrule
  {\Gamma; \Sigma \vdash v_1 \ni \texttt{def} \; m:S' \rightarrow T' \\
  	\Gamma; \Sigma \vdash v_1 : T_1 \\
  	\Gamma; \Sigma \vdash v_2 : S' \\
  	\Gamma; \Sigma \vdash T' <: U}
  {	\Gamma; \Sigma \vdash v_1.m_U(v_2) : U}
\end{mathpar}
By Lemma \ref{lem:path_type_preservation} we know that 
$\Gamma; \Sigma \vdash l : T_1$. From Lemma \ref{lem:subtype:decl} 
we can show that $\Gamma; \Sigma \vdash [v_1/z](\texttt{def} \; m:S \rightarrow T) <:
\texttt{def} \; m:S' \rightarrow T'$ which implies 
$\Gamma; \Sigma \vdash [v_1/z] T <: T'$. Since 
$\Gamma; \Sigma \vdash e : T$, by Lemma \ref{lem:subst} 
we get $\Gamma; \Sigma \vdash [v_1/z,v_2 \unlhd S/x]e : [v_1/z,v_2 \unlhd S/x]T$.
Since $T$ is well-formed in the absence of $x$, 
$[v_1/z,v_2 \unlhd S/x]T = [v_1/z]T$. Thus we get 
$\Gamma; \Sigma \vdash [v_1/z,v_2 \unlhd S/x]e : [v_1/z]T$, and by 
\textsc{T-Sub} we have 
$\Gamma; \Sigma \vdash [v_1/z,v_2 \unlhd S/x]e : T'$ and subsequently 
$\Gamma; \Sigma \vdash [v_1/z,v_2 \unlhd S/x]e : U$. 
Now, let $\Sigma' = \Sigma$, and by \textsc{T-Type} 
we have $\Gamma; \Sigma' \vdash [v_1/z,v_2 \unlhd S/x]e \unlhd U : U$

\end{casethm}
\begin{casethm}[\textsc{R-Context}]
\begin{mathpar}
\inferrule
  {	\mu \; | \; e \; \rightarrow \; \mu' \; | \; e'}
  {\mu \; | \; E[e] \; \rightarrow \mu' \; | \; E[e']}
\end{mathpar}
By case analysis on $E[e]$.
\begin{subcase}[$\mu \; | \; e.m_U(e_S) \; \rightarrow \mu' \; | \; e'.m_U(e_S)$]
\begin{mathpar}
\inferrule
  {\Gamma; \Sigma \vdash e \ni \texttt{def} \; m:S' \rightarrow T' \\
  	\Gamma; \Sigma \vdash e : T \\
  	\Gamma; \Sigma \vdash e_S : S' \\
  	\Gamma; \Sigma \vdash T' <: U}
  {	\Gamma; \Sigma \vdash e.m_U(e_S) : U}
\end{mathpar}
By our induction hypothesis we assume 
$\Gamma; \Sigma' \vdash e' : T$. We now need to investigate 
method membership for $e'$. Since $e$ reduces to $e'$, we can be
sure that $e$ is not a path. By inversion on 
$\Gamma; \Sigma \vdash e \ni \texttt{def} \; m:S' \rightarrow T'$ it
follows that since $e$ is not a path, $z \notin \texttt{def} \; m:S' \rightarrow T'$. 
Thus $\Gamma; \Sigma' \vdash e' \ni \texttt{def} \; m:S' \rightarrow T'$, since 
no substitution is required. By simple weakening of the store type (\hl{TODO?})
the following hold.
\begin{mathpar}
\inferrule
  {\Gamma; \Sigma' \vdash e_S : S' \\
  	\Gamma; \Sigma' \vdash T' <: U}
  {}
\end{mathpar}
By \textsc{T-Meth} we have $\Gamma; \Sigma' \vdash e.m_U(e_S) : U$
\end{subcase}

\begin{subcase}[$\mu \; | \; p.m_U(e) \; \rightarrow \mu' \; | \; p.m_U(e')$]
\begin{mathpar}
\inferrule
  {\Gamma; \Sigma \vdash p \ni \texttt{def} \; m:S \rightarrow T \\
  	\Gamma; \Sigma \vdash p : T_p \\
  	\Gamma; \Sigma \vdash e : S \\
  	\Gamma; \Sigma \vdash T <: U}
  {	\Gamma; \Sigma \vdash p.m_U(e) : U}
\end{mathpar}
By our induction hypothesis we assume 
$\exists \Sigma' : \Gamma; \Sigma \vdash e : S$, where 
$\Sigma'$ extends $\Sigma$. By simple weakening of the store type (\hl{TODO?})
we get
\begin{mathpar}
\inferrule
  {\Gamma; \Sigma' \vdash p \ni \texttt{def} \; m:S \rightarrow T \\
  	\Gamma; \Sigma' \vdash p : T_p \\
  	\Gamma; \Sigma' \vdash T <: U}
  {}
\end{mathpar}
Thus by 
\textsc{T-Meth}, $\Gamma; \Sigma' \vdash p.m_U(e') : U$
\end{subcase}

\begin{subcase}[$\mu \; | \; e.f \; \rightarrow \mu' \; | \; e'.f$]
\begin{mathpar}
\inferrule
  {	\Gamma; \Sigma \vdash e : S \\
  	\Gamma; \Sigma \vdash e \ni \texttt{val} \; f:T}
  {	\Gamma; \Sigma \vdash e.f : T}
\end{mathpar}
By our induction hypothesis we assume $\Gamma; \Sigma \vdash e' : S$. 
Since $e$ reduces to $e'$ it follows that $e$ is not a path. By inversion 
on the derivation of $\Gamma; \Sigma \vdash e \ni \texttt{val} \; f:T$, 
$z \notin \texttt{val} \; f:T$, and it follows that 
$\Gamma; \Sigma' \vdash e' \ni \texttt{val} \; f:T$. Now, by \textsc{T-Field} 
we have $\Gamma; \Sigma' \vdash e'.f : S$
\end{subcase}

\begin{subcase}[$\mu \; | \; e \unlhd T \; \rightarrow \mu' \; | \; e' \unlhd T$]
\begin{mathpar}
\inferrule
  {	\Gamma; \Sigma \vdash e : T}
  {	\Gamma; \Sigma \vdash e \unlhd T : T}
\end{mathpar}
By our induction hypothesis we assume $\Gamma; \Sigma' \vdash e' : T$. 
Now by the simple application of \textsc{T-Type} we have 
$\Gamma; \Sigma' \vdash e' \unlhd : T$
\end{subcase}

\end{casethm}
\end{proof}
\qed

\newpage

\begin{lemma}[Subtype Transitivity]\label{thm:trans}
If $\varnothing; \Sigma; \Gamma_1 \vdash S <: T \dashv \Gamma_2$ and
   $\varnothing; \Sigma; \Gamma_2 \vdash T <: U \dashv \Gamma_3$ then
	$\varnothing; \Sigma; \Gamma_1 \vdash S <: U \dashv \Gamma_3$.
\end{lemma}
\begin{proof}
By structural induction on 
$\varnothing; \Sigma; \Gamma_1 \vdash S <: T \dashv \Gamma_2$.
\begin{casethm}[\textsc{S-Refl}]
\begin{mathpar}
\inferrule
  {}
  {\varnothing; \Sigma; \Gamma \vdash T\; \texttt{<:}\; T \dashv \Gamma}
\end{mathpar}
Trivial.
\end{casethm}

\begin{casethm}[\textsc{S-Assume}]
\begin{mathpar}
\inferrule
  {(S <: T) \in \varnothing}
  {\varnothing; \Sigma; \Gamma_1 \vdash S\; \texttt{<:}\; T \dashv \Gamma_2}
\end{mathpar}
Trivial.
\end{casethm}

\begin{casethm}[\textsc{S-Rec}]
\begin{mathpar}
\inferrule
  {A' = \varnothing,(\{z \Rightarrow \overline{\sigma}\} <: \{z \Rightarrow \overline{\sigma}'\})\\
  	A'; \Sigma; \Gamma_1, z : \{z \Rightarrow \overline{\sigma}\} \vdash \overline{\sigma} <:\; \overline{\sigma}'  \dashv \Gamma_2, z : \{z \Rightarrow \overline{\sigma}'\}}
  {\varnothing; \Sigma; \Gamma_1 \vdash \{z \Rightarrow \overline{\sigma}\}\; <:\; \{z \Rightarrow \overline{\sigma}'\}\dashv \Gamma_2}
\end{mathpar}
\end{casethm}

\begin{casethm}[\textsc{S-Select-Refl}]
\begin{mathpar}
\inferrule
  {\varnothing; \Sigma; \Gamma_1 \vdash p \ni \texttt{type} \; L : S_1 .. U_1\\
  	\varnothing; \Sigma; \Gamma_2 \vdash p \ni \texttt{type} \; L : S_2 .. U_2\\
  	\varnothing; \Sigma; \Gamma_2 \vdash S_2 \; \texttt{<:}\; S_1 \dashv \Gamma_1 \\
  	\varnothing; \Sigma; \Gamma_1 \vdash U_1 \; \texttt{<:}\; U_2 \dashv \Gamma_2}
  {\varnothing; \Sigma; \Gamma_1 \vdash p.L \; \texttt{<:}\; p.L \dashv \Gamma_2}
\end{mathpar}
\end{casethm}

\begin{casethm}[\textsc{S-Select-Upper}]
\begin{mathpar}
\inferrule
  {\varnothing; \Sigma; \Gamma_1 \vdash p \ni \texttt{type} \; L : S .. U\\
  	\varnothing; \Sigma; \Gamma_1 \vdash S <: U \dashv \Gamma_1 \\
  	\varnothing; \Sigma; \Gamma_1 \vdash U <: U' \dashv \Gamma_2}
  {\varnothing; \Sigma; \Gamma_1 \vdash p.L\; <:\; U' \dashv \Gamma_2}
\end{mathpar}
\end{casethm}

\begin{casethm}[\textsc{S-Select-Lower}]
\begin{mathpar}
\inferrule
  {\varnothing; \Sigma; \Gamma_2 \vdash p \ni \texttt{type} \; L : S .. U \\
  	\varnothing; \Sigma; \Gamma_2 \vdash S <: U \dashv \Gamma_2 \\
  	\varnothing; \Sigma; \Gamma_1 \vdash S' <: S \dashv \Gamma_2}
  {\varnothing; \Sigma; \Gamma_1 \vdash S'\; <:\; p.L \dashv \Gamma_2}
\end{mathpar}
\end{casethm}

\begin{casethm}[\textsc{S-Top}]
\begin{mathpar}
\inferrule
  {}
  {\varnothing; \Sigma; \Gamma_1 \vdash T\; \texttt{<:}\; \top \dashv \Gamma_2}
\end{mathpar}
Trivial?
\end{casethm}

\begin{casethm}[\textsc{S-Bottom}]
\begin{mathpar}
\inferrule
  {}
  {\varnothing; \Sigma; \Gamma_1 \vdash \bot\; \texttt{<:}\; T \dashv \Gamma_2}
\end{mathpar}
Trivial.
\end{casethm}

\end{proof}
\qed

\newpage

\begin{lemma} \label{lem:path_type_preservation}
If $\varnothing; \Sigma; \Gamma \vdash v : T$, 
$\Sigma \vdash  \mu$ and $\mu \vdash v \leadsto l$ then 
$\varnothing; \Sigma; \Gamma \vdash l : S$ where
$\varnothing; \Sigma; \Gamma \vdash S <: T \dashv \Gamma$.
\end{lemma}
\begin{proof}
By induction on the derivation of $\mu \vdash v \leadsto l$.
\begin{casethm}[\textsc{L-Loc}]
\begin{mathpar}
\inferrule
  {v = l}
  {}
  \and
\inferrule
  {}
  {\mu \vdash l \leadsto l}
\end{mathpar}
Trivial.
\end{casethm}
\begin{casethm}[\textsc{L-Type}]
\begin{mathpar}
  {v = v' \unlhd T}
  {}
  \and
\inferrule
  {\mu \vdash v' \leadsto l}
  {\mu \vdash v' \unlhd T \leadsto l}
\end{mathpar}
Since we know that $\varnothing; \Sigma; \Gamma \vdash v' \unlhd T: T$, 
by \textsc{T-Type} we can infer $\varnothing; \Sigma; \Gamma \vdash v' : S'$ 
and $\varnothing; \Sigma; \Gamma \vdash S' <: T \dashv \Gamma$.
By our induction hypothesis, we assume that if 
$\varnothing; \Sigma; \Gamma \vdash v' : S'$ then 
$\varnothing; \Sigma; \Gamma \vdash l : S$, where 
$\varnothing; \Sigma; \Gamma \vdash S <: S' \dashv \Gamma$.
By \hl{Subtype Transitivity} we have 
$\varnothing; \Sigma; \Gamma \vdash S <: T \dashv \Gamma$.
\end{casethm}
\begin{casethm}[\textsc{L-Path}]
\begin{mathpar}
\inferrule
  {\mu \vdash v' \leadsto l' \\
	\mu(l') = \{z \Rightarrow ..., \texttt{val} f : T = l, ...\}}
  {\mu \vdash v'.f \leadsto l}
\end{mathpar}
By inversion on the derivation of $\Gamma; \Sigma \vdash v'.f : T$ we 
have 
\begin{mathpar}
\inferrule
  {	\varnothing; \Sigma; \Gamma \vdash v' : S \\
  	\varnothing; \Sigma; \Gamma \vdash v' \ni \texttt{val} \; f:T}
  {	\varnothing; \Sigma; \Gamma \vdash v'.f : T}
\end{mathpar}
By assumption $\Sigma \vdash \mu$, which implies $\varnothing; \Sigma; \varnothing 
\vdash l : S$, where 
$\varnothing; \Sigma; \varnothing \vdash S <: S' \dashv \varnothing$. 
Environment weakening gives us 
$\varnothing; \Sigma; \Gamma \vdash l : T$
and $\varnothing; \Sigma; \Gamma \vdash S <: S' \dashv \Gamma$
\end{casethm}
\qed
\end{proof}

\newpage

Now we present the proof for expression substitution. 
During method call reduction we need to substitute 
method parameters with arguments into the method body.
We want the
substitution of paths to maintain the expression 
typing relation. That is, if $e$ has type $T$, then 
$[p\unlhd U/x]e$ has type $[p\unlhd U/x]T$, where $U$ is 
the type of $x$ in the current environment.
Since types may contain 
variables as part of a selection type we must also 
perform a substitution on the environment and the assumption context. 
The full lemma is stated below.
\begin{mathpar}
\inferrule
  {A; \Sigma; \Gamma_1, (x : U), \Gamma_2 \vdash e : T \\
  	x \notin \, \Gamma_1\\
  	A; \Sigma; \Gamma_1 \vdash p : S \\
  	A; \Sigma; \Gamma_1 \vdash S <: U \dashv \Gamma_1}
  {[p \unlhd U/x]A; \Sigma; , \Gamma_1, [p \unlhd U/x]\Gamma_2 \vdash [p \unlhd U/x]e : [p \unlhd U/x]T}
  \quad (\textsc {$:$ - Substitution})
\end{mathpar}
If we have an expression $e$ that has type $T$ in an environment
containing a variable $x$ of type $U$, then the substitution 
of a path $p$ of the appropriate type maintains the 
typing relation assuming the correct substitutions 
are made in the environment and assumption context.
We make the restriction that $p$ be well-typed in the 
absence of $x$. Since the typing relation is defined in a mutually 
dependent manner with the subtyping, membership and 
expansion relations, we must perform a mutual induction
on these relations. We now need to construct a similar 
lemma for each of these relations.
\begin{mathpar}
\inferrule
  {A; \Sigma; \Gamma_1, (x : U), \Gamma_2 \vdash T <: T' \dashv\Gamma_1, (x : U), \Gamma_2' \\
  	x \notin \, \Gamma_1\\\\
  	A; \Sigma; \Gamma_1 \vdash p : S \\
  	A; \Sigma; \Gamma_1 \vdash S <: U \dashv \Gamma_1}
  {[p \unlhd U/x]A; \Sigma; , \Gamma_1, [p \unlhd U/x]\Gamma_2 \vdash [p \unlhd U/x]T <: [p \unlhd U/x]T' \dashv \Gamma_1, [p \unlhd U/x]\Gamma_2'}
  \quad (\textsc {$<:$ - Substitution})
  \and
\inferrule
  {A; \Sigma; \Gamma_1, (x : U), \Gamma_2 \vdash e \ni \sigma \\
  	x \notin \, \Gamma_1\\
  	A; \Sigma; \Gamma_1 \vdash p : S \\
  	A; \Sigma; \Gamma_1 \vdash S <: U \dashv \Gamma_1}
  {[p \unlhd U/x]A; \Sigma; , \Gamma_1, [p \unlhd U/x]\Gamma_2 \vdash [p \unlhd U/x]e \ni [p \unlhd U/x]\sigma}
  \quad (\textsc {$\ni$ - Substitution})
  \and
\inferrule
  {A; \Sigma; \Gamma_1, (x : U), \Gamma_2 \vdash T \prec_z \overline{\sigma} \\
  	x \notin \, \Gamma_1\\
  	A; \Sigma; \Gamma_1 \vdash p : S \\
  	A; \Sigma; \Gamma_1 \vdash S <: U \dashv \Gamma_1}
  {[p \unlhd U/x]A; \Sigma; , \Gamma_1, [p \unlhd U/x]\Gamma_2 \vdash [p \unlhd U/x]T \prec_z [p \unlhd U/x]\overline{\sigma}}
  \quad (\textsc {$\prec$ - Substitution})
\end{mathpar}
Each of these lemmas is just a similar statement of expression 
substitution for each relation. The only major difference is 
for \textsc{$<:$ - Substitution}. Since we work with a double 
headed subtyping relation that maintains two environments, 
we have two different environments in the statement of this
lemma. While we don't use different environments in any other 
parts of the type system, this is done for the purposes of 
induction that can be seen in case \ref{case:subst:s-rec} below.

\begin{lemma}[Expression Substitution]\label{lem:subst}
\begin{mathpar}
\inferrule
  {A; \Sigma; \Gamma_1, (x : U), \Gamma_2 \vdash e : T \\
  	x \notin \, \Gamma_1\\
  	A; \Sigma; \Gamma_1 \vdash p : S \\
  	A; \Sigma; \Gamma_1 \vdash S <: U \dashv \Gamma_1}
  {[p \unlhd U/x]A; \Sigma; , \Gamma_1, [p \unlhd U/x]\Gamma_2 \vdash [p \unlhd U/x]e : [p \unlhd U/x]T}
  \quad (\textsc {$:$ - Substitution})
  \and
\inferrule
  {A; \Sigma; \Gamma_1, (x : U), \Gamma_2 \vdash T <: T' \dashv\Gamma_1, (x : U), \Gamma_2' \\
  	x \notin \, \Gamma_1\\\\
  	A; \Sigma; \Gamma_1 \vdash p : S \\
  	A; \Sigma; \Gamma_1 \vdash S <: U \dashv \Gamma_1}
  {[p \unlhd U/x]A; \Sigma; , \Gamma_1, [p \unlhd U/x]\Gamma_2 \vdash [p \unlhd U/x]T <: [p \unlhd U/x]T' \dashv \Gamma_1, [p \unlhd U/x]\Gamma_2'}
  \quad (\textsc {$<:$ - Substitution})
  \and
\inferrule
  {A; \Sigma; \Gamma_1, (x : U), \Gamma_2 \vdash e \ni \sigma \\
  	x \notin \, \Gamma_1\\
  	A; \Sigma; \Gamma_1 \vdash p : S \\
  	A; \Sigma; \Gamma_1 \vdash S <: U \dashv \Gamma_1}
  {[p \unlhd U/x]A; \Sigma; , \Gamma_1, [p \unlhd U/x]\Gamma_2 \vdash [p \unlhd U/x]e \ni [p \unlhd U/x]\sigma}
  \quad (\textsc {$\ni$ - Substitution})
  \and
\inferrule
  {A; \Sigma; \Gamma_1, (x : U), \Gamma_2 \vdash T \prec_z \overline{\sigma} \\
  	x \notin \, \Gamma_1\\
  	A; \Sigma; \Gamma_1 \vdash p : S \\
  	A; \Sigma; \Gamma_1 \vdash S <: U \dashv \Gamma_1}
  {[p \unlhd U/x]A; \Sigma; , \Gamma_1, [p \unlhd U/x]\Gamma_2 \vdash [p \unlhd U/x]T \prec_z [p \unlhd U/x]\overline{\sigma}}
  \quad (\textsc {$\prec$ - Substitution})
\end{mathpar}
\end{lemma}
\begin{proof}
By structural induction on 
$\varnothing; \Sigma; \Gamma, (x : U) \vdash e : T$.
\begin{casethm}[\textsc{T-Var}]
\begin{mathpar}
\inferrule
  {e = y \\
  	T = \Gamma(y)}
  {}
  \and
\inferrule
  {y \in dom(\Gamma_1, (x : U), \Gamma_2)}
  {	A; \Sigma; \Gamma_1, (x : U), \Gamma_2 \vdash y : \Gamma_1, (x : U), \Gamma_2(y)}
\end{mathpar}
Case analysis on $x = y$.
\begin{subcase}[$x = y$]
\begin{mathpar}
\inferrule
  {[p\unlhd U/x]y = p\unlhd U \\
  	[p\unlhd U/x]U = U}
  {}
\end{mathpar}
The substitutions are given above. 
Since $U$ is typed absent 
$x$, $[p\unlhd U/x]U = U$.
By assumption we have $A; \Sigma; \Gamma_1 \vdash p : S$ 
and $A; \Sigma; \Gamma_1 \vdash S <: U \dashv \Gamma_1$.
\hl{Weakening} gives us 
\begin{mathpar}
\inferrule
  {A; \Sigma; \Gamma_1, [p\unlhd U/x]\Gamma_2 \vdash p : S \\
  	A; \Sigma; \Gamma_1, [p\unlhd U/x]\Gamma_2 \vdash S <: U \dashv \Gamma_1, [p\unlhd U/x]\Gamma_2}
  {}
\end{mathpar}
Thus by \textsc{T-Type}, 
$\varnothing; \Sigma; \Gamma_1, [p\unlhd U/x]\Gamma_2 \vdash p \unlhd U : U$.
\end{subcase}
\begin{subcase}[$x \neq y$]
\begin{mathpar}
\inferrule
  {[p\unlhd U/x]y = y}
  {}
\end{mathpar}
The substitutions are given above.Thus by \textsc{T-Var},
$A; \Sigma; \Gamma_1, [p\unlhd U/x]\Gamma_2 \vdash y : \Gamma_1, [p\unlhd U/x]\Gamma_2(y)$.
\end{subcase}
\end{casethm}

\begin{casethm}[\textsc{T-Loc}]
\begin{mathpar}
\inferrule
  {	l \in dom(\Sigma)}
  {	A; \Sigma; \Gamma, (x : U) \vdash l : \Sigma(l)}
\end{mathpar}
Trivial.
\end{casethm}

\begin{casethm}[\textsc{T-New}]
\begin{mathpar}
\inferrule
  {A; \Sigma; \Gamma_1, (x : U), \Gamma_2, z : \{z \Rightarrow \overline{\sigma}\} 
  \vdash \overline{d} : \overline{\sigma}}
  {A; \Sigma; \Gamma_1, (x : U), \Gamma_2 \vdash \texttt{new} \; \{z \Rightarrow \overline{d}\} : 
  \{z \Rightarrow \overline{\sigma}\}}
\end{mathpar}
By our induction hypothesis we assume 
\begin{mathpar}
\inferrule
  {A; \Sigma; \Gamma_1, [p\unlhd U/x]\Gamma_2, z : \{z \Rightarrow [p\unlhd U/x] \overline{\sigma}\} 
  \vdash [p\unlhd U/x]\overline{d} : [p\unlhd U/x]\overline{\sigma}}
  {}
\end{mathpar}
Thus by \textsc{T-New} we have 
\begin{mathpar}
\inferrule
  {A; \Sigma; \Gamma_1, [p\unlhd U/x]\Gamma_2 \vdash \texttt{new} \; \{z \Rightarrow [p\unlhd U/x]\overline{d}\} : 
  \{z \Rightarrow [p\unlhd U/x]\overline{\sigma}\}}
  {}
\end{mathpar}
\end{casethm}

\begin{casethm}[\textsc{T-Meth}]
\begin{mathpar}
\inferrule
  {A; \Sigma; \Gamma_1, (x : U), \Gamma_2 \vdash e_0 \ni \texttt{def} \; m:S \rightarrow T \\
  	A; \Sigma; \Gamma_1, (x : U), \Gamma_2 \vdash e_0 : T_0 \\
  	A; \Sigma; \Gamma_1, (x : U), \Gamma_2 \vdash e_1 : S' \\
  	A; \Sigma; \Gamma_1, (x : U), \Gamma_2 \vdash S' <: S \dashv \Gamma_1, (x : U), \Gamma_2 \\
  	A; \Sigma; \Gamma_1, (x : U), \Gamma_2 \vdash T <: T' \dashv \Gamma_1, (x : U), \Gamma_2}
  {A; 	\Sigma; \Gamma_1, (x : U), \Gamma_2 \vdash e_0.m_T'(e_1) : T'}
\end{mathpar}
By our mutual induction hypothesis we assume 
\begin{mathpar}
\inferrule
  {[p\unlhd U/x]A; \Sigma; \Gamma_1, [p\unlhd U/x]\Gamma_2 \vdash [p\unlhd U/x]e_0 \ni [p\unlhd U/x](\texttt{def} \; m:S \rightarrow T) \\
  	[p\unlhd U/x]A; \Sigma; \Gamma_1, [p\unlhd U/x]\Gamma_2 \vdash [p\unlhd U/x]e_0 : [p\unlhd U/x]T_0 \\
  	[p\unlhd U/x]A; \Sigma; \Gamma_1, [p\unlhd U/x]\Gamma_2 \vdash [p\unlhd U/x]e_1 : [p\unlhd U/x]S' \\
  	[p\unlhd U/x]A; \Sigma; \Gamma_1, [p\unlhd U/x]\Gamma_2 \vdash [p\unlhd U/x]S' <: [p\unlhd U/x]S \dashv \Gamma_1, [p\unlhd U/x]\Gamma_2 \\
  	A; \Sigma; \Gamma_1, [p\unlhd U/x]\Gamma_2 \vdash [p\unlhd U/x]T <: [p\unlhd U/x]T' \dashv \Gamma_1, [p\unlhd U/x]\Gamma_2}
  {}
\end{mathpar}
It follows then by \textsc{T-Meth} that 
$[p\unlhd U/x]A; \Sigma; \Gamma_1, [p\unlhd U/x]\Gamma_2 \vdash [p\unlhd U/x]e_0.m_{[p\unlhd U/x]T'}([p\unlhd U/x]e_1) : [p\unlhd U/x]T'$.
\end{casethm}

\begin{casethm}[\textsc{T-Acc}]
\begin{mathpar}
\inferrule
  {A; 	\Sigma; \Gamma_1, (x : U), \Gamma_2 \vdash e : S \\
  	A; 	\Sigma; \Gamma_1, (x : U), \Gamma_2 \vdash e \ni \texttt{val} \; f:T}
  {A; 	\Sigma; \Gamma_1, (x : U), \Gamma_2 \vdash e.f : T}
\end{mathpar}
By our induction hypothesis, we assume
\begin{mathpar}
\inferrule
  {[p\unlhd U/x]A; \Sigma; \Gamma_1, [p\unlhd U/x]\Gamma_2 \vdash [p\unlhd U/x]e : [p\unlhd U/x]S \\
  	[p\unlhd U/x]A; \Sigma; \Gamma_1, [p\unlhd U/x]\Gamma_2 \vdash [p\unlhd U/x]e \ni [p\unlhd U/x](\texttt{val} \; f:T)}
  {}
\end{mathpar}
It then follows by \textsc{T-Acc} that
$[p\unlhd U/x]A; \Sigma; \Gamma_1, [p\unlhd U/x]\Gamma_2 \vdash [p\unlhd U/x]e.f : [p\unlhd U/x]T$.
\end{casethm}

\begin{casethm}[\textsc{T-Type}]
\begin{mathpar}
\inferrule
  {A;\Sigma; \Gamma_1, (x : U), \Gamma_2 \vdash e : S \\
   A;\Sigma; \Gamma_1, (x : U), \Gamma_2 \vdash S <: T \dashv \Gamma_1, (x : U), \Gamma_2}
  {A;\Sigma; \Gamma_1, (x : U), \Gamma_2 \vdash e \unlhd T : T}
\end{mathpar}
By our induction hypothesis we assume
\begin{mathpar}
\inferrule
  {[p\unlhd U/x]A;\Sigma; \Gamma_1, [p\unlhd U/x]\Gamma_2 \vdash [p\unlhd U/x]e : [p\unlhd U/x]S \\
   [p\unlhd U/x]A;\Sigma; \Gamma_1, [p\unlhd U/x]\Gamma_2 \vdash [p\unlhd U/x]S <: [p\unlhd U/x]T \dashv \Gamma_1, [p\unlhd U/x]\Gamma_2}
  {}
\end{mathpar}
It then follows by \textsc{T-Type} that
$[p\unlhd U/x]A;\Sigma; \Gamma_1, [p\unlhd U/x]\Gamma_2 \vdash [p\unlhd U/x]e \unlhd T : [p\unlhd U/x]T$.
\end{casethm}

\begin{casethm}[\textsc{S-Assume}]
\begin{mathpar}
\inferrule
  {(S <: T) \in A}
  {A; 	\Sigma; \Gamma_1, (x : U), \Gamma_2 \vdash S\; \texttt{<:}\; T \dashv \Gamma_1, (x : U), \Gamma_2'}
\end{mathpar}
Trivial.
\end{casethm}

\begin{casethm}[\textsc{S-Rec}]\label{case:subst:s-rec}
\begin{mathpar}
\inferrule
  {A; 	\Sigma; \Gamma_1, (x : U), \Gamma_2, z : \{z \Rightarrow \overline{\sigma}\} \vdash \overline{\sigma} <:\; \overline{\sigma}'  \dashv \Gamma_1, (x : U), \Gamma_2', z : \{z \Rightarrow \overline{\sigma}'\}}
  {A; 	\Sigma; \Gamma_1, (x : U), \Gamma_2 \vdash \{z \Rightarrow \overline{\sigma}\}\; <:\; \{z \Rightarrow \overline{\sigma}'\}\dashv \Gamma_1, (x : U), \Gamma_2'}
\end{mathpar}
By our induction hypothesis we assume
\begin{mathpar}
\inferrule
  {[p\unlhd U/x]A; \Sigma; \Gamma_1, [p\unlhd U/x]\Gamma_2, z : [p\unlhd U/x]\{z \Rightarrow \overline{\sigma}\} \vdash [p\unlhd U/x]\overline{\sigma} <:\; [p\unlhd U/x]\overline{\sigma}'  \dashv \Gamma_1, [p\unlhd U/x]\Gamma_2', z : [p\unlhd U/x]\{z \Rightarrow \overline{\sigma}'\}}
  {}
\end{mathpar}
It then follows by \textsc{S-Rec}
\begin{mathpar}
\inferrule
  {}
  {[p\unlhd U/x]A; 	\Sigma; \Gamma_1, [p\unlhd U/x]\Gamma_2 \vdash [p\unlhd U/x]\{z \Rightarrow \overline{\sigma}\}\; <:\; [p\unlhd U/x]\{z \Rightarrow \overline{\sigma}'\}\dashv \Gamma_1, [p\unlhd U/x]\Gamma_2'}
\end{mathpar}
\end{casethm}

\begin{casethm}[\textsc{S-Select-Refl}]
\begin{mdframed}[hidealllines=true,backgroundcolor=yellow]
Julian: This is the part I am most unsure of. The issue is the size of the 
sub-derivation. The premises for \textsc{S-Select-Refl} require the 
sub-derivation of a subtype relation with an expanded assumption context.
Now, my argument for this would be that neither of the base cases, 
\textsc{S-Top} and \textsc{S-Bottom} depend on the assumption context. 
The \textsc{S-Assume} base case does however, and this is where my 
brain fails me. Can we extend the induction hypothesis to premises not 
dealing with $A$, but rather the larger $A,(p'.L <: p'.L)$?

Since our induction is on the size of the judgement derivation, and not 
the size of the assumption context, I feel that this is alright.
I'm not sure if this logic holds up, or even if it does what the best way 
to express it is. This case is done assuming my logic holds.
\end{mdframed}
\begin{mathpar}
\inferrule
  {A; 	\Sigma; \Gamma_1, (x : U), \Gamma_2 \vdash p' \ni \texttt{type} \; L : S_1 .. U_1\\
  	A; 	\Sigma; \Gamma_1, (x : U), \Gamma_2' \vdash p' \ni \texttt{type} \; L : S_2 .. U_2\\
  	A,(p'.L <: p'.L); 	\Sigma; \Gamma_1, (x : U), \Gamma_2' \vdash S_2 \; \texttt{<:}\; S_1 \dashv \Gamma_1, (x : U), \Gamma_2 \\
  	A,(p'.L <: p'.L); 	\Sigma; \Gamma_1, (x : U), \Gamma_2 \vdash U_1 \; \texttt{<:}\; U_2 \dashv \Gamma_1, (x : U), \Gamma_2'}
  {A; 	\Sigma; \Gamma_1, (x : U), \Gamma_2 \vdash p'.L \; \texttt{<:}\; p'.L \dashv \Gamma_1, (x : U), \Gamma_2'}
\end{mathpar}
By our induction hypothesis we assume
\begin{mathpar}
\inferrule
  {[p\unlhd U/x]A; 	\Sigma; \Gamma_1, [p\unlhd U/x]\Gamma_2 \vdash p' \ni [p\unlhd U/x](\texttt{type} \; L : S_1 .. U_1)\\
  	[p\unlhd U/x]A; 	\Sigma; \Gamma_1, [p\unlhd U/x]\Gamma_2' \vdash p' \ni [p\unlhd U/x](\texttt{type} \; L : S_2 .. U_2)\\
  	[p\unlhd U/x](A,(p'.L <: p'.L)); 	\Sigma; \Gamma_1, [p\unlhd U/x]\Gamma_2' \vdash [p\unlhd U/x]S_2 \; \texttt{<:}\; [p\unlhd U/x]S_1 \dashv \Gamma_1, [p\unlhd U/x]\Gamma_2 \\
  	[p\unlhd U/x](A,(p'.L <: p'.L)); 	\Sigma; \Gamma_1, [p\unlhd U/x]\Gamma_2 \vdash [p\unlhd U/x]U_1 \; \texttt{<:}\; [p\unlhd U/x]U_2 \dashv \Gamma_1, [p\unlhd U/x]\Gamma_2'}
  {}
\end{mathpar}
Since 
\begin{mathpar}
\inferrule
  {[p\unlhd U/x](A,(p'.L <: p'.L)) = [p\unlhd U/x]A,([p\unlhd U/x]p'.L <: [p\unlhd U/x]p'.L)}
  {}
\end{mathpar}
it then follows by \textsc{S-Select-Refl},
\begin{mathpar}
\inferrule
  {}
  {[p\unlhd U/x]A; 	\Sigma; \Gamma_1, [p\unlhd U/x]\Gamma_2 \vdash [p\unlhd U/x]p'.L \; \texttt{<:}\; [p\unlhd U/x]p'.L \dashv \Gamma_1, [p\unlhd U/x]\Gamma_2'}
\end{mathpar}
\end{casethm}

\begin{casethm}[\textsc{S-Select-Upper}] 
\begin{mathpar}
\inferrule
  {A; 	\Sigma; \Gamma_1, (x : U), \Gamma_2 \vdash p' \ni \texttt{type} \; L : S' .. U'\\
  	A; 	\Sigma; \Gamma_1, (x : U), \Gamma_2 \vdash S' <: U' \dashv \Gamma_1, (x : U), \Gamma_2 \\
  	A; 	\Sigma; \Gamma_1, (x : U), \Gamma_2 \vdash U' <: T \dashv \Gamma_1, (x : U), \Gamma_2'}
  {A; 	\Sigma; \Gamma_1, (x : U), \Gamma_2 \vdash p'.L\; <:\; T \dashv \Gamma_1, (x : U), \Gamma_2'}
\end{mathpar}
By our induction hypothesis we assume
\begin{mathpar}
\inferrule
  {[p\unlhd U/x]A; 	\Sigma; \Gamma_1, [p\unlhd U/x]\Gamma_2 \vdash [p\unlhd U/x]p' \ni [p\unlhd U/x](\texttt{type} \; L : S' .. U')\\
  	[p\unlhd U/x]A; 	\Sigma; \Gamma_1, [p\unlhd U/x]\Gamma_2 \vdash [p\unlhd U/x]S' <: [p\unlhd U/x]U' \dashv \Gamma_1, [p\unlhd U/x]\Gamma_2 \\
  	[p\unlhd U/x]A; 	\Sigma; \Gamma_1, [p\unlhd U/x]\Gamma_2 \vdash [p\unlhd U/x]U' <: [p\unlhd U/x]T \dashv \Gamma_1, [p\unlhd U/x]\Gamma_2'}
  {}
\end{mathpar}
It then follows by \textsc{S-Select-Upper},
\begin{mathpar}
\inferrule
  {}
  {[p\unlhd U/x]A; 	\Sigma; \Gamma_1, [p\unlhd U/x]\Gamma_2 \vdash [p\unlhd U/x]p'.L\; <:\; [p\unlhd U/x]T \dashv \Gamma_1, [p\unlhd U/x]\Gamma_2'}
\end{mathpar}
\end{casethm}

\begin{casethm}[\textsc{S-Select-Lower}]
\begin{mathpar}
\inferrule
  {A; 	\Sigma; \Gamma_1, (x : U), \Gamma_2' \vdash p' \ni \texttt{type} \; L : S' .. U' \\
  	A; 	\Sigma; \Gamma_1, (x : U), \Gamma_2' \vdash S' <: U' \dashv \Gamma_1, (x : U), \Gamma_2' \\
  	A; 	\Sigma; \Gamma_1, (x : U), \Gamma_2 \vdash S <: S' \dashv \Gamma_1, (x : U), \Gamma_2'}
  {A; 	\Sigma; \Gamma_1, (x : U), \Gamma_2 \vdash S\; <:\; p'.L \dashv \Gamma_1, (x : U), \Gamma_2'}
\end{mathpar}
By our induction hypothesis we assume
\begin{mathpar}
\inferrule
  {[p\unlhd U/x]A; 	\Sigma; \Gamma_1, [p\unlhd U/x]\Gamma_2' \vdash [p\unlhd U/x]p' \ni [p\unlhd U/x](\texttt{type} \; L : S' .. U') \\
  	[p\unlhd U/x]A; 	\Sigma; \Gamma_1, [p\unlhd U/x]\Gamma_2' \vdash [p\unlhd U/x]S' <: [p\unlhd U/x]U' \dashv \Gamma_1, [p\unlhd U/x]\Gamma_2' \\
  	[p\unlhd U/x]A; 	\Sigma; \Gamma_1, [p\unlhd U/x]\Gamma_2 \vdash [p\unlhd U/x]S <: [p\unlhd U/x]S' \dashv \Gamma_1, [p\unlhd U/x]\Gamma_2'}
  {}
\end{mathpar}
It then follows by \textsc{S-Select-Lower},
\begin{mathpar}
\inferrule
  {}
  {[p\unlhd U/x]A; 	\Sigma; \Gamma_1, [p\unlhd U/x]\Gamma_2 \vdash [p\unlhd U/x]S'\; <:\; [p\unlhd U/x]p.L \dashv \Gamma_1, [p\unlhd U/x]\Gamma_2'}
\end{mathpar}
\end{casethm}

\begin{casethm}[\textsc{S-Top}]
\begin{mathpar}
\inferrule
  {}
  {A; 	\Sigma; \Gamma_1, (x : U), \Gamma_2 \vdash T\; \texttt{<:}\; \top \dashv \Gamma_1, (x : U), \Gamma_2'}
\end{mathpar}
Trivial.
\end{casethm}

\begin{casethm}[\textsc{S-Bottom}]
\begin{mathpar}
\inferrule
  {}
  {A; 	\Sigma; \Gamma_1, (x : U), \Gamma_2 \vdash \bot\; \texttt{<:}\; T \dashv \Gamma_1, (x : U), \Gamma_2'}
\end{mathpar}
Trivial.
\end{casethm}

\begin{casethm}[\textsc{M-Path}]
\begin{mathpar}
\inferrule
  {A; 	\Sigma; \Gamma_1, (x : U), \Gamma_2 \vdash p' : T \\
  	A; 	\Sigma; \Gamma_1, (x : U), \Gamma_2 \vdash T \prec_z \overline{\sigma}\\
  	\sigma_i \in \overline{\sigma}}
  {A; 	\Sigma; \Gamma_1, (x : U), \Gamma_2 \vdash p' \ni [p'/z]\sigma_i}
\end{mathpar}
By our induction hypothesis we assume
\begin{mathpar}
\inferrule
  {[p\unlhd U/x]A; 	\Sigma; \Gamma_1, [p\unlhd U/x]\Gamma_2 \vdash [p\unlhd U/x]p' : [p\unlhd U/x]T \\
  	[p\unlhd U/x]A; 	\Sigma; \Gamma_1, [p\unlhd U/x]\Gamma_2 \vdash [p\unlhd U/x]T \prec_z [p\unlhd U/x]\overline{\sigma}\\
  	[p\unlhd U/x]\sigma_i \in [p\unlhd U/x]\overline{\sigma}}
  {}
\end{mathpar}
It then follows by \textsc{M-Path} that
\begin{mathpar}
\inferrule
  {}
  {[p\unlhd U/x]A; 	\Sigma; \Gamma_1, [p\unlhd U/x]\Gamma_2 \vdash [p\unlhd U/x]p' \ni [[p\unlhd U/x]p'/z]([p\unlhd U/x]\sigma_i)}
\end{mathpar}
Since $[[p\unlhd U/x]p'/z]([p\unlhd U/x]\sigma_i) = [p\unlhd U/x]([p'/z]\sigma_i)$
we have the desired result. \hl{Do we? ... Substitution?}
\end{casethm}

\begin{casethm}[\textsc{M-Exp}]
\begin{mathpar}
\inferrule
  {A; 	\Sigma; \Gamma_1, (x : U), \Gamma_2 \vdash e : T \\
  	A; 	\Sigma; \Gamma_1, (x : U), \Gamma_2 \vdash T \prec_z \overline{\sigma}\\
  	\sigma_i \in \overline{\sigma} \\
  	z \notin \sigma_i}
  {A; 	\Sigma; \Gamma_1, (x : U), \Gamma_2 \vdash e \ni \sigma_i}
\end{mathpar}
By our induction hypothesis we assume
\begin{mathpar}
\inferrule
  {[p\unlhd U/x]A; 	\Sigma; \Gamma_1, [p\unlhd U/x]\Gamma_2 \vdash [p\unlhd U/x]e : [p\unlhd U/x]T \\
  	[p\unlhd U/x]A; 	\Sigma; \Gamma_1, [p\unlhd U/x]\Gamma_2 \vdash [p\unlhd U/x]T \prec_z [p\unlhd U/x]\overline{\sigma}\\
  	[p\unlhd U/x]\sigma_i \in [p\unlhd U/x]\overline{\sigma} \\
  	z \notin [p\unlhd U/x]\sigma_i}
  {}
\end{mathpar}
It then follows by \textsc{M-Exp} that
\begin{mathpar}
\inferrule
  {}
  {[p\unlhd U/x]A; 	\Sigma; \Gamma_1, [p\unlhd U/x]\Gamma_2 \vdash [p\unlhd U/x]e \ni [p\unlhd U/x]\sigma_i}
\end{mathpar}
\hl{Work needed to show $z$ is not in $[p\unlhd U/x]$}
\end{casethm}

\begin{casethm}[E-Rec]
\begin{mathpar}
\inferrule
  {}
  {A; 	\Sigma; \Gamma_1, (x : U), \Gamma_2 \vdash \{z \Rightarrow \overline{\sigma}\} \prec_z \overline{\sigma}}
\end{mathpar}
Trivial.
\end{casethm}

\begin{casethm}[E-Rec]
\begin{mathpar}
\inferrule
  {A; 	\Sigma; \Gamma_1, (x : U), \Gamma_2 \vdash p \ni \texttt{type} \; L : S..U \\
  	A; 	\Sigma; \Gamma_1, (x : U), \Gamma_2 \vdash U \prec_z \overline{\sigma}}
  {A; 	\Sigma; \Gamma_1, (x : U), \Gamma_2 \vdash p.L \prec_z \overline{\sigma}}
\end{mathpar}
\end{casethm}

\begin{casethm}[E-Rec]
\begin{mathpar}
\inferrule
  {}
  {A; 	\Sigma; \Gamma_1, (x : U), \Gamma_2 \vdash \top \prec_z \varnothing}
\end{mathpar}
Trivial.
\end{casethm}

\end{proof}
\qed

\newpage

\begin{lemma}[Subtype Transitivity] \label{lem:subtype_trans}
If $A; \Sigma; \Gamma_1 \vdash S <: T \dashv \Gamma_2$ and 
   $A; \Sigma; \Gamma_2 \vdash T <: U \dashv \Gamma_3$ then
   $A; \Sigma; \Gamma_1 \vdash S <: U \dashv \Gamma_3$.
\end{lemma}
\begin{proof}
By induction on the derivation of $A; \Sigma; \Gamma_1 \vdash S <: T \dashv \Gamma_2$.
\begin{casethm}[\textsc{S-Assume}]
\begin{mathpar}
\inferrule
  {(S <: T) \in A}
  {A; \Sigma; \Gamma_1 \vdash S\; \texttt{<:}\; T \dashv \Gamma_2}
\end{mathpar}
\end{casethm}

\begin{casethm}[\textsc{S-Rec}]
\begin{mathpar}
\inferrule
  {S = \{z \Rightarrow \overline{\sigma}_1\} \\
  	T = \{z \Rightarrow \overline{\sigma}_2\}}
  {}
  \and
\inferrule
  {A; \Sigma; \Gamma_1, z : \{z \Rightarrow \overline{\sigma}_1\} \vdash \overline{\sigma}_1 <:\; \overline{\sigma}_2  \dashv \Gamma_2, z : \{z \Rightarrow \overline{\sigma}_2\}}
  {A; \Sigma; \Gamma_1 \vdash \{z \Rightarrow \overline{\sigma}_1\}\; <:\; \{z \Rightarrow \overline{\sigma}_2\}\dashv \Gamma_2}
\end{mathpar}
\end{casethm}

\begin{casethm}[\textsc{S-Select-Refl}]
\begin{mathpar}
\inferrule
  {S = p.L \\
  	T = p.L}
  {}
  \and
\inferrule
  {A; \Sigma; \Gamma_1 \vdash p \ni \texttt{type} \; L : S_1 .. U_1\\
  	A; \Sigma; \Gamma_2 \vdash p \ni \texttt{type} \; L : S_2 .. U_2\\
  	A, (p.L <: p.L); \Sigma; \Gamma_2 \vdash S_2 \; \texttt{<:}\; S_1 \dashv \Gamma_1 \\
  	A, (p.L <: p.L); \Sigma; \Gamma_1 \vdash U_1 \; \texttt{<:}\; U_2 \dashv \Gamma_2}
  {A; \Sigma; \Gamma_1 \vdash p.L \; \texttt{<:}\; p.L \dashv \Gamma_2}
\end{mathpar}
\end{casethm}

\begin{casethm}[\textsc{S-Select-Upper}]
\begin{mathpar}
\inferrule
  {S = p.L}
  {}
  \and
\inferrule
  {A; \Sigma; \Gamma_1 \vdash p \ni \texttt{type} \; L : S_1 .. U_1\\
  	A; \Sigma; \Gamma_1 \vdash S_1 <: U_1 \dashv \Gamma_1 \\
  	A; \Sigma; \Gamma_1 \vdash U_1 <: T \dashv \Gamma_2}
  {A; \Sigma; \Gamma_1 \vdash p.L\; <:\; T \dashv \Gamma_2}
\end{mathpar}
Since $A; \Sigma; \Gamma_2 \vdash T <: U \dashv \Gamma_3$, 
by our inductive hypothesis, we assume that 
\begin{mathpar}
\inferrule
  {A; \Sigma; \Gamma_1 \vdash U_1 <: U \dashv \Gamma_2 }
  {}
\end{mathpar}
Thus, by \textsc{S-Select-Upper} it folows that 
$A; \Sigma; \Gamma_2 \vdash p.L <: U \dashv \Gamma_3$.
\end{casethm}

\begin{casethm}[\textsc{S-Select-Lower}]
\begin{mathpar}
\inferrule
  {T = p.L}
  {}
  \and
\inferrule
  {A; \Sigma; \Gamma_2 \vdash p \ni \texttt{type} \; L : S_2 .. U_2 \\
  	A; \Sigma; \Gamma_2 \vdash S_2 <: U_2 \dashv \Gamma_2 \\
  	A; \Sigma; \Gamma_1 \vdash S <: S_2 \dashv \Gamma_2}
  {A; \Sigma; \Gamma_1 \vdash S\; <:\; p.L \dashv \Gamma_2}
\end{mathpar}
By our inductive hypothesis, we assume
\begin{mathpar}
\inferrule
  {A; \Sigma; \Gamma_1 \vdash S_2 <: U \dashv \Gamma_2 }
  {}
\end{mathpar}
By case analysis on the the derivation of 
$A; \Sigma; \Gamma_2 \vdash p.L <: U \dashv \Gamma_3$.
\begin{subcase}[\textsc{S-Select-Refl}]
\begin{mathpar}
\inferrule
  {U = p.L}
  {}
  \and
\inferrule
  {A; \Sigma; \Gamma_2 \vdash p \ni \texttt{type} \; L : S_2 .. U_2\\
  	A; \Sigma; \Gamma_3 \vdash p \ni \texttt{type} \; L : S_3 .. U_3\\
  	A, (p.L <: p.L); \Sigma; \Gamma_3 \vdash S_3 \; \texttt{<:}\; S_2 \dashv \Gamma_2 \\
  	A, (p.L <: p.L); \Sigma; \Gamma_2 \vdash U_2 \; \texttt{<:}\; U_3 \dashv \Gamma_3}
  {A; \Sigma; \Gamma_1 \vdash p.L \; \texttt{<:}\; p.L \dashv \Gamma_3}
\end{mathpar}
\end{subcase}
\begin{subcase}[\textsc{S-Select-Upper}]
\begin{mathpar}
\inferrule
  {A; \Sigma; \Gamma_2 \vdash p \ni \texttt{type} \; L : S_2 .. U_2\\
  	A; \Sigma; \Gamma_2 \vdash S_2 <: U_2 \dashv \Gamma_1 \\
  	A; \Sigma; \Gamma_2 \vdash U_2 <: T \dashv \Gamma_3}
  {A; \Sigma; \Gamma_2 \vdash p.L\; <:\; T \dashv \Gamma_3}
\end{mathpar}
\end{subcase}
\begin{subcase}[\textsc{S-Select-Lower}]
\end{subcase}
\begin{subcase}[\textsc{S-Top}]
Trivial.
\end{subcase}
\end{casethm}

\begin{casethm}[\textsc{S-Top}]
\begin{mathpar}
\inferrule
  {T = \top}
  {}
  \and
\inferrule
  {}
  {A; \Sigma; \Gamma_1 \vdash S\; \texttt{<:}\; \top \dashv \Gamma_2}
\end{mathpar}
Since $T = \top$, by case analysis on the derivation of 
$A; \Sigma; \Gamma_2 \vdash \top <: U \dashv \Gamma_3$,
we can see that $U = \top$. Thus by \textsc{S-Top} it follows that
$A; \Sigma; \Gamma_1 \vdash S <: U \dashv \Gamma_3$.
\end{casethm}

\begin{casethm}[\textsc{S-Bottom}]
\begin{mathpar}
\inferrule
  {S = \bot}
  {}
  \and
\inferrule
  {}
  {A; \Sigma; \Gamma_1 \vdash \bot\; \texttt{<:}\; T \dashv \Gamma_2}
\end{mathpar}
\end{casethm}
Trivial.
\end{proof}
\qed

\newpage

\begin{lemma}[Leadsto type Preservation] \label{lem:path_type_preservation}
If $\varnothing; \Sigma; \Gamma \vdash v : T$, 
$\Sigma \vdash  \mu$ and $\mu \vdash v \leadsto l$ then 
$\varnothing; \Sigma; \Gamma \vdash l : S$ where
$\varnothing; \Sigma; \Gamma \vdash S <: T \dashv \Gamma$.
\end{lemma}
\begin{proof}
By induction on the derivation of $\mu \vdash v \leadsto l$.
\begin{casethm}[\textsc{L-Loc}]
\begin{mathpar}
\inferrule
  {v = l}
  {}
  \and
\inferrule
  {}
  {\mu \vdash l \leadsto l}
\end{mathpar}
Trivial.
\end{casethm}
\begin{casethm}[\textsc{L-Type}]
\begin{mathpar}
  {v = v' \unlhd T}
  {}
  \and
\inferrule
  {\mu \vdash v' \leadsto l}
  {\mu \vdash v' \unlhd T \leadsto l}
\end{mathpar}
Since we know that $\varnothing; \Sigma; \Gamma \vdash v' \unlhd T: T$, 
by \textsc{T-Type} we can infer $\varnothing; \Sigma; \Gamma \vdash v' : S'$ 
and $\varnothing; \Sigma; \Gamma \vdash S' <: T \dashv \Gamma$.
By our induction hypothesis, we assume that if 
$\varnothing; \Sigma; \Gamma \vdash v' : S'$ then 
$\varnothing; \Sigma; \Gamma \vdash l : S$, where 
$\varnothing; \Sigma; \Gamma \vdash S <: S' \dashv \Gamma$.
By \hl{Subtype Transitivity} we have 
$\varnothing; \Sigma; \Gamma \vdash S <: T \dashv \Gamma$.
\end{casethm}
\begin{casethm}[\textsc{L-Path}]
\begin{mathpar}
\inferrule
  {\mu \vdash v' \leadsto l' \\
	\mu(l') = \{z \Rightarrow ..., \texttt{val} f : T' = l, ...\}}
  {\mu \vdash v'.f \leadsto l}
\end{mathpar}
By inversion on the derivation of $\Gamma; \Sigma \vdash v'.f : T$ we 
have 
\begin{mathpar}
\inferrule
  {	\varnothing; \Sigma; \Gamma \vdash v' : S' \\
  	\varnothing; \Sigma; \Gamma \vdash v' \ni \texttt{val} \; f:T}
  {	\varnothing; \Sigma; \Gamma \vdash v'.f : T}
\end{mathpar}
By assumption $\Sigma \vdash \mu$, which implies $\varnothing; \Sigma; \varnothing 
\vdash l : S$, where 
$\varnothing; \Sigma; \varnothing \vdash S <: T \dashv \varnothing$. 
Environment weakening gives us 
$\varnothing; \Sigma; \Gamma \vdash l : T$
and $\varnothing; \Sigma; \Gamma \vdash S <: S' \dashv \Gamma$
\hl{TODO: expansion preservation}
\end{casethm}
\end{proof}
\qed

\newpage

\begin{theorem}[Preservation]
If $\varnothing; \Sigma; \Gamma \vdash e : T$, 
   	$\mu \; | \; e \; \rightarrow \mu' \; | \; e'$ where
	$\Sigma \vdash \mu \; \tt{\bf{wf}}$ then 
 	$\exists \Sigma'$ s.t. 
	$\Sigma'$ extends $\Sigma$, 
	$\Sigma' \vdash \mu' \; \tt{\bf{wf}}$, 
	$\varnothing; \Sigma'; \Gamma \vdash e' : T$.
\end{theorem}
\begin{proof}
By structural induction on 
$\mu \; | \; e \; \rightarrow \mu' \; | \; e'$.
\begin{casethm}[\textsc{R-New}]
\begin{mathpar}
\inferrule
  {\mu \vdash \overline{d_v} \leadsto \overline{d} \\
  	l \notin dom(\mu) \\
  	\mu' = \mu, l \mapsto \{\texttt{z} \Rightarrow \overline{d}\}}
  {\mu \; | \; \texttt{new} \; \{\texttt{z} \Rightarrow \overline{d_v}\} \; \rightarrow \mu' \; | \; l}
\end{mathpar}
Trivial.
\end{casethm}

\begin{casethm}[\textsc{R-Meth}]
\begin{mathpar}
\inferrule
  {\mu \vdash v_1 \leadsto l \\
  	\mu(l) = \{\texttt{z} \Rightarrow ...,m:T(x:S)=e,...\}}
  {\mu \; | \; v_1.m_U(v_2) \;\rightarrow \mu \; | \; [l/\texttt{z},v_2 \unlhd S/x]e \unlhd U}
\end{mathpar}
Trivial.
\end{casethm}

\begin{casethm}[\textsc{R-Context}]
\begin{mathpar}
\inferrule
  {	\mu \; | \; e \; \rightarrow \; \mu' \; | \; e'}
  {\mu \; | \; E[e] \; \rightarrow \mu' \; | \; E[e']}
\end{mathpar}
\end{casethm}

\end{proof}
\qed

\newpage

\section{Abstract}





\bibliographystyle{plain}
\bibliography{bib}

\end{document}

\end{document}