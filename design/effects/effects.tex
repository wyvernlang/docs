\documentclass{llncs}

\usepackage{listings}
\usepackage{proof}
\usepackage{amssymb}
\usepackage[margin=.9in]{geometry}
\usepackage{amsmath}
\usepackage[english]{babel}
\usepackage[utf8]{inputenc}
\usepackage{enumitem}
\usepackage{filecontents}
\usepackage{calc}
\usepackage[linewidth=0.5pt]{mdframed}
\usepackage{changepage}
\allowdisplaybreaks

\usepackage{fancyhdr}
\renewcommand{\headrulewidth}{0pt}
\pagestyle{fancy}
 \fancyhf{}
\rhead{\thepage}

\lstset{tabsize=3, basicstyle=\ttfamily\small, commentstyle=\itshape\rmfamily, numbers=left, numberstyle=\tiny, language=java,moredelim=[il][\sffamily]{?},mathescape=true,showspaces=false,showstringspaces=false,columns=fullflexible,xleftmargin=5pt,escapeinside={(@}{@)}, morekeywords=[1]{objtype,module,import,let,in,fn,var,type,rec,fold,unfold,letrec,alloc,ref,application,policy,external,component,connects,to,meth,val,where,return,group,by,within,count,connect,with,attr,html,head,title,style,body,div,keyword,unit,def}}
\lstloadlanguages{Java,VBScript,XML,HTML}

\newcommand{\keywadj}[1]{\mathtt{#1}}
\newcommand{\kwa}[1]
{\keywadj{#1}}
\newcommand{\keyw}[1]{\keywadj{#1}~}
\newcommand{\kw}[1]
{\keyw{#1}~}
\newcommand{\reftt}{\mathtt{ref}~}
\newcommand{\Reftt}{\mathtt{Ref}~}
\newcommand{\inttt}{\mathtt{int}~}
\newcommand{\Inttt}{\mathtt{Int}~}
\newcommand{\stepsto}{\leadsto}
\newcommand{\todo}[1]{\textbf{[#1]}}
\newcommand{\intuition}[1]{#1}
\newcommand{\hyphen}{\hbox{-}}

%\newcommand{\intuition}[1]{}

\newlist{pcases}{enumerate}{1}
\setlist[pcases]{
  label=\fbox{\textit{Case}}\protect\thiscase\textit{:}~,
  ref=\arabic*,
  align=left,
  labelsep=0pt,
  leftmargin=0pt,
  labelwidth=0pt,
  parsep=0pt
}
\newcommand{\pcase}[1][]{

  \if\relax\detokenize{#1}\relax
    \def\thiscase{}
  \else
    \def\thiscase{~\fbox{#1:}}
  \fi
  \item
}

\newcommand{\thm}[3]{
	\begin{large}
		\bf{#1}
	\end{large} \\\\
	\fbox{Statement.} ~ #2
	\fbox{Proof.}~ #3 \qed
}

\newcommand{\proofcase}[2]{
	\begin{adjustwidth}{1.5em}{0pt}
		\fbox{Case.}~~#1. \\ ~#2
	\end{adjustwidth}
}

\newcommand{\subcase}[1] {
	\begin{adjustwidth}{2.2em}{0pt}
		\underline{Subcase.} #1
	\end{adjustwidth}
}

\newcommand{\stmt}[1] {

\begin{adjustwidth}{2.5em}{0pt}
	#1
\end{adjustwidth}

}
\newcommand{\type}[2]{
	#1~\keyw{with} #2
}

\newcommand{\newd}[0]{
	\keywadj{new}_d~x \Rightarrow \overline{d = e}
}

\newcommand{\newsig}[0]{
	\keywadj{new}_\sigma~x \Rightarrow \overline{\sigma = e}
}

\newcommand{\defn}[1]{
	\textbf{Definition. (#1)~~}
}

\begin{document}

\today

\section{Stanford Encyclopaedia of Philosophy}

\subsection{Extensionality}

\noindent
Accessed: 02/09/2016 \\

\noindent
\defn{Extension} The thing to which a reference points. For a term, this is exactly the referent; for the predicate $P$, it is the set of things for which $P$ holds; for the sentence, it is the truth-value. \\

\noindent
\defn{Intension} The sense or meaning of a reference provided by a semantics; the aspect of the semantics determining its extension. \\

\noindent
\defn{Logic} A formal language with a semantics, a semantics being a theory which provides rigorous definitions of truth, validity, and logical consequence inside the language. A logic is \textit{extensional} if the truth-value of every sentence is determined entirely by its form; otherwise it is \textit{intentional}. Extensional logics obey the \textit{substitution principle}, while intensional logics do not (because truth appeals to something higher than form). \\

\noindent
Propositional/predicate logic are extensional. Suppose the following sentences are true:

\begin{enumerate}
	\item ``All John's dogs are mammals'': $\forall x (D x \rightarrow Mx)$.
	\item ``All John's pets are mammals: $\forall x (P x \rightarrow Mx)$.
\end{enumerate}

\noindent
Because both are true, they have the same extension. Therefore, we may replace one with the other inside a third sentence, and truth will be preserved. Once we add modalities, this no longer holds.

\begin{enumerate}
	\item ``Necessarily, all of John's dogs are mammals'': $\Box (\forall x (Dx \rightarrow Mx))$
	\item ``Necessarily, all of John's pets are mammals'': $\Box (\forall x (Px \rightarrow Mx))$
\end{enumerate}

\noindent
The first sentence is true, but the second is not. This shows modal logic is intensional, and that the substitution principle does not hold.

\subsection{Possible Worlds}

To regain extensionality, possible world semantics was developed, which describes the semantics of modal sentences in terms of first-order logic (which is extensional). Modal operators are interpreted as quantifiers over possible worlds.

\begin{itemize}
	\item $\Box \phi$ is true if and only if $\phi$ is true in every possible world.
	\item $\Diamond \phi$ is true if and only if $\phi$ is true in some possible world.
\end{itemize}


\end{document}

